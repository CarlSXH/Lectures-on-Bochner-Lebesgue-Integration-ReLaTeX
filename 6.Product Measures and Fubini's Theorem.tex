

Throughout this chapter we will let $(X,S,\mu)$ and $(Y,T,\nu)$ denote measure spaces with $\mu$ and $\nu$ non-negative (extended) real-valued. The objective of this chapter is to show how to define a measure, $\mu\otimes\nu$, on the $\sigma$-ring of subsets of $X\times Y$ generated by the sets $E\times F$, $E\in S$, $F\in T$, which has the property that \[(\mu\otimes\nu)(E\times F)=\mu(E)\nu(F)\] for all $E\in S,F\in T$. We then derive some relations between integration with respect to $\mu\otimes\nu$ and integration with respect to $\mu$ and $\nu$.

\section{Product Measures}

\begin{definition}
We will let $S\times T$ denote the family of subsets of $X\times Y$ of the form $E\times F$ where $E\in S$ and $F\in T$. The elements of $S\times T$ will be called the \defline{measurable rectangles} of $X\times Y$. We will denote the $\sigma$-ring generated by $S\times T$ by $S\otimes T$. The pair $(X\times Y,S\otimes T)$ will be called the \defline{product measurable space} of the measurable spaces $(X,S)$ and $(Y,T)$.
\end{definition}

If we let $\rho$ be the function on $S\times T$ defined by $\rho(E\times F)=\mu(E)\nu(F)$ (with the convention that $0\cdot\infty=0$), the objective of this section is to show that we can extend $\rho$ to a measure on $S\otimes T$. We do this by showing that the results of Chapter \ref{ch:measures}, particularly Theorem \ref{thm:measure extension thm}, are applicable to the set function $\rho$. In order to do this it will be useful to have the definition and some properties of ``sections''.

\begin{definition}
If $A\subseteq X\times Y$ and if $x\in X$, then the \defline{$x$-section of $A$}, which is denoted by $\xsec{A}{x}$, is defined by \[\xsec{A}{x}=\brc{y\in Y:(x,y)\in A}.\] Note that an $x$-section is a subset of $Y$. Similarly, if $y\in Y$, then the \defline{$y$-section of $A$}, denoted by $\ysec{A}{y}$, is defined by \[\ysec{A}{y}=\brc{x\in X:(x,y)\in A}.\] It is, of course, a subset of $X$.
\end{definition}

\begin{proposition}\label{prop:section preserve union intersection}
Let $A_n$ be a sequence of subsets of $X\times Y$. Then $\xsec{\br{\bigcup_{n=1}^\infty A_n}}{x}=\bigcup_{n=1}^\infty\xsec{A_n}{x}$, $\xsec{\br{\bigcup_{n=1}^\infty A_n}}{x}=\bigcup_{n=1}^\infty\xsec{A_n}{x}$, and $\xsec{(A_1\sd A_2)}{x}=\xsec{A_1}{x}\sd\xsec{A_2}{x}$. Similar results hold for $y$-sections.
\end{proposition}

\begin{proof}
Let $x^j$ denote the function from $Y$ to $X\times Y$ defined by $x^j(y)=(x, y)$. For any $A\subseteq X\times Y$ it is clear that $\xsec{A}{x}=(x^j)^{-1}(A)$. The proposition now follows immediately from the fact that the formation of preimages preserves unions, intersections and differences.
\end{proof}


\begin{definition}
If $h$ is a function on $X\times Y$ and if $x\in X$, then the \defline{$x$-section of $h$}, which is denoted by $\xsec{h}{x}$, is the function on $Y$ defined by $(\xsec{h}{x})(y)=h(x, y)$. The definition of \defline{$y$-sections of $h$}, which we denote by $\ysec{h}{y}$, is similar.
\end{definition}

\begin{proposition}\label{prop:section basic properties}
\begin{enumerate}
    \item\label{prop:item:section linear}
    If $g$ and $h$ are $B$-valued functions on $X\times Y$ and if $r$ and $s$ are scalars, then for all $x\in X$ we have \[\xsec{(rg+sh)}{x}=r(\xsec{g}{x})+s(\xsec{h}{x}).\]

    \item\label{prop:item:section limit}
    If $h_n$ is a sequence of functions on $X\times Y$ which converges to $h$ pointwise, then for every $x$ the sequence $\xsec{h_n}{x}$ converges to $\xsec{h}{x}$ pointwise.
    
    \item\label{prop:item:section of idf}
    If $A\subseteq X\times Y$, then $\xsec{(\idf{A})}{x}=\idf{(\xsec{A}{x})}$
\end{enumerate}
Similar results hold for $y$-sections
\end{proposition}

\begin{proof}
The proposition follows immediately from the observation that if $h$ is a function on $X\times Y$ then $\xsec{h}{x}=h\circ x^j$, where $x^j$ is the function defined in the proof of Proposition \ref{prop:section preserve union intersection}.
\end{proof}

\begin{proposition}
\begin{enumerate}
    \item\label{prop:item:sections of sets meas}
    If $A\in S\otimes T$, then $\xsec{A}{x}\in T$ for every $x\in X$, and $\ysec{A}{y}\in S$ for every $y\in Y$.
    
    \item If $h$ is an $(S\otimes T)$-measurable function on $X\times Y$, then $\xsec{h}{x}$ is $T$-measurable for every $x\in X$ and $\ysec{h}{y}$ is $S$-measurable for every $y\in Y$.
\end{enumerate}
\end{proposition}

\begin{proof}
Let $R$ be the set of those $A\in S\otimes T$ such that $\xsec{A}{x}\in T$ for every $x\in X$ and $\ysec{A}{y}\in S$ for every $y\in Y$. Clearly $S\times T\subseteq R$, and $R$ is a $\sigma$-ring by Proposition \ref{prop:section preserve union intersection}. Therefore $R=S\otimes T$. Turning to the second part of the proposition, it follows from Proposition \ref{prop:item:section linear} and \ref{prop:item:section of idf} that $x$-sections of simple $(S\otimes T)$-measurable functions are simple $T$-measurable functions, and similarly for $y$-sections. The second part of the proposition now follows from Proposition \ref{prop:item:section limit} and the definition of measurable functions.
\end{proof}

\begin{lemma}
$S\times T$ is a semiring.
\end{lemma}

\begin{proof}
This is a consequence of the following easily verified set theoretic equalities.
\begin{enumerate}
    \item $\bigcap_{n=1}^\infty(E_n\times F_n)=\br{\bigcap_{n=1}^\infty E_n}\times\br{\bigcap_{n=1}^\infty F_n}$.
    
    \item
    \begin{align*}
        E_1\times F_1\sd E_2\times F_2&=\sbr{(E_1\sd E_2)\times F_1}\du[(E_1\cap E_2)\times(F_1\sd F_2)]\\
        &=[E_1\times(F_1\sd F_2)]\du[(E_1\sd E_2)\times(F_1\cap F_2)]
    \end{align*}
\end{enumerate}
\end{proof}

\begin{theorem}\label{thm:product premeasure is premeasure}
Let $\mu$ and $\nu$ be arbitrary non-negative measures. Then the set function $\rho$ defined on $S\times T$ by $\rho(E\times F)=\mu(E)\nu(F)$, (with the convention that $0\cdot\infty=0$) is a premeasure, and so extends to a non-negative measure on $S\otimes T$.
\end{theorem}

\begin{proof}
We must show that $\rho$ is countably additive. Suppose that $E\times F=\bigdu_{n=1}^\infty E_n\times F_n$, where $E\times F$ and the $E_n\times F_n$ are in $S\times T$. We must show that $\rho(E\times F)=\sum_{n=1}^\infty\rho(E_n\times F_n)$. For each $m$ let $A_m=\bigdu_{n=1}^mE_n\times F_n$, and let $A=E\times F$. Then $A_m$ increases to $A$, and so $\xsec{\idf{A_m}}{x}$ increases to $\xsec{\idf{A}}{x}$ for each $x$. If we use Definition \ref{def:extended real meas integrable} for the integral of non-negative functions which are not necessarily integrable, and apply Corollary \ref{cor:mct extended real}, we find that $\int\xsec{\idf{A_m}}{x}\dd\nu$ increases to $\int\xsec{\idf{A}}{x}\dd\nu$ for each $x$. Evaluating these integrals, we find that $\sum_{n=1}^m\idf{E_n}(x)\nu(F_n)$ increases to $\idf{E}(x)\nu(F)$ for each $x$ (where we must again use the convention that $0\cdot\infty=0$). Note that these functions may now take the value $\infty$, but are clearly measurable in the sense of Definition \ref{def:extended real meas integrable}. Furthermore it is easily seen that if we apply the definition of the integral given in Definition \ref{def:extended real meas integrable}, then we find that $\int\idf{E}\nu(F)\dd\mu=\rho(E\times F)$ and $\int\sum_{n=1}^m\idf{E_n}\nu(F_n)\dd\mu=\sum_{n=1}^m\rho(E_n\times F_n)$. Applying Corollary \ref{cor:mct extended real} again, we thus conclude that $\sum_{n=1}^m\rho(E_n\times F_n)$ converges to $\rho(E\times F)$ as $m$ goes to $\infty$, so that $\rho$ is a premeasure. Theorem \ref{thm:measure extension thm} is now applicable.
\end{proof}

We invite the reader to contemplate the possibility of a proof of Theorem \ref{thm:product premeasure is premeasure} which does not use integration theory (in the form of the Monotone Convergence Theorem) but only results of Chapter \ref{ch:measures}.

\begin{definition}
The measure obtained from $\rho$ by applying Theorem \ref{thm:measure extension thm} is called the \defline{product} of the measures $\mu$ and $\nu$. We will denote it by $\mu\otimes\nu$. The measure space $(X\times Y, S\otimes T,\mu\otimes\nu)$ is called the \defline{product} of the measure spaces $(X,S,\mu)$ and $(Y,T,\nu)$.
\end{definition}

In order to know that the extension of $\rho$ to a measure on $S\otimes T$ is unique (by applying Theorem \ref{thm:unique extension}) we must know that $\rho$ is $\sigma$-finite.

\begin{proposition}\label{prop:sigma finite product}
If $\mu$ and $\nu$ are both $\sigma$-finite, then so is $\rho$.
\end{proposition}

\begin{proof}
Let $E\times F\in S\times T$. Since $\mu$ and $\nu$ are assumed $\sigma$-finite, we have $E=\bigcup_{m=1}^\infty E_m$ and $F=\bigcup_{n=1}^\infty F_n$ where the $E_m$ and the $F_n$ have finite measure. Then $E\times F\subseteq\bigcup_{m,n}E_m\times F_n$ and $\rho(E_m\times F_n)<\infty$ for all $m$ and $n$.
\end{proof}

From Proposition \ref{prop:sigma finite product} and Theorem \ref{thm:unique extension} we immediately obtain:

\begin{proposition}
If $\mu$ and $\nu$ are $\sigma$-finite then so is $\mu\otimes\nu$, and it is the only measure on $S\otimes T$ which has the property that $(\mu\otimes\nu)(E\times F)=\mu(E)\nu(F)$ for all $E\in S$ and $F\in T$.
\end{proposition}

\begin{corollary}
If $\mu$ and $\nu$ are arbitrary non-negative measures, and if $E$ and $F$ are $\sigma$-finite elements of $S$ and $T$ respectively, then the product of the restrictions of $\mu$ and $\nu$ to $E$ and $F$ respectively coincides with the restriction of $\mu\otimes\nu$ to $E\times F$.
\end{corollary} 

\begin{proposition}\label{prop:totally finite product}
If $\mu$ and $\nu$ are both finite (or totally finite, or totally $\sigma$-finite), then so is $\mu\otimes\nu$.
\end{proposition}

\begin{proof}
Note that every element of $S\otimes T$ is contained in a measurable rectangle (since the collection of sets having this property is clearly a $\sigma$-ring containing $S\times T$). So if $A\in S\otimes T$, let $A\subseteq E\times F$ for some $E\in S$ and $F\in T$. Then if $\mu$ and $\nu$ are finite $(\mu\otimes\nu)(A)\leq(\mu\otimes\nu)(E\times F)=\mu(E)\nu(F)<\infty$. In particular if $\mu(X)<\infty$ and $\nu(Y)<\infty$ then $(\mu\otimes\nu)(X\times Y)<\infty$, and similarly in the $\sigma$-finite case.
\end{proof}

\section{Fubini's Theorem}

In this section we examine the relation between integration with respect to $\mu\otimes\nu$ and integration with respect to $\mu$ and $\nu$. The main result is known as Fubini's theorem.

In order to prove the main theorems we will need to make some finiteness assumptions. For this reason it is sufficient in the preliminary lemmas to assume that the measures $\mu$ and $\nu$ are finite.

Let $R$ denote the collection of all finite disjoint unions of elements of $S\times T$. Then from Corollary \ref{cor:ring generated by semiring} we see that $R$ is just the ring generated by $S\times T$. The first step in the proof of Fubini's theorem is to prove a version of it just for this ring.

\begin{lemma}\label{lem:fubini finite union}
Let $\mu$ and $\nu$ be finite. If $A$ is any element of $R$ (the ring generated by $S\times T$) then
\begin{enumerate}[label=\arabic*),ref=\arabic*)]
    \item\label{lem:item:sections of ring integrable}
    $\xsec{\idf{A}}{x}$ and $\ysec{\idf{A}}{y}$ are measurable, in fact integrable, functions with respect to $\nu$ and $\mu$ respectively,

    \item\label{lem:item:integral of section of ring integrable}
    $\int\xsec{\idf{A}}{x}\dd\nu$ and $\int\ysec{\idf{A}}{y}\dd\mu$ are measurable, in fact integrable, functions of $x$ and $y$ respectively, and

    \item $\int\br{\int\xsec{\idf{A}}{x}\dd\nu}\dd\mu(x)=\int\idf{A}\dd(\mu\otimes\nu)=\int\br{\int\ysec{\idf{A}}{y}\dd\mu}\dd\nu(y)$.
\end{enumerate}
\end{lemma}

\begin{proof}
Let $A=\bigdu_{i=1}^nE_i\times F_i$ where $E_i\times F_i\in S\times T$ for each $i$. It follows that $\idf{A}(x,y)=\sum_{i=1}^n\idf{E_i}(x)\idf{F_i}(y)$ for all $x$ and $y$. From this it is clear that each section is an ISF, which proves \ref{lem:item:sections of ring integrable}. It is also clear that $\int\xsec{\idf{A}}{x}\dd\nu=\sum_{i=1}^n\idf{E_i}(x)\nu(F_i)$, and similarly for the integral with respect to $\mu$, and so they too are both ISF, proving \ref{lem:item:integral of section of ring integrable}. Finally, it is now clear that both iterated integrals turn out to be $\sum_{i=1}^n\mu(E_i)\nu(F_i)$, that is, $\int\idf{A}\dd(\mu\otimes\nu)$.
\end{proof}

The main step in the proof of Fubini's theorem is to extend Lemma \ref{lem:fubini finite union} to the case in which $A$ is an arbitrary element of $S\otimes T$. But it is sufficient to still work with finite measures.

\begin{keylemma}\label{lem:fubini idf}
Let $\mu$ and $\nu$ be finite and let $A\in S\otimes T$. Then
\begin{enumerate}[label=\arabic*),ref=\arabic*)]
    \item\label{lem:item:sections of sigma ring integrable}
    $\xsec{\idf{A}}{x}$ and $\ysec{\idf{A}}{y}$ are measurable, in fact integrable, functions with respect to $\mu$ and $\nu$ respectively.

    \item\label{lem:item:integral of section of sigma ring integrable}
    $\int\xsec{\idf{A}}{x}\dd\nu$ and $\int\ysec{\idf{A}}{y}\dd\mu$ are measurable, in fact integrable, functions of $x$ and $y$ respectively.

    \item\label{lem:item:iterated sigma ring integral}
    $\int\br{\int\xsec{\idf{A}}{x}\dd\nu}\dd\mu(x)=\int\idf{A}\dd(\mu\otimes\nu)=\int\br{\int\ysec{\idf{A}}{y}\dd\mu}\dd\nu(y)$.
\end{enumerate}
\end{keylemma}

\begin{proof}
By Proposition \ref{prop:item:section of idf} and \ref{prop:item:sections of sets meas}, the functions in \ref{lem:item:sections of sigma ring integrable} are characteristic functions of measurable sets, and so, since $\mu$ and $\nu$ are finite measures, they are integrable, This proves property \ref{lem:item:sections of sigma ring integrable}.

Note that because the integrands in property \ref{lem:item:integral of section of sigma ring integrable} are characteristic functions and the measures are finite, the two functions in property \ref{lem:item:integral of section of sigma ring integrable} are bounded. Thus if we can prove that these functions are measurable, it will follow that they are integrable.

Let $M$ be the collection of all sets $A\in S\otimes T$ for which properties \ref{lem:item:sections of sigma ring integrable}, \ref{lem:item:integral of section of sigma ring integrable} and \ref{lem:item:iterated sigma ring integral} are true. We wish to show that $M=S\otimes T$. Let $R$ be the set of finite disjoint unions of elements of $S\times T$. (so that by Corollary \ref{cor:ring generated by semiring} $R$ is the ring generated by $S\times T$). Then Lemma \ref{lem:fubini finite union} says exactly that $R\subseteq M$. Let us investigate further the properties of $M$.

Suppose that $A_n$ is a sequence of elements of $M$, and that $A_n$ decreases to a set $A$. Then $A\in S\otimes T$ and so property \ref{lem:item:sections of sigma ring integrable} holds for $A$. Now $\xsec{\idf{A_n}}{x}$ and $\ysec{\idf{A_n}}{y}$ will decrease to $\xsec{\idf{A}}{x}$ and $\ysec{\idf{A}}{y}$, and so we can apply Corollary \ref{cor:mct} of the Monotone Convergence Theorem to conclude that
\begin{align*}
\int\xsec{\idf{A_n}}{x}\dd\nu&\text { decreases to }\int\xsec{\idf{A}}{x}\dd\nu\text{ and}\\
\int\ysec{\idf{A_n}}{y}\dd\mu&\text { decreases to }\int\ysec{\idf{A}}{y}\dd\mu.
\end{align*}
Since property \ref{lem:item:integral of section of sigma ring integrable} holds for the sets $A_n$, it follows that the functions $\int\xsec{\idf{A}}{x}\dd\nu$ and $\int\ysec{\idf{A}}{y}\dd\mu$ are measurable. Thus property \ref{lem:item:integral of section of sigma ring integrable} holds for $A$ also. Using Corollary \ref{cor:mct} again, we see that
\begin{align*}
\int\br{\int\xsec{\idf{A_n}}{x}\dd\nu}\dd\mu(x)&\text { decreases to }\int\br{\int\xsec{\idf{A}}{x}\dd\nu}\dd\mu(x)\text{ and}\\
\int\br{\int\ysec{\idf{A_n}}{y}\dd\mu}\dd\nu(y)&\text { decreases to }\int\br{\int\ysec{\idf{A}}{y}\dd\mu}\dd\nu(y).
\end{align*}
Since property \ref{lem:item:iterated sigma ring integral} holds for the sets $A_n$, the left hand sides are just $\int\idf{A_n}\dd(\mu\otimes\nu)$. But $\int\idf{A_n}\dd(\mu\otimes\nu)$ decreases to $\int\idf{A}\dd(\mu\otimes\nu)$ by Proposition 4.43, and so property \ref{lem:item:iterated sigma ring integral} holds for $A$ also. A similar proof shows that if instead the sequence $A_n$ increases to $A$, then again $A\in M$.

We are thus led to make the following definition:

\begin{definition}
A collection $M$ of sets is called a \defline{monotone class} if it is closed under the formation of countable increasing unions and countable decreasing intersections.
\end{definition}

Thus what we have shown above is that the collection $M$ of the proof of Lemma \ref{lem:fubini idf} is a monotone class which contains the ring generated by the semiring $S\times T$. It is then clear that the proof of Lemma \ref{lem:fubini idf} will be completed once we have proven the following lemma:

\begin{lemma}[The Lemma on Monotone Classes]
Let $M$ be a monotone class and let $P$ be a semiring. If $M$ contains the ring generated by $P$, then $M$ contains the $\sigma$-ring generated by $P$.
\end{lemma}

\begin{proof}
Note first that the intersection of any collection of monotone classes is again a monotone class, so that any collection of sets is contained in a smallest monotone class, which it is said to generate. Thus we can (and will) assume that $M$ is the monotone class generated by the ring, $R$, generated by $P$. Thus we wish to show that $M$ coincides with the $\sigma$-ring, $\sring{P}$, generated by $P$. For this it suffices to show that $M$ is a $\sigma$-ring, and for this it suffices to show that $M$ is closed under taking differences and finite unions (since $\bigcup_{n=1}^\infty F_n=\bigcup_{m=1}^\infty\br{\bigcup_{n=1}^mF_n}$ which is an increasing union).

Thus for each $E\in M$ define a subset, $K(E)$, of $M$ by \[K(E)=\brc{F\in M: E\sd F, F\sd E,E\cup F\in M}.\] %I changed
What we need to show is that $K(E)=M$ for all $E\in M$. We divide the proof of this fact into 6 short steps.
\begin{enumerate}[label=\arabic*),ref=\arabic*)]
    \item\label{lem:mcl s1}
    If $F\in K(E)$ then $E\in K(F)$. This is clear from the definitions of $K(E)$ and $K(F)$.
    
    \item\label{lem:mcl s2}
    If $E\in R$ then $R\subseteq K(E)$. This is clear from the definition of a ring.
    
    \item\label{lem:mcl s3}
    $K(E)$ is a monotone class for every $E\in M$. To see this, suppose that $F_n$ is a sequence of elements of $K(E)$ which increases to a set $F$. Then $F_n\sd E$ increases to $F\sd E, E\sd F_n$ decreases to $E\sd F$, and $F_n\cup E$ increases to $F\cup E$, so that $F\sd E, E\sd F$ and $E\cup F$ are in $M$. Thus $F\in K(E)$. A similar argument works if instead the sequence $F_n$ decreases to $F$.

    \item\label{lem:mcl s4}
    If $E\in R$ then $K(E)=M$. This follows from \ref{lem:mcl s2} and \ref{lem:mcl s3} and the fact that we have assumed that $M$ is the monotone class generated by $R$.

    \item\label{lem:mcl s5}
    $R\subseteq K(E)$ for all $E\in M$. This follows from \ref{lem:mcl s4} and \ref{lem:mcl s1}.

    \item $K(E)=M$ for all $E\in M$. This follows from \ref{lem:mcl s5} and \ref{lem:mcl s3}.
\end{enumerate}
\end{proof}
%I changed
Thus Lemma \ref{lem:fubini idf} is proved.
\end{proof}

The next step in the proof of Fubini's theorem is to extend Lemma \ref{lem:fubini idf} to the case of non-negative measurable functions. We do not need to assume that $\mu$ and $\nu$ are finite any more, but we do need to assume that they are $\sigma$-finite. %(see exercise). %FIX where is the fucking exercise REEEEEEEEEEE
Because we will want to use Corollary \ref{cor:mct} of the Monotone convergence theorem we need the following fact:

\begin{proposition}\label{prop:meas as increase limit}
If $f$ is a non-negative measurable function, then there is an increasing sequence of non-negative simple measurable functions which converges to $f$ pointwise.
\end{proposition}

\begin{proof}
Define $f_n$ by \[f_n(x)=\begin{cases}(k-1)/2^n&\text{if }(k-1)/2^n\leq f(x)<k/2^n, k=1,\dots,n2^n\\n&\text{if }n\leq f(x)\end{cases}.\]
\end{proof}

\begin{theorem}[(Fubini, Tonelli)]\label{thm:fubini pos}
Let $f$ be a non-negative $(S\otimes T)$-measurable function. Then the following conditions are equivalent:
\begin{enumerate}[label=\arabic*)]
    \item\label{thm:cond:fubini pos integrable}
    $f$ is integrable,

    \item\label{thm:cond:fubini pos integrable x sections}
    $\xsec{f}{x}$ is integrable for almost all $x\in X$, and $x\mapsto\int\xsec{f}{x}\dd\nu$ is an (almost everywhere defined) $\mu$-integrable function on $X$,

    \item\label{thm:cond:fubini pos integrable y sections}
    $\ysec{f}{y}$ is integrable for almost all $y\in Y$, and $y\mapsto\int\ysec{f}{y}\dd\mu$ is an (almost everywhere defined) $\nu$-integrable function on $Y$.
\end{enumerate}
If any of these three conditions holds then \[\int\br{\int\xsec{f}{x}\dd\nu}\dd\mu=\int f\dd(\mu\otimes\nu)=\int\br{\int\ysec{f}{y}\dd\mu}\dd\nu.\]
\end{theorem}

\begin{proof}
Since $f$ is a measurable function, $\car{f}$ is an $(S\otimes T)$-measurable set. Since we are assuming that the measures $\mu$ and $\nu$ are $\sigma$-finite, it is easy to see that there is a rectangle $E\times F\in S\times T$ and a sequence $E_n\times F_n$ of elements of $S\times T$ such that $\car{f}\subseteq E\times F$, the sequence $E_n\times F_n$ increases to $E\times F$, and $\mu(E_n)<\infty$, $\nu(F_n)<\infty$ for all $n$. Using Proposition \ref{prop:meas as increase limit} let $f_n$ be an increasing sequence of non-negative simple $(S\otimes T)$-measurable functions which converges to $f$ pointwise. If we let $g_n=\idf{E_n\times F_n}f_n$ for each $n$, then the $g_n$ form a sequence of non-negative ISF increasing to $f$ pointwise.

For the moment fix $n$, and consider $\mu$, $\nu$, and $\mu\otimes\nu$ restricted to $E_n$, $F_n$, and $E_n\times F_n$ respectively. According to Proposition \ref{prop:totally finite product} the product of the restrictions of $\mu$ and $\nu$ to $E_n$ and $F_n$ respectively coincides with $\mu\otimes\nu$ restricted to $E_n\times F_n$, and all these restrictions are finite. We can thus apply Key Lemma \ref{lem:fubini idf} to conclude that conditions \ref{thm:cond:fubini pos integrable x sections} and \ref{thm:cond:fubini pos integrable y sections} and the equality between integrals in Theorem \ref{thm:fubini pos} are true for the characteristic function of any measurable subset of $E_n\times F_n$, in fact true even with the qualification ``almost'' omitted. Now since $\car{g_n}\subseteq E_n\times F_n$ and $g_n$ is just a finite sum of characteristic functions of measurable subsets of $E_n\times F_n$, it follows that conditions \ref{thm:cond:fubini pos integrable x sections} and \ref{thm:cond:fubini pos integrable y sections} and the equality between integrals in Theorem \ref{thm:fubini pos} are true for $g_n$ also, again even with the qualification ``almost'' omitted. In particular, $\int\xsec{g_n}{x}\dd\nu$ is an integrable function of $x$ for each $n$, and the $\int\xsec{g_n}{x}\dd\nu$ form an increasing sequence of non-negative integrable functions.

Now if condition \ref{thm:cond:fubini pos integrable} holds, then \[\int\br{\int\xsec{g_n}{x}\dd\nu}\dd\mu=\int g_n\dd(\mu\otimes\nu)\leq\int f\dd(\mu\otimes\nu)\] for all $n$. On the other hand, if condition \ref{thm:cond:fubini pos integrable x sections} holds, then $\int\xsec{g_n}{x}\dd\nu\leq\int\xsec{f}{x}\dd\nu$ a.e., and so \[\int\br{\int\xsec{g_n}{x}\dd\nu}\dd\mu\leq\int\br{\int\xsec{f}{x}\dd\nu}\dd\mu\] for all $n$. Thus in either case the sequence of the $L^1$-norms of the functions $\int\xsec{g_n}{x}\dd\nu$ is bounded above, and so we can apply the Monotone Convergence Theorem (Theorem \ref{thm:mct}) to conclude that the sequence $\int\xsec{g_n}{x}\dd\nu$ converges a.e. and in mean to an integrable function, $h$. Let $N$ be the null set off of which $\int\xsec{g_n}{x}\dd\nu$ converges to $h(x)$ for all $x$. If $x\notin N$ then $\int\xsec{g_n}{x}\dd\nu\leq h(x)$ for all $n$, and so the sequence of the norms of the $\xsec{g_n}{x}$ is bounded. But $\xsec{g_n}{x}$ increases to $\xsec{f}{x}$, and so $\xsec{f}{x}$ is integrable and $\int\xsec{f}{x}\dd\nu=\lim_n\int\xsec{g_n}{x}\dd\nu$ for all $x\notin N$ by Corollary \ref{cor:mct}. (Thus we see that $N$ is exactly the set of those $x\in X$ for which $\xsec{f}{x}$ is not integrable.) It follows that $\int\xsec{f}{x}\dd\nu=h(x)$ for all $x\in N$, and in particular that the function $\int\xsec{f}{x}\dd\nu$ (defined off of $N$) is integrable. Thus if it is condition \ref{thm:cond:fubini pos integrable} which holds, then we see that we have shown that it follows that condition \ref{thm:cond:fubini pos integrable x sections} holds also.

Now since $\int\xsec{f}{x}\dd\nu=h(x)$ for $x\notin N$, it follows that the sequence of functions $\int\xsec{g_n}{x}\dd\nu$ converges to $\int\xsec{f}{x}\dd\nu$ in mean, so that \[\int\br{\int\xsec{f}{x}\dd\nu}\dd\mu=\lim\int\br{\int\xsec{g_n}{x}\dd\nu}\dd\mu=\lim\int g_n\dd(\mu\otimes\nu).\] Suppose now that it is condition \ref{thm:cond:fubini pos integrable x sections} which holds. Then we see from the above equation that the sequence of $L^1$-norms of the $g_n$ is bounded above, and so, since $g_n$ increases to $f$, that $f$ is integrable and $g_n$ converges to $f$ in mean by Corollary \ref{cor:mct}. In particular, condition \ref{thm:cond:fubini pos integrable} is seen to follow. Finally if either condition, so both, hold, we see that \[\int f\dd(\mu\otimes\nu)=\lim_n\int g_n\dd(\mu\otimes\nu)=\int\br{\int\xsec{f}{x}\dd\nu}\dd\mu.\]

Of course an entirely parallel argument shows that conditions \ref{thm:cond:fubini pos integrable} and \ref{thm:cond:fubini pos integrable y sections} are equivalent and that the second equality between integrals holds.
\end{proof}

We remark that the above proof could have been shortened somewhat by allowing functions to take the value $\infty$, and in particular by defining $\int\xsec{f}{x}\dd\nu$ to be $\infty$ whenever $\xsec{f}{x}$ is not integrable. But this seems to obscure somewhat the role of the null set $N$, which is an essential aspect of this theorem. Furthermore, the final version of Fubini's theorem which we shall give involves functions with values in a Banach space and again there will be a null set, $N$, on which $\xsec{f}{x}$ is not integrable. But in this setting it is no longer natural to set $\int\xsec{f}{x}\dd\nu$ equal to $\infty$ for $x\in N$. Thus even in the present setting we prefer simply to say that $\int\xsec{f}{x}\dd\nu$ is undefined for $x\in N$.

We also remark that the part of the above theorem which is usually called Tonelli's theorem is the fact that conditions \ref{thm:cond:fubini pos integrable x sections} or \ref{thm:cond:fubini pos integrable y sections} imply condition \ref{thm:cond:fubini pos integrable}. For applications this is by far the most useful method of trying to show that a measurable function on a product space is integrable. Notice that Tonelli's theorem is immediately applicable to functions with values in a Banach space, since if $f$ is such a function and is measurable, then to show that $f$ is integrable it suffices by Theorem \ref{thm:bochner characterization integrable} to show that the non-negative measurable function $\norm{f(\imarg)}$ is integrable. %A counter-example for Tonelli's theorem when one of the measures $\mu$ and $\nu$ is not $\sigma$-finite can be found in exercise.
%FIX no exercises REEEEEEE again

Notation which is more commonly used in stating Fubini's theorem than that which we have used above is as follows:

\begin{definition}
If $f$ is a measurable function on $X\times Y$, if $\int\xsec{f}{x}\dd\nu$ is defined a.e. and measurable, and if $\int\br{\int\xsec{f}{x}\dd\nu}\dd\mu$ exists, then we will write $\iint f(x,y)\dd\nu(y)\dd\mu(x)$ or $\iint f\dd\nu\dd\mu$ instead of $\int\br{\int\xsec{f}{x}\dd\nu}\dd\mu$. Under similar conditions we will write $\iint f(x,y)\dd\mu(x)\dd\nu(y)$ or $\iint f\dd\mu\dd\nu$ instead of $\int\br{\int\ysec{f}{y}\dd\mu}\dd\nu$. These expressions are called \defline{iterated integrals} to distinguish them from $\int f\dd(\mu\otimes\nu)$, which is called the \defline{double integral} of $f$.
\end{definition}

From Fubini's theorem we obtain the following convenient characterization of null sets with respect to $\mu\otimes\nu$:

\begin{corollary}\label{cor:product null sets}
If $C\in\nring{\mu\otimes\nu}$ (the collection of null sets for $\mu\otimes\nu$), then $\xsec{C}{x}\in\nring{\nu}$ for almost all $x\in X$ and $\ysec{C}{y}\in\nring{\mu}$ for almost all $y\in Y$. Conversely if $A\in S\otimes T$ and $\xsec{A}{x}\in\nring{\nu}$ a.e. (or $\ysec{A}{y}\in\nring{\mu}$ a.e.), then $A\in\nring{\mu\otimes\nu}$.
\end{corollary}

\begin{proof}
Since $C\in\nring{\mu\otimes\nu}$, there exists $A\in S\otimes T$. such that $C\subseteq A$ and $(\mu\otimes\nu)(A)=0$. By Theorem \ref{thm:fubini pos} we have \[\int\br{\int\xsec{\idf{A}}{x}\dd\nu}\dd\mu=\int\idf{A}\dd(\mu\otimes\nu)=(\mu\otimes\nu)(A)=0.\] Thus $\nu(\xsec{A}{x})=\int\xsec{\idf{A}}{x}\dd\nu=0$ for almost all $x\in X$. Since $\xsec{C}{x}\subseteq\xsec{A}{x}$ for all $x\in X$ we are done. The converse is clear.
\end{proof}

We now come to our final version of Fubini's theorem, which involves functions with values in a Banach space.

\begin{theorem}[Fubini]
If $\mu$ and $\nu$ are $\sigma$-finite non-negative measures and if $f$ is a $B$-valued $(\mu\otimes\nu)$-integrable function, then
\begin{enumerate}[label=\arabic*),ref=\arabic*)]
    \item\label{thm:item:fubini sections integrable}
    $\xsec{f}{x}$ and $\ysec{f}{y}$ are integrable functions for almost all $x\in X$ and almost all $y\in Y$,

    \item\label{thm:item:fubini integral of sections integrable}
    $\int\xsec{f}{x}\dd\nu$ and $\int\ysec{f}{y}\dd\mu$ are (almost everywhere defined) integrable functions of $x$ and $y$ respectively, and

    \item\label{thm:item:fubini integral}
    $\iint f\dd\nu\dd\mu=\int f\dd(\mu\otimes\nu)=\iint f\dd\mu\dd\nu$.
\end{enumerate}
\end{theorem}

\begin{proof}
Since $f$ is $(\mu\otimes\nu)$-integrable, $\norm{f(\imarg)}$ is a non-negative $(\mu\otimes\nu)$-integrable function. Choose (as in the proof of Theorem \ref{thm:bochner characterization integrable}) a sequence, $f_n$, of simple $(S\otimes T)$-measurable functions which converges to $f$ a.e. and such that $\norm{f_n}\leq 2\norm{f}$ for all $n$. Then by the Lebesgue Dominated Convergence Theorem (Theorem \ref{thm:dct}) the sequence $f_n$ is a mean Cauchy sequence of ISF which converges to $f$ in mean. In particular, \[\int f\dd(\mu\otimes\nu)=\lim_n\int f_n\dd(\mu\otimes\nu).\]

Now by Theorem \ref{thm:fubini pos},$\xsec{(\norm{f(\imarg)})}{x}=\norm{f(x,\imarg)}=\norm{\xsec{f}{x}}$ is a $\nu$-integrable function on $Y$ for almost all $x\in X$, say for $x\notin N_1\in\nring{\mu}$. It follows from Theorem \ref{thm:bochner characterization integrable} that part \ref{thm:item:fubini sections integrable} is proved. Let $C$ be the null set off of which $f_n$ converges to $f$ pointwise. By the definition of the $f_n$'s the sequence $\xsec{f_n}{x}$ converges to $\xsec{f}{x}$ for all $x\notin N_2$, where $N_2\in\nring{\mu}$ is the set of those $x\in X$ for which $\nu(\xsec{C}{x})\neq 0$ (see Corollary \ref{cor:product null sets}). Therefore, since $\norm{\xsec{f_n}{x}}\leq2\norm{\xsec{f}{x}}$, we can use the Lebesgue Dominated Convergence Theorem again to conclude that $\xsec{f_n}{x}$ and $\xsec{f}{x}$ are $\nu$-integrable and that $\int\xsec{f_n}{x}\dd\nu$ converges to $\int\xsec{f}{x}\dd\nu$ for all $x\notin N_1\cup N_2$. Now from Key Lemma \ref{lem:fubini idf} it follows (as in the proof of Theorem \ref{thm:fubini pos}) that the present theorem is true for ISF, and so in particular, $\int\xsec{f_n}{x}\dd\nu$ is a measurable (and, in fact, integrable) function on $X$. Since \[\norm{\int\xsec{f_n}{x}\dd\nu}\leq2\int\xsec{(\norm{f})}{x}\dd\nu\] for all $x\notin N_1$ and all $n$, and since by Theorem \ref{thm:fubini pos} the right hand side is a $\mu$-integrable function of $x$, we can again apply the Lebesgue Dominated Convergence Theorem to conclude that $\int\xsec{f}{x}\dd\nu$ is a $\mu$-integrable function on $X$ (proving part \ref{thm:item:fubini integral of sections integrable}), and that \[\iint f\dd\nu\dd\mu=\lim_n\iint f_n\dd\nu\dd\mu.\] But as was noted before, Fubini's Theorem holds for the ISF, and so \[\int f\dd(\mu\otimes\nu)=\lim_n\int f_n\dd(\mu\otimes\nu)=\lim_n\iint f_n\dd\nu\dd\mu.\] Thus part \ref{thm:item:fubini integral} and the theorem are proved.
\end{proof}

We remark again that in most applications of this theorem one will first have to invoke Tonelli's theorem to show that $f$ is integrable.

We conclude this chapter by indicating how to extend Fubini's theorem to the cases in which $\mu$ and $\nu$ are complex-valued measures. To begin with, when $\mu$ is a real measure, define $\mu^+$ and $\mu^-$ by \[\mu^+=\frac{|\mu|+\mu}2\text{ and }\mu^-=\frac{|\mu|-\mu}2.\] Then $\mu^+$and $\mu^-$are easily seen to be non-negative measures and $\mu=\mu^+-\mu^-$. Thus if both $\mu$ and $\nu$ are real measures it is natural to define $\mu\otimes\nu$ by \[\mu\otimes\nu=\mu^+\otimes\nu^+-\mu^+\otimes\nu^--\mu^-\otimes\nu^++\mu^-\otimes\nu^-.\] It is also easy to show that this is what the product measure should be, in the sense that it does the right thing on rectangles, and that by linearity Fubini's theorem is true with this definition of $\mu\otimes\nu$.

Similarly, if $\mu$ is a complex measure, define $\mu_r$ and $\mu_i$ by \[\mu_r=\frac{\mu+\overline{\mu}}2\text{ and }\mu_i=\frac{\mu-\overline{\mu}}{2i}\] (where the bar denotes complex conjugation and $\overline{\mu}$ is defined by $\overline{\mu}(E)=\overline{\mu(E)}$). Then $\mu_r$ and $\mu_i$ are real measures and $\mu=\mu_r+i\mu_i$. Thus if $\mu$ and $\nu$ are both complex measures, we define $\mu\otimes\nu$ by \[\mu\otimes\nu=\mu_r\otimes\nu_r-\mu_i\otimes\nu_i+i(\mu_r\otimes\nu_i+\mu_i\otimes\nu_r).\] Again it is easily seen that this is the appropriate product measure and that Fubini's theorem holds for it.