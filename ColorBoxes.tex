\usepackage[many]{tcolorbox}
\tcbuselibrary{theorems}
 
\makeatletter
\newcommand\notefont{\normalfont\sffamily}
\def\tcb@theo@title#1#2#3{%
    \ifdefempty{#2}{\setbox\z@=\hbox{#1}}{\setbox\z@=\tcb@theo@form{#1}{#2}}%
    \def\temp@a{#3}%
    \ifx\temp@a\@empty\relax%
    \unhbox\z@\kvtcb@terminatorsign%
    \else%
    \setbox\z@=\hbox{\unhbox\z@\kvtcb@separatorsign\ }%
    \hangindent\wd\z@%
    \hangafter=1%
    \mbox{\unhbox\z@}{\notefont\kvtcb@desc@delim@left#3\kvtcb@desc@delim@right\kvtcb@terminatorsign}%
    \fi%
}
\makeatother
 
\newtcbtheorem[number within = section]{theorem}{Theorem}{%
    colback=Red!7,
    colframe=Black!40!Red,
    sharp corners,
    theorem style = break,
    fonttitle = \bfseries\sffamily,
    coltitle = Black!40!Red,
    separator sign dash,
    terminator sign none,
    description delimiters none,
    boxrule = 1pt,
    parbox = false,
}{theorem}
 
\newtcbtheorem[use counter from = theorem]{definition}{Definition}{%
    colback=RoyalBlue!7,
    colframe=NavyBlue!60!RoyalBlue,
    theorem style = plain,
    fonttitle = \bfseries\sffamily,
    coltitle = NavyBlue!60!RoyalBlue,
    sharp corners,
    separator sign none,
    terminator sign dash,
    leftrule = 2pt,
    bottomrule = 0pt,
    toprule = 0pt,
    rightrule = 0pt,
    parbox = false,
}{def}
 
\newtcbtheorem{keyterms}{Key Terms}{%
    colback=ForestGreen!5,
    colframe=ForestGreen!70!black,
    theorem style = plain,
    fonttitle = \bfseries\sffamily,
    theorem name,
    coltitle = ForestGreen!70!black,
    sharp corners,
    separator sign none,
    terminator sign dash,
    leftrule = 2pt,
    bottomrule = 0pt,
    toprule = 0pt,
    rightrule = 0pt,
    parbox = false,
}{def}
 
\newtcbtheorem[use counter from = theorem]{example}{Example}{%
    colback=ForestGreen!5,
    colframe=ForestGreen!70!black,
    theorem style = break,
    fonttitle = \bfseries\sffamily,
    coltitle = ForestGreen!70!black,
    separator sign dash,
    terminator sign none,
    description delimiters none,
    boxrule = 0.8pt,
    parbox = false,
    breakable,
}{exam}
 
\let\homework\undefined
\newtcbtheorem{homework}{Homework}{%
    colback=Plum!12,
    colframe=BlueViolet!80!Black,
    theorem style = break,
    fonttitle = \bfseries\sffamily,
    coltitle = BlueViolet!80!Black,
    separator sign dash,
    terminator sign none,
    sharp corners,
    description delimiters none,
    boxrule = 0pt,
    theorem name,
    parbox = false,
}{hw}
 
\newtcbtheorem{homeworksolution}{Solution}{%
    colback=Plum!12,
    colframe=BlueViolet!80!Black,
    theorem style = plain,
    fonttitle = \bfseries\sffamily,
    theorem name,
    coltitle = BlueViolet!80!Black,
    sharp corners,
    separator sign none,
    terminator sign dash,
    leftrule = 2pt,
    bottomrule = 0pt,
    toprule = 0pt,
    rightrule = 0pt,
    parbox = false,
    breakable,
}{def}
 
\newtcbtheorem[use counter from = theorem]{lemma}{Lemma}{%
    colback=BurntOrange!8,
    colframe=RawSienna!50,
    sharp corners,
    theorem style = break,
    fonttitle = \bfseries\sffamily,
    coltitle = RawSienna,
    separator sign dash,
    terminator sign none,
    description delimiters none,
    boxrule = 1pt,
    parbox = false,
}{lem}
 
\newtcbtheorem[use counter from = theorem]{corollary}{Corollary}{%
    colback=BurntOrange!8,
    colframe=RawSienna!50,
    sharp corners,
    theorem style = break,
    fonttitle = \bfseries\sffamily,
    coltitle = RawSienna,
    separator sign dash,
    terminator sign none,
    description delimiters none,
    boxrule = 1pt,
    parbox = false,
}{lem}
 
\newtcbtheorem{remark}{Remark}{%
    colback=Black!4,
    colframe=Black,
    sharp corners,
    theorem style = plain,
    fonttitle = \bfseries\sffamily,
    theorem name,
    coltitle = Black,
    separator sign none,
    terminator sign dash,
    description delimiters none,
    boxrule = 0pt,
    parbox = false,
}{rem}
 
\newtcbtheorem[use counter from = theorem]{proposition}{Proposition}{%
    colback=Black!4,
    colframe=Black,
    sharp corners,
    theorem style = break,
    fonttitle = \bfseries\sffamily,
    coltitle = Black,
    separator sign dash,
    terminator sign none,
    description delimiters none,
    boxrule = 0pt,
    parbox = false,
}{prop}
 
\newtcbtheorem[use counter from = theorem]{fact}{Fact}{%
    colback=Black!4,
    colframe=Black,
    sharp corners,
    theorem style = break,
    fonttitle = \bfseries\sffamily,
    coltitle = Black,
    theorem name,
    separator sign dash,
    terminator sign none,
    description delimiters none,
    boxrule = 0pt,
    parbox = false,
}{prop}
 
\newtcbtheorem{question}{Question}{%
    colback=Plum!12,
    colframe=BlueViolet!80!Black,
    theorem style = break,
    fonttitle = \bfseries\sffamily,
    coltitle = BlueViolet!80!Black,
    separator sign dash,
    terminator sign none,
    description delimiters none,
    boxrule = 0.8pt,
    theorem name,
    parbox = false,
}{hw}
 
\newtcbtheorem{exercise}{Exercise}{%
    colback=Plum!12,
    colframe=BlueViolet!80!Black,
    theorem style = break,
    fonttitle = \bfseries\sffamily,
    coltitle = BlueViolet!80!Black,
    separator sign dash,
    terminator sign none,
    description delimiters none,
    boxrule = 0.8pt,
    parbox = false,
}{hw}
 
\newtcbtheorem{hwq}{Problem}{%
    colback=Black!4,
    colframe=Black,
    sharp corners,
    theorem style = break,
    fonttitle = \bfseries\sffamily,
    coltitle = Black,
    separator sign none,
    terminator sign none,
    description delimiters none,
    boxrule = 0pt,
    parbox = false,
}{hw}
 
\newtcbtheorem{hwb}{}{%
    colback=Black!4,
    colframe=Black,
    sharp corners,
    theorem style = break,
    fonttitle = \bfseries\sffamily,
    coltitle = Black,
    theorem name,
    separator sign none,
    terminator sign none,
    description delimiters none,
    boxrule = 0pt,
    parbox = false,
}{hw}
 
\newtheorem*{claim}{Claim}
\newtheorem{abuse}{Abuse of Notation}