
\section{Restrictions of Measures}

\begin{definition}
A pair $(X,S)$ is called a measurable space if $X$ is a set and $S$ is a $\sigma$-ring of subsets of $X$. A triple $(X,S,\mu)$ is called a measure space if $(X, S)$ is a measurable space and $\mu$ is a measure on $S$.
\end{definition}

\begin{definition}
Let $(X, S)$ be a measurable space. We say that $E \subseteq X$ is \underline{$S$-measurable} (or just \underline{measurable}) if $E \in S$. We say that $E \subseteq X$ is \underline{locally $S$-measurable} if $E\cap F \in S$ for all $F \in S$. If $\mu$ is a non-negative measure on $(X, S)$, then we say that $E \subseteq X$ is \underline{$\mu$-measurable} if $E \in S\du\nring{\mu}$ (which was defined in Proposition 1.42). We say that $I \subseteq X$ is \underline{locally $\mu$-measurable} if $A\cap F$ is $\mu$-measurable for every $\mu$-measurable set $F$.
\end{definition}

We remark that if $\mu$ is $\sigma$-finite, then the $\mu$-measurable sets are exactly the sets which are measurable with respect to the outer measure determined by $\mu$, as is seen from Theorem 1.43.

It is easily seen that if $(X, S)$ is a measurable space, then the family of all locally $S$-measurable sets is a $\sigma$-field, as is the family of all locally $\mu$-measurable sets if $\mu$ is a non-negative measure on $S$.

If $X$ is any set, $P$ is a collection of subsets of $X$, and $E \subseteq X$, then by $P \cap E$ we will mean the collection $\brc{F \cap E: F \in P}$. For example, if $S$ is a $\sigma$-ring and $E$ is a locally $S$-measurable set, then it is easily seen that $S \cap E$ will form a $\sigma$-ring which is contained in $S$. If $\mu$ is a measure on $S$, then we can obtain a measure on $S \cap E$ by restricting the domain of $\mu$ to $S \cap E$

\begin{definition}
If $\mu$ is a measure on $S$ and if $E$ is a locally $S$-measurable set, then the measure obtained by restricting the domain of $\mu$ to $S \cap E$ will be called \underline{the restriction of $\mu$ to $E$}.
\end{definition}

Given a measure $\mu'$ on $S \cap E$ we can enlarge its domain to obtain a measure $\mu$ on $S$ by letting $\mu(F)=\mu'(F \cap E)$ for all $F \in S$. Note that if we start with $\mu$ on $S$, restrict $\mu$ to $E$ and then enlarge back to a measure on $S$, we do not necessarily obtain $\mu$ back again. In particular, there will in general be many other ways of enlarging the domain of $\mu'$ to obtain a measure on $S$

\begin{proposition}
If $X$ is a set, $P$ is a family of subsets of $X$ and $E \subseteq X$, then $\sring{P \cap E}=\sring{P} \cap E$
\end{proposition}


\begin{proof}
Since $\sring{P} \cap E$ is a $\sigma$-ring which contains $P \cap E$, it follows that $\sring{P \cap E} \subseteq \sring{P} \cap E$. We must show the reverse inclusion. Let $T$ be the class of all sets of the form $A\du(B\sd E)$ where $A \in \sring{P \cap E}$ and $B \in \sring{P}$. Symbolically we may write $T=\sring{P \cap E}\du(\sring{P}\cap E')$. It is easy to verify that $T$ is a $\sigma$-ring. If $F \in P$, then the relation $F=(F \cap E) \du(F\sd E)$ and the fact that $F \cap E \in P \cap E \subseteq \sring{P \cap E}$ show that $F \in T$, and therefore that $P \subseteq T$. It follows that $\sring{P} \subseteq T$, and therefore that $\sring{P} \cap E \subseteq T \cap E$. Since, however, it is clear that $T \cap E=\sring{P \cap E}$, it follows that $\sring{P} \cap E \subset \sring{P \cap E}$.
\end{proof}

For example, this proposition shows that the two natural ways of defining the Borel sets in the interval $[0,1]$ coincide; we can either take the intersections with $[0,1]$ of the Borel sets in $\bR$, or we can apply directly to $[0,1]$ the definition of the Borel sets of a topological space.

\begin{corollary}
If $\mu$ is a $\sigma$-finite premeasure on a semiring $P$, if $E \in P$ and if $\bar{\mu}$ is the extension of $\mu$ to $\sring{P}$ while $\hat{\mu}$ is the extension to $\sring{P \cap E}$ of $\mu$ restricted to $P \cap E$, then $\hat{\mu}$ is just the restriction of $\bar{\mu}$ to $\sring{P} \cap E$
\end{corollary}

For example, this corollary shows that the two ways in which one might define Lebesgue measure on the interval $[0, 1)$ coincide.

\section{The Total Variation of a Measure}

In Chapter 1, most of the results which we proved involved non-negative measures. The purpose of this section is to define the total variation of an arbitrary measure. This gives us a way to obtain a non-negative measure which is closely related to a given arbitrary measure, and this will enable us to extend some of our earlier definitions and results about non-negative measures to arbitrary measures.

\begin{definition}
If $\mu$ is an arbitrary measure on a $\sigma$-ring $S$, then the total variation, $|\mu|$, of $\mu$ is defined by $$|\mu|(E)=\sup \brc{\sum_{i=1}^n\norm{\mu(E_i)}: E=\bigdu_{i=1}^nE_i, E_i\in S}$$ for each $E \in S$. (Of course, if $\mu$ is extended real-valued, we let $\norm{\infty}=\infty$.)
\end{definition}


\begin{theorem}
The total variation, $|\mu|$, of a measure $\mu$ is a non-negative measure.
\end{theorem}

\begin{proof}
It is clear from its definition that $|\mu|$ is a non-negative extended real valued function, on $S$. We remark that it is also clear that $|\mu|$ is monotone, that is, that if $E \subseteq F$, where $E, F \in S$, then $|\mu|(E)\leq|\mu|(F)$. To prove the theorem we need to show that $|\mu|$ is countably additive.

Suppose first that $E=F\du G$ with $F,G\in S$. If $F=\bigdu_{i=1}^mF_i$ and $G=\bigdu_{j=1}^nG_j$, then $E=\bigdu_{i=1}^mF_i\du\bigdu_{j=1}^nG_j$, and so $$|\mu|(E)\geq\sum_{i=1}^m\norm{\mu(F_i)}+\sum_{j=1}^n\norm{\mu(G_j)}.$$ It follows that $|\mu|(E)\geq|\mu|(F)+|\mu|(G)$. By induction it follows that if $E=\bigdu_{i=1}^nE_i$, then $|\mu|(E)\geq\sum_{i=1}^n|\mu|(E_i)$.

Suppose now that $E=\bigdu_{i=1}^\infty E_i$. Then $E\supseteq\bigdu_{i=1}^nE_i$ for all $n$, and so, since $|\mu|$ is monotone, $|\mu|(E)\geq|\mu|\br{\bigdu_{i=1}^nE_i}=\sum_{i=1}^n|\mu|(E_i)$ for all $n$. Thus $|\mu|(E)\geq\sum_{i=1}^\infty|\mu|(E_i)$.

To prove the opposite inequality, suppose that $E=\bigdu_{j=1}^\infty E_j$ and also that $E=\bigdu_{i=1}^nF_i$. Then
\begin{align*}
    \sum_{i=1}^n\norm{\mu(F_i)}&=\sum_{i=1}^n\norm{\mu\br{F_i\cap\bigdu_{j=1}^\infty E_j}}=\sum_{i=1}^n\norm{\mu\br{\bigdu_{j=1}^\infty F_i\cap E_j}}\\
    &=\sum_{i=1}^n\norm{\sum_{j=1}^\infty\mu(F_i\cap E_j)}\leq\sum_{i=1}^n\sum_{j=1}^\infty\norm{\mu(F_i\cap E_j)}\\
    &=\sum_{j=1}^\infty\sum_{i=1}^n\norm{\mu(F_i\cap E_j)}\leq\sum_{j=1}^\infty|\mu|(E_j),
\end{align*}
since $E_j=\bigdu_{i=1}^n(F_i\cap E_j)$ for each $j$. Thus $|\mu|(E)\leq\sum_{j=1}^\infty|\mu|(E_j)$, and so $|\mu|$ is countably additive.
\end{proof}

Note that $\norm{\mu(E)}\leq|\mu|(E)$ for all $E \in S$, so that $|\mu|$ may be thought of as a non-negative measure which in a sense dominates $\mu$. In exercise 1 at the end of this chapter you will be asked to show that $|\mu|$ is the smallest non-negative measure having this property.

As a result of Theorem 2.7 we can extend some of our earlier definitions which were originally made only for non-negative measures.

\begin{definition}
An arbitrary measure $\mu$ is said to be \underline{$\sigma$-finite } (\underline{totally $\sigma$-finite}) if $|\mu|$ is $\sigma$-finite (totally $\sigma$-finite).
\end{definition}

\begin{definition}
If $\mu$ is an arbitrary measure on a $\sigma$-ring $S$, then a set $E \subseteq X$ is said to be (\underline{locally}) \underline{$\mu$-measurable} if it is (locally) measurable (see Definition 2.2). A set $F \in \hring{S}$ is called a \underline{$\mu$-null set} (or just a \underline{null set}) if $|\mu|^*(F)=0$. We denote the family of $\mu$-null sets by $\nring{\mu}$.
\end{definition}

We remark again that Theorem 1.43 shows that if $\mu$ is $\sigma$-finite, then the $\mu$-measurable sets are exactly the sets which are measurable with respect to the outer measure determined by $|\mu|$.

\begin{definition}
An arbitrary measure $\mu$ is said to be \underline{complete} if every $\mu$-null set is in the domain of $\mu$.
\end{definition}

The fact that we have used the term $\mu$-measurable in Definition 2.9 suggests that we should be able to extend $\mu$ to a measure on the $\sigma$-ring of $\mu$-measurable sets. The following theorem, which generalizes Proposition 1.42, shows that this is in fact the case.

\begin{theorem}
Let $(X, S, \mu)$ be an arbitrary measure space. Define $\hat{\mu}$ on the $\sigma$-ring of $\mu$-measurable sets, $S\du\nring{\hat{\mu}}$, by $\hat{\mu}(E\du F)=\mu(E)$ where $E\in S$ and ${F} \in \nring{\mu}$. Then $\hat{\mu}$ is a well defined complete measure which extends $\mu$.
\end{theorem}

\begin{proof}
If $E \in S$ and $|\mu|(E)=0$, then $\mu(E)=0$. As a consequence, it is easily seen that the proof of Proposition 1.42 applies to this case also.
\end{proof}

\section{Bounded Measures}
\begin{definition}
A measure $\mu$ is said to be \underline{bounded} if it does not take the value $+\infty$, that is, if all its values are in a Banach space. If $\mu$ is bounded and if in addition $X\in S$ (so that $\norm{\mu(X)}<\infty$, then $\mu$ is said to be \underline{totally bounded}.
\end{definition}

We remark that a bounded measure need not be $\sigma$-finite. An example of such a measure will be given in exercise of Chapter 5.

The following proposition justifies the name "bounded".

\begin{proposition}
If $\mu$ is a bounded measure, then there exists a constant $K<\infty$ such that $\norm{\mu(E)}<K$ for all $E \in$ domain $\mu$.
\end{proposition}

\begin{proof}
Suppose not. Then for each $n$ we can find a measurable set $E_n$ such that $\norm{\mu(E_n)}\geq n$. Let $E=\bigcup_{n=1}^\infty E_n$, so that $E \in S$. By construction, $\mu$ is unbounded on $E$, that is, for each $a\in\bR$ there exists a measurable set $F \subseteq E$ such that $\norm{\mu(F)}\geq a$. If the $E_n$ were disjoint we would easily get a contradiction. We now construct three sequences of measurable sets, $F_n, G_n$ and $H_n$, by induction, so that the $G_n$ acts like the $E_n$ but in addition are disjoint, and so that $\mu$ is unbounded on each $H_n$. Choose $F_1 \subseteq E$ such that $\norm{\mu(F_1)}\geq\norm{\mu(E)}+1$. Then $\norm{\mu(F_1)}\geq1$ and $\norm{\mu(E\sd F_1)}\geq1$ (since $\norm{\mu(E\sd F_1)}=\norm{\mu(E)-\mu(F_1)}\geq|\norm{\mu(E)}-\norm{\mu(F_1)}|\geq1$.) It is easily seen that $\mu$ must be unbounded on either $F_1$ or $E\sd F$ since otherwise it would not be unbounded on $E$. Let $H_1$ be one of $F_1$ or $E\sd F_1$ in such a way that $\mu$ is unbounded on $H_1$, and let $G_1$ be the other of the two sets. We have thus defined $F_1$, $G_1$, and $H_1$. If $F_{n-1}$, $G_{n-1}$, and $H_{n-1}$ have been chosen, choose $F_n\subseteq H_{n-1}$ such that $\norm{\mu(F_n)}\geq\norm{\mu(H_{n-1})}+1$. Then, as before, $\norm{\mu(F_n)}\geq1$, $\norm{\mu(H_{n-1}\sd F_n)}\geq1$, and $\mu$ is unbounded on either $F_n$ or $H_{n-1}\sd F_n$. Let $H_n$ be one of $F_n$ or $H_{n-1}\sd F_n$ in such a way that $\mu$ is unbounded on $H_n$, and let $G_n$ be the other of the two sets. The important thing to notice is that not only is $\norm{\mu(G_n)}\geq1$ for each $n$, but also the $G_n$ are all disjoint. Let $G=\bigdu_{n=1}^\infty G_n$. Then $\mu(G)=\sum_{n=1}^\infty\mu(G_n)$, and so $\sum_{n=1}^\infty\mu(G_n)$ is a series which converges to a finite (since $\mu$ is assumed bounded) value, but this is impossible since $\norm{\mu(G_n)}\geq1$ for each $n$.
\end{proof}

\begin{definition}
Let $\mu$ be a measure on a $\sigma$-ring $S$. A locally $S$-measurable set $E$ is said to \underline{carry} $\mu$ if $\mu(E)=\mu(F\cap E)$ for all $F\in S$, that is, if $F\cap E=\varnothing$ implies that $\mu(F)=0$. We also sometimes say that $\mu$ \underline{lives} on $E$.
\end{definition}

\begin{proposition}
Let $\mu$ be a measure on a $\sigma$-ring S. If $\mu$ is a bounded measure, then there exists $E \in S$ on which $\mu$ lives.
\end{proposition}

\begin{proof}
By Proposition 2.13 we know that $\sup\brc{\norm{\mu(F)}: F \in S}<\infty$. We define a sequence, $E_n$, of elements of $S$ by induction. Choose $E_1 \in S$ so that $\norm{\mu(E_1)}\geq\frac12 \sup\brc{\norm{\mu(F)}: F \in S}$. If $E_{n-1}, \dots, E_{n-1}$ have been chosen, choose $E_{n}$ so that $E_n \cap \bigdu_{i=1}^{n-1}E_{i}=\varnothing$ and $\norm{\mu(E_n)}\geq \frac12\sup\brc{\norm{\mu(F)}: F\cap\bigdu_{i=1}^{n-1}E_i=\varnothing,F\in S}$. Let $E=\bigdu_{n=1}^{\infty} E_n \in S$. Then $\mu(E)=\sum_{n=1}^{\infty} \mu(E_{n})$, so the series converge to a finite value, and so $\norm{\mu(E_n)}$ converges to $0$ as $n$ goes to $\infty$. We show that $E$ carries $\mu$. Suppose that $F\in S$ and $F\cap E=\varnothing$. Then $F\cap E_n=\varnothing$ for each $n$, and so, by the definition of the $E_n$, we have $\norm{\mu(E_n)}\geq\frac12\norm{\mu(F)}$ for every $n$. Since the $\norm{\mu(E_n)}$ converges to $0$, it follows that $\mu(F)=0$.
\end{proof} %FIXME

In view of Proposition 2.15, it is natural to extend a bounded measure $\mu$ to a measure $\mu'$ on the $\sigma$-field of all locally measurable sets $F$ by setting $\mu'(F)=\mu(F\cap E)$, where $E$ carries $\mu$. It is then easily seen that if $\mu$ is $\sigma$-finite so is $\mu'$. Thus, in a sense, the only time we must work with $\sigma$-rings instead of $\sigma$-fields in order to preserve $\sigma$-finiteness is when we have an extended real valued measure.

If $\mu$ is a non-negative measure then it too can be extended to the $\sigma$-field of locally measurable sets $F$ by letting $\mu'(F)=\sup\brc{\mu(E):E\subseteq F,E\in\text{domain }\mu}$. However, in general this $\mu'$ will not be $\sigma$-finite even if $\mu$ is $\sigma$-finite, and so it is usually not useful to make this extension. 

\section{Convergence Properties of Measures}

The following two propositions will be very useful in later chapters.

\begin{proposition}

Let $\mu$ be a measure on a $\sigma$-ring $S$. If $\brc{E_n}_{n=1}^\infty$ is a sequence of elements of $S$, and if $E_n\uparrow E$, that is, if $E_n\subseteq E_{n+1}$ for each $n$ and $\bigcup_{n=1}^\infty E_n=E$, then $\mu(E_n)$ converges to $\mu(E)$ as $n$ goes to $\infty$. 
\end{proposition}

\begin{proof}
Clearly $E=E_1\du\bigdu_{i=1}^\infty(E_i\sd E_{i-1})$ and $E_n=E_1\du\bigdu_{i=1}^n(E_i-E_{i-1})$, and so
\begin{align*}
\mu(E)&=\mu(E_1) + \sum_{i=1}^\infty \mu(E_i-E_{i-1})\\
&=\lim_{n}\br{\mu(E_1)+\sum_{i=1}^n\mu(E_i-E_{i-1})}=\lim_n\mu(E_n)
\end{align*}
as desired.
\end{proof}

\begin{proposition}
Let $\mu$ be a measure on a $\sigma$-ring $S$. If $\brc{E_n}_{n=1}^\infty$ is a sequence of elements of $S$ such that $E_n\downarrow E$, that is, $E_n \supseteq E_{n+1}$ for each $n$ and $\bigcap_{n=1}^\infty E_n=E$, and if $\norm{\mu(E_k)}<\infty$ for some $k$, in case $\mu$ is an extended real-valued measure, then $\mu(E_n)$ converges to $\mu(E)$ as $n$ goes to $\infty$.
\end{proposition}

\begin{proof}
We can assume that $\norm{\mu(E_1)}<\infty$ since we can ignore a finite number of terms if we wish. But $(E_1\sd E_n)\uparrow(E_1\sd E)$, so by Proposition 2.16, $\mu(E_1\sd E_n)$ converges to $\mu(E_1\sd E)$ as $n$ goes to $\infty$. Thus, since $\mu(E_1)-\mu(E_n)=\mu(E_1\sd E_n)$ and $\mu(E_1)-\mu(E)=\mu(E_1\sd E)$, we find that $\mu(E_1)-\mu(E_n)$ converges to $\mu(E_1)-\mu(E)$ as $n$ goes to $\infty$. Since $\mu(E_1)$ is assumed finite, it follows that $\mu(E_n)$ converges to $\mu(E)$ as $n$ goes to $\infty$.
\end{proof}

Note that Proposition 2.17 is not true without the hypothesis that $\norm{\mu(E_n)}<\infty$ for some $n$. As an example, let $\mu$ be Lebesgue measure on $\mathbb{R}$ and let $E_n=[n, \infty)$. 

\section{Exercises}
\begin{enumerate}[label=\arabic*)]

\item If $\mu$ is a vector valued measure, show that $|\mu|$ is the smallest extended real valued measure such that $\norm{\mu(E)}\leq|\mu|(E)$ for all $E \in S$, that is, if $\nu$ is a non-negative measure on the domain of $\mu$ such that $\norm{\mu(E)}\leq\nu(E)$ for all $E$ in the domain $\mu$, then $|\mu|(E)\leq\nu(E)$ for all $E$ in the domain $\mu$

\item A measure $\mu$ is said to be of \underline{bounded variation} if $|\mu|$ is a bounded measure. Show that any measure with values in a finite dimensional Banach space is of bounded variation. (You may assume that $B=\bR^n$ with the usual Euclidean norm, since it can be shown that all norms on a finite dimensional vector space are equivalent, that is, any two norms, $\norm{\cdot}$ and $\norm{\cdot}_0$ satisfy $k\norm{\cdot}\leq\norm{\cdot}_0\leq K\norm{\cdot}$ for suitable constants $K$.)

\item Compute the total variations of those set functions in problem 1 of Chapter 1 which are measures. How does you result compare with problem 2 above?

\item Let $B$ be a Banach space and let $(X, S)$ be a measurable space. Then the collection of $B$-valued measures on $S$ forms a vector space when $\mu+\nu$ is defined by $(\mu+\nu)(E)=\mu(E)+\nu(E)$ for all $E \in S$, and $\alpha\mu$ is defined by $(\alpha \mu)(E)=\alpha(\mu(E))$ for $E \in S$. (The field of scalars for the vector space of measures is taken to be the same as the field scalars for B. Note that we cannot form a vector space in this way if we admit measures taking the value $+\infty$.) Let $M$ be the collection of all $B$-valued measures on $S$ which are of bounded variation. It is easy to see that $M$ is a subspace of the vector space of all $B$-valued measures on $S$. Furthermore, we can define a function $\norm{\cdot}$ from $M$ to $\bR$ by $$\norm{\mu}=\sup\brc{|\mu|(E):E\in S}.$$ Show that $\norm{\cdot}$ is a norm on $M$. (This is called the total variation norm.), and show that $(M,\norm{\cdot})$ is a Banach space. (Be sure to show that the set functions which you claim are measures really are.)

\item Two measures $\mu$ and $\nu$ on $(X, S)$ are said to be \underline{mutually singular} if there are disjoint locally measurable sets $E$, and $F$ such that $E$ carries $\mu$ and $F$ carries $\nu$.
\begin{enumerate}
\item Show that if $\mu$ and $\nu$ are mutually singular then so are $|\mu|$ and $|\nu|$, and that if $\mu$ and $\nu$ have bounded variation, then $\norm{\mu+\nu}=\norm{\mu}+\norm{\nu}=\norm{\mu-\nu}$.
\item Find Borel measures (i.e. measures on Borel sets) $\mu$ and $\nu$ on $[0,1]$ which are mutually singular but are not carried on disjoint \underline{closed} sets.
\end{enumerate}
\end{enumerate}