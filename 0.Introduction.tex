
\hspace*{0pt}\hfill------ {\it Marc Rieffel}

These notes are based on the first year graduate level course in measure theory which we have taught several times at the University of California, Berkeley. The aspect in which our treatment of measure theory differs from that given in most current texts is that, from the beginning, we develop the theory of the Bochner-Lebesgue integral, that is, of the integration of functions which have values in a Banach space. Our reason for wishing to introduce students to the Bochner-Lebesgue integral is, of course, that it is a useful tool in many areas of analysis. However, we would not consider it appropriate for first year graduate students to learn this material were it not for the fact that it requires scarcely more effort than is required to learn the usual theory of the integration of real-valued functions. We do not assume that students beginning the study of this material have ever knowingly seen a Banach space before, and in fact in these notes very little of the general theory of Banach spaces is developed. (For example, linear functionals are never mentioned.) Put another way, if a student wishes to substitute ``real or complex numbers'' for ``Banach space'' throughout the notes, they will obtain a fairly standard treatment of (real or complex valued) integration theory. Only a few proofs will simplify somewhat by taking advantage of the order properties of the real numbers. However, we do feel that treating the case of functions with values in a Banach space has the pedagogical advantage of forcing a certain amount of clarification of some of the ideas involved.

The main prerequisite for reading these notes is a good background in undergraduate analysis, in particular a knowledge of the basic theorems concerning completeness and compactness in metric spaces, and also the rudiments of the theory of vector spaces. It would also be helpful for motivational purposes if the reader had some slight previous exposure to the Lebesgue integral on the real line. We expected that such was the case with students taking the course on which these notes are based, and so we did not spend much time motivating the early material of the course. But, in fact, a fair number of the students did not have such a background, and they did not seem to suffer much from this situation. In particular, from a logical point of view these notes do not at all require such an exposure to the theory of the Lebesgue integral. Students taking the course on which these notes are based were also expected to take concurrently (or have taken) a course in general topology. But the material from such a course is not needed until the last chapter of these notes, which is concerned with integration on locally compact topological spaces, though in some of the earlier chapters we have included occasional inessential examples and exercises which depend on the material of such a course (as well as, in a few other exercises, an acquaintance with such topics as ordinal arithmetic or group theory).

The first two times we taught the course on which these notes are based the course lasted a semester, and we had no difficulty covering the material of these notes. The third time we taught the course it lasted only a quarter, and we were able to cover only the first seven chapters, and at that, we had to treat a few peripheral topics somewhat rapidly. Because the last four chapters are essentially independent, there is some flexibility as to what material is covered and in what order.

Anyone who is familiar with the exposition of measure theory given by Halmos in his book ``Measure Theory'' will be aware of our indebtness to him. In particular, the general lines of the exposition in these notes follow those of his book fairly closely. We have also been aided by chapter III of ``Linear Operators Vol. I'' by Dunford and Schwartz and ``Int\'{e}gration'' by Bourbaki, which are among the few books which contain an exposition of the Bochner-Lebesgue integral. Finally, the material on the Radon-Nikodym theorem as well as a fair number of the exercises were developed while we were writing our paper ``The Radon-Nikodym theorem for the Bochner integral'' (Tran. A.M.S. 131 (1968) 466-487), although these notes contain some improvements over the treatment of the Radon-Nikodym theorem given in that paper.



\hspace*{0pt}\hfill------ {\it Carl Sun}


I LaTeX'ed these notes during summer because some notation seem slightly strange (for example $\otimes$ for disjoint union instead of $\sqcup$) and some parts are just too hard to read, but also because I was sort of bored and wanted something to do. I actually didn't quite understand Bochner integrals when I took the course Math 202a during 2022 Fall at Berkeley. I thought it was too general to consider a $\sigma$-ring of measurable sets (where the whole set might not be measurable) and define the integral for Banach space valued functions. Now that the course is over, I sort of got a new appreciation for Bochner integrals. In some sense it's similar to how people first see the definition of a limit: even though it's long and convoluted, it's the bare minimum necessary to define the concept. In any case, this note took me way too long and way too many sleepless nights and around 40 hours to complete. I spent around 24 hours typing out all the chapters, 13 hours proof reading and adding clickable links, and around 3 hours for fixing miscellaneous issues and making the document look nicer. However, I do feel like I learned a lot about LaTeX in general, so I wouldn't say that this was a complete waste of time. 