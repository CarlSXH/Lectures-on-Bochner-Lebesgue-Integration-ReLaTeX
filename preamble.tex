
\usepackage{soul}

%\theorembodyfont{\rm}
\newtheoremstyle{definition}%             % Name
  {}%                                     % Space above
  {}%                                     % Space below
  {}%                                     % Body font
  {}%                                     % Indent amount
  {\bfseries}%                            % Theorem head font
  {.}%                                    % Punctuation after theorem head
  { }%                                    % Space after theorem head, ' ', or \newline
  {\underline{\thmname{#2}\thmnumber{  #1}\thmnote{: #3}}}% Theorem head spec (can be left empty, meaning `normal')
\theoremstyle{definition}
\newtheorem{definition}{Definition}[chapter]
\newtheorem{theorem}[definition]{Theorem}
\newtheorem{proposition}[definition]{Proposition}
\newtheorem{lemma}[definition]{Lemma}
\newtheorem{examples}[definition]{Examples}
\newtheorem{example}[definition]{Example}
\newtheorem{corollary}[definition]{Corollary}
\newtheorem{keylemma}[definition]{Key Lemma}
\newtheorem*{exercises}{Exercises}


\providecommand{\defline}[1]{{\ul{\it #1}}}%\ul}




\newcommand{\floor}[1]{\left\lfloor#1\right\rfloor}
\renewcommand{\abs}[1]{\left\lvert#1\right\rvert}
\newcommand{\br}[1]{\left(#1\right)}
\newcommand{\sbr}[1]{\left[#1\right]}
\newcommand{\brc}[1]{\left\{#1\right\}}
\newcommand{\brk}[1]{\left\langle#1\right\rangle}
\renewcommand{\bf}[1]{\mathbf{#1}}


\newcommand{\bC}{\mathbb{C}}
\newcommand{\bR}{\mathbb{R}}
\newcommand{\bQ}{\mathbb{Q}}
\newcommand{\bN}{\mathbb{N}}
\newcommand{\ep}{\varepsilon}

% not my favorite notations so I can change lol
%\newcommand{\setdiff}[2]{#1\setminus #2}

%set difference
\newcommand{\sd}{\setminus}
%complement of a set
\newcommand{\comp}[1]{{#1}^c}
%disjoint union
\newcommand{\du}{\sqcup}
%big disjoint union
\newcommand{\bigdu}{\bigsqcup}
%indicator function
\newcommand{\idf}[1]{{\raisebox{\depth}{$\chi$}}_{#1}}
%carrier of
\DeclareMathOperator{\carrier}{carrier}
\newcommand{\car}[1]{\carrier(#1)}
%essential range
\newcommand{\er}[2]{\text{er}_{#2}(#1)}
%curly L
\newcommand{\cL}{{\cal L}}
%curly F
\newcommand{\cF}{{\cal F}}
%symmetric difference
\newcommand{\symd}{\triangle}

%the dot for implicit argument
\newcommand{\imarg}{\,\cdot\,}

%x y section
\newcommand{\xsec}[2]{{}^{#2}#1}
\newcommand{\ysec}[2]{#1^{#2}}



% definition that should be flexible
% the ring generated by a collection of subsets
\newcommand{\sring}[1]{{\cal S}(#1)}
% the sigma-ring of hereditary subsets
\newcommand{\hring}[1]{{\cal H}(#1)}
% the sigma-ring of measurable sets of an outer measure
\newcommand{\mring}[1]{{\cal M}(#1)}
% the sigma-ring of null sets
\newcommand{\nring}[1]{{\cal N}(#1)}