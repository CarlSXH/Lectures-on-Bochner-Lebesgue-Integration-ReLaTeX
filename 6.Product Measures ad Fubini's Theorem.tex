
Note: Following Lemma 1.17 of Chapter $1$ we should have had:


The numbers of subsequent items in Chapter I must thus all be increased by 1 . In Chapter 6 references to items in Chapter $I$ will be numbered accordingly.

Also, pages $4.27$ and $4.28$ of Chapter 4 should be replaced by the following pages: non-decreasing, and are clearly bounded above by $c$, and so they form a Cauchy sequence. But $f_{n}-f_{m}$ is either positive a.e. or negative a.e., depending on whether or not $n$ is larger than $m$, and so we have

$$
\left\|f_{n}-f_{m}\right\|_{1}=\int f_{n}-f_{m}|\alpha| \mu||=\left|\int\left(f_{n}-f_{m}\right) d\right| \mu||=\left|\int f_{n} d\right| \mu\left|-\int f_{m} d\right| \mu|| .
$$

Thus $f_{n}$ is a mean Cauchy sequence also. Since $\mathcal{L}^{L}$ is complete, there exists an $f \in \mathcal{L}^{I}$ to which the sequence $f_{n}$ converges in mean, and hence in measure. But then by Corollary $3.27$ there must be a subsequence of the $f_{n}$ which converges to $f$ a.e. Since the $f_{n}$ are nondecreasing, it follows that the sequence $f_{n}$ itself also converges to $f$ a.e.ll

$4.43$ Corollary. If $f_{n}$ is a sequence of real-valued integrable functions which is non-decreasing a.e. and which converges a.e. to a function $f$, and if the sequence of the norms of the $f_{n}$ is bounded, then $f$ is integrable and the sequence $f_{n}$ converges to $f$ in mean. In particular, $\int f_{n} d \mu$ converges to $\int f d \mu$. The same result holds if instead the sequence $f_{n}$ is non-increasing a.e.

Proof. By Theorem $4.42$ the sequence $f_{n}$ will converge a.e. and in mean to an integrable function $h$, which must of course equal $f$ a.e. $/$ 

![](https://cdn.mathpix.com/cropped/2022_07_04_30ee69d6663b0900c88eg-03.jpg?height=244&width=1016&top_left_y=181&top_left_x=158)

Proof. We will let the reader verify that $f$ must be $\mu-$ measurable, since in the applications we make of this corollary in these notes we will always know in advance that $f$ is measurable. We remark next that if $g$ and $\mathrm{h}$ are $\mu$-measurable non-negative extended real-valued functions such that $g \geq h$ a.e. then it follows from Theorem $4.41$ and Proposition $4 .$ I 8 e that $\int g d \mu \geq \int$ haj. Thus if $\int f_{n} d \mu=\infty$ for at least one $n$, then it is 

![](https://cdn.mathpix.com/cropped/2022_07_04_30ee69d6663b0900c88eg-04.jpg?height=302&width=986&top_left_y=164&top_left_x=165)

$$
\int_{E}\left(1 i m \inf f_{n}\right) d \mu \leq \lim \inf \int f_{n} d \mu .
$$

In particular, if the right hand is finite, then $\lim \inf f_{n}$ is integrable.

Proof. Let $\left.g_{n}=i n f f_{i}: n \leq i<\infty\right\} .$ Now $g_{n}$ is the limit as $m$ goes to $\infty$ of the decreasing sequence $\inf \left\{f_{i}: n \leq i \leq m\right\}$. But $f r o m$ Proposition $3.7$ it follows that the infinum of a finite number of measurable functions is measurable. Thus $g_{n}$ is measurable for each $n$. Since lim inf $f_{n}=$ lim $g_{n}$, we see that lim inf $f_{n}$ is also measurable. Now $g_{n}$ is a non-decreasing sequence, and so Iim $\int g_{n} d \mu=\int$ lim inf $f\left(\mu_{n} b y\right.$ Corollary 4.44. But $g_{n} \leq f_{n}$, and so $\int g_{n} d \mu \leq \int f_{n} d \mu$, for all $n$. Because the sequence $\int g_{n} d \mu$ is non-decreasing, we obtain the inequality Iim inf $\int f_{n} d \mu \geq$ Iim $\int g_{n} d \mu=\int$ Iim inf $f_{n} d \mu$ 

\section{Chapter 6 - Product Measures and Fubini's Theorem}

![](https://cdn.mathpix.com/cropped/2022_07_04_30ee69d6663b0900c88eg-05.jpg?height=248&width=980&top_left_y=264&top_left_x=166)

$$
(\mu \otimes \nu)(E \times F)=\mu(E) \nu(F)
$$

for all $E \in S, F \in T$. We then derive some relations between integration with respect to $\mu \otimes v$ and integration with respect to $\mu$ and $v$.

\section{A. Product Measures}

6.I Definition. We will let SxT denote the family of subsets of $X \times Y$ of the form $E \times F$ where $E \in S$ and $F \in T$. The elements of $S \times I$ will be called the measurable rectangles of $X \times Y$. We will denote the $\sigma$-ring generated by $\mathrm{S} \times \mathrm{T}$ by $\mathrm{S} \otimes \mathrm{T} \mathrm{~ . ~ T h e ~ p a i r ~ ( X}$ product measurable space of the measurable spaces $(X, S)$ and (Y, $Y)$.

If we let $\rho$ be the function on $S \times T$ defined by $\rho(E \times F)=\mu(E) \nu(F)$ (with the convention that $0 \cdot \infty=0$ ), the objective of this section is to show that we can extend $\rho$ to a measure on $S \otimes T$. We do this by showing that the results of Chapter 1, particularly Theorem $1.33$, are applicable to the set function $p$. In order to do this it will be useful to have the definition and some properties of "sections".

6.2 Definition. If $A \subseteq X \times Y$ and if $x \in X$, then the $x$-section of $A$, which is denoted by ${ }^{x} A$, is defined by 

$$
x_{A}=\{y \in Y:(x, y) \in A\} .
$$

Note that an $\mathrm{x}$-section is a subset of $\mathrm{Y}$. Similarly, if $\mathrm{y} \in \mathrm{Y}$, then the $\underline{y \text {-section of } A \text {, denoted by } A^{\mathrm{y}} \text {, is defined by }$

$$
A^{y}=\{x \in X:(x, y) \in A\} .
$$

It is, of course, a subset of $\mathrm{X}$.

$6.3$ Proposition. Let $A_{n}$ be a sequence of subsets of $X \times Y$. Then $\left.\left.x_{n=1}^{\infty} \bigcup_{n}\right)=\bigcup_{n=1}^{\infty} x_{A_{n}}, x_{n=1}^{\infty} A_{n}\right)=\bigcap_{n=1}^{\infty} x_{A_{n}}$, and ${ }^{x}\left(A_{1}-A_{2}\right)=x_{A_{1}}-x_{A_{2}}$. Similar results hold for y-sections.

$\underline{\text { Proof. }}$. Let $\mathrm{x}^{j}$ denote the function from $Y$ to $X \times Y$ defined by $x^{j}(y)=(x, y)$. For any $A \subseteq X \times Y$ it is clear that $x_{A}=\left(x^{j}\right)^{-1}(A)$. The proposition now follows immediately from the fact that the formation of preimages preserves unions, intersections and differences.//

6.4 Definition. If $h$ is a function on $X \times Y$ and if $x \in X$, then the $\underline{x \text {-section of } h}$, which is denoted by $x, h$, is the function on $Y$ defined by $(x h)(y)=h(x, y)$. The definition of $y$-sections of $h$, which we denote by $h^{y}$, is similar.

6.5 Proposition. a) If $g$ and $h$ are B-valued functions on $X \times Y$ and if $r$ and $s$ are scalars, then for all $x \in X$ we have

$$
{ }^{x}(r g+s h)=r\left({ }^{x} g\right)+s\left({ }^{x} h\right) .
$$

b) If $h_{n}$ is a sequence of functions on $X \times Y$ which converges to $h$ pointwise, then for every $x$ the sequence $x_{h}$ converges to $x_{h}$ pointwise. c) If $A \subseteq X Y$, then ${ }^{x}\left(X_{A}\right)=\chi_{\left({ }^{x} A\right)}$.

Similar results hold for y-sections.

Proof. The proposition follows immediately from the observation that if $\mathrm{h}$ is a function on $X \times Y$ then $x_{h}=h 0_{x}^{j}$, where $x^{j}$ is the function defined in the proof of Proposition 6.3.//

6.6 Proposition. a) If $A \in S \otimes T$, then ${ }^{x} A \in T$ for every $x \in x$, and $A^{\mathrm{y}} \in S$ for every $y \in Y$.

b) If $h$ is an $(S \otimes T)$-measurable function on $X \times Y$, then $x_{h}$ is T-measurable for every $\mathrm{x} \in \mathrm{X}$ and $\mathrm{h}^{\mathrm{Y}}$ is S-measurable for every $\mathrm{y} \in \mathrm{Y}$.

Proof. Let $R$ be the set of those $A \in S \otimes T$ such that ${ }^{x_{A}} \in T$ for every $x \in X$ and $A^{y} \in S$ for every $y \in Y$. Clearly $S \times T \subseteq R$, and $R$ is a o-ring by Proposition 6.3. Therefore $R=S \otimes T$. Turning to the second part of the proposition, it follows from parts a) and c) of Proposition $6.5$ that x-sections of simple $\mathrm{S} \otimes$ T-measurable functions are simple T-measurable functions, and similarly for y-sections. The second part of the proposition now follows from part $b$ ) of Proposition $6.5$ and the definition of measurable functions.//

$\underline{6.7 \text { Lemma. }} . \quad \mathrm{S} \times \mathrm{T}$ is a semiring.

Proof. This is a consequence of the following easily verified set theoretic equalities.

a) $\bigcap_{n=1}^{\infty}\left(E_{n} \times F_{n}\right)=\left(\bigcap_{n=1}^{\infty} E_{n}\right) \times\left(\bigcap_{n=1}^{\infty} F_{n}\right)$. b) $E_{1} \times F_{1}-E_{2} \times F_{2}=\left[\left(E_{1}-E_{2}\right) \times F_{1}\right] \oplus\left[\left(E_{1} \cap E_{2}\right) \times\left(F_{1}-F_{2}\right)\right]$

$$
=\left[E_{1} \times\left(F_{1}-F_{2}\right)\right] \oplus\left[\left(E_{1}-E_{2}\right) \times\left(F_{1} \cap F_{2}\right)\right] . \|
$$

$6.8$ Theorem. Let $\mu$ and $v$ be arbitrary non-negative measures. Then the set function $\rho$ defined on $S \times T$ by $\rho(E \times F)=\mu(E) \nu(F)$, (with the convention that $0 \cdot \infty=0$ ) is a premeasure, and so extends to a non-negative measure on $\mathrm{S} \otimes \mathrm{T}$.

$\underline{\text { Proof. We must show that } \rho \text { is countably additive. Suppose that }$ $E \times F=\underset{n=1}{\oplus} E_{n} \times F_{n}$, where $E \times F$ and the $E_{n} \times F_{n}$ are in $S \times T$. We must show that $\rho(E \times F)=\sum_{n=1}^{\infty} \rho\left(E_{n} \times F_{n}\right)$. For each m let $A_{m}=\prod_{n=1}^{m} E_{n} \times F_{n}$, and let $A=E \times F$. Then $A_{m}$ increases to $A$, and so ${ }^{x} X_{A_{m}}$ increases to ${ }^{x} x_{A}$ for each $x$. If we use Definition $4.44$ for the integral of non-negative functions which are not necessarily integrable, and apply Corol.lary $4.45$, we find that $\int^{x} x_{A} d v$ increases to $\int{ }^{x} x_{A} d v$ for each $x$. Evaluating these integrals, we find that $\sum_{n=1}^{m} \chi_{E}(x) v\left(F_{n}\right)$ increases to $x_{E}(x) v(F)$ for each $x$ (where we must again use the convention that $0 \cdot \infty=0$ ). Note that these functions may now take the value $\infty$, but are clearly measurable in the sense of Definition 4.44. Furthermore it is easily seen that if we apply the definition of the integral given in Definition $4.44$, then we find that $\int\left(\chi_{E} v(F)\right) d \mu=\rho(E \times F)$ and $\int\left(\sum_{n=1}^{m} X_{E} v_{n} v\left(F_{n}\right)\right) d \mu=\sum_{m=1}^{m} \rho\left(E_{n} \times F_{m}\right)$. Applying Corollary $4.45$ again, we thus conclude that $\sum_{n=1}^{m} \rho\left(E n_{n} \times F_{m}\right)$ converges to $\rho(E \times F)$ as $m$ goes to $\infty$, so that $\rho$ is a premeasure. Theorem $1.33$ is now applicable. We invite the reader to contemplate the possibility of a proof of. Theorem $6.8$ which does not use integration theory (in the form of the Monotone Convergence Theorem) but only results of Chapter $I$.

6.9 Definition. The measure obtained from $\rho$ by applying Theorem $1.33$ is called the product of the measures $\mu$ and $v$. We will denote it by $\mu \otimes v$. The measure space $(X \times Y, S \otimes T, \mu \otimes v)$ is called the product of the measure spaces $(X, S, T)$ and $(Y, T, V)$.

In order to know that the extension of $\rho$ to a measure on $S \otimes T$ is unique (by applying Theorem 1.39) we must know that $\rho$ is $\sigma$-finite.

6.10 Proposition. If $\mu$ and $v$ are both $\sigma$-finite, then so is $\rho$.

$\underline{\text { Proof. }}$. Let $E \times F \in S \times T$. Since $\mu$ and $v$ are assumed $\sigma$-finite, we have $E=\bigcup_{m=1}^{\infty} E_{m}$ and $F=\bigcup_{n=1}^{\infty} F_{n}$ where the $E_{m}$ and the $F_{n}$ have finite measure. Then $E \times F \subseteq \bigcup_{m, n} E_{m} \times F_{n}$ and $\rho\left(E_{m} \times F{ }_{n}\right)<\infty$ for all $m$ and $n . \|$ From Proposition 6.10 and Theorem 1.39 we immediately obtain:

$\underline{6.11}$ Proposition. If $\mu$ and $v$ are $\sigma$-finite then so is $\mu \otimes v$, and it is the only measure on $S \otimes T$ which has the property that $(\mu \otimes V)(E \times F)$ $=\mu(E) \nu(F)$ for all $E \in S$ and $F \in T$.

$\underline{6.12 \text { Corollary. If } \mu \text { and } v \text { are arbitrary non-negative measures, }$ and if $E$ and $F$ are o-finite elements of $S$ and $T$ respectively, then the product of the restrictions of $\mu$ and $\nu$ to $E$ and $F$ respectively coincides with the restriction of $\mu \otimes v$ to $E \times F$. 6.13 Proposition. If $\mu$ and $v$ are both finite (or totally finite, or totally $\sigma$-finite), then so is $\mu \otimes v$.

Proof. Note that every element of $\mathrm{S} \otimes T$ is contained in a measurable rectangle (since the collection of sets having this property is clearly a o-ring containing $S \times T$ ). So if $A \in S \otimes T$, let $A \subseteq E \times F$ for some $E \in S$ and $F \in T$. Then if $\mu$ and $v$ are finite $(\mu \otimes \nu)(A) \leq(\mu \otimes V)(E \times F)$ $=\mu(E) \nu(F)<\infty$. In particular if $\mu(X)<\infty$ and $v(Y)<\infty$ then $(\mu \otimes v)(X \times Y)<\infty$, and similarly in the $\sigma$-finite case. $\|$ 

\section{B. Fubini's Theorem}

In this section we examine the relation between integration with respect to $\mu \otimes \nu$ and integration with respect to $\mu$ and $\nu$. The main result is known as Fubini's theorem.

In order to prove the main theorems we will need to make some finiteness assumptions. For this reason it is sufficient in the preliminary lemmas to assume that the measures $\mu$ and $v$ are finite.

Let $R$ denote the collection of all finite disjoint unions of elements of SXI. Then from Corollary $1.18$ we see that $R$ is just the ring generated by SXI. The first step in the proof of Fubini's theorem is to prove a version of it just for this ring.

6. I4 Lemma. Let $\mu$ and $v$ be finite. If $A$ is any element of $R$ (the ring generated by $S \times T$ ) then

1) ${ }^{x} X_{A}$ and $X_{A}^{y}$ are measurable, in fact integrable, functions with respect to $v$ and $\mu$ respectively,

2) $\int{ }^{\mathrm{x}} X_{A} d v$ and $\int x_{A}{ }^{y} d \mu$ are measurable, in fact integrable, functions of $x$ and $y$ respectively, and

3) $\int\left(\int x_{A} d v\right) d \mu(x)=\int x_{A} d(\mu \geq v)=\int\left(\int y_{A} d \mu\right) d v(y)$.

Proof. Let $A=\prod_{i=1}^{n} E_{i} \times F_{i}$ where $E_{i} \otimes F_{i} \in S \times T$ for each i. It follows that $X_{A}(x, y)=\sum_{i=1}^{n} X_{E_{i}}(x) X_{F_{i}}(y)$ for all $x$ and $y$. From this it is clear that each section is an ISF, which proves I). It is also clear . that $\int^{x} X_{A} d v=\sum_{i=1}^{n} X_{E_{i}}(x) v\left(F_{i}\right)$, and similarly for the integral with respect to $\mu$, and so they too are both ISF, proving 2). Finally, it is now clear that both iterated integrals turn out to be $\sum_{i=1}^{n} \mu\left(E_{i}\right) \cup\left(F_{i}\right)$, that is, $\int x_{A} d(\mu \otimes v) . \|$

The main step in the proof of Fubini's theorem is to extend Lemma $6.14$ to the case in which A is an arbitrary element of $\mathrm{S} \otimes \mathrm{T}$. But it is sufficient to still work with finite measures.

$6.15 \mathrm{Key}$ Lemma. Let $\mu$ and $\nu$ be finite and let $A \in S \otimes T$. Then 1) ${ }^{x} \chi_{A}$ and $x_{A}^{y}$ are measurable, in fact integrable, functions with respect to $\mu$ and $v$ respectively.

2) $\int x_{A}^{x} d \nu$ and $\int X_{A}^{y} d \mu$ are measurable, in fact integrable, functions of $x$ and $y$ respectively.

3) $\int\left(\int^{x} x_{A} d v\right) d \mu(x)=\int x_{A} d(\mu \otimes v)=\int\left(\int x_{A}^{y} d \mu\right) d v(y)$.

Proof. By Proposition 6.5c) and 6.6a), the functions in 1) are characteristic functions of measurable sets, and so, since $\mu$ and $v$ are finite measures, they are integrable, This proves property l).

Note that because the integrands in property 2) are characteristic functions and the measures are finite, the two functions in property 2) are bounded. Thus if we can prove that these functions are measurable, it will follow that they are integrable.

Let $M$ be the collection of all sets $A \in S \otimes T$ for which properties 1), 2) and 3) are true. We wish to show that $M=S \otimes T$. Let $R$ be the set of finite disjoint unions of elements of sxT. (so that by Corollary 1.18 $\vec{R}$ is the ring generated by $S \times T$ ). Then Lemma 6.14 says exactly that $R \subseteq M$. Let us investigate further the properties of $M$.

Suppose that $A_{n}$ is a sequence of elements of $M$, and that $A_{n}$ decreases to a set A. Then $A \in S \otimes T$ and so property l) holds for A. Now ${ }^{\mathrm{x}} \chi_{A_{n}}$ and $x_{A_{n}}^{y}$ will decrease to ${ }^{x} X_{A}$ and $x_{A}^{y}$, and so we can apply Corollary $4.43$ of the Monotone Convergence Theorem to conclude that

$$
\begin{aligned}
&\int{ }^{x_{A}} d v \text { decreases to } \int x^{x} x_{A} d v \\
&\int x_{A_{n}}^{y} d \mu \text { decreases to } \int x_{A}^{y} d \mu .
\end{aligned}
$$

Since property 2) holds for the sets $A_{n}$, it follows that the functions $\int^{x} x_{A} d \nu$ and $\int \chi_{A}^{y} d \mu$ are measurable. Thus property 2) holds for A also. Using Corollary $4.43$ again, we see that

$$
\begin{aligned}
&\int\left(\int x_{A_{n}}^{x_{1}} d v\right) d \mu(x) \text { decreases to } \int\left(\int^{x} x_{A} d v\right) d \mu(x) \text { and } \\
&\int\left(\int x_{A_{n}}^{y} d \mu\right) d \nu(y) \text { decreases to } \int\left(\int x_{A}^{y} d \mu\right) d v(y) .
\end{aligned}
$$

Since property 3) holds for the sets $A_{n}$, the left hand sides are just $\int x_{A_{n}} d(\mu \otimes v)$. But $\int x_{A_{n}} d(\mu \otimes v)$ decreases to $\int x_{A} a(\mu \otimes v)$ by Proposition 4.43, and so property 3) holds for A also. A similar proof shows that if instead the sequence $A_{n}$ increases to $A$, then again $A \in M$.

We are thus led to make the following definition:

6.16 Definition. A collection $M$ of sets is called a monotone class if it is closed under the formation of countable increasing unions and countable decreasing intersections. Thus what we have shown above is that the collection $M$ of the proof of Lemma $6.15$ is a monotone class which contains the ring generated by the semiring $S \times T$. It is then clear that the proof of Lemma $6.15$ will be completed once we have proven the following lemma:

6.17 Lemma (The Lerama on Monotone Classes). Let M be a monotone class and let $P$ be a semiring. If $M$ contains the ring generated by $P$, then $M$ contains the $\sigma$-ring generated by $P$.

Proof. Note first that the intersection of any collection of monotone classes is again a monotone class, so that any collection of sets is contained in a smallest monotone class, which it is said to generate. Thus we can (and willl) assume that $M$ is the monotone class generated by the ring, $R$, generated by $P$. Thus we wish to show that $M$ coincides with the $\sigma$-ring, $S(P)$, generated by $P$. For this it suffices to show that $M$ is a $\sigma$-ring, and for this it suffices to show that $M$ is closed under taking differences and finite unions (since $\bigcup_{n=1}^{\infty} F_{n}=\bigcup_{m=1}^{\infty}\left(\bigcup_{n=1}^{m} F_{n}\right.$ ) which is an increasing union).

Thus for each $E \in M$ define a subset, $K(E)$, of $M$ by $K(E)=\{F \in M: E-F, F-E$, and $E \cup F$ are in $M\}$.

What we need to show is that $K(E)=M$ for all $E \in M$. We divide the proof of this fact into 6 short steps.

1) If $F \in K(E)$ then $E \in K(F)$. This is clear from the definitions of $K(E)$ and $K(F)$. 2) If $E \in R$ then $R \subseteq K(E)$. This is clear from the definition of a ring.

3) $K(E)$ is a monotone class for every $E \in M$. To see this, suppose that $F_{n}$ is a sequence of elements of $K(E)$ which increases to a set $F$. Then $F_{n}-E$ increases to $F-E, E-F_{n}$ decreases to $E-F$, and $F_{n} \cup E$ increases to $F \cup E$, so that $F-E, E-F$ and $E \cup F$ are in $M$. Thus $F \in K(E)$. A similar argument works if instead the sequence $F_{n}$ decreases to $F$.

4) If $E \in R$ then $K(E)=M$. This follows from 2) and 3) and the fact that we have assumed that $M$ is the monotone class generated by $R$.

5) $R \subseteq K(E)$ for all $E \in M$. This follows from 4) and $I)$.

6) $K(E)=M$ for all $E \in M$. This follows from 5) and 3).//

The next step in the proof of Fubini's theorem is to extend Lemma $6.15$ to the case of non-negative measurable functions. We do not need to assume that $\mu$ and $\nu$ are finite any more, but we do need to assume that they are o-finite (see exercise ). Because we will want to use Corollary $4.43$ of the Monotone convergence theorem we need the following fact:

6.18 Proposition. If $f$ is a non-negative measurable function, then there is an increasing sequence of non-negative simple measurable functions which converges to $f$ pointwise.

$\underline{\text { Proof. }}$. Define $\mathrm{f}_{\mathrm{n}}$ by

$$
f_{n}(x)= \begin{cases}(k-1) / 2^{n} & \text { if }(k-1) / 2^{n} \leq f(x)<k / 2^{n} \quad k=1, \ldots, n 2^{n} \\ n & \text { if } n \leq f(x) .\end{cases}
$$

6.19 Theorem (Fubini, Tonelli). Let $f$ be a non-negative $S \otimes T-$ measurable function. Then the following conditions are equivalent:

1) $f$ is integrable,

2) $x_{f}$ is integrable for almost all $x \in X$, and $x \mapsto \int x_{f d \nu}$ is an (almost everywhere defined) $\mu$-integrable function on $\mathrm{X}$,

3) $f^{y}$ is integrable for almost all $y \in Y$, and $y \mapsto \int f^{y} d \mu$ is an (a)most everywhere defined) $v$-integrable function on $Y$.

If any of these three conditions holds then

$$
\int\left(\int x_{f d v}\right) d \mu=\int f d(\mu \otimes v)=\int\left(\int f^{y} d \mu\right) d v .
$$

Proof. Since $f$ is a measurable function, $C(f)$ is an $S \otimes T$-measurable set. Since we are assuming that the measures $\mu$ and $v$ are $\sigma$-finite, it is easy to see that there is a rectangle $E \times F \in S \times T$ and a sequence $E_{n} \times F$ of elements of $S \times T$ such that $C(f) \subseteq E \times F$, the sequence $E_{n} \times F_{n}$ increases to $E \times F$, and $\mu\left(E_{n}\right)<\infty, V\left(F_{n}\right)<\infty$ for all $n$. Using Proposition $6.18$ let $f_{n}$ be an increasing sequence of non-negative simple $s \otimes T$-measurable functions which converges to $f$ pointwise. If we let $g_{n}=\chi_{E_{n} \times F_{n}} f_{n}$ for each $n$, then the $g_{n}$ form a sequence of non-negative ISF increasing to $f$ pointwise.

For the moment fix $n$, and consider $\mu, v$ and $\mu \otimes \nu$ restricted to $E_{n}, F_{n}$ and $E_{n} \times F_{n}$ respectively. According to Proposition $6.13$ the product of the restrictions of $\mu$ and $\nu$ to $E_{n}$ and $F_{n}$ respectively coincides with $\mu \otimes \nu$ restricted to $E_{n} \times F_{n}$, and all these restrictions are finite. We can thus apply key Lemma $6.15$ to conclude that conditions 2) and 3) and the equality between integrals in Theorem $6.19$ are true for the characteristic function of any measurable subset of $E_{n} \times F_{n}$, in fact true even with the qualification "almost" omitted. Now since $C\left(g_{n}\right) \subseteq E_{n} \times F$ m and $g_{n}$ is just a finite sum of characteristic functions of measurable subsets of $E_{n} \times F_{n}$, it follows that conditions 2) and 3) and the equality between integrals in Theorem $6.19$ are true for. $g_{n}$ also, again even with the qualification "almost" omitted. In particular, $\int{ }_{g_{n}} d \nu$ is an integrable function of $x$ for each $n$, and the $\int{ }_{g_{n}} d \nu$ form an increasing sequence of non-negative integrable functions.

Now if condition 1 ) holds, then

$$
\int\left(\int \mathrm{g}_{\mathrm{n}} d v\right) d \mu=\int \mathrm{g}_{\mathrm{n}} d(\mu \otimes v) \leq \int f d(\mu \otimes v)
$$

for all n. On the other hand, if condition 2). holds, then $\int{ }^{x_{n}} d v \leq \int x_{f d v}$ a.e., and so

$$
\int\left(\int{ }_{g_{n}} d v\right) d \mu \leq \int\left(\int x_{f} d v\right) d \mu
$$

for all n. Thus in either case the sequence of the $L^{I}$-norms of the functions $\int x_{g_{n}} d v$ is bounded above, and so we can apply the Monotone Convergence Theorem (Theorem 4.42) to conclude that the sequence $\int{ }_{g_{n}} d v$ converges a.e. and in mean to an integrable function, $\mathrm{h}$. Let. $N$ be the null set off of which $\int x_{g_{n}} d v$ converges to $h(x)$ for all $x$. If $x \notin N$ then $\int{ }^{g_{n}} d \nu \leq h(x)$ for all $n$, and so the sequence of the norms of the ${ }^{x_{n}} g_{n}$ is bounded. But $\mathrm{x}_{\mathrm{g}_{\mathrm{n}}}$ increases to $\mathrm{x}_{f}$, and so $\mathrm{x}_{f}$ is integrable and $\int{ }^{x_{f d \nu}}=\lim \int{ }^{x_{n}} d \nu$ for all $x \notin N$ by Corollary 4.43. (Thus we see that $\mathbb{N}$ is exactly the set of those $x \in X$ for which ${ }_{x} f$ is not integrable.) 

![](https://cdn.mathpix.com/cropped/2022_07_04_30ee69d6663b0900c88eg-18.jpg?height=198&width=940&top_left_y=197&top_left_x=167)

longer natural to set $\int x_{\text {fav }}$ equal to $\infty$ for $x \in \mathbb{N}$. Thus even in the present setting we prefer simply to say that $\int{ }^{x}$ fd is undefined for $x \in \mathbb{N}$.

We also remark that the part of the above theorem which is usually called Tonelli's theorem is the fact that conditions 2) or 3) imply condition 1). For applications this is by far the most useful method of trying to show that a measurable function on a product space is integrable. Notice that Tonelli's theorem is immediately applicable to functions with values in a Banach space, since if $f$ is such a function and is measurable, then to show that $f$ is integrable it suffices by Theorem 4.4I to show that the nonnegative measurable function $\|f(\cdot)\|$ is integrable. A counter-example for Tonelli's theorem when one of the measures $\mu$ and $\nu$ is not $\sigma$-finite can be found in exercise .

Notation which is more commoniy used in stating Fubini's theorem than that which we have used above is as follows:

6.20 Definition. 'If $f$ is a measurable function on $X \times Y$, if $\int x_{f d v}$ is defined a.e. and measurable, and if $\int\left(\int x_{f d v) d \mu}\right.$ exists, then we will write $\int f(x, y) d \nu(y) d \mu(x)$ or $\int f d v d \mu$ instead of $\int\left(\int x_{f d v}\right) d \mu$. Under similar conditions we will write $\int f(x, y) d \mu(x) d \nu(y)$ or $\int f d \mu d \nu$ instead of $\int\left(\int f^{y} d \mu\right) d v$. These expressions are called iterated integrals to distinguish them from $\int f(\mu \otimes \nu)$, which is called the double integral of $f$.

From Fubini's theorem we obtain the following convenient characterization of null sets with respect to $\mu \otimes \nu$ : 6.21 Corollary. If. $C \in \mathbb{N}(\mu \otimes v)$ (the collection of null sets for $\mu \otimes v)$, then $\mathrm{x}_{\mathrm{C}} \in \mathbb{N}(v)$ for almost all $\mathrm{x} \in \mathrm{X}$ and $C^{\mathrm{y}} \in \mathbb{N}(\mu)$ for almost all $y \in Y$. Conversely if $A \in S \otimes T$ and ${ }^{x} A \in \mathbb{N}(\nu)$ a.e. (or $A^{y} \in \mathbb{N}(\mu)$ a.e.), then $A \in \mathbb{N}(\mu \otimes v)$.

Proof. Since $c \in \mathbb{N}(\mu \otimes v)$, there exists $A \in S \otimes T$. such that $c \subseteq A$ and $(\mu \otimes \nu)(A)=0$. By Theorem $6.19$ we have

$$
\int\left(\int x^{x} X_{A} d \nu\right) d \mu=\int x_{A} d(\mu \otimes \nu)=(\mu \otimes \nu)(A)=0 .
$$

Thus $v\left({ }^{\mathrm{x}} \mathrm{A}\right)=\int{ }^{x_{A}} X_{A} v=0$ for almost all $\mathrm{x} \in \mathrm{X}$. Since ${ }^{\mathrm{x}_{C}} \subseteq{ }^{\mathrm{x}_{A}}$ for all $\mathbf{x} \in \mathrm{X}$ we are done. The converse is clear.ll

We now come to our final version of Fubini's theorem, which involves functions with values in a Banach space.

6.22 Theorem (Fubini). If $\mu$ and $v$ are $\sigma$-finite non-negative measures and if $f$ is a B-valued $(\mu \otimes v)$-integrable function, then

1) $x_{f}$ and $f^{y}$ are integrable functions for almost all $x \in x$ and almost all y $\in Y$,

2) $\int x^{x} d v$ and $\int f^{y} d \mu$ are (almost everywhere defined) integrable functions of $\mathrm{x}$ and $\mathrm{y}$ respectively, and

3) $\iint f d v d \mu=\int f(\mu \otimes v)=\iint f d \mu d \nu$.

Proof. Since $f$ is $(\mu \otimes v)$-integrable, $\|f(\cdot)\|$ is a non-negative ( $\mu \otimes v$ )-integrable function. Choose (as in the proof of Theorem 4.4I) a sequence, $f_{n}$, of simple $S \otimes T$-measurable functions which converges to $f$ a.e. and such that $\left\|_{n}\right\| \leq 2\|f\|$ for all $n$. Then by the Lebesgue Dominated 

![](https://cdn.mathpix.com/cropped/2022_07_04_30ee69d6663b0900c88eg-21.jpg?height=196&width=912&top_left_y=156&top_left_x=166)

$$
\left\|\int_{f_{n}} d v\right\| \leq 2 \int^{x}(\|f\|) d v
$$

for all $\mathrm{x} \notin \mathrm{N}_{1}$ and $a l l \mathrm{n}$, and since by Theorem $6.19$ the right hand side is a $\mu$-integrable function of $\mathrm{x}$, we can again apply the Lebesgue Dominated Convergence Theorem to conclude that $\int \cdot{ }^{x} f d \nu$ is a $\mu$-integrable function on $X$ (proving part 2)), and that

$$
\iint f d v d \mu=\lim \iint f_{n} d v d \mu
$$

But as was noted before, Fubini's Theorem holds for the ISF, and so

$$
\int f d(\mu \otimes v)=\lim \int f_{n} d(\mu \otimes v)=\lim \iint f_{n} d v d \mu .
$$

Thus part 3) and the theorem are proved.II

![](https://cdn.mathpix.com/cropped/2022_07_04_30ee69d6663b0900c88eg-22.jpg?height=88&width=930&top_left_y=292&top_left_x=221)

$$
\mu^{+}=\frac{|\mu|+\mu}{2} \text { and } \mu^{-}=\frac{|\mu|-\mu}{2} \text {. }
$$

Then $\mu^{+}$and $\mu^{-}$are easily seen to be non-negative measures and $\mu=\mu^{+}-\mu^{-}$. Thus if both $\mu$ and $v$ are real measures it is natural to define $\mu \otimes v$ by

$$
\mu \otimes \nu=\mu^{+} \otimes \nu^{+}-\mu^{+} \otimes v^{-}-\mu^{-} \otimes v^{+}+\mu^{-} \otimes v^{-} .
$$

It is also easy to show that this is what the product measure should be, in the sense that it does the right thing on rectangles, and that by linearity Fubini's theorem is true with this definition of $\mu \otimes v$.

Similarly, if $\mu$ is a complex measure, define $\mu_{r}$ and $\mu_{i}$ by

$$
\mu_{r}=\frac{\mu+\bar{\mu}}{2} \text { and } \mu_{i}=\frac{\mu-\bar{\mu}}{2 i}
$$

(where the bar denotes complex conjugation and $\bar{\mu}$ is defined by $\left.\bar{\mu}(E)=(\mu(E))^{-}\right)$.

Then $\mu_{r}$ and $\mu_{i}$ are real measures and $\mu=\mu_{r}+i_{i_{i}}$. Thus if $\mu$ and $v$ are both complex measures, we define $\mu \otimes \nu$ by

$$
\mu \otimes v_{r} \mu_{r} \otimes v_{r}-\mu_{i} \otimes v_{i}+i\left(\mu_{r} \otimes v_{i}+\mu_{i} \otimes v_{r}\right) \text {. }
$$

Again it is easily seen that this is the appropriate product measure and that Fubini's theorem holds for it.