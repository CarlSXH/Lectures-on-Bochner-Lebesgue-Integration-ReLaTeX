


\section{The Domain of a Measure}


Given a set $X$, a measure on $X$ will be a rule which in some sense tells us the size of certain subsets of $X$. Thus a measure will be a function whose domain is a suitable collection of subsets of $X$. By "suitable" we mean that the family is closed under certain set operations.

\begin{definition}
A nonempty collection $R$ of subsets of $X$ is called a \underline{ring} if $R$ is closed under the formation of the union and difference of any two elements of $R$, that is, if $E,F\in R$, then $E\cup F$ and $E\sd F\in R$ (where $E\sd F = E\cap \comp{F}$, where $\comp{F}$ is the complement of $F$). 
\end{definition}

It is easy to see that any ring is also closed under the formation of the intersection of any two of its members, since $E\cap F = E\sd(E\sd F) = F\sd (F\sd E)$. Furthermore, it can be shown by simple inductive proofs that any ring is closed under the formation of arbitrary finite unions and finite intersections, that is, if $E_1,\dots,E_n\in R$ then $\bigcup_{i=1}^nE_i$ and $\bigcap_{i=1}^nE_i\in R$. Note that any ring $R$ contains $\varnothing$ (the empty set), since if $E\in R$, then $\varnothing=E\sd E\in R$. However, a ring does not necessarily contain the whole set $X$.

\begin{definition}
If a ring $R$ contains the set $X$, then $R$ is called a \underline{field} (also sometimes called an algebra).
\end{definition}

In order to define a measure we actually need to be able to form countable unions of subsets of $X$. Thus we need the domain of a measure to be more than just a ring or a field.

\begin{definition}
A collection $S$ of subsets of $X$ is called a \underline{$\sigma$-ring} if $S$ is a ring and if $S$ is closed under the formation of countable unions, i.e. if $E_i\in S$, $i=1,2,\dots$ then $\bigcup_{i=1}^\infty E_i\in S$.
\end{definition}

Since, if $E=\bigcup_{i=1}^\infty E_i$, then $\bigcap_{i=1}^\infty E_i=E\sd\br{\bigcup_{i=1}^\infty(E\sd E_i)}$, it is clear that a $\sigma$-ring is also closed under formation of countable intersections.

\begin{definition}
A $\sigma$-ring $S$ is called a \underline{$\sigma$-field} (also called a $\sigma$-algebra) if it is also a field, that is, if $X\in S$.
\end{definition}

\begin{proposition}
The intersection of any collection of rings (fields, $\sigma$-rings or $\sigma$-fields) on a set $X$ is again a ring (field, $\sigma$-ring or $\sigma$-field).
\end{proposition}
\begin{proof}
We give the proof only for rings, since the proofs for the other cases are similar.

Let $\brc{R_\alpha}_{\alpha\in A}$ be a collection of rings, where $A$ is some index set, and let $R = \bigcap_{\alpha\in A}R_\alpha$. If $E,F\in R$, then $E,F\in R_\alpha$ for all $\alpha\in A$, and so $E\cup F$ and $E\sd F\in R_\alpha$ for all $\alpha\in A$, and so $E\cup F$, $E\sd F\in R$. Thus $R$ is a ring.
\end{proof}

\begin{corollary}
Given any collection $P$ of subsets of $X$ there exists a smallest ring (field, $\sigma$-ring or $\sigma$-field) containing $P$.
\end{corollary}
\begin{proof}
By "smallest" we mean "which is contained in any ring (field, $\sigma$-ring or $\sigma$-field) which contains $P$". Thus by the proposition just proved it is enough to show that there is some ring (field, $\sigma$-ring or $\sigma$-field) which contains $P$, for the smallest one will be just the intersection of all those which contain $P$. But it is obvious that the collection of all subsets of $X$ is a $\sigma$-field (and hence a ring, field and $\sigma$-ring) which contains $P$.
\end{proof}

This corollary allows us to make the following definition.

\begin{definition}
The smallest ring (field, $\sigma$-algebra or $\sigma$-field) containing $P$ is called \underline{the ring (field, $\sigma$-ring or $\sigma$-field) generated by $P$}. The $\sigma$-ring generated by $P$ will be denoted as $\sring{P}$
\end{definition}

Using this definition we can give some important examples.

\begin{example}
Let $X$ be a topological space, and let $P$ be the collection of open subsets of $X$. Then $\sring{P}$ is called \underline{the $\sigma$-ring of Borel sets of $X$}.
\end{example}

There is also a different definition of Borel sets which is in common use. If $X$ is a locally compact space, then the $\sigma$-ring generated by the compact subsets of $X$ is also frequently called the $\sigma$-ring of Borel subsets of $X$. The reader should be able to check that for the real line these two definitions are equivalent. In fact they are equivalent for any locally compact space which are $\sigma$-compact, that is, which is the union of a countable number of compact subsets. However, for an uncountable space with the discrete topology the two definitions do not coincide.

\begin{example}
Let $X$ be a locally compact space, and let $P$ be the collection of compact $G_\delta$'s contained in $X$. (Recall that  $G_\delta$ is a set which is the intersection of countable collection of open sets.) In this case $\sring{P}$ is called \underline{the $\sigma$-ring of Baire sets of $X$.} (If $X$ is a locally compact metric space, then the reader should be able to verify that the Baire sets are the same as the Borel sets (second definition).)
\end{example}

\begin{example}
Let $X=\bR$ (the real line) and let $P$ be the collection of left closed right open finite intervals, $[a,b)$. It is not hard to show that $\sring{P}$ is the $\sigma$-ring of Borel sets of $\bR$. This can be done by showing that $\sring{P}$ contains all compact subsets of $\bR$ (in fact all open or closed subsets) and that the $\sigma$-ring of Borel sets contains the half open intervals. For example, $[a,b)=\bigcup_{n=1}^\infty\sbr{a,b-\frac1n}$ so $P\subseteq\sigma$-ring of Borel sets of $\bR$. The remaining details are left to the reader as an exercise. This example shows, in particular, that two collections of sets which have no members in common can nevertheless generate the same $\sigma$-field. Note that the $\sigma$-ring of Borel sets of $\bR$ is actually a $\sigma$-field since $\bigcup_{n=1}^\infty[-n,n)=\bR$.
\end{example}

We remark that while Definition 1.7 is a simple definition, it hides a great deal. It is often very difficult to decide whether a given set is in the $\sigma$-ring generated by a given collection of sets. For example, it should be far from clear to the reader at this point whether or not every subset of the real line is a Borel set. (But see exercise 9.) %REFFIX


\section{The Definition of a Measure}


A sequence, $E_i$, $i=1,2,\dots$ of subsets of a set is said to consist of disjoint sets if $E_i\cap E_j=\varnothing$ whenever $i\neq j$. We will denote the union of such a sequence of disjoint sets by $\bigdu_{i=1}^\infty E_i$ instead of $\bigcup_{i=1}^\infty E_i$, to emphasize that the sets are disjoint.

\begin{definition}
Let $P$ be an arbitrary collection of subsets of a set $X$, and let $\mu$ be a function from $P$ into a Banach space, or into $\bR^\infty$ ($=\bR\cup\brc{+\infty}$). Then $\mu$ is said to be \underline{countably additive} (or $\sigma$-additive) on $P$ if for every sequence $E_i$, $i=1,2,\dots$, of disjoint elements of $P$ such that $\bigdu_{i=1}^\infty E_i$ is also in $P$, we have $\mu\br{\bigdu_{i=1}^\infty E_i}=\sum_{i=1}^\infty\mu(E_i)$. 
\end{definition}

The right hand side of this last equation is defined to be the limit in the norm topology of the sequence $s_n$, where $s_n$ is the $n$th partial sum in the of the series $\sum_{i=1}^\infty\mu(E_i)$. In $\bR^\infty$, we allow this limits to be $+\infty$ in the obvious sense. Note that implicit in the definition of countable additivity is the requirement that the sum of the series must exist for all sequences of disjoint elements of $P$ whose union is in $P$. Note also that in order to verify the countable additivity of a function it is necessary only to consider disjoint sequences of elements of $P$ whose union is again in $P$. In some important cases, for example the $P$ of Example 1.10, there will be relatively few such sequences, which will simplify the verification of countable additivity. We could also consider functions which take values in $\bR^{-\infty} = \bR\cup\brc{-\infty}$ but this case is virtually the same as that for $\bR^\infty$, and so we will not discuss it. However we cannot consider $\sigma$-additivity of functions with values in $\bR\cup\brc{-\infty,\infty}$ since there is no suitable definition of $\infty+(-\infty)$. %REFFIX

\begin{definition}
A \underline{measure} is a function $\mu$, whose domain is a $\sigma$-ring $S$, of subsets of a set $X$, whose range is contained either in a Banach space or in $\bR^\infty$, and which is countably additive on $S$. If the range of $\mu$ is in $\bR^\infty$, then $\mu$ will be said to be an \underline{extended real valued measure}. Of particular interest will be non-negative extended real valued measures. We will call these \underline{non-negative measures}, leaving "extended real valued" as understood.
\end{definition}

One example of a non-negative measure is the function which has value $\infty$ on all members of $S$. This example is quite uninteresting. However we remark that every other measure has the property that $\mu(\varnothing)=0$. For if a measure $\mu$, has finite value on at least one set $E$, (in particular, if its range is contained in a Banach space) then $\mu(E)+\mu(\varnothing) = \mu(E)$, and so $\mu(\varnothing)=0$.

Since the domain $S$ of a measure is a $\sigma$-ring, every sequence of disjoint sets in $S$ is such that its union is in $S$. Thus this part of the definition of countable additivity is automatically satisfied. Also, the property of having a $\sigma$-ring as the domain allows us to "disjointize" an arbitrary sequence, $E_i$, $i = 1, 2,\dots$, of members of $S$. Namely, we can construct a sequence of disjoint elements of $S$, $F_n$, $n = 1, 2,\dots$, such that $F_n\subseteq E_n$ and $\bigcup_{i=1}^\infty E_i=\bigdu_{n=1}^\infty F_n$, by letting $F_1 = E_1$, and $F_n=E_n\sd\bigcup_{i=1}^{n-1}E_i$ for $n > 1$. This procedure will be useful in a number of places.

The first problem which arises in connection with the above definitions is to describe means by which interesting and useful measures can be constructed. In the rest of this chapter, we will describe some ways of doing this which will involve extending appropriate set functions so that they become measures. In later chapters we will describe various ways of obtaining new measures from old ones.


\section{An Example - Borel-Stieltjes Premeasures}

Before going on to state and prove our extension theorems, we will give an example of the kind of set function which we wish to extend, and we will use this example to give an idea of the properties which we will use to construct extensions. When extended, the set function of our example will be Borel-Stieltjes (or Lebesgue-Stieltjes) measure on the real line (depending on the $\sigma$-ring to which we extend).

\begin{definition} (\underline{Borel-Stieltjes premeasure})
The domain of the set functions which we will define will be just the family $P$ of Example 1.10, that is, the family of left closed, right open finite intervals, $[a, b)$, on the real line. Note that $[a, a)=\varnothing $. The functions are defined as follows: Let $\alpha$ be a real-valued non-decreasing left continuous function on $\bR$. Define a function, $\mu_\alpha$, from $P$ to $\bR$ by $\mu_\alpha([a, b))=\alpha(b)-\alpha(a)$.
\end{definition}

The assumption that $\alpha$ is left continuous will be quite important. It is not hard to show that the theorem which we will prove in this section is false without this requirement. We remark however that it can be shown that any non-decreasing function can have at most a countable number of discontinuities, and so from any non-decreasing function we can obtain a left continuous non-decreasing function by changing its values at most at a countable number of points.

\begin{theorem}
The function $\mu_\alpha$ is countably additive on $P$.
\end{theorem}

\begin{proof}
Let $E_n=\left[a_n, b_n\right)$ $n=1, 2, \dots$ be a sequence of disjoint elements of $P$, and suppose that $\bigdu_{n=1}^\infty E_n=E$, where $E=[a_0, b_0)$. We need to show that $\mu_\alpha(E)=\sum_{n=1}^\infty\mu_\alpha(E_n)$, that is, that $\alpha(b_0)-\alpha(a_0)=\sum_{n=1}^\infty(\alpha(b_n)-\alpha(a_n))$.

We begin by showing that $\mu_\alpha(E) \geq \sum_{n=1}^\infty \mu_\alpha\left(E_n\right)$. Now it is sufficient for this to show that $\mu_\alpha(E) \geq \sum_{n=1}^m \mu_\alpha\left(E_n\right)$ for each finite $m$. Given any $m$, order the intervals $E_n$, $n=1, \dots, m$, according to their left endpoint, that is, re-index them so that $a_i \leq a_{i+1}$ for all $i=1, \dots, m-1$. Since the intervals are disjoint, it follows that $b_i \leq a_{i+1}$. We must show that $\alpha\left(b_0\right)-\alpha\left(a_0\right) \geq \sum_{n=1}^m\left(\alpha\left(b_n\right)-\alpha\left(a_n\right)\right)$. Since clearly $b_0 \geq b_i$ and $a_0 \leq a_i$ for $i$, we have $\alpha\left(b_0\right)-\alpha\left(a_0\right) \geq \alpha\left(b_m\right)-\alpha\left(a_1\right)$. Now $\sum_{n=1}^m\left(\alpha\left(b_n\right)-\alpha\left(a_n\right)\right)=\alpha\left(b_m\right)-\alpha\left(a_1\right)+\sum_{n=1}^m\left(\alpha\left(b_n\right)-\alpha\left(a_{n+1}\right)\right)$, and $\sum_{n=1}^{m-1}(\alpha(b_n)-\alpha(a_{n+1})) \leq 0$ because of the way we have ordered the disjoint intervals and because $\alpha$ is non-decreasing. Hence we obtain the desired inequality.

Now we need to show that $\mu_\alpha(E) \leq \sum_{n=1}^\infty \mu_\alpha(E_n)$. Choose $\varepsilon>0$, choose $b_0'<b_0$ such that $\alpha(b_0') \geq \alpha(b_0)-\frac{\varepsilon}2$, and for each $n$ choose $a_n'<a_n$ such that $\alpha(a_n')\geq \alpha(a_n)-\varepsilon_n$, where the $\varepsilon_n$ are positive numbers such that $\sum_{n=1}^\infty\varepsilon_n=\frac{\varepsilon}2$. For example we could let $\varepsilon_n=\varepsilon/2^{n+1}$. (Recall that $\sum_{n=1}^\infty 1/2^n=1$. We will use this fact repeatedly.) We can choose such $a_n'$ only because $\alpha$ is assumed to be left continuous. Then $[a_0, b_0'] \subseteq[a_0, b_0)=\bigdu_{n=1}^\infty[a_n, b_n) \subseteq \bigcup_{n=1}^\infty(a_n', b_n) .$ Since $[a_0, b_0']$ is compact and the $(a_n', b_n)$ are open, there exists a finite integer $m$ such that $[a_0, b_0'] \subseteq \bigcup_{n=1}^m(a_n', b_n)$. Re-index the intervals so that the first contains $a_0$, the second contains the right endpoint of the first, the third contains the right endpoint of the second and so on until we get $b_0'$ contained in an interval of the string. There may be some intervals left over, and we discard them as they are superfluous. We see that we have now arranged matters so that $b_n \geq a_{n+1}'$ for $n=1, \dots, m$ and $a_1' \leq a_0$ and $b_0' \leq b_m .$ We also reindex the $\varepsilon_n$ in the same way. Then $\alpha(b_0)-\alpha(a_0) \leq \alpha(b_0')-\alpha(a_0)+\frac{\varepsilon}2 \leq \alpha(b_m)-\alpha(a_1')+\frac{\varepsilon}2\leq \alpha(b_m)-\alpha(a_1')+\sum_{n=1}^{m-1}(\alpha(b_n)-\alpha(a_{n+1}'))+\frac{\varepsilon}2=\sum_{n=1}^m(\alpha(b_n)-\alpha(a_n'))+\frac{\varepsilon}2\leq \sum_{n=1}^m(\alpha(b_n)-\alpha(a_n)+\varepsilon_n)+\frac{\varepsilon}2 \leq \sum_{n=1}^\infty(\alpha(b_n)-\alpha(a_n))+\varepsilon.$. Since $\varepsilon$ was arbitrary we have shown that $\alpha(b_0)-\alpha(a_0) \leq \sum_{n=1}^\infty(\alpha(b_n)-\alpha(a_n))$ which is what we needed.
\end{proof}

\section{Semirings and Premeasures}

Motivated by the properties of the collection $P$ of the previous section, we make:

\begin{definition}
A collection $P$ of subsets of a set $X$ is called a \underline{semiring} if
\begin{enumerate}[label=\arabic*)]
    \item $\varnothing \in P$
    \item if $E, F \in P$ then $E \cap F \in P$
    \item if $E, F \in P$ then there exist $E_1, \dots, E_m \in P$ such that $E\sd F=\bigdu_{n=1}^m E_n$.
\end{enumerate}
\end{definition}

This definition of semiring is a slight variation of the one first given by von Neumann. We leave to the reader the trivial verification that the collection $P$ of the previous section (or of Example 1.10) is a semiring. Another important example of a semiring, which in a sense generalizes the example $P$ just considered, is the collection of all differences of compact subsets of a topological space. This example will be particularly useful to us when we consider measure theory on locally compact topological spaces.

\begin{definition}
A non-negative (extended real valued) function $\mu$ defined on a semiring $P$ is called a \underline{premeasure} on $P$ if $\mu$ is countably additive on $P$.
\end{definition}

Thus Theorem 1.14 says that the functions $\mu_\alpha$ of that theorem are premeasures. We also remark that every non-negative measure is also a premeasure. In the definition of a premeasure we could also have allowed the values to be in a Banach space. But we do not know whether the theorems which we will prove shortly about extending a premeasure to a measure are true in the case of Banach space valued premeasures. Certainly our proofs will use strongly the fact that the values of a premeasure are in $\bR^\infty$.

As was the case for measures, it is easily seen that if a premeasure $\mu$, has finite value on at least one set, then $\mu(\varnothing)=0$.

The extension theorem which we wish to prove states in part that a premeasure $\mu$, on a semiring $P$ can be extended to a measure on $\sring{P}$, the $\sigma$-ring generated by $P$. But before beginning the discussion of this theorem, we need three lemmas concerning premeasures. In each of the following lemmas, $P$ is an arbitrary semiring and $\mu$ is an arbitrary premeasure on $P$.

\begin{lemma}
If $E, E_1, E_2, \dots, E_m\in P$ then there exist $F_i \in P$, $i=1, \dots, k$ such that $(((E\sd E_1)\sd E_2)\sd \cdots\sd E_m)=\bigdu_{i=1}^k F_i$ 
\end{lemma}
\begin{proof}
We use induction on $m$. If $m=1$ then the lemma is true by the definition of a semiring. Now suppose that the lemma is true for $m-1$ where $m>1$. Then there exist $F_i'$, $i=1, \dots, k'$ such that $((((E\sd E_1)\sd E_2)\sd\cdots\sd E_{m-1})\sd E_m)=\br{\bigdu_{i=1}^{k'}F_i'}\sd E_m$. But $\br{\bigdu_{i=1}^{k'}F_i'}\sd E_m=\bigdu_{i=1}^{k'}(F_i'\sd E_m)$. Thus, using the definition of a semiring to express each $F_i'\sd E_m$ as the disjoint union of a finite number of elements of $P$, we obtain the desired result.
\end{proof}

\begin{lemma}
If $E \supseteq \bigdu_{i=1}^\infty E_i$ where $E, E_i \in P$, then $\mu(E) \geq \sum_{i=1}^\infty \mu(E_i)$. (As a special case we see that a premeasure is monotone, that is, if $E \subseteq F$, and $E, F \in P$, then $\mu(E) \leq \mu(F)$.)
\end{lemma}

\begin{proof}
We note first that we have not assumed that $\bigdu_{i=1}^\infty E_i$ is in $P$. Now, to prove the lemma it is sufficient to show that $\mu(E) \geq \sum_{i=1}^m\mu(E_i)$ for each finite $m$. By Lemma 1.17, $(((E\sd E_1)\sd E_2)\sd\cdots\sd E_m) =\bigdu_{i=1}^k F_i$ where $F_i \in P$. Then $E=E_1 \du E_2 \du \cdots \du E_m\du\bigdu_{i=1}^kF_i$. Thus $\mu(E)=\mu(E_1)+\cdots+\mu(E_m)+\sum_{i=1}^k\mu(F_i)$. Since $\mu$ is non-negative, it follows that $\mu(E) \geq \sum_{i=1}^m \mu(E_i)$.
\end{proof}

\begin{lemma}
A premeasure is countably subadditive, that is, if $E \subseteq \bigcup_{i=1}^\infty E_i$ where $E, E_i \in P$, then $\mu(E)\leq\sum_{i=1}^\infty\mu(E_i)$ 
\end{lemma}

\begin{proof}
Clearly $E=\bigcup_{i=1}^\infty(E \cap E_i)$. Let $E \cap E_i=E_i'$ so that $E=\bigcup_{i=1}^\infty E_i'$. Using Lemma 1.17 repeatedly we obtain 
\begin{align*}
    E=\bigcup_{i=1}^\infty E_i'&=E_1'\du(E_2'\sd E_1')\du\cdots\du\br{E_m'\sd\bigcup_{j=1}^{m-1}E_j'}\du\cdots\\
    &=E_1'\du\bigdu_{i=1}^{k_2}F_{2i}\du\cdots\du\bigdu_{i=1}^{k_m} F_{mi}\du\cdots\\
\end{align*}
Therefore, $\mu(E)=\mu(E_1')+\sum_{i=1}^{k_2}\mu(F_{2i})+\cdots+\sum_{i=1}^{k_m}\mu(F_{mi})+\cdots$. But $\bigdu_{i=1}^{k_m}F_{mi}\subseteq E_m'\subseteq E_m$, so by Lemma 1.18, $\sum_{i=1}^{k_m}\mu(F_{mi})\leq\mu(E_m)$. Thus $\mu(E)\leq\sum_{i=1}^\infty\mu(E_i)$.
\end{proof}

\section{The Extension of Premeasures to Outer Measures}

In order to extend a premeasure to a measure we first extend the premeasure to a set function, which is not necessarily a measure, but which has a very large domain. We then restrict this set function to a smaller domain in such a way that the restricted function is a measure.

\begin{definition}
If $P$ is any collection of subsets of a set $X$ we say $E\subseteq X$ is \underline{countable covered by} $P$ if there is a countable collection $\brc{E_n}_{n=1}^\infty$ of elements of $P$ such that $E\subseteq\bigcup_{n=1}^\infty E_n$.
\end{definition}

It is easily seen that if $P$ is any collection of sets, then the collection $\hring{P}$, of all those sets which are countably covered by $P$ is a $\sigma$-ring with the additional property that if $E\in\hring{P}$ and $F\subseteq E$ then $F\in\hring{P}$.

\begin{definition}
A $\sigma$-ring $H$ with the property that if $E\in H$ and $F\subseteq E$ then $F\in H$ is called a \underline{hereditary $\sigma$-ring}.
If $P$ is any collection of sets, then $\hring{P}$ will be called the \underline{hereditary $\sigma$-ring generated by $P$}.
\end{definition}

\begin{definition}
An \underline{outer measure} is a non-negative (extended real valued) function, $\mu^*$, whose domain is a hereditary $\sigma$-ring $H$, such that
\begin{enumerate}[label=\arabic*)]
    \item $\mu^*$ is monotone, that is, if $E\subseteq F$, $F\in H$, then $\mu^*(E)\leq\mu^*(F)$
    \item $\mu^*$ is countably subadditive, that is, if $E\subseteq\bigcup_{n=1}^\infty E_n$, $E_n\in H$, then $\mu^*(E)\leq\sum_{n=1}^\infty\mu^*(E_n)$.
\end{enumerate}
\end{definition}

\begin{theorem}
If $\mu$ is a premeasure on a semiring $P$ and if $\mu^*$ is defined on $\hring{P}$ by $$\mu^*(A)=\inf \brc{\sum_{n=1}^\infty\mu(E_n): A\subseteq\bigcup_{n=1}^\infty E_n, E_n \in P},$$ then $\mu^*$ is an outer measure on $\hring{P}$ which extends $\mu$ (that is, $\mu^*(E)=\mu(E)$ for each $E\in P$.)
\end{theorem}

\begin{proof}
Property 1) in the definition of an outer measure is clearly satisfied by $\mu^*$. So we need first to prove property 2), that is, that if $A\subseteq\bigcup_{i=1}^\infty A_i$, where each $A_i\in\hring{P}$ then $\mu^*(A)\leq\sum_{i=1}^\infty \mu^*(A_i)$. Now if $\mu^*(A_i)=\infty$ for some $i$, then the result is certainly true, and so we may assume that $\mu^*(A_i)<\infty$ for every $i$.

Let $\varepsilon>0$ be given. Choose $\brc{E_{ij}}_{i, j=1}^\infty\subseteq P$ such that $A_i\subseteq\bigcup_{j=1}^\infty E_{ij}$ and $\mu^*(A_i)\geq\sum_{j=1}^\infty\mu(E_{ij})-\frac{\varepsilon}{2^i}$. Then $A\subseteq\bigcup_{i,j=1}^\infty E_{i j}$ and $\mu^*(A)\leq\sum_{i,j=1}^\infty\mu(E_{ij})=\sum_{i=1}^\infty\br{\sum_{j=1}^\infty\mu(E_{i j})}\leq \sum_{i=1}^\infty\br{\mu^*(A_i)+\frac{\varepsilon}{2^i}}=\sum_{i=1}^\infty\mu^*(A_i)+\varepsilon$. Since $\varepsilon$ was arbitrary it follows that $\mu^*$ is countably subadditive. Thus $\mu^*$ is an outer measure.

Now we need to show that $\mu^*$ extends $\mu$. Clearly $\mu^*(A)\leq\mu(A)$ for all $A\in P$ by the definition of $\mu^*$, so all that we need to show is that $\mu^*(A)\geq\mu(A)$ if $A\in P$, that is, that $\sum_{i=1}^\infty\mu(E_i)\geq\mu(A)$ if $A\subseteq \bigcup_{i=1}^\infty E_i$ and $A, E_i \in P$. But this is just the fact that a premeasure is countably subadditive.
\end{proof}

\begin{definition}
The outer measure, $\mu^*$, defined in the statement of Theorem 1.23 is called the \underline{outer measure determined by the premeasure $\mu$}.
\end{definition}

\section{Measures from Outer Measures}

Outer measures will in general not be measures. We now begin the process of restricting the domain of an outer measure so as to obtain a measure.

\begin{definition}
Given a hereditary $\sigma$-ring $H$ and an outer measure $\mu^*$ on $H$ we call a set $E\in H$ \underline{$\mu^*$-measurable} if for every $A\in H$ we have the equality $\mu^*(A)=\mu^*(A\cap E)+\mu^*(A\sd E)$. We will denote the collection of all $\mu^*$-measurable sets by $\mring{\mu^*}$.
\end{definition}

This ingenious definition is due to Caratheodory in 1918. It says that $E$ splits all elements of $H$ nicely with respect to $\mu^*$. Note that it is always true that $\mu^*(A) \leq \mu^*(A\cap E)+\mu^*(A\sd E)$ because of the subadditivity of $\mu^*$. Thus to show that a set $E$ is $\mu^*$-measurable we need only prove the opposite inequality, $\mu^*(A)\geq\mu^*(A\cap E)+\mu^*(A\sd E)$, for all $A\in H$

\begin{theorem}
If $\mu^*$ is an outer measure on a hereditary $\sigma$-ring $H$, then either $\mring{\mu^*}$ is empty, or it is a $\sigma$-ring and the restriction of $\mu^*$ to $\mring{\mu^*}$ is a measure.
\end{theorem}

\begin{proof}

We divide the proof into two main steps.

\begin{lemma}
If $\mring{\mu^*}$ is not empty, then it is a ring.
\end{lemma}
\begin{proof}
We need to prove that if $E,F \in \mring{\mu^*}$, then $E \cup F$ and $E\sd F \in \mring{\mu^*}$. Let ${A} \in {H}$ be arbitrary. Then, using the fact that $E$ and ${F}$ are $\mu^*$-measurable, we obtain the following two strings of inequalities:
\begin{align*}
    \mu^*(A)&\leq\mu^*(A \cap(E \cup F))+\mu^*(A\sd(E \cup F)) \\
    &=\mu^*((A \cap E) \du((A-E) \cap F))+\mu^*((A\sd E)\sd F) \\
    & \leq \mu^*(A \cap E)+\mu^*((A\sd E) \cap F)+\mu^*((A\sd E)\sd F) \\
    &=\mu^*(A \cap E)+\mu^*(A\sd E)=\mu^*(A)
\end{align*}
and
\begin{align*}
\mu^*(A) & \leq \mu^*(A \cap(E\sd F))+\mu^*(A\sd (E\sd F)) \\
&=\mu^*((A \cap E)\sd F)+\mu^*((A\sd E) \cup(A \cap F)) \\
&=\mu *((A \cap E)\sd F)+\mu^*((A\sd E) \du(A \cap E \cap F)) \\
& \leq \mu^*((A \cap E)\sd F)+\mu^*(A\sd E)+\mu^*(A \cap E \cap F) \\
&=\mu^*(A \cap E)+\mu^*(A\sd E)=\mu^*(A) .
\end{align*}
Since both strings begin and end with $\mu^*(A)$, all inequalities are actually equalities. Thus the first line of the first string shows that $E \cup F \in \mring{\mu^*}$ and the first line of the second string shows that $E-F \in \mring{\mu^*}$. Therefore $\mring{\mu^*}$ is a ring if it is not empty.
\end{proof}

\begin{lemma}
Either $\mring{\mu^*}$ is empty, or it is a $\sigma$-ring, in which case if $E=\bigdu_{i=1}^\infty E_i$ where $E_i\in\mring{\mu^*}$, then $\mu^*(A \cap E)=\sum_{j=1}^\infty \mu^*(A \cap E_i)$ for all ${A} \in {H}$
\end{lemma}
\begin{proof}
We remark first that if $E$ and $F$ are disjoint elements of $\mring{\mu^*}$, then $$\mu^*(A \cap(E \du F))=\mu^*(A \cap E)+\mu^*(A \cap F),$$ for this is easily seen to be the same as the statement that $E$ splits the set $A \cap(E \du F)$ nicely with respect to $\mu^*$.

Now to prove that $\mring{\mu^*}$ is a $\sigma$-ring we only need to show that $\mring{\mu^*}$ is closed under formation of countable unions. Suppose that $E=\bigcup_{i=l}^\infty E_i$, where each $E_i \in \mring{\mu^*}$. We wish to show that $E \in \mring{\mu^*}$. Now we can assume that the $E_i$'s are disjoint, since if they are not, we can disjointize them by letting $E_k'=E_k-\bigcup_{i=1}^{k-1} E_i$. Clearly each $E_{k}' \in \mring{\mu^*}$, since $\mring{\mu^*}$ is a ring.

Let $A \in {H}$. Then we have
\begin{align*}
    \mu^*(A)&=\mu^*\br{A\cap\br{\bigdu_{i=1}^mE_i}}+\mu^*\br{A\sd\br{\bigdu_{i=1}^mE_i}}& &\text{since }\mring{\mu^*}\text{ is a ring}\\
    &\geq\mu^*\br{A\cap\br{\bigdu_{i=1}^mE_i}}+\mu^*(A\sd E)& &\text{by monotonicity of }\mu^*\\
    &=\sum_{i=1}^m \mu^*(A \cap E_i)+\mu^*(A\sd E)& &\text{by lemma 1.}
\end{align*}
Since this holds for every finite $m$ and since $\bigdu_{i=1}^\infty(A \cap E_i)=A \cap E$ we thus have $\mu^*(A) \geq \sum_{i=1}^\infty \mu^*\left(A \cap E_i\right)+\mu^*(A\sd E) \geq \mu^*(A \cap E)+\mu^*(A\sd E)$ since $\mu^*$ is subadditive. Thus $E$ is $\mu^*$-measurable, and consequently $\mring{\mu^*}$ is a $\sigma$-ring if it is not empty.

To prove the second part of the lemma we note that we can also conclude from the above that $\mu^*(A)=\sum_{i=1}^\infty \mu^*\left(A \cap E_i\right)+\mu^*(A\sd E)$. Then if we replace $A$ by $A \cap E$ we obtain $\mu^*(A \cap E)=\sum_{i=1}^\infty \mu^*\left(A \cap E_i\right)+\mu^*(\varnothing)=\sum_{i=1}^\infty \mu^*\left(A \cap E_i\right)$ as desired, where we have used the fact that if $\mring{\mu^*}$ is not empty then some set splits $\varnothing$ nicely, so that either $\mu^*(\varnothing)=0$ or else $\mu^*(E)=\infty$ for all $E \in H$.
\end{proof}

To conclude the proof of Theorem 1.26 all that we need to show is that $\mu^*$ restricted to $\mring{\mu^*}$ is a measure, that is, that if $E=\bigdu_{i=1}^\infty E_i$, with each $E_i\in\mring{\mu^*}$, then $\mu^*(E)=\sum_{i=1}^\infty \mu^*(E_i)$. But to obtain this we need only let $A=E$ in the second statement of Lemma 1.28.
\end{proof}

We remark that it is entirely possible for $\mring{\mu^*}$ to be empty (see exercise 4).

\begin{definition}
A non-negative measure $\mu$ on a $\sigma$-ring ${S}$ is \underline{complete} if whenever $F \subseteq E$, $E \in S$, and $\mu(E)=0$, then $F \in S $. (And, of course, $\mu(F)=0$.)
\end{definition}

\begin{proposition}
If $\mu^*$ is an outer measure, if $\mring{\mu^*} \neq \varnothing$, and if $\tilde{\mu}$ is the restriction of $\mu^*$ to $\mring{\mu^*}$, then $\tilde{\mu}$ is a complete measure.
\end{proposition}

\begin{proof}
It is sufficient to show that if $\mu^*(E)=0$ then $E \in \mring{\mu^*}$. Let $A \in H$. Then clearly $\mu^*(A) \geq \mu^*(A \cap E)+\mu^*(A\sd E)$, since $\mu^*(A) \geq \mu^*(A\sd E)$ and $\mu^*(A \cap E)=0$ by the monotonicity of $\mu^*$. Therefore $E \in \mring{\mu^*}$.
\end{proof}

At this point we do not yet know that we have proved a really useful theorem. For example, we do not know whether $\mring{\mu^*}$ is ever non-empty. Also, we do not have an extension theorem yet. We solve these problems with the following theorem.

\begin{theorem}
If $\mu$ is a premeasure on a semiring $P$, and if $\mu^*$, defined on $\hring{P}$, is the outer measure which $\mu$ determines, then $P \subseteq \mring{\mu^*} .$
\end{theorem}
\begin{proof}
We need to show that if $E \in P$, then $E \in \mring{\mu^*}$, that is, that for all $A \in \hring{P}, \mu^*(A)=\mu^*(A \cap E)+\mu^*(A\sd E)$. If $\mu^*(A)=\infty$ then we are clearly done since it is then clear that $\mu^*(A) \geq \mu^*(A \cap E)+\mu^*(A\sd E)$. Thus, we may assume that $\mu^*(A)<\infty$. Suppose we are given $\varepsilon>0$. Then, since $A \in \hring{P}$, we can choose elements $F_i$ of $P$ such that $A \subseteq \bigcup_{i=1}^\infty F_i$ and $\mu^*(A)+\varepsilon \geq \sum_{i=1}^\infty \mu(F_i)$. Now $F_i=(E \cap F_i) \du(F_i\sd E)$, so by the definition of a semiring there exist $G_{ij} \in P$, such that $F_i=(E \cap F_i) \du\br{\bigdu_{j=1}^{k_i}G_{ij}}$. Then
\begin{align*}
\sum_{i=1}^\infty \mu(F_i)&=\sum_{i=1}^\infty(\mu(E \cap F_i)+\sum_{j=1}^{k_i} \mu(G_{i j})) \\
&=\sum_{i=1}^\infty \mu(E \cap F_i)+\sum_{i=1}^\infty \sum_{j=1}^{k_i} \mu(G_{i j}).
\end{align*}
Now we note that $A \cap E \subseteq \bigcup_{i=1}^\infty\left(F_i \cap E\right)$ and that $A\sd E \subseteq \bigcup_{i=1}^\infty\left(F_i\sd E\right)=\bigcup_{i=1}^\infty \bigcup_{j=1}^{k_i} G_{ij} $. Thus $$\mu^*(A)+\varepsilon \geq \sum_{i=1}^\infty \mu(F_i)=\sum_{i=1}^\infty \mu\left(F_i \cap E\right)+\sum_{i=1}^\infty \sum_{j=1}^{k_i} \mu\left(G_{i j}\right) \geq \mu^*(A \cap E)+\mu^*(A\sd E).$$
Since $\varepsilon$ is arbitrary, we obtain the desired result.
\end{proof}

As a corollary of the above results we obtain:

\begin{theorem}
\underline{The Extension Theorem.} If $\mu$ is a premeasure on a semiring $P$ and if $\mu^*$, defined on $\hring{P}$, is the outer measure determined by $\mu$, then 1) if $\bar{\mu}$ is the restriction of $\mu^*$ to $\sring{P}$, then $\bar{\mu}$ is a measure which extends $\mu$. 2) if $\tilde{\mu}$ is the restriction of $\mu^*$ to $\mring{\mu^*}$, then $\tilde{\mu}$ is a complete measure which extends $\bar{\mu}$ and so $\mu$.
\end{theorem}

\begin{definition}
If $P, \alpha$ and $\mu_\alpha$ are as in 1.13, then $\bar{\mu}_\alpha$ is called the \underline{Stieltjes-Borel measure} on $R$ corresponding to $\alpha$ (recall that $\sring{P}$ is the $\sigma$-ring of Borel sets of $R$), and $\tilde{\mu}_\alpha$ is called the \underline{Stieltjes-Lebesgue measure} on ${R}$ corresponding to $\alpha$. If $\alpha(x)=x$, then $\bar{\mu}_\alpha$ is called Borel measure on $R, \tilde{\mu}_\alpha$ is called \underline{Lebesgue measure} on $R$, and $\mring{\mu_\alpha^*}$ is called the $\sigma$-ring of \underline{Lebesgue measurable subsets} of $R$.
\end{definition}

Implicit in the wording of the last paragraph is the fact that, given $\alpha$, the corresponding $\bar{\mu}_\alpha$ and $\tilde{\mu}_\alpha$ are unique. We have not yet proved anything like this, but the uniqueness of extensions is the subject of the next section, and uniqueness of Stieltjes-Borel and Stieltjes-Lebesgue measure will follow from the results obtained there.

\section{Uniqueness of Extensions}

Before going on to prove uniqueness we first show that we cannot extend a premeasure $\mu$, to a measure on a larger $\sigma$-ring than $\mring{\mu^*}$ by iterating the process described above.

\begin{proposition}
Let $\mu$ be a premeasure on a semiring $P$ and let $\mu^*$ be the outer measure determined by $\mu$. Then $\hring{P}=\hring{\mring{\mu^*}}$, and the outer measures determined by $\bar{\mu}$ and $\tilde{\mu}$ are both just $\mu^*$. More precisely, if $E \in \hring{P}$, then $\mu^*(E)=\inf \brc{\bar{\mu}(F): E \subseteq F, F \in \sring{P}}=\inf \brc{\tilde{\mu}(F): E \subseteq F, F \in \mring{\mu^*}}$.
\end{proposition}

\begin{proof}

$\hring{P}=\hring{\mring{\mu^*}}$ because $P \subseteq \mring{\mu^*} \subseteq \hring{P}$. To prove the second statement it suffices to show that for each $E \in \hring{P}$ the following string of inequalities holds:
\begin{align*}
    \mu^*(E) &\geq \inf\brc{\sum_{i=1}^\infty\mu(E_i): E \subseteq \bigcup_{i=1}^\infty E_i, E_i \in P}\\
    &\geq \inf \brc{\bar{\mu}(F): E \subseteq F, F \in \sring{P}}\\
    &\geq \inf \brc{\tilde{\mu}(F): E \subseteq F, {F} \in \mring{\mu^*}}\\
    &\geq \mu^*(E).
\end{align*}
Now the first inequality follows from the definition of $\mu^*$. The second inequality follows from the fact that if $F=\bigcup_{i=1}^\infty E_i$ where the $E_i \in P$, then $\bar{\mu}(F) \leq \sum_{i=1}^\infty \mu\left(E_i\right)$ by the countable subadditivity of $\mu^*$. The third inequality holds because the term on the right is the infimum of a larger set. Finally, the last inequality follows from the monotonicity of $\mu^*$.
\end{proof}

\begin{corollary}
If $E \in \hring{P}$, then there exists $F \in \sring{P}$ such that $E \subseteq F$ and $\bar{\mu}(F)=\mu^*(E)$
\end{corollary}
\begin{proof}
By Proposition 1.34 there must exist a sequence $\left\{{F}_i\right\}_{i=1}^\infty$ of elements of $\sring{P}$ such that $E \subseteq F_i$ for each $i$ and $\lim _{i \to \infty} \bar{\mu}\left(F_i\right)=\mu^*(E)$ Let $F=\bigcap_{i=1}^\infty F_i$.
\end{proof}

In order to prove that the extensions we have obtained are unique we need to make an additional hypothesis (see Exercise 12).

\begin{definition}
Let $\mu$ be a non-negative set function (such as a premeasure, measure, or outer measure) defined on a collection $P$, of subsets of $X$. Then $E \subseteq X$ is said to be \underline{$\sigma$-finite} for $\mu$ if there exist $E_i \in P$ such that $\mu\left(E_i\right)<\infty$ for each $i$, and $E \subseteq \bigcup_{i=1}^\infty E_i$. If each $E \in P$ is $\sigma$-finite, then $\mu$ itself is said to be \underline{$\sigma$-finite}. If in fact $X$ is $\sigma$-finite, then $\mu$ is said to be \underline{totally $\sigma$-finite}.
\end{definition}

Being $\sigma$-finite will be an important hypothesis in many theorems in addition to the theorem concerning the uniqueness of extensions. Many pathologies occur with measures that are not $\sigma$-finite.

\begin{proposition}
If $\mu$ is a $\sigma$-finite premeasure then $\mu^*$ is $\sigma$-finite, so, in particular, $\bar{\mu}$ and $\tilde{\mu}$ are $\sigma$-finite.
\end{proposition}
\begin{proof}
Obvious
\end{proof}

\begin{theorem}(Uniqueness of extensions) If $\mu$ is a $\sigma$-finite premeasure on a semiring $P$, if $S$ is a $\sigma$-ring such that $\sring{P} \subseteq S \subseteq \mring{\mu^*}$ and if $\nu$ is a non-negative extension of $\mu$ to a measure on ${S}$, then $\nu$ coincides with the restriction of $\mu^*$ to ${S}$.
\end{theorem}

\begin{proof}
If $E \in S$ and $E \subseteq \bigcup_{i=1}^\infty E_i$ where each $E_i \in P$, then $\nu(E) \leq \sum_{i=1}^\infty \nu(E_i)=\sum_{i=1}^\infty \mu(E_i)$ by the countable subadditivity of non-negative measures and by the fact that $\nu$ extends $\mu$. Thus $\nu(E) \leq \mu^*(E)$ for all $E \in S$. Thus we need to show that $\mu^*(E) \leq \nu(E)$ for all $E \in S$.

\underline{Case 1}. Suppose first that $E$ is such that there exists $F \in P$ for which $E \subseteq F$ and $\mu(F)<\infty$. Then, since $F=E \du(F\sd E)$, we have $\nu(F)=\mu(F)=\mu^*(F)=\mu^*(E)+\mu^*(F\sd E) \geq \nu(E)+\nu(F\sd E)=\nu(F)$. Thus $\mu^*(E)+\mu^*(F\sd E)=\nu(E)+\nu(F\sd E)$. But $\mu^*(E) \geq \nu(E)$ and $\mu^*(F\sd E) \geq \nu(F\sd E)$ and these quantities are all assumed to be finite, so we must have $\mu^*(E)=v(E) .$

\underline{Case 2}. Suppose now that $E \in S$ is arbitrary. Then, since $\mu$ is assumed to be $\sigma$-finite, there exist $F_i \in P$ such that $\mu\left(F_i\right)<\infty$ for each $i$ and $E \subseteq \bigcup_{j=1}^\infty F_i .$ Using the usual disjointizing process, we can then obtain sets ${G}_i$ in $\sring{P}$ such that $\bar{\mu}\left({G}_i\right)<\infty$ for each $i$ and $E \subseteq \bigdu_{i=1}^\infty {G}_i$. Then, using Case 1, we see that $\nu(E)=\sum_{i=1}^\infty \nu\left(E \cap {G}_i\right)=\sum_{i=1}^\infty \mu^*\left(E \cap G_i\right)=\mu^*(E)$.
\end{proof}

\begin{corollary}
Any non-negative measure $\mu$ defined on the Borel subsets of the real line which is finite on finite intervals (and so is $\sigma$-finite) is $\bar{\mu}_\alpha$ for some monotone non-decreasing left continuous function $\alpha$.
\end{corollary}

\begin{proof}
Define $\alpha$ by $\alpha(t)=\mu([0, t))$ if $t \geq 0$ and $\alpha(t)=-\mu([t, 0))$ if $t<0$. Then $\alpha$ is clearly non-decreasing, and it is easily seen, using the countable additivity of $\mu$, that $\alpha$ is left continuous. Clearly $\bar{\mu}_\alpha$ agrees with $\mu$ on the family of left closed, right open finite intervals, and so $\bar{\mu}_\alpha$ agrees with $\mu$ on the class of Borel sets, by the theorem concerning the uniqueness of extensions.
\end{proof}

We will now examine the structure of $\mring{\mu^*}$ more closely and show that if $\mu^*$ is determined by a $\sigma$-finite premeasure, then $\mring{\mu^*}$ has a rather simple form.

\begin{definition}
Let $\mu$ be a non-negative measure on a $\sigma$-ring $S$. A subset $E$ of $X$ is called a \underline{null set} with respect to $\mu$ if there exists $F \in S$ such that $E \subseteq F$ and $\mu(F)=0$ (or equivalently, if $\mu^*(E)=0$). The null sets form a hereditary $\sigma$-ring which we will denote by $\nring{\mu}$.
\end{definition}

Our first result does not involve $\sigma$-finiteness.

\begin{proposition}
Let $\mu$ be a non-negative measure on a $\sigma$-ring $S$. Let $S\du\nring{\mu}=\{E\du F: E \in S, F \in \nring{\mu}\}$ and define $\hat{\mu}$ on $S \du \nring{\mu}$ by $\hat{\mu}(E \du F)=\mu(E)$ where $E \in S$ and $F \in \nring{\mu}$. Then $S \du \nring{\mu}$ is a $\sigma$-ring, $\hat{\mu}$ is well defined and $\hat{\mu}$ is a completion of $\mu$, that is, $\hat{\mu}$ is an extension of $\mu$ to a complete measure.
\end{proposition}

\begin{proof}
We leave to the reader the trivial proof that $S\du \nring{\mu}$ is a $\sigma$-ring. The only slightly non-trivial thing to prove is that $\hat{\mu}$ is well defined. Let $E_1 \du F_1=E_2 \du F_2$, where $E_1, E_2 \in S$ and $F_1, F_2 \in \nring{\mu}$. Then we must show that $\mu(E_1)=\mu(E_2)$. But $E_1\sd E_2=E_1 \cap F_2 \subseteq F_2$ and $E_2\sd E_1=E_2 \cap F_1 \subseteq F_1$, and so $\mu(E_1\sd E_2)=\mu(E_2\sd E_1)=0$. Since $\mu(E_1)$ $=\mu(E_1 \cap E_2)+\mu(E_1\sd E_2)$ and $\mu(E_2)=\mu(E_1 \cap E_2)+\mu(E_2\sd E_1)$, it follows that $\mu(E_1)=\mu(E_2)$.

\end{proof}

\begin{theorem}
If $\mu$ is a $\sigma$-finite non-negative measure on a $\sigma$-ring $S$, then $\mring{\mu^*}=S \du \nring{\mu}$
\end{theorem}

\begin{proof}
Clearly ${S} \du \nring{\mu} \subseteq \mring{\mu^*}$. Thus we must show that $\mring{\mu^*} \subseteq S \du \nring{\mu}$

\underline{Case 1}. Suppose first that $E \in \mring{\mu^*}$ is such that $\mu^*(E)<\infty$. By Corollary $1.35$ there exists $F \in S$ such that $E \subseteq F$ and $\mu(F)=\mu^*(E)$. Then $\mu(F)=\mu^*(E)+\mu^*(F\sd E)$ so $\mu^*(F\sd E)=\mu(F)-\mu^*(E)=0$. Using the corollary again, we see that there exists $G \in S$ such that $F\sd E \subseteq G$ and $\mu(G)=0$. Since $F\sd E \subseteq G$, we have $E\sd G=F\sd G$, and so $E=(E\sd G) \du(E \cap G)=(F\sd G) \du(E \cap G)$. But $F\sd G \in S$ and $E \cap G \in \nring{\mu}$, and so $E \in S \du \nring{\mu}$

\underline{Case 2}. Suppose now that $E \in \mring{\mu^*}$ is arbitrary. Since $\mu$ is assumed to be $\sigma$-finite, so is $\mu^*$, and so there exist sets $E_i \in \mring{\mu^*}$ such that $\mu^*(E_i)<\infty$ for each $i$ and $E=\bigcup_{i=1}^\infty E_i .$ (To get $E=\bigcup_{i=1}^\infty E_i$ rather than $E \subseteq \bigcup_{i=1} E_i$ we can take intersections with $E$ if necessary.) By Case $1$, $E_i \in {S} \du\nring{\mu}$ for each $i$. Since $S \du \nring{\mu}$ is a $\sigma$-ring, it follows that $E \in S \du\nring{\mu}$.
\end{proof}

\begin{corollary}
If $\mu$ is a $\sigma$-finite premeasure on a semiring $P$, then $\mring{\mu^*}=\sring{P} \du\nring{\bar{\mu}}$.
\end{corollary}

Since Lebesgue measure is the most important example of a measure, it is of interest to know whether there exist subsets of the real line which are not Lebesgue measurable. It is not very difficult to construct such sets, and we suggest such a construction in Exercise 14 at the end of this chapter. However, an interesting feature of all known examples (including that of Exercise 14 ) is that they seem to require the use of the axiom of choice. In fact, R. Solovay has recently shown that there exist models for the Zermelo-Frankel axioms of set theory other than the uncountable axiom of choice, such that, in these models, it is true that every subset of the real Iine is Lebesgue measurable.

It is also of interest to know that the Borel subsets of the real line form a proper subset of the class of all Lebesgue measurable subsets of the real line. For this fact, see problem 10 at the end of this chapter.

\section{Exercises}

\begin{enumerate}[label=\arabic*)]
\item Let $X$ be the set of positive integers and let $S$ be the family of all subsets of $X$. Determine which of the following are measures on $S$ (not all of them are).
\begin{enumerate}
\item $\mu$ has values in $\ell^\infty$, and, for each $E \in S$, $\mu(E)$ is the sequence whose $n$th term is $1 / n$ if $n \in E$ and 0 if $n \notin E$.
\item $\mu$ has values in $\ell^\infty$ and, for each $E \in S$, $\mu(E)$ is the sequence whose $n$th term is $1$ if $n \in E$ and $0$ if $n \notin E$.
\item $\mu$ is defined as in b) but its values are viewed as being elements of the Banach space of sequences a such that $\sum_{n=1}|a_n|/n^2<\infty$. In this space, $\norm{a}=\sum_{n=1}^\infty|a_n|/n^2$.
\item $\mu$ has values in $\ell^1$ and, for each $E \in {S}$, $\mu(E)$ is the sequence whose $n$th term is $\frac1{n^2}$ if $n \in$ E and $0$ if $n \notin$ E.
\end{enumerate}

\item Let $T$ be a collection of sets which is closed under taking finite unions and intersections. Show that the collection of sets of the form $E\sd F$ where $E, F \in T$ forms a semiring. Many interesting semirings arise in this way. For example, this shows that the collection of differences of compact subsets (or of open subsets) of a topological space forms a semiring. Show that the semiring of left-closed right open intervals of the real line also arises in this way.

\item In showing that the set function $\mu_\alpha$ defined on the left closed right open intervals of $\bR$ in terms of a non-decreasing function $\alpha$ is countably additive, we had to use a compactness argument and so the completeness of $\bR$, and we had to assume that $\alpha$ is left continuous.

\begin{enumerate}
\item Show that if $\alpha(x) \equiv x$, but that $\bR$ is replaced by $\bQ$, the rational numbers, then the corresponding function, $\mu_\alpha$, defined on the left closed right open subsets of $\bQ$, is not countably additive.
\item Show that if $\alpha$ is not left continuous then $\mu_\alpha$ is not countably additive.
\end{enumerate}

\item Let $X$ be a set and let ${S}$ be the family of all subsets of $X$. Define $\mu^*$ on ${S}$ by $\mu^*(E)=1$ for all $E \in {S}$. Show that $\mu^*$ is an outer measure, and determine its measurable sets.

\item A measure $\mu$ is called continuous if $\mu(\{x\})=0$ for every point ${x}$. If $\bar{\mu}_\alpha$ is a Borel-Stieltjes measure, show that $\bar{\mu}_\alpha$ is a continuous measure if and only if $\alpha$ is a continuous function.

\item If $\mu$ is a measure on a $\sigma$-ring $S$, then $E \in S$ is called an atom for $\mu$ if $\mu(E) \neq 0$ and for any $F \subseteq E, \mu(F)=\mu(E)$ or $\mu(F)=0$.
\begin{enumerate}
\item If $\bar{\mu}_\alpha$ is a Borel-Stieltjes measure, determine its atoms in terms of $\alpha$.
\item A measure is called \underline{purely atomic} if every measurable set of nonzero measure is a union of atoms. Determine which $\bar{\mu}_\alpha$ are purely atomic in terms of $\alpha$.
\item If $\alpha$ is such that $\bar{\mu}_\alpha$ is purely atomic, determine the measurable sets for $\bar{\mu}_\alpha$ (that is, for the corresponding outer measure).
\end{enumerate}


\item Unfortunately atoms need not be associated with points as above. Let ${X}$ be an uncountable set and let ${S}$ be the collection of subsets of ${X}$ which are either countable or whose complement is countable. Show that ${S}$ is a $\sigma$-field. Define $\mu$ on ${S}$ by $\mu(E)=0$ if ${X}$ is countable and $\mu(E)=1$ if $E$ is uncountable. Show that $\mu$ is a measure, and find its atoms.

\item A measure is called \underline{purely non-atomic} if it has no atoms. Prove that if $\mu$ is a purely non-atomic non-negative measure on a $\sigma$-ring $S$, if $E \in S$, and if $c$ is any constant such that $0 \leq c \leq \mu(E)$, then there exists an $F \in S$ such that $F \subseteq E$ and $\mu(F)=c$.

\item Prove that the cardinality of the collection of Borel subsets of the real line is equal to the cardinality of the real line, so that not every subset of the real line can be a Borel subset. Hit: let $P$ be the collection of open intervals, let $P^u$ be the collection of countable unions of elements of $P$, let $P^{uc}$ be the union of $P^u$ with the complements of elements of $P^u$, let $P^{ucu}$ be countable unions of elements of $P^{uc}$, etc. Show taht each collection of sets so obtained has the cardinality of the real line, and that the union of this sequence of collections is the $\sigma$-ring of Borel sets.

\item The Cantor set. Expand every number $x$ in the closed unit interval $X=[0,1]$ in the ternary system, that is, if $x\in X$, write $x=\sum_{n=1}^\infty\frac{\alpha_n}{3^n}$, with each $\alpha_n=0,1$ or $2$, and let $C$ be the set of all those numbers $x$ in whose expansion the digit $1$ is not needed. (Observe that if, motivated by the customary decimal notation, we write $\alpha_1\alpha_2\dots$ for $\sum_{n=1}^\infty\frac{\alpha_n}{3^n}$, then for instance $\frac13=.10000\dots=.02222\dots,$ and therefore $\frac13\in C$, but that since $\frac12=.111\dots$ and since this is the only ternary expansion of $\frac12$, therefore $\frac12\notin C$.) The set $C$ is called the Cantor set. Let $X_1$ be the open middle third of $X$, $X_1=\br{\frac13,\frac23}$; let $X_2,X_3$ be the open middle thirds of the two closed intervals which make up $X\sd X_1$, that is $X_2=\br{\frac19\frac29}$ and $X_3=\br{\frac79,\frac89}$; let $X_4,X_5,X_6,$ and $X_7$ be the open middle thirds of the four closed intervals which make up $X\sd(X_1\du X_2\du X_3)$, and so on. Prove the following statements.
\begin{enumerate}
\item $C=X\sd\bigdu_{n=1}^\infty X_n$.
\item If $\mu$ is Lebesgue measure on $\bR$ then $\mu(C)=0$.
\item The cardinality of $C$ is the same as the cardinality of the real line.
\item Conclude that the cardinality of the collection of Lebesgue measurable set is $2^c$ where $c$ is the cardinality of the real line. In view of exercise 9 this shows that there exists Lebesgue measurable sets which are not Borel sets.
\end{enumerate}

\item An outer measure $\mu^*$ is called \underline{regular} (an overused word) if for any $A\in H$ (its domain) there is a $\mu^*$-measurable set $E$ such that $E\supseteq A$ and $\mu^*(E)=\mu^*(A)$. We have seen (Corollary 1.35) that any outer measure induce by a premeasure is regular. Show that if $\mu^*$ is regular, if $A\in H$ and if $\mu^*$ is finitely additive on $\sring{\mring{\mu^*}\cup\brc{A}}$, then $A\in\mring{\mu^*}$. Thus $\mring{\mu^*}$ is the largest $\sigma$-ring on which $\mu^*$ is a measure (although the restriction of $\mu^*$ to $\mring{\mu^*}$ can sometimes be extended to be a measure on larger $\sigma$-rings in other ways).

\item Let $P$ be the semiring of left closed right open intervals of $\bQ$, the rational numbers, and let $\mu$ be defined on $P$ by $\mu(\varnothing)=0$, and $\mu(E)=\infty$ if $E\in P$, $E\neq\varnothing$. Show that $\mu$ is a premeasure which has many different extensions to a measure on $\sring{P}$. Thus some hypothesis such as $\sigma$-finiteness is needed to prove uniqueness of extensions.

\item If $E$ and $F$ are sets, then their symmetric difference, denoted by $E\triangle F$, is defined to be $(E\sd F)\cup(F\sd E)$. If $R$ is a ring of sets, show that, with $\triangle$ as addition and $\cap$ as multiplication, $R$ becomes a ring in the usual algebraic sense. If $R$ is a $\sigma$-ring and $\mu$ is a non-negative measure on $R$, show that the null sets of $\mu$ which are in $R$ form an ideal in $R$.

\item Let $\mu$ be Lebesgue measure on the real line, and let $\mu^*$ be the corresponding outer measure.
\begin{enumerate}
\item Show that $\mu$ and $\mu^*$ are translation invariant, that is, if $E\subseteq\bR$ and $r\in\bR$, and if $r+E=\brc{r+x:x\in E}$, then $\mu^*(r+E)=\mu^*(E)$, and similarly for $\mu$. (You must check the translation invariance of the domain of $\mu$.)
\item Let $G=[0,1)$ and define an addition, $\hat{+}$, on $G$ as follows (this is addition modulo $1$): if $r,s\in G$, then $r\hat{+}s=r+s$ if $r+s<1$, and $r\hat{+}s=r+s-1$ if $r+s\geq1$. Show that $G$ is a group under $\hat{+}$. (In fact the function $r\mapsto\text{exp}(2\pi ir)$ is an isomorphism of the group $G$ onto the multiplicative group of complex numbers of modulus 1). Show that restriction of $\mu$ and $\mu^*$ to $G$ are translation invariant with respect to $\hat{+}$.
\item Show that there is a subset of $[0,1)$ which is not Lebesgue measurable, as follows. Let $G_\bQ$ denote the subset of $G$ consisting of the rational numbers in $G$. Then $G_\bQ$ is a subgroup of $G$. Using axiom of choice, pick one element from each coset of $G_\bQ$ in $G$, and let $E$ be the set consisting of all these elements, that is $E$ is a set of coset representatives for $G_\bQ$. Show that $E$ cannot be Lebesgue measurable.
\end{enumerate}

\end{enumerate}