\documentclass[11pt]{book}
\usepackage[utf8]{inputenc}
\usepackage{physics}
\usepackage{amsthm}
\usepackage{amssymb}
\usepackage{amsfonts}
\usepackage{enumitem}
\usepackage{hyperref}
\hypersetup{
    colorlinks=true,
    linkcolor=blue,
    filecolor=magenta,      
    urlcolor=cyan,
    pdftitle={Overleaf Example},
    pdfpagemode=FullScreen,
    }
\usepackage[english]{babel}

\title{Lectures on Bochner-Lebesgue Integration}
\author{Marc Rieffel}
\date{October 2021}

%\binoppenalty=10000
%\relpenalty=10000
%\sloppy


%\theorembodyfont{\rm}
\newtheoremstyle{definition}%             % Name
  {}%                                     % Space above
  {}%                                     % Space below
  {}%                                     % Body font
  {}%                                     % Indent amount
  {\bfseries}%                            % Theorem head font
  {.}%                                    % Punctuation after theorem head
  { }%                                    % Space after theorem head, ' ', or \newline
  {\thmname{#1}\thmnumber{ #2}\thmnote{: #3}}%                         % Theorem head spec (can be left empty, meaning `normal')
\theoremstyle{definition}
\newtheorem{definition}{Definition}[chapter]  
\newtheorem{theorem}[definition]{Theorem}
\newtheorem{proposition}[definition]{Proposition}
\newtheorem{lemma}[definition]{Lemma}
\newtheorem{examples}[definition]{Examples}
\newtheorem{example}[definition]{Example}
\newtheorem{corollary}[definition]{Corollary}
\newtheorem*{exercises}{Exercises}


\renewcommand{\thesection}{\Alph{section}}
\renewcommand{\theenumi}{\alph{enumi})}
\renewcommand{\thedefinition}{\arabic{chapter}.\arabic{definition}}




\newcommand{\floor}[1]{\left\lfloor#1\right\rfloor}
\renewcommand{\abs}[1]{\left\lvert#1\right\rvert}
\newcommand{\br}[1]{\left(#1\right)}
\newcommand{\sbr}[1]{\left[#1\right]}
\newcommand{\brc}[1]{\left\{#1\right\}}
\newcommand{\brk}[1]{\left\langle#1\right\rangle}
\renewcommand{\bf}[1]{\mathbf{#1}}



\newcommand{\bC}{\mathbb{C}}
\newcommand{\bR}{\mathbb{R}}
\newcommand{\bQ}{\mathbb{Q}}
\newcommand{\bN}{\mathbb{N}}
\newcommand{\ep}{\varepsilon}

% not my favorite notations so I can change lol
%\newcommand{\setdiff}[2]{#1\setminus #2}

%set difference
\newcommand{\sd}{\setminus}
%complement of a set
\newcommand{\comp}[1]{{#1}^c}
%disjoint union
\newcommand{\du}{\sqcup}
%big disjoint union
\newcommand{\bigdu}{\bigsqcup}
%indicator function
\newcommand{\idf}[1]{\chi_{#1}}
%carrier of
\newcommand{\car}[1]{C(#1)}
%essential range
\newcommand{\er}[2]{\text{er}_{#2}(#1)}
%curly L
\newcommand{\cL}{{\cal L}}
%curly F
\newcommand{\cF}{{\cal F}}
%symmetric difference
\newcommand{\symd}{\Delta}

% definition that should be flexible
\newcommand{\sring}[1]{R(#1)}
\newcommand{\hring}[1]{{\cal H}(#1)}
\newcommand{\mring}[1]{{\cal M}(#1)}
\newcommand{\nring}[1]{{\cal N}(#1)}



\begin{document}

\maketitle

\setcounter{chapter}{-1}


\chapter*{Introduction}
%These notes are based on the first year graduate level course in measure theory which we have taught several times at the University of California, Berkeley. The aspect in which our treatment of measure theory differs from that given in most current texts is that, from the beginning, we develop the theory of the Bochner-Lebesgue integral, that is, of the integration of functions which have values in a Banach space. Our reason for wishing to introduce students to the Bochner-Lebesgue integral is, of course, that it is a useful tool in many areas of analysis. However, we would not consider it appropriate for first year graduate students to learn this material were it not for the fact that it requires scarcely more effort than is required to learn the usual theory of the integration of real-valued functions. We do not assume that students beginning the study of this material have ever knowingly seen a Banach space before, and in fact in these notes very little of the general theory of Banach spaces is developed. (For example, linear functionals are never mentioned). Put another way, if a student wishes to substitute "real or complex numbers" for "Banach space" throughout the notes, he will obtain a fairly standard treatment of (real or complex valued) integration theory. Only a few proofs will simplify somewhat by taking advantage of the order properties of the real numbers. However, we do feel that treating the case of functions with values in a Banach space has the pedagogical advantage of forcing a certain amount of clarification of some of the ideas involved.

The main prerequisite for reading these notes is a good background in undergraduate analysis, in particular a knowledge of the basic theorems concerning completeness and compactness in metric spaces, and also the rudiments of the theory of vector spaces.  It would also be helpful for motivational purposes if the reader had some slight previous exposure to the Lebesgue integral on the real line.  We expected that such was the case with students taking the course on which these notes are based, and so we did not spend much time motivating the early material of the course. But, in fact, a fair number of the students did not have such a background, and they did not seem to suffer much from this situation. In particular, from a logical point of view these notes do not at all require such an exposure to the theory of the Lebesgue integral.  Students taking the course on which these notes are based were also expected to take concurrently (or have taken) a course in general topology.  But the material from such a course is not needed until the last chapter of these notes, which is concerned with integration on locally compact topological spaces, though in some of the earlier chapters we have included occasional inessential examples and exercises which depend on the material of such a course (as well as, in a few other exercises, an acquaintance with such topics as ordinal arithmetic or group theory).

The first two times we taught the course on which these notes are based the course lasted a semester, and we had no difficulty covering the material of these notes. The third time we taught the course it lasted only a quarter, and we were able to cover only the first seven chapters, and at that, we had to treat a few peripheral topics somewhat rapidly. Because the last four chapters are essentially independent, there is some flexibility as to what material is covered and in what order.

Anyone who is familiar with the exposition of measure theory given by Halmos in his book "Measure Theory" will be aware of our indebtness to him.  In particular, the general lines of the exhibition in these notes follow those of his book fairly closely.  We have also been aided by chapter III of "Linear Operators Vol. I" by Dunford and Schwartz and "Int\'{e}gration" by Bourbaki, which are among the few books which contain an exposition of the Bochner-Lebesgue integral. Finally, the material on the Radon-Nikodym theorem as well as a fair number of the exercises were developed while we were writing our paper "The Radon-Nikodym theorem for the Bochner integral" (Tran. A.M.S. 131 (1968) 466-487), although these notes contain some improvements over the treatment of the Radon-Nikodym theorem given in that paper.


\tableofcontents

\chapter{Preliminaries on Banach spaces}

In this preliminary chapter we will define what is meant by a Banach space, give a few examples, and describe those few elementary properties of Banach spaces which will be needed in later chapters. We assume that the reader is familiar with the theory of metric spaces. In a few places before Chapter 8 we will mention topological spaces, but a reader who is not familiar with topological spaces will have no difficulty if at every occurrence of the words ``topological space'' they substitute the words ``metric space''.

\begin{definition}\label{def:norm}
Let $V$ be a vector space (over the real or complex numbers). A function, $\norm{\imarg}$, from $V$ into the non-negative real numbers is called a \defline{seminorm} on $V$ if it satisfies the following two properties:
\begin{enumerate}[label=\alph*)]
    \item $\norm{v+w}\leq\norm{v}+\norm{w}$ for all $v,w\in V$. (this inequality is called the triangle inequality)
    \item $\norm{sv}=|s|\norm{v}$ for all $v\in V$ and all scalars $s$.
\end{enumerate}
A seminorm, $\norm{\imarg}$, is called a \defline{norm} if, in addition, it satisfies the following property:
\begin{enumerate}[label=\alph*)]
    \setcounter{enumi}{2}
    \item If $v\in V$ and $\norm{v}=0$, then $v=0$.
\end{enumerate}
\end{definition}

\begin{definition}\label{def:norm space}
A \defline{normed vector space} (or, more briefly, a \defline{normed space}) is a vector space $V$ together with a particular norm on $V$.
\end{definition}



\begin{examples}
The following are examples of normed spaces:
\begin{enumerate}
    \item\label{ex:norm 1}
    The field, $\mathbb{R}$, of real numbers, or the field, $\bC$, of complex numbers, where the norm of an element is taken to be its absolute value.
    
    \item\label{ex:norm 2}
    The vector spaces $\bR^n$ and $\bC^n$, consisting of all $n$-tuples of real or complex numbers, with the norm of an $n$-tuple, $(s_1,\dots,s_n)$, defined by \[\norm{(s_1,\dots,s_n)}=\br{\sum_{i=1}^n|s_i|^2}^{1/2}.\] This norm is called the Euclidean norm.
    
    \item\label{ex:norm 3}
    The same vector spaces as in example \ref{ex:norm 2} above, but with the norm defined by \[\norm{(s_1,\dots,s_n)}=\sum_{i=1}^n|s_i|.\]
    
    \item The same vector spaces as in example \ref{ex:norm 2} and \ref{ex:norm 3} above, but with the norm defined by \[\norm{(s_1,\dots,s_n)}=\max\brc{|s_1|,\dots,|s_n|}.\]
    
    \item\label{ex:norm last}
    Let $X$ be a topological space, and let $C(X)$ denote the vector space of all complex-valued (or real-valued) bounded functions on $X$, with the norm of an element $f\in C(X)$ defined by \[\norm{f}=\sup\brc{|f(x)|:x\in X}.\] If, instead, we choose one fixed point, $x_0$, of $X$, and define $\norm{\imarg}$ by \[\norm{f}=|f(x_0)|,\] then we obtain an example of a seminorm on $C(X)$.
    
    \item\label{ex:norm non banach}
    Let $X$ be the closed unit interval, $[0,1]$, so that $C(X)$ is the vector space of all continuous functions on $[0,1]$ (real valued if desired), and define the norm of an element, $f$, of $C(X)$ to be \[\norm{f}=\int_0^1|f(x)|\dd{x},\] where the integral is a Riemann integral.
\end{enumerate}
\end{examples}

A norm, $\norm{\imarg}$, on a vector space $V$ can always be used to define a metric, $d$, on $V$, by setting \[d(v,w)=\norm{v-w}\] for any $v,w\in V$. We will leave to the reader the verification that $d$, defined in this way, is a metric. If, instead, $\norm{\imarg}$ is only a seminorm, then $d$, defined as above, will only be a semimetric (pseudometric), that is, it will satisfy all the properties of a metric except that the distance between some pairs of different points may be zero.

It follows from the above considerations that every normed space is a metric space, when we equip it with the metric obtained from its norm. We can thus discuss open sets, convergence of sequences, Cauchy sequences, completeness, compactness, etc., in a normed space. In later chapters, when we do mention such concepts in relation to a normed space, it will always be understood that they are defined with respect to the metric obtained from the norm of the normed space. In particular, whenever we speak of the topology of a normed space we will always mean the collection of all its subsets which are open with respect to the metric obtained from its norm.

\begin{definition}
A \defline{Banach space} is a normed space which is complete (with respect to the metric obtained from its norm).
\end{definition}

In other words, a Banach apace is a normed space which has the property that every Cauchy sequence of elements of the space has a limit.


\begin{examples}
\begin{enumerate}
    \item\label{ex:banach 1}
    The reader should have little difficulty in verifying that the normed spaces described in Examples \ref{ex:norm 1} through \ref{ex:norm last} are, in fact, Banach spaces.
    
    \item\label{ex:banach non ex}
    However, the reader should be able to convince themselves that the normed space described in Example \ref{ex:norm non banach} is not a Banach space. In fact, one of the principle results of the theory of the Lebesgue integral is to give a useful description of the (metric space) completion of the normed space of this example.
    
    \item\label{ex:banach 2}
    Let $\ell^1$ denote the vector space of all complex-valued sequences, $\brc{s_i}_{i=1}^\infty$, which have the property that $\sum_{i=1}^\infty|s_i|$ is finite, and define the norm of any such sequence to be this sum.
    
    \item\label{ex:banach 3}
    If $X$ is any set, then we will let $\ell^\infty(X)$ denote the vector space of all bounded complex-valued functions on $X$, and define the norm of an element, $f$, of $\ell^\infty(X)$ to be \[\norm{f}=\sup\brc{|f(x)|:x\in X}.\] If $X$ is viewed as being a metric space with the discrete metric, then this example is a special case of Example \ref{ex:norm last}. If $X$ is the set $\bN$ of positive integers, then we will write just $\ell^\infty$ instead of $\ell^\infty(\bN)$. Thus $\ell^\infty$ is the space of all bounded sequences.
\end{enumerate} 
\end{examples}

The reader should not have much difficulty in proving the fact that these last two examples are Banach spaces.

We conclude with a simple consequence of the triangle inequality which will be useful in a number of places.

\begin{proposition}
Let $V$ be a normed space. Then for any $v,w\in V$ we have \[|\norm{v}-\norm{w}|\leq\norm{v-w}.\]
\end{proposition}

\begin{proof}
Using triangle inequality, we obtain \[\norm{v}=\norm{(v-w)+w}\leq\norm{v-w}+\norm{w},\] so that $\norm{v}-\norm{w}\leq\norm{v-w}$. Reversing the roles of $v$ and $w$, we obtain the fact that $\norm{w}-\norm{v}\leq\norm{v-w}.$
\end{proof}

\subsection*{Exercises}
\begin{enumerate}[label=\arabic*)]
    \item Verify that the normed spaces of Examples \ref{ex:banach 1}, \ref{ex:banach 2}, and \ref{ex:banach 3} are Banach spaces.
    \item Verify that the normed space of Example \ref{ex:banach non ex} is not a Banach space.
\end{enumerate}

\chapter{Measure}



\section{The Domain of a Measure}

Given a set $X$, a measure on $X$ will be a rule which in some sense tells us the size of certain subsets of $X$. Thus a measure will be a function whose domain is a suitable collection of subsets of $X$. By ``suitable'' we mean that the family is closed under certain set operations.

\begin{definition}
A nonempty collection $R$ of subsets of $X$ is called a \defline{ring}\index{ring} if $R$ is closed under the formation of the union and difference of any two elements of $R$, that is, if $E,F\in R$, then $E\cup F$ and $E\sd F\in R$ (where $E\sd F=E\cap\comp{F}$, where $\comp{F}$ is the complement of $F$). 
\end{definition}

It is easy to see that any ring is also closed under the formation of the intersection of any two of its members, since $E\cap F=E\sd(E\sd F) = F\sd (F\sd E)$. Furthermore, it can be shown by simple inductive proofs that any ring is closed under the formation of arbitrary finite unions and finite intersections, that is, if $E_1,\dots,E_n\in R$ then $\bigcup_{i=1}^nE_i$ and $\bigcap_{i=1}^nE_i\in R$. Note that any ring $R$ contains $\varnothing$ (the empty set), since if $E\in R$, then $\varnothing=E\sd E\in R$. However, a ring does not necessarily contain the whole set $X$.

\begin{definition}
If a ring $R$ contains the set $X$, then $R$ is called a \defline{field} (also sometimes called an \defline{algebra})\index{field, algebra}.
\end{definition}

In order to define a measure we actually need to be able to form countable unions of subsets of $X$. Thus we need the domain of a measure to be more than just a ring or a field.

\begin{definition}
A collection $S$ of subsets of $X$ is called a \defline{$\sigma$-ring}\index{sigma-ring@$\sigma$-ring} if $S$ is a ring and if $S$ is closed under the formation of countable unions, i.e. if $E_i\in S$, $i=1,2,\dots$ then $\bigcup_{i=1}^\infty E_i\in S$.
\end{definition}

Since, if $E=\bigcup_{i=1}^\infty E_i$, then $\bigcap_{i=1}^\infty E_i=E\sd\br{\bigcup_{i=1}^\infty(E\sd E_i)}$, it is clear that a $\sigma$-ring is also closed under formation of countable intersections.

\begin{definition}
A $\sigma$-ring $S$ is called a \defline{$\sigma$-field} (also called a \defline{$\sigma$-algebra}) if it is also a field, that is, if $X\in S$. \index{sigma-field, sigma-algebra@$\sigma$-field, $\sigma$-algebra}
\end{definition}

\begin{proposition}
The intersection of any collection of rings (fields, $\sigma$-rings, or $\sigma$-fields) on a set $X$ is again a ring (field, $\sigma$-ring, or $\sigma$-field).
\end{proposition}
\begin{proof}
We give the proof only for rings, since the proofs for the other cases are similar.

Let $\brc{R_\alpha}_{\alpha\in A}$ be a collection of rings, where $A$ is some index set, and let $R =\bigcap_{\alpha\in A}R_\alpha$. If $E,F\in R$, then $E,F\in R_\alpha$ for all $\alpha\in A$, and so $E\cup F$ and $E\sd F\in R_\alpha$ for all $\alpha\in A$, and so $E\cup F$, $E\sd F\in R$. Thus $R$ is a ring.
\end{proof}

\begin{corollary}
Given any collection $P$ of subsets of $X$ there exists a smallest ring (field, $\sigma$-ring, or $\sigma$-field) containing $P$.
\end{corollary}
\begin{proof}
By ``smallest'' we mean ``which is contained in any ring (field, $\sigma$-ring, or $\sigma$-field) which contains $P$''. Thus by the proposition just proved it is enough to show that there is some ring (field, $\sigma$-ring, or $\sigma$-field) which contains $P$, for the smallest one will be just the intersection of all those which contain $P$. But it is obvious that the collection of all subsets of $X$ is a $\sigma$-field (and hence a ring, field, and $\sigma$-ring) which contains $P$.
\end{proof}

This corollary allows us to make the following definition.

\begin{definition}\label{def:generate sets}
The smallest ring (field, $\sigma$-algebra, or $\sigma$-field) containing $P$ is called \defline{the ring} (\defline{field}, \defline{$\sigma$-ring}, or \defline{$\sigma$-field}) \defline{generated by $P$} \index{ring generated by}. The $\sigma$-ring generated by $P$ will be denoted as $\sring{P}$.
\end{definition}

Using this definition we can give some important examples.

\begin{example}
Let $X$ be a topological space, and let $P$ be the collection of open subsets of $X$. Then $\sring{P}$ is called \defline{the $\sigma$-ring of Borel sets of $X$}. \index{Borel sets}
\end{example}

There is also a different definition of Borel sets which is in common use. If $X$ is a locally compact space, then the $\sigma$-ring generated by the compact subsets of $X$ is also frequently called the $\sigma$-ring of Borel subsets of $X$. The reader should be able to check that for the real line these two definitions are equivalent. In fact they are equivalent for any locally compact space which are $\sigma$-compact, that is, which is the union of a countable number of compact subsets. However, for an uncountable space with the discrete topology the two definitions do not coincide.

\begin{example}
Let $X$ be a locally compact space, and let $P$ be the collection of compact $G_\delta$'s contained in $X$. (Recall that a $G_\delta$ is a set which is the intersection of a countable collection of open sets.) In this case $\sring{P}$ is called \defline{the $\sigma$-ring of Baire sets of $X$}. (If $X$ is a locally compact metric space, then the reader should be able to verify that the Baire sets are the same as the Borel sets (second definition).)
\end{example}

\begin{example}\label{ex:borel for reals}
Let $X=\bR$ (the real line) and let $P$ be the collection of left-closed right-open finite intervals, $[a,b)$. It is not hard to show that $\sring{P}$ is the $\sigma$-ring of Borel sets of $\bR$. This can be done by showing that $\sring{P}$ contains all compact subsets of $\bR$ (in fact all open or closed subsets) and that the $\sigma$-ring of Borel sets contains the half open intervals. For example, $[a,b)=\bigcup_{n=1}^\infty\sbr{a,b-\frac1n}$ so $P\subseteq\sigma$-ring of Borel sets of $\bR$. The remaining details are left to the reader as an exercise. This example shows, in particular, that two collections of sets which have no members in common can nevertheless generate the same $\sigma$-field. Note that the $\sigma$-ring of Borel sets of $\bR$ is actually a $\sigma$-field since $\bigcup_{n=1}^\infty[-n,n)=\bR$.
\end{example}

We remark that while Definition \ref{def:generate sets} is a simple definition, it hides a great deal. It is often very difficult to decide whether a given set is in the $\sigma$-ring generated by a given collection of sets. For example, it should be far from clear to the reader at this point whether or not every subset of the real line is a Borel set. (But see Exercise \ref{exer:construct borel for reals}.)


\section{The Definition of a Measure}


A sequence, $E_i$, $i=1,2,\dots$ of subsets of a set is said to consist of disjoint sets if $E_i\cap E_j=\varnothing$ whenever $i\neq j$. We will denote the union of such a sequence of disjoint sets by $\bigdu_{i=1}^\infty E_i$ instead of $\bigcup_{i=1}^\infty E_i$, to emphasize that the sets are disjoint.

\begin{definition}
Let $P$ be an arbitrary collection of subsets of a set $X$, and let $\mu$ be a function from $P$ into a Banach space, or into $\bR^\infty$ ($=\bR\cup\brc{+\infty}$). Then $\mu$ is said to be \defline{countably additive}\index{countably additive} (or \defline{$\sigma$-additive}) on $P$ if for every sequence $E_i$, $i=1,2,\dots$, of disjoint elements of $P$ such that $\bigdu_{i=1}^\infty E_i$ is also in $P$, we have $\mu\br{\bigdu_{i=1}^\infty E_i}=\sum_{i=1}^\infty\mu(E_i)$. 
\end{definition}

The right hand side of this last equation is defined to be the limit in the norm topology of the sequence $s_n$, where $s_n$ is the $n$th partial sum in the of the series $\sum_{i=1}^\infty\mu(E_i)$. In $\bR^\infty$, we allow this limits to be $+\infty$ in the obvious sense. Note that implicit in the definition of countable additivity is the requirement that the sum of the series must exist for all sequences of disjoint elements of $P$ whose union is in $P$. Note also that in order to verify the countable additivity of a function it is necessary only to consider disjoint sequences of elements of $P$ whose union is again in $P$. In some important cases, for example the $P$ of Example \ref{ex:borel for reals}, there will be relatively few such sequences, which will simplify the verification of countable additivity. We could also consider functions which take values in $\bR^{-\infty}=\bR\cup\brc{-\infty}$ but this case is virtually the same as that for $\bR^\infty$, and so we will not discuss it. However we cannot consider $\sigma$-additivity of functions with values in $\bR\cup\brc{-\infty,\infty}$ since there is no suitable definition of $\infty+(-\infty)$.

\begin{definition}
A \defline{measure}\index{measure} is a function $\mu$, whose domain is a $\sigma$-ring $S$, of subsets of a set $X$, whose range is contained either in a Banach space or in $\bR^\infty$, and which is countably additive on $S$. If the range of $\mu$ is in $\bR^\infty$, then $\mu$ will be said to be an \defline{extended real valued measure}\index{extended real valued measure}. Of particular interest will be non-negative extended real valued measures. We will call these \defline{non-negative measures}\index{non-negative measures}, leaving ``extended real valued'' as understood.
\end{definition}

One example of a non-negative measure is the function which has value $\infty$ on all members of $S$. This example is quite uninteresting. However we remark that every other measure has the property that $\mu(\varnothing)=0$. For if a measure $\mu$, has finite value on at least one set $E$, (in particular, if its range is contained in a Banach space) then $\mu(E)+\mu(\varnothing)=\mu(E)$, and so $\mu(\varnothing)=0$.

Since the domain $S$ of a measure is a $\sigma$-ring, every sequence of disjoint sets in $S$ is such that its union is in $S$. Thus this part of the definition of countable additivity is automatically satisfied. Also, the property of having a $\sigma$-ring as the domain allows us to ``disjointize'' an arbitrary sequence, $E_i$, $i = 1, 2,\dots$, of members of $S$. Namely, we can construct a sequence of disjoint elements of $S$, $F_n$, $n = 1, 2,\dots$, such that $F_n\subseteq E_n$ and $\bigcup_{i=1}^\infty E_i=\bigdu_{n=1}^\infty F_n$, by letting $F_1 = E_1$, and $F_n=E_n\sd\bigcup_{i=1}^{n-1}E_i$ for $n > 1$. This procedure will be useful in a number of places.

The first problem which arises in connection with the above definitions is to describe means by which interesting and useful measures can be constructed. In the rest of this chapter, we will describe some ways of doing this which will involve extending appropriate set functions so that they become measures. In later chapters we will describe various ways of obtaining new measures from old ones.


\section{An Example - Borel-Stieltjes Premeasures}

Before going on to state and prove our extension theorems, we will give an example of the kind of set function which we wish to extend, and we will use this example to give an idea of the properties which we will use to construct extensions. When extended, the set function of our example will be Borel-Stieltjes (or Lebesgue-Stieltjes) measure on the real line (depending on the $\sigma$-ring to which we extend).

\begin{definition}[Borel-Stieltjes premeasure]\label{def:stieltjes premeasure}
The domain of the set functions which we will define will be just the family $P$ of Example \ref{ex:borel for reals}, that is, the family of left closed, right open finite intervals, $[a, b)$, on the real line. Note that $[a, a)=\varnothing $. The functions are defined as follows: Let $\alpha$ be a real-valued non-decreasing left continuous function on $\bR$. Define a function, $\mu_\alpha$, from $P$ to $\bR$ by $\mu_\alpha([a, b))=\alpha(b)-\alpha(a)$.
\end{definition}

The assumption that $\alpha$ is left continuous will be quite important. It is not hard to show that the theorem which we will prove in this section is false without this requirement. We remark however that it can be shown that any non-decreasing function can have at most a countable number of discontinuities, and so from any non-decreasing function we can obtain a left continuous non-decreasing function by changing its values at most at a countable number of points.

\begin{theorem}\label{thm:stieltjes sigma additive}
The function $\mu_\alpha$ is countably additive on $P$.
\end{theorem}

\begin{proof}
Let $E_n=[a_n, b_n)$, $n=1,2,\dots$ be a sequence of disjoint elements of $P$, and suppose that $\bigdu_{n=1}^\infty E_n=E$, where $E=[a_0, b_0)$. We need to show that $\mu_\alpha(E)=\sum_{n=1}^\infty\mu_\alpha(E_n)$, that is, that $\alpha(b_0)-\alpha(a_0)=\sum_{n=1}^\infty(\alpha(b_n)-\alpha(a_n))$.

We begin by showing that $\mu_\alpha(E)\geq\sum_{n=1}^\infty\mu_\alpha(E_n)$. Now it is sufficient for this to show that $\mu_\alpha(E)\geq\sum_{n=1}^m\mu_\alpha(E_n)$ for each finite $m$. Given any $m$, order the intervals $E_n$, $n=1,\dots,m$, according to their left endpoint, that is, re-index them so that $a_i\leq a_{i+1}$ for all $i=1,\dots,m-1$. Since the intervals are disjoint, it follows that $b_i\leq a_{i+1}$. We must show that $\alpha(b_0)-\alpha(a_0)\geq\sum_{n=1}^m(\alpha(b_n)-\alpha(a_n))$. Since clearly $b_0\geq b_i$ and $a_0\leq a_i$ for $i$, we have $\alpha(b_0)-\alpha(a_0)\geq\alpha(b_m)-\alpha(a_1)$. Now $\sum_{n=1}^m(\alpha(b_n)-\alpha(a_n))=\alpha(b_m)-\alpha(a_1)+\sum_{n=1}^m(\alpha(b_n)-\alpha(a_{n+1}))$, and $\sum_{n=1}^{m-1}(\alpha(b_n)-\alpha(a_{n+1}))\leq0$ because of the way we have ordered the disjoint intervals and because $\alpha$ is non-decreasing. Hence we obtain the desired inequality.

Now we need to show that $\mu_\alpha(E)\leq\sum_{n=1}^\infty\mu_\alpha(E_n)$. Choose $\ep>0$, choose $b_0'<b_0$ such that $\alpha(b_0')\geq\alpha(b_0)-\frac\ep2$, and for each $n$ choose $a_n'<a_n$ such that $\alpha(a_n')\geq\alpha(a_n)-\ep_n$, where the $\ep_n$ are positive numbers such that $\sum_{n=1}^\infty\ep_n=\frac\ep2$. For example we could let $\ep_n=\ep/2^{n+1}$. (Recall that $\sum_{n=1}^\infty 1/2^n=1$. We will use this fact repeatedly.) We can choose such $a_n'$ only because $\alpha$ is assumed to be left continuous. Then $[a_0,b_0']\subseteq[a_0,b_0)=\bigdu_{n=1}^\infty[a_n,b_n)\subseteq\bigcup_{n=1}^\infty(a_n',b_n)$. Since $[a_0,b_0']$ is compact and the $(a_n',b_n)$ are open, there exists a finite integer $m$ such that $[a_0,b_0']\subseteq\bigcup_{n=1}^m(a_n',b_n)$. Re-index the intervals so that the first contains $a_0$, the second contains the right endpoint of the first, the third contains the right endpoint of the second and so on until we get $b_0'$ contained in an interval of the string. There may be some intervals left over, and we discard them as they are superfluous. We see that we have now arranged matters so that $b_n\geq a_{n+1}'$ for $n=1,\dots,m$ and $a_1'\leq a_0$ and $b_0'\leq b_m$. We also re-index the $\ep_n$ in the same way. Then
\begin{align*}
    \alpha(b_0)-\alpha(a_0)&\leq\alpha(b_0')-\alpha(a_0)+\frac\ep2\leq\alpha(b_m)-\alpha(a_1')+\frac\ep2\\
    &\leq\alpha(b_m)-\alpha(a_1')+\sum_{n=1}^{m-1}(\alpha(b_n)-\alpha(a_{n+1}'))+\frac\ep2\\
    &=\sum_{n=1}^m(\alpha(b_n)-\alpha(a_n'))+\frac\ep2\\
    &\leq\sum_{n=1}^m(\alpha(b_n)-\alpha(a_n)+\ep_n)+\frac\ep2\\
    &\leq\sum_{n=1}^\infty(\alpha(b_n)-\alpha(a_n))+\ep.
\end{align*}Since $\ep$ was arbitrary we have shown that $\alpha(b_0)-\alpha(a_0)\leq\sum_{n=1}^\infty(\alpha(b_n)-\alpha(a_n))$ which is what we needed.
\end{proof}

\section{Semirings and Premeasures}

Motivated by the properties of the collection $P$ of the previous section, we make:

\begin{definition}
A collection $P$ of subsets of a set $X$ is called a \defline{semiring}\index{semiring} if
\begin{enumerate}[label=\arabic*)]
    \item $\varnothing\in P$;
    \item if $E,F\in P$ then $E\cap F\in P$;
    \item if $E,F\in P$ then there exist $E_1,\dots,E_m\in P$ such that $E\sd F=\bigdu_{n=1}^mE_n$.
\end{enumerate}
\end{definition}

This definition of semiring is a slight variation of the one first given by von Neumann. We leave to the reader the trivial verification that the collection $P$ of the previous section (or of Example \ref{ex:borel for reals}) is a semiring. Another important example of a semiring, which in a sense generalizes the example $P$ just considered, is the collection of all differences of compact subsets of a topological space. This example will be particularly useful to us when we consider measure theory on locally compact topological spaces.

\begin{definition}
A non-negative (extended real valued) function $\mu$ defined on a semiring $P$ is called a \defline{premeasure}\index{premeasure} on $P$ if $\mu$ is countably additive on $P$.
\end{definition}

Thus Theorem \ref{thm:stieltjes sigma additive} says that the functions $\mu_\alpha$ of that theorem are premeasures. We also remark that every non-negative measure is also a premeasure. In the definition of a premeasure we could also have allowed the values to be in a Banach space. But we do not know whether the theorems which we will prove shortly about extending a premeasure to a measure are true in the case of Banach space valued premeasures. Certainly our proofs will use strongly the fact that the values of a premeasure are in $\bR^\infty$.

As was the case for measures, it is easily seen that if a premeasure $\mu$, has finite value on at least one set, then $\mu(\varnothing)=0$.

The extension theorem which we wish to prove states in part that a premeasure $\mu$, on a semiring $P$, can be extended to a measure on $\sring{P}$, the $\sigma$-ring generated by $P$. But before beginning the discussion of this theorem, we need three lemmas concerning premeasures. In each of the following lemmas, $P$ is an arbitrary semiring and $\mu$ is an arbitrary premeasure on $P$.

\begin{lemma}\label{lem:repeated set diff}
If $E,E_1,E_2,\dots,E_m\in P$ then there exist $F_i\in P$, $i=1,\dots,k$ such that $(((E\sd E_1)\sd E_2)\sd\cdots\sd E_m)=\bigdu_{i=1}^kF_i$.
\end{lemma}

\begin{proof}
We use induction on $m$. If $m=1$ then the lemma is true by the definition of a semiring. Now suppose that the lemma is true for $m-1$ where $m>1$. Then there exist $F_i'$, $i=1,\dots,k'$ such that $((((E\sd E_1)\sd E_2)\sd\cdots\sd E_{m-1})\sd E_m)=\br{\bigdu_{i=1}^{k'}F_i'}\sd E_m$. But $\br{\bigdu_{i=1}^{k'}F_i'}\sd E_m=\bigdu_{i=1}^{k'}(F_i'\sd E_m)$. Thus, using the definition of a semiring to express each $F_i'\sd E_m$ as the disjoint union of a finite number of elements of $P$, we obtain the desired result.
\end{proof}


\begin{corollary}\label{cor:ring generated by semiring}
The ring generated by a semiring $P$ consists of the finite disjoint unions of elements of $P$.
\end{corollary}

\begin{proof}
By Lemma \ref{lem:repeated set diff} we have \[\bigdu_mE_m\sd\bigdu_nF_n=\bigdu_m\br{E_m\sd\bigdu_nF_n}=\bigdu_m\bigdu_{j=1}^{k_m}G_{mj}\] where the $E_m, F_n$ and $G_{mj}$ are elements of $P$. Thus the collection of finite disjoint unions of elements of $P$ is closed under taking differences. But this collection is also clearly closed under taking finite disjoint unions. It follows that it is closed under taking finite unions, since $E\cup F=E\du(F\sd E)$.
\end{proof}


\begin{lemma}\label{lem:countable superadditive}
If $E\supseteq\bigdu_{i=1}^\infty E_i$ where $E,E_i\in P$, then $\mu(E)\geq\sum_{i=1}^\infty\mu(E_i)$. (As a special case we see that a premeasure is monotone, that is, if $E\subseteq F$, and $E,F\in P$, then $\mu(E)\leq\mu(F)$.)
\end{lemma}

\begin{proof}
We note first that we have not assumed that $\bigdu_{i=1}^\infty E_i$ is in $P$. Now, to prove the lemma it is sufficient to show that $\mu(E)\geq\sum_{i=1}^m\mu(E_i)$ for each finite $m$. By Lemma \ref{lem:repeated set diff}, $(((E\sd E_1)\sd E_2)\sd\cdots\sd E_m)=\bigdu_{i=1}^kF_i$ where $F_i\in P$. Then $E=E_1\du E_2\du\cdots\du E_m\du\bigdu_{i=1}^kF_i$. Thus $\mu(E)=\mu(E_1)+\cdots+\mu(E_m)+\sum_{i=1}^k\mu(F_i)$. Since $\mu$ is non-negative, it follows that $\mu(E)\geq\sum_{i=1}^m\mu(E_i)$.
\end{proof}

\begin{lemma}
A premeasure is countably subadditive, that is, if $E\subseteq\bigcup_{i=1}^\infty E_i$ where $E,E_i\in P$, then $\mu(E)\leq\sum_{i=1}^\infty\mu(E_i)$.
\end{lemma}

\begin{proof}
Clearly $E=\bigcup_{i=1}^\infty(E\cap E_i)$. Let $E\cap E_i=E_i'$ so that $E=\bigcup_{i=1}^\infty E_i'$. Using Lemma \ref{lem:repeated set diff} repeatedly we obtain 
\begin{align*}
    E=\bigcup_{i=1}^\infty E_i'&=E_1'\du(E_2'\sd E_1')\du\cdots\du\br{E_m'\sd\bigcup_{j=1}^{m-1}E_j'}\du\cdots\\
    &=E_1'\du\bigdu_{i=1}^{k_2}F_{2i}\du\cdots\du\bigdu_{i=1}^{k_m} F_{mi}\du\cdots\\
\end{align*}
Therefore, $\mu(E)=\mu(E_1')+\sum_{i=1}^{k_2}\mu(F_{2i})+\cdots+\sum_{i=1}^{k_m}\mu(F_{mi})+\cdots$. But $\bigdu_{i=1}^{k_m}F_{mi}\subseteq E_m'\subseteq E_m$, so by Lemma \ref{lem:countable superadditive}, $\sum_{i=1}^{k_m}\mu(F_{mi})\leq\mu(E_m)$. Thus $\mu(E)\leq\sum_{i=1}^\infty\mu(E_i)$.
\end{proof}

\section{The Extension of Premeasures to Outer Measures}

In order to extend a premeasure to a measure we first extend the premeasure to a set function, which is not necessarily a measure, but which has a very large domain. We then restrict this set function to a smaller domain in such a way that the restricted function is a measure.

\begin{definition}
If $P$ is any collection of subsets of a set $X$ we say $E\subseteq X$ is \defline{countable covered by}\index{countable covered by} $P$ if there is a countable collection $\brc{E_n}_{n=1}^\infty$ of elements of $P$ such that $E\subseteq\bigcup_{n=1}^\infty E_n$.
\end{definition}

It is easily seen that if $P$ is any collection of sets, then the collection $\hring{P}$ of all those sets which are countably covered by $P$ is a $\sigma$-ring with the additional property that if $E\in\hring{P}$ and $F\subseteq E$ then $F\in\hring{P}$.

\begin{definition}
A $\sigma$-ring $H$ with the property that if $E\in H$ and $F\subseteq E$ then $F\in H$ is called a \defline{hereditary $\sigma$-ring}\index{hereditary sigma-ring@hereditary $\sigma$-ring}.
If $P$ is any collection of sets, then $\hring{P}$ will be called the \defline{hereditary $\sigma$-ring generated by}\index{ring generated by} $P$.
\end{definition}

\begin{definition}
An \defline{outer measure}\index{outer measure} is a non-negative (extended real valued) function, $\mu^*$, whose domain is a hereditary $\sigma$-ring $H$, such that
\begin{enumerate}[label=\arabic*)]
    \item\label{def:outer measure 1}
    $\mu^*$ is monotone, that is, if $E\subseteq F$, $F\in H$, then $\mu^*(E)\leq\mu^*(F)$;
    \item\label{def:outer measure 2}
    $\mu^*$ is countably subadditive, that is, if $E\subseteq\bigcup_{n=1}^\infty E_n$, $E_n\in H$, then $\mu^*(E)\leq\sum_{n=1}^\infty\mu^*(E_n)$.
\end{enumerate}
\end{definition}

\begin{theorem}\label{thm:outer measure of premeasure}
If $\mu$ is a premeasure on a semiring $P$ and if $\mu^*$ is defined on $\hring{P}$ by \[\mu^*(A)=\inf\brc{\sum_{n=1}^\infty\mu(E_n):A\subseteq\bigcup_{n=1}^\infty E_n,E_n\in P},\] then $\mu^*$ is an outer measure on $\hring{P}$ which extends $\mu$ (that is, $\mu^*(E)=\mu(E)$ for each $E\in P$.)
\end{theorem}

\begin{proof}
Property \ref{def:outer measure 1} in the definition of an outer measure is clearly satisfied by $\mu^*$. So we need first to prove property \ref{def:outer measure 2}, that is, that if $A\subseteq\bigcup_{i=1}^\infty A_i$, where each $A_i\in\hring{P}$ then $\mu^*(A)\leq\sum_{i=1}^\infty\mu^*(A_i)$. Now if $\mu^*(A_i)=\infty$ for some $i$, then the result is certainly true, and so we may assume that $\mu^*(A_i)<\infty$ for every $i$.

Let $\ep>0$ be given. Choose $\brc{E_{ij}}_{i, j=1}^\infty\subseteq P$ such that $A_i\subseteq\bigcup_{j=1}^\infty E_{ij}$ and $\mu^*(A_i)\geq\sum_{j=1}^\infty\mu(E_{ij})-\frac{\ep}{2^i}$. Then $A\subseteq\bigcup_{i,j=1}^\infty E_{i j}$ and $\mu^*(A)\leq\sum_{i,j=1}^\infty\mu(E_{ij})=\sum_{i=1}^\infty\br{\sum_{j=1}^\infty\mu(E_{ij})}\leq\sum_{i=1}^\infty\br{\mu^*(A_i)+\frac{\ep}{2^i}}\allowbreak=\sum_{i=1}^\infty\mu^*(A_i)+\ep$. Since $\ep$ was arbitrary it follows that $\mu^*$ is countably subadditive. Thus $\mu^*$ is an outer measure.

Now we need to show that $\mu^*$ extends $\mu$. Clearly $\mu^*(A)\leq\mu(A)$ for all $A\in P$ by the definition of $\mu^*$, so all that we need to show is that $\mu^*(A)\geq\mu(A)$ if $A\in P$, that is, that $\sum_{i=1}^\infty\mu(E_i)\geq\mu(A)$ if $A\subseteq\bigcup_{i=1}^\infty E_i$ and $A,E_i\in P$. But this is just the fact that a premeasure is countably subadditive.
\end{proof}

\begin{definition}
The outer measure, $\mu^*$, defined in the statement of Theorem \ref{thm:outer measure of premeasure} is called the \defline{outer measure determined by the premeasure $\mu$}\index{outer measure determined by}.
\end{definition}

\section{Measures from Outer Measures}

Outer measures will in general not be measures. We now begin the process of restricting the domain of an outer measure so as to obtain a measure.

\begin{definition}
Given a hereditary $\sigma$-ring $H$ and an outer measure $\mu^*$ on $H$ we call a set $E\in H$ \defline{$\mu^*$-measurable}\index{mu-measurable@$\mu$-measurable} if for every $A\in H$ we have the equality $\mu^*(A)=\mu^*(A\cap E)+\mu^*(A\sd E)$. We will denote the collection of all $\mu^*$-measurable sets by $\mring{\mu^*}$.
\end{definition}

This ingenious definition is due to Caratheodory in 1918. It says that $E$ splits all elements of $H$ nicely with respect to $\mu^*$. Note that it is always true that $\mu^*(A)\leq\mu^*(A\cap E)+\mu^*(A\sd E)$ because of the subadditivity of $\mu^*$. Thus to show that a set $E$ is $\mu^*$-measurable we need only prove the opposite inequality, $\mu^*(A)\geq\mu^*(A\cap E)+\mu^*(A\sd E)$, for all $A\in H$.

\begin{theorem}\label{thm:restriction of outer measure to meas sets}
If $\mu^*$ is an outer measure on a hereditary $\sigma$-ring $H$, then either $\mring{\mu^*}$ is empty, or it is a $\sigma$-ring and the restriction of $\mu^*$ to $\mring{\mu^*}$ is a measure.
\end{theorem}

\begin{proof}

We divide the proof into two main steps.

\begin{lemma}
If $\mring{\mu^*}$ is not empty, then it is a ring.
\end{lemma}
\begin{proof}
We need to prove that if $E,F\in\mring{\mu^*}$, then $E\cup F$ and $E\sd F\in\mring{\mu^*}$. Let $A\in H$ be arbitrary. Then, using the fact that $E$ and $F$ are $\mu^*$-measurable, we obtain the following two strings of inequalities:
\begin{align*}
    \mu^*(A)&\leq\mu^*(A\cap(E\cup F))+\mu^*(A\sd(E\cup F)) \\
            &=\mu^*((A\cap E)\du((A\sd E)\cap F))+\mu^*((A\sd E)\sd F) \\
            &\leq\mu^*(A\cap E)+\mu^*((A\sd E)\cap F)+\mu^*((A\sd E)\sd F) \\
            &=\mu^*(A\cap E)+\mu^*(A\sd E)=\mu^*(A)
\end{align*}
and
\begin{align*}
    \mu^*(A)&\leq\mu^*(A\cap(E\sd F))+\mu^*(A\sd (E\sd F)) \\
            &=\mu^*((A\cap E)\sd F)+\mu^*((A\sd E)\cup(A\cap F)) \\
            &=\mu^*((A\cap E)\sd F)+\mu^*((A\sd E)\du(A\cap E\cap F)) \\
            &\leq\mu^*((A\cap E)\sd F)+\mu^*(A\sd E)+\mu^*(A\cap E\cap F) \\
            &=\mu^*(A\cap E)+\mu^*(A\sd E)=\mu^*(A) .
\end{align*}
Since both strings begin and end with $\mu^*(A)$, all inequalities are actually equalities. Thus the first line of the first string shows that $E\cup F\in\mring{\mu^*}$ and the first line of the second string shows that $E\sd F\in\mring{\mu^*}$. Therefore $\mring{\mu^*}$ is a ring if it is not empty.
\end{proof}

\begin{lemma}\label{lem:outer measure sigma additivity}
Either $\mring{\mu^*}$ is empty, or it is a $\sigma$-ring, in which case if $E=\bigdu_{i=1}^\infty E_i$ where $E_i\in\mring{\mu^*}$, then $\mu^*(A\cap E)=\sum_{j=1}^\infty\mu^*(A\cap E_i)$ for all $A\in H$.
\end{lemma}
\begin{proof}
We remark first that if $E$ and $F$ are disjoint elements of $\mring{\mu^*}$, then \[\mu^*(A\cap(E\du F))=\mu^*(A\cap E)+\mu^*(A\cap F),\] for this is easily seen to be the same as the statement that $E$ splits the set $A\cap(E\du F)$ nicely with respect to $\mu^*$.

Now to prove that $\mring{\mu^*}$ is a $\sigma$-ring we only need to show that $\mring{\mu^*}$ is closed under formation of countable unions. Suppose that $E=\bigcup_{i=1}^\infty E_i$, where each $E_i\in\mring{\mu^*}$. We wish to show that $E\in\mring{\mu^*}$. Now we can assume that the $E_i$'s are disjoint, since if they are not, we can disjointize them by letting $E_k'=E_k\sd\bigcup_{i=1}^{k-1} E_i$. Clearly each $E_k'\in\mring{\mu^*}$, since $\mring{\mu^*}$ is a ring.

Let $A\in H$. Then we have
\begin{align*}
    \mu^*(A)&=\mu^*\br{A\cap\br{\bigdu_{i=1}^mE_i}}+\mu^*\br{A\sd\br{\bigdu_{i=1}^mE_i}}& &\text{since }\mring{\mu^*}\text{ is a ring}\\
            &\geq\mu^*\br{A\cap\br{\bigdu_{i=1}^mE_i}}+\mu^*(A\sd E)& &\text{by monotonicity of }\mu^*\\
            &=\sum_{i=1}^m\mu^*(A\cap E_i)+\mu^*(A\sd E)& &\text{by first remark}
\end{align*}
Since this holds for every finite $m$ and since $\bigdu_{i=1}^\infty(A\cap E_i)=A\cap E$ we thus have $\mu^*(A)\geq\sum_{i=1}^\infty\mu^*(A\cap E_i)+\mu^*(A\sd E)\geq\mu^*(A\cap E)+\mu^*(A\sd E)$ since $\mu^*$ is subadditive. Thus $E$ is $\mu^*$-measurable, and consequently $\mring{\mu^*}$ is a $\sigma$-ring if it is not empty.

To prove the second part of the lemma we note that we can also conclude from the above that $\mu^*(A)=\sum_{i=1}^\infty\mu^*(A\cap E_i)+\mu^*(A\sd E)$. Then if we replace $A$ by $A\cap E$ we obtain $\mu^*(A\cap E)=\sum_{i=1}^\infty\mu^*(A\cap E_i)+\mu^*(\varnothing)=\sum_{i=1}^\infty\mu^*(A\cap E_i)$ as desired, where we have used the fact that if $\mring{\mu^*}$ is not empty then some set splits $\varnothing$ nicely, so that either $\mu^*(\varnothing)=0$ or else $\mu^*(E)=\infty$ for all $E\in H$.
\end{proof}

To conclude the proof of Theorem \ref{thm:restriction of outer measure to meas sets} all that we need to show is that $\mu^*$ restricted to $\mring{\mu^*}$ is a measure, that is, that if $E=\bigdu_{i=1}^\infty E_i$, with each $E_i\in\mring{\mu^*}$, then $\mu^*(E)=\sum_{i=1}^\infty\mu^*(E_i)$. But to obtain this we need only let $A=E$ in the second statement of Lemma \ref{lem:outer measure sigma additivity}.
\end{proof}

We remark that it is entirely possible for $\mring{\mu^*}$ to be empty (see Exercise \ref{exer:empty meas sets}).

\begin{definition}
A non-negative measure $\mu$ on a $\sigma$-ring $S$ is \defline{complete}\index{complete measure} if whenever $F\subseteq E$, $E\in S$, and $\mu(E)=0$, then $F\in S$. (And, of course, $\mu(F)=0$.)
\end{definition}

\begin{proposition}
If $\mu^*$ is an outer measure, if $\mring{\mu^*}\neq\varnothing$, and if $\widetilde{\mu}$ is the restriction of $\mu^*$ to $\mring{\mu^*}$, then $\widetilde{\mu}$ is a complete measure.
\end{proposition}

\begin{proof}
It is sufficient to show that if $\mu^*(E)=0$ then $E\in\mring{\mu^*}$. Let $A\in H$. Then clearly $\mu^*(A)\geq\mu^*(A\cap E)+\mu^*(A\sd E)$, since $\mu^*(A)\geq\mu^*(A\sd E)$ and $\mu^*(A\cap E)=0$ by the monotonicity of $\mu^*$. Therefore $E\in\mring{\mu^*}$.
\end{proof}

At this point we do not yet know that we have proved a really useful theorem. For example, we do not know whether $\mring{\mu^*}$ is ever non-empty. Also, we do not have an extension theorem yet. We solve these problems with the following theorem.

\begin{theorem}
If $\mu$ is a premeasure on a semiring $P$, and if $\mu^*$, defined on $\hring{P}$, is the outer measure which $\mu$ determines, then $P\subseteq\mring{\mu^*}$.
\end{theorem}
\begin{proof}
We need to show that if $E\in P$, then $E\in\mring{\mu^*}$, that is, that for all $A\in\hring{P},\mu^*(A)=\mu^*(A\cap E)+\mu^*(A\sd E)$. If $\mu^*(A)=\infty$ then we are clearly done since it is then clear that $\mu^*(A)\geq\mu^*(A\cap E)+\mu^*(A\sd E)$. Thus, we may assume that $\mu^*(A)<\infty$. Suppose we are given $\ep>0$. Then, since $A\in\hring{P}$, we can choose elements $F_i$ of $P$ such that $A\subseteq\bigcup_{i=1}^\infty F_i$ and $\mu^*(A)+\ep\geq\sum_{i=1}^\infty\mu(F_i)$. Now $F_i=(E\cap F_i)\du(F_i\sd E)$, so by the definition of a semiring there exist $G_{ij}\in P$, such that $F_i=(E\cap F_i)\du\br{\bigdu_{j=1}^{k_i}G_{ij}}$. Then
\begin{align*}
\sum_{i=1}^\infty\mu(F_i)&=\sum_{i=1}^\infty\br{\mu(E\cap F_i)+\sum_{j=1}^{k_i}\mu(G_{i j})} \\
&=\sum_{i=1}^\infty\mu(E\cap F_i)+\sum_{i=1}^\infty\sum_{j=1}^{k_i}\mu(G_{i j}).
\end{align*}
Now we note that $A\cap E\subseteq\bigcup_{i=1}^\infty(F_i\cap E)$ and that $A\sd E\subseteq\bigcup_{i=1}^\infty(F_i\sd E)=\bigcup_{i=1}^\infty\bigcup_{j=1}^{k_i}G_{ij} $. Thus
\begin{align*}
    \mu^*(A)+\ep&\geq\sum_{i=1}^\infty\mu(F_i)=\sum_{i=1}^\infty\mu(F_i\cap E)+\sum_{i=1}^\infty\sum_{j=1}^{k_i}\mu(G_{i j})\\
    &\geq\mu^*(A\cap E)+\mu^*(A\sd E).
\end{align*}
Since $\ep$ is arbitrary, we obtain the desired result.
\end{proof}

As a corollary of the above results we obtain:

\begin{theorem}[The Extension Theorem]\label{thm:measure extension thm}\index{Extension Theorem}
If $\mu$ is a premeasure on a semiring $P$ and if $\mu^*$, defined on $\hring{P}$, is the outer measure determined by $\mu$, then 1) if $\overline{\mu}$ is the restriction of $\mu^*$ to $\sring{P}$, then $\overline{\mu}$ is a measure which extends $\mu$. 2) if $\widetilde{\mu}$ is the restriction of $\mu^*$ to $\mring{\mu^*}$, then $\widetilde{\mu}$ is a complete measure which extends $\overline{\mu}$ and so $\mu$.
\end{theorem}

\begin{definition}
If $P,\alpha$ and $\mu_\alpha$ are as in \ref{def:stieltjes premeasure}, then $\overline{\mu_\alpha}$ is called the \defline{Borel-Stieltjes measure}\index{Borel-Stieltjes measure} on $\bR$ corresponding to $\alpha$ (recall that $\sring{P}$ is the $\sigma$-ring of Borel sets of $\bR$), and $\widetilde{\mu_\alpha}$ is called the \defline{Lebesgue-Stieltjes measure}\index{Lebesgue-Stieltjes measure} on $\bR$ corresponding to $\alpha$. If $\alpha(x)=x$, then $\overline{\mu_\alpha}$ is called \defline{Borel measure}\index{Borel measure} on $\bR$, $\widetilde{\mu_\alpha}$ is called \defline{Lebesgue measure}\index{Lebesgue measure} on $\bR$, and $\mring{\mu_\alpha^*}$ is called the $\sigma$-ring of \defline{Lebesgue measurable subsets}\index{Lebesgue measurable} of $\bR$.
\end{definition}

Implicit in the wording of the last paragraph is the fact that, given $\alpha$, the corresponding $\overline{\mu_\alpha}$ and $\widetilde{\mu_\alpha}$ are unique. We have not yet proved anything like this, but the uniqueness of extensions is the subject of the next section, and uniqueness of Borel-Stieltjes and Lebesgue-Stieltjes measure will follow from the results obtained there.

\section{Uniqueness of Extensions}

Before going on to prove uniqueness we first show that we cannot extend a premeasure $\mu$, to a measure on a larger $\sigma$-ring than $\mring{\mu^*}$ by iterating the process described above.

\begin{proposition}\label{prop:repeated outer measure}
Let $\mu$ be a premeasure on a semiring $P$ and let $\mu^*$ be the outer measure determined by $\mu$. Then $\hring{P}=\hring{\mring{\mu^*}}$, and the outer measures determined by $\overline{\mu}$ and $\widetilde{\mu}$ are both just $\mu^*$. More precisely, if $E\in\hring{P}$, then $\mu^*(E)=\inf\brc{\overline{\mu}(F):E\subseteq F,F\in\sring{P}}=\allowbreak\inf\brc{\widetilde{\mu}(F):E\subseteq F,F\in\mring{\mu^*}}$.
\end{proposition}

\begin{proof}

$\hring{P}=\hring{\mring{\mu^*}}$ because $P\subseteq\mring{\mu^*}\subseteq\hring{P}$. To prove the second statement it suffices to show that for each $E\in\hring{P}$ the following string of inequalities holds:
\begin{align*}
    \mu^*(E)&\geq\inf\brc{\sum_{i=1}^\infty\mu(E_i):E\subseteq\bigcup_{i=1}^\infty E_i,E_i\in P}\\
            &\geq\inf\brc{\overline{\mu}(F):E\subseteq F,F\in\sring{P}}\\
            &\geq\inf\brc{\widetilde{\mu}(F):E\subseteq F,F\in\mring{\mu^*}}\\
            &\geq\mu^*(E).
\end{align*}
Now the first inequality follows from the definition of $\mu^*$. The second inequality follows from the fact that if $F=\bigcup_{i=1}^\infty E_i$ where the $E_i\in P$, then $\overline{\mu}(F)\leq\sum_{i=1}^\infty\mu(E_i)$ by the countable subadditivity of $\mu^*$. The third inequality holds because the term on the right is the infimum of a larger set. Finally, the last inequality follows from the monotonicity of $\mu^*$.
\end{proof}

\begin{corollary}\label{cor:measure of hereditary wrt outer measure}
If $E\in\hring{P}$, then there exists $F\in\sring{P}$ such that $E\subseteq F$ and $\overline{\mu}(F)=\mu^*(E)$.
\end{corollary}

\begin{proof}
By Proposition \ref{prop:repeated outer measure} there must exist a sequence $\brc{F_i}_{i=1}^\infty$ of elements of $\sring{P}$ such that $E\subseteq F_i$ for each $i$ and $\lim _{i\to\infty}\overline{\mu}(F_i)=\mu^*(E)$ Let $F=\bigcap_{i=1}^\infty F_i$.
\end{proof}

In order to prove that the extensions we have obtained are unique we need to make an additional hypothesis (see Exercise \ref{exer:non unique extension measure}).

\begin{definition}
Let $\mu$ be a non-negative set function (such as a premeasure, measure, or outer measure) defined on a collection $P$, of subsets of $X$. Then $E\subseteq X$ is said to be \defline{$\sigma$-finite}\index{sigma-finite@$\sigma$-finite} for $\mu$ if there exist $E_i\in P$ such that $\mu(E_i)<\infty$ for each $i$, and $E\subseteq\bigcup_{i=1}^\infty E_i$. If each $E\in P$ is $\sigma$-finite, then $\mu$ itself is said to be \defline{$\sigma$-finite}. If in fact $X$ is $\sigma$-finite, then $\mu$ is said to be \defline{totally $\sigma$-finite}\index{totally sigma-finite@totally $\sigma$-finite}.
\end{definition}

Being $\sigma$-finite will be an important hypothesis in many theorems in addition to the theorem concerning the uniqueness of extensions. Many pathologies occur with measures that are not $\sigma$-finite.

\begin{proposition}
If $\mu$ is a $\sigma$-finite premeasure then $\mu^*$ is $\sigma$-finite, so, in particular, $\overline{\mu}$ and $\widetilde{\mu}$ are $\sigma$-finite.
\end{proposition}
\begin{proof}
Obvious
\end{proof}

\begin{theorem}[Uniqueness of extensions]\label{thm:unique extension}
If $\mu$ is a $\sigma$-finite premeasure on a semiring $P$, if $S$ is a $\sigma$-ring such that $\sring{P}\subseteq S\subseteq\mring{\mu^*}$ and if $\nu$ is a non-negative extension of $\mu$ to a measure on $S$, then $\nu$ coincides with the restriction of $\mu^*$ to $S$.
\end{theorem}

\begin{proof}
If $E\in S$ and $E\subseteq\bigcup_{i=1}^\infty E_i$ where each $E_i\in P$, then $\nu(E)\leq\sum_{i=1}^\infty\nu(E_i)=\sum_{i=1}^\infty\mu(E_i)$ by the countable subadditivity of non-negative measures and by the fact that $\nu$ extends $\mu$. Thus $\nu(E)\leq\mu^*(E)$ for all $E\in S$. Thus we need to show that $\mu^*(E)\leq\nu(E)$ for all $E\in S$.

\underline{Case 1}. Suppose first that $E$ is such that there exists $F\in P$ for which $E\subseteq F$ and $\mu(F)<\infty$. Then, since $F=E\du(F\sd E)$, we have $\nu(F)=\mu(F)=\mu^*(F)=\mu^*(E)+\mu^*(F\sd E)\geq\nu(E)+\nu(F\sd E)=\nu(F)$. Thus $\mu^*(E)+\mu^*(F\sd E)=\nu(E)+\nu(F\sd E)$. But $\mu^*(E)\geq\nu(E)$ and $\mu^*(F\sd E)\geq\nu(F\sd E)$ and these quantities are all assumed to be finite, so we must have $\mu^*(E)=v(E)$.

\underline{Case 2}. Suppose now that $E\in S$ is arbitrary. Then, since $\mu$ is assumed to be $\sigma$-finite, there exist $F_i\in P$ such that $\mu(F_i)<\infty$ for each $i$ and $E\subseteq\bigcup_{j=1}^\infty F_i$. Using the usual disjointizing process, we can then obtain sets $G_i$ in $\sring{P}$ such that $\overline{\mu}(G_i)<\infty$ for each $i$ and $E\subseteq\bigdu_{i=1}^\infty G_i$. Then, using Case 1, we see that $\nu(E)=\sum_{i=1}^\infty\nu(E\cap G_i)=\sum_{i=1}^\infty\mu^*(E\cap G_i)=\mu^*(E)$.
\end{proof}

\begin{corollary}
Any non-negative measure $\mu$ defined on the Borel subsets of the real line which is finite on finite intervals (and so is $\sigma$-finite) is $\overline{\mu_\alpha}$ for some monotone non-decreasing left continuous function $\alpha$.
\end{corollary}

\begin{proof}
Define $\alpha$ by $\alpha(t)=\mu([0, t))$ if $t\geq 0$ and $\alpha(t)=-\mu([t, 0))$ if $t<0$. Then $\alpha$ is clearly non-decreasing, and it is easily seen, using the countable additivity of $\mu$, that $\alpha$ is left continuous. Clearly $\overline{\mu_\alpha}$ agrees with $\mu$ on the family of left closed, right open finite intervals, and so $\overline{\mu_\alpha}$ agrees with $\mu$ on the class of Borel sets, by the theorem concerning the uniqueness of extensions.
\end{proof}

We will now examine the structure of $\mring{\mu^*}$ more closely and show that if $\mu^*$ is determined by a $\sigma$-finite premeasure, then $\mring{\mu^*}$ has a rather simple form.

\begin{definition}
Let $\mu$ be a non-negative measure on a $\sigma$-ring $S$. A subset $E$ of $X$ is called a \defline{null set}\index{null set} with respect to $\mu$ if there exists $F\in S$ such that $E\subseteq F$ and $\mu(F)=0$ (or equivalently, if $\mu^*(E)=0$). The null sets form a hereditary $\sigma$-ring which we will denote by $\nring{\mu}$.
\end{definition}

Our first result does not involve $\sigma$-finiteness.

\begin{proposition}\label{prop:null set completion}
Let $\mu$ be a non-negative measure on a $\sigma$-ring $S$. Let $S\du\nring{\mu}=\{E\du F:E\in S,F\in\nring{\mu}\}$ and define $\widehat{\mu}$ on $S\du\nring{\mu}$ by $\widehat{\mu}(E\du F)=\mu(E)$ where $E\in S$ and $F\in\nring{\mu}$. Then $S\du\nring{\mu}$ is a $\sigma$-ring, $\widehat{\mu}$ is well defined and $\widehat{\mu}$ is a completion of $\mu$, that is, $\widehat{\mu}$ is an extension of $\mu$ to a complete measure.
\end{proposition}

\begin{proof}
We leave to the reader the trivial proof that $S\du\nring{\mu}$ is a $\sigma$-ring. The only slightly non-trivial thing to prove is that $\widehat{\mu}$ is well defined. Let $E_1\du F_1=E_2\du F_2$, where $E_1,E_2\in S$ and $F_1,F_2\in\nring{\mu}$. Then we must show that $\mu(E_1)=\mu(E_2)$. But $E_1\sd E_2=E_1\cap F_2\subseteq F_2$ and $E_2\sd E_1=E_2\cap F_1\subseteq F_1$, and so $\mu(E_1\sd E_2)=\mu(E_2\sd E_1)=0$. Since $\mu(E_1)=\mu(E_1\cap E_2)+\mu(E_1\sd E_2)$ and $\mu(E_2)=\mu(E_1\cap E_2)+\mu(E_2\sd E_1)$, it follows that $\mu(E_1)=\mu(E_2)$.

\end{proof}

\begin{theorem}\label{thm:null set completion sigma finite}
If $\mu$ is a $\sigma$-finite non-negative measure on a $\sigma$-ring $S$, then $\mring{\mu^*}=S\du\nring{\mu}$.
\end{theorem}

\begin{proof}
Clearly $S\du\nring{\mu}\subseteq\mring{\mu^*}$. Thus we must show that $\mring{\mu^*}\subseteq S\du\nring{\mu}$.

\underline{Case 1}. Suppose first that $E\in\mring{\mu^*}$ is such that $\mu^*(E)<\infty$. By Corollary \ref{cor:measure of hereditary wrt outer measure} there exists $F\in S$ such that $E\subseteq F$ and $\mu(F)=\mu^*(E)$. Then $\mu(F)=\mu^*(E)+\mu^*(F\sd E)$ so $\mu^*(F\sd E)=\mu(F)-\mu^*(E)=0$. Using the corollary again, we see that there exists $G\in S$ such that $F\sd E\subseteq G$ and $\mu(G)=0$. Since $F\sd E\subseteq G$, we have $E\sd G=F\sd G$, and so $E=(E\sd G)\du(E\cap G)=(F\sd G)\du(E\cap G)$. But $F\sd G\in S$ and $E\cap G\in\nring{\mu}$, and so $E\in S\du\nring{\mu}$.

\underline{Case 2}. Suppose now that $E\in\mring{\mu^*}$ is arbitrary. Since $\mu$ is assumed to be $\sigma$-finite, so is $\mu^*$, and so there exist sets $E_i\in\mring{\mu^*}$ such that $\mu^*(E_i)<\infty$ for each $i$ and $E=\bigcup_{i=1}^\infty E_i$. (To get $E=\bigcup_{i=1}^\infty E_i$ rather than $E\subseteq\bigcup_{i=1}E_i$ we can take intersections with $E$ if necessary.) By Case $1$, $E_i\in S\du\nring{\mu}$ for each $i$. Since $S\du\nring{\mu}$ is a $\sigma$-ring, it follows that $E\in S\du\nring{\mu}$.
\end{proof}

\begin{corollary}
If $\mu$ is a $\sigma$-finite premeasure on a semiring $P$, then $\mring{\mu^*}=\sring{P}\du\nring{\bar{\mu}}$.
\end{corollary}

Since Lebesgue measure is the most important example of a measure, it is of interest to know whether there exist subsets of the real line which are not Lebesgue measurable. It is not very difficult to construct such sets, and we suggest such a construction in Exercise \ref{exer:lebesgue measure rotation inv} at the end of this chapter. However, an interesting feature of all known examples (including that of Exercise \ref{exer:lebesgue measure rotation inv}) is that they seem to require the use of the axiom of choice. In fact, R. Solovay has recently shown that there exist models for the Zermelo-Frankel axioms of set theory other than the uncountable axiom of choice, such that, in these models, it is true that every subset of the real line is Lebesgue measurable.

It is also of interest to know that the Borel subsets of the real line form a proper subset of the class of all Lebesgue measurable subsets of the real line. For this fact, see Exercise \ref{exer:cantor set} at the end of this chapter.

\section{Exercises}

\begin{enumerate}[label=\arabic*),ref=\arabic*]
\item\label{exer:ex of measures}
Let $X$ be the set of positive integers and let $S$ be the family of all subsets of $X$. Determine which of the following are measures on $S$ (not all of them are).
\begin{enumerate}[label=\alph*),ref=\alph*)]
    \item $\mu$ has values in $\ell^\infty$, and, for each $E\in S$, $\mu(E)$ is the sequence whose $n$th term is $1/n$ if $n\in E$ and 0 if $n\notin E$.
    
    \item\label{exer:which are measure ell infty}
    $\mu$ has values in $\ell^\infty$ and, for each $E\in S$, $\mu(E)$ is the sequence whose $n$th term is $1$ if $n\in E$ and $0$ if $n\notin E$.
    
    \item $\mu$ is defined as in \ref{exer:which are measure ell infty} but its values are viewed as being elements of the Banach space of sequences $a$ such that $\sum_{n=1}|a_n|/n^2<\infty$. In this space, $\norm{a}=\sum_{n=1}^\infty|a_n|/n^2$.
    
    \item $\mu$ has values in $\ell^1$ and, for each $E\in S$, $\mu(E)$ is the sequence whose $n$th term is $\frac1{n^2}$ if $n\in E$ and $0$ if $n\notin E$.
\end{enumerate}

\item Let $T$ be a collection of sets which is closed under taking finite unions and intersections. Show that the collection of sets of the form $E\sd F$ where $E,F\in T$ forms a semiring. Many interesting semirings arise in this way. For example, this shows that the collection of differences of compact subsets (or of open subsets) of a topological space forms a semiring. Show that the semiring of left-closed right-open intervals of the real line also arises in this way.

\item In showing that the set function $\mu_\alpha$ defined on the left-closed right-open intervals of $\bR$ in terms of a non-decreasing function $\alpha$ is countably additive, we had to use a compactness argument and so the completeness of $\bR$, and we had to assume that $\alpha$ is left continuous.

\begin{enumerate}
\item Show that if $\alpha(x)\equiv x$, but that $\bR$ is replaced by $\bQ$, the rational numbers, then the corresponding function, $\mu_\alpha$, defined on the left-closed right-open subsets of $\bQ$, is not countably additive.
\item Show that if $\alpha$ is not left continuous then $\mu_\alpha$ is not countably additive.
\end{enumerate}

\item\label{exer:empty meas sets}
Let $X$ be a set and let $S$ be the family of all subsets of $X$. Define $\mu^*$ on $S$ by $\mu^*(E)=1$ for all $E\in S$. Show that $\mu^*$ is an outer measure, and determine its measurable sets.

\item A measure $\mu$ is called \defline{continuous}\index{continuous measure} if $\mu(\brc{x})=0$ for every point $x$. If $\bar{\mu_\alpha}$ is a Borel-Stieltjes measure, show that $\overline{\mu_\alpha}$ is a continuous measure if and only if $\alpha$ is a continuous function.

\item If $\mu$ is a measure on a $\sigma$-ring $S$, then $E\in S$ is called an \defline{atom}\index{atom} for $\mu$ if $\mu(E)\neq 0$ and for any $F\subseteq E,\mu(F)=\mu(E)$ or $\mu(F)=0$.
\begin{enumerate}
\item If $\overline{\mu_\alpha}$ is a Borel-Stieltjes measure, determine its atoms in terms of $\alpha$.
\item A measure is called \defline{purely atomic}\index{purely atomic} if every measurable set of nonzero measure is a union of atoms. Determine which $\overline{\mu_\alpha}$ are purely atomic in terms of $\alpha$.
\item If $\alpha$ is such that $\overline{\mu_\alpha}$ is purely atomic, determine the measurable sets for $\overline{\mu_\alpha}$ (that is, for the corresponding outer measure).
\end{enumerate}


\item Unfortunately atoms need not be associated with points as above. Let $X$ be an uncountable set and let $S$ be the collection of subsets of $X$ which are either countable or whose complement is countable. Show that $S$ is a $\sigma$-field. Define $\mu$ on $S$ by $\mu(E)=0$ if $X$ is countable and $\mu(E)=1$ if $E$ is uncountable. Show that $\mu$ is a measure, and find its atoms.

\item\label{exer:non atomic measure}
A measure is called \defline{purely non-atomic}\index{purely non-atomic} if it has no atoms. Prove that if $\mu$ is a purely non-atomic non-negative measure on a $\sigma$-ring $S$, if $E\in S$, and if $c$ is any constant such that $0\leq c\leq\mu(E)$, then there exists an $F\in S$ such that $F\subseteq E$ and $\mu(F)=c$.

\item\label{exer:construct borel for reals}
Prove that the cardinality of the collection of Borel subsets of the real line is equal to the cardinality of the real line, so that not every subset of the real line can be a Borel subset. Hint: let $P$ be the collection of open intervals, let $P^u$ be the collection of countable unions of elements of $P$, let $P^{uc}$ be the union of $P^u$ with the complements of elements of $P^u$, let $P^{ucu}$ be countable unions of elements of $P^{uc}$, etc. Show that each collection of sets so obtained has the cardinality of the real line, and that the union of this sequence of collections is the $\sigma$-ring of Borel sets.

\item\label{exer:cantor set}
The Cantor set\index{cantor set}. Expand every number $x$ in the closed unit interval $X=[0,1]$ in the ternary system, that is, if $x\in X$, write $x=\sum_{n=1}^\infty\frac{\alpha_n}{3^n}$, with each $\alpha_n=0,1$ or $2$, and let $C$ be the set of all those numbers $x$ in whose expansion the digit $1$ is not needed. (Observe that if, motivated by the customary decimal notation, we write $\alpha_1\alpha_2\dots$ for $\sum_{n=1}^\infty\frac{\alpha_n}{3^n}$, then for instance $\frac13=.10000\dots=.02222\dots,$ and therefore $\frac13\in C$, but that since $\frac12=.111\dots$ and since this is the only ternary expansion of $\frac12$, therefore $\frac12\notin C$.) The set $C$ is called the Cantor set. Let $X_1$ be the open middle third of $X$, $X_1=\br{\frac13,\frac23}$; let $X_2,X_3$ be the open middle thirds of the two closed intervals which make up $X\sd X_1$, that is $X_2=\br{\frac19,\frac29}$ and $X_3=\br{\frac79,\frac89}$; let $X_4,X_5,X_6,$ and $X_7$ be the open middle thirds of the four closed intervals which make up $X\sd(X_1\du X_2\du X_3)$, and so on. Prove the following statements.
\begin{enumerate}
    \item $C=X\sd\bigdu_{n=1}^\infty X_n$.
    \item If $\mu$ is Lebesgue measure on $\bR$ then $\mu(C)=0$.
    \item The cardinality of $C$ is the same as the cardinality of the real line.
    \item Conclude that the cardinality of the collection of Lebesgue measurable set is $2^c$ where $c$ is the cardinality of the real line. In view of Exercise \ref{exer:construct borel for reals} this shows that there exists Lebesgue measurable sets which are not Borel sets.
\end{enumerate}

\item An outer measure $\mu^*$ is called \defline{regular}\index{regular measure} (an overused word) if for any $A\in H$ (its domain) there is a $\mu^*$-measurable set $E$ such that $E\supseteq A$ and $\mu^*(E)=\mu^*(A)$. We have seen (Corollary \ref{cor:measure of hereditary wrt outer measure}) that any outer measure induce by a premeasure is regular. Show that if $\mu^*$ is regular, if $A\in H$ and if $\mu^*$ is finitely additive on $\sring{\mring{\mu^*}\cup\brc{A}}$, then $A\in\mring{\mu^*}$. Thus $\mring{\mu^*}$ is the largest $\sigma$-ring on which $\mu^*$ is a measure (although the restriction of $\mu^*$ to $\mring{\mu^*}$ can sometimes be extended to be a measure on larger $\sigma$-rings in other ways).

\item\label{exer:non unique extension measure} Let $P$ be the semiring of left-closed right-open intervals of $\bQ$, the rational numbers, and let $\mu$ be defined on $P$ by $\mu(\varnothing)=0$, and $\mu(E)=\infty$ if $E\in P$, $E\neq\varnothing$. Show that $\mu$ is a premeasure which has many different extensions to a measure on $\sring{P}$. Thus some hypothesis such as $\sigma$-finiteness is needed to prove uniqueness of extensions.

\item\label{exer:null set as ideal}
If $E$ and $F$ are sets, then their symmetric difference, denoted by $E\symd F$, is defined to be $(E\sd F)\cup(F\sd E)$. If $R$ is a ring of sets, show that, with $\symd$ as addition and $\cap$ as multiplication, $R$ becomes a ring in the usual algebraic sense. If $R$ is a $\sigma$-ring and $\mu$ is a non-negative measure on $R$, show that the null sets of $\mu$ which are in $R$ form an ideal in $R$.

\item\label{exer:lebesgue measure rotation inv}
Let $\mu$ be Lebesgue measure on the real line, and let $\mu^*$ be the corresponding outer measure.
\begin{enumerate}[label=\alph*),ref=\theenumi\alph*)]
    \item\label{exer:item:translation inv}
    Show that $\mu$ and $\mu^*$ are translation invariant, that is, if $E\subseteq\bR$ and $r\in\bR$, and if $r+E=\brc{r+x:x\in E}$, then $\mu^*(r+E)=\mu^*(E)$, and similarly for $\mu$. (You must check the translation invariance of the domain of $\mu$.)
    
    \item Let $G=[0,1)$ and define an addition, $\widehat{+}$, on $G$ as follows (this is addition modulo $1$): if $r,s\in G$, then $r\widehat{+}s=r+s$ if $r+s<1$, and $r\widehat{+}s=r+s-1$ if $r+s\geq1$. Show that $G$ is a group under $\widehat{+}$. (In fact the function $r\mapsto\text{exp}(2\pi ir)$ is an isomorphism of the group $G$ onto the multiplicative group of complex numbers of modulus 1.) Show that restriction of $\mu$ and $\mu^*$ to $G$ are translation invariant with respect to $\widehat{+}$.
    
    \item Show that there is a subset of $[0,1)$ which is not Lebesgue measurable, as follows. Let $G_\bQ$ denote the subset of $G$ consisting of the rational numbers in $G$. Then $G_\bQ$ is a subgroup of $G$. Using axiom of choice, pick one element from each coset of $G_\bQ$ in $G$, and let $E$ be the set consisting of all these elements, that is, $E$ is a set of coset representatives for $G_\bQ$. Show that $E$ cannot be Lebesgue measurable.
\end{enumerate}

\end{enumerate}

\chapter{Properties of Measures}

\section{Restrictions of Measures}

\begin{definition}
A pair $(X,S)$ is called a \defline{measurable space}\index{measurable space} if $X$ is a set and $S$ is a $\sigma$-ring of subsets of $X$. A triple $(X,S,\mu)$ is called a \defline{measure space}\index{measure space} if $(X,S)$ is a measurable space and $\mu$ is a measure on $S$.
\end{definition}

\begin{definition}\label{def:measure space set locally}
Let $(X, S)$ be a measurable space. We say that $E\subseteq X$ is \defline{$S$-measurable}\index{S-measurable@$S$-measurable} (or just \defline{measurable}) if $E\in S$. We say that $E\subseteq X$ is \defline{locally $S$-measurable}\index{locally S-measurable@locally $S$-measurable} if $E\cap F\in S$ for all $F\in S$. If $\mu$ is a non-negative measure on $(X, S)$, then we say that $E\subseteq X$ is \defline{$\mu$-measurable}\index{mu-measurable@$\mu$-measurable} if $E\in S\du\nring{\mu}$ (which was defined in Proposition \ref{prop:null set completion}). We say that $E\subseteq X$ is \defline{locally $\mu$-measurable}\index{locally mu-measurable@locally $\mu$-measurable} if $E\cap F$ is $\mu$-measurable for every $\mu$-measurable set $F$.
\end{definition}

We remark that if $\mu$ is $\sigma$-finite, then the $\mu$-measurable sets are exactly the sets which are measurable with respect to the outer measure determined by $\mu$, as is seen from Theorem \ref{thm:null set completion sigma finite}.

It is easily seen that if $(X, S)$ is a measurable space, then the family of all locally $S$-measurable sets is a $\sigma$-field, as is the family of all locally $\mu$-measurable sets if $\mu$ is a non-negative measure on $S$.

If $X$ is any set, $P$ is a collection of subsets of $X$, and $E\subseteq X$, then by $P\cap E$ we will mean the collection $\brc{F\cap E:F\in P}$. For example, if $S$ is a $\sigma$-ring and $E$ is a locally $S$-measurable set, then it is easily seen that $S\cap E$ will form a $\sigma$-ring which is contained in $S$. If $\mu$ is a measure on $S$, then we can obtain a measure on $S\cap E$ by restricting the domain of $\mu$ to $S\cap E$

\begin{definition}
If $\mu$ is a measure on $S$ and if $E$ is a locally $S$-measurable set, then the measure obtained by restricting the domain of $\mu$ to $S\cap E$ will be called \defline{the restriction of $\mu$ to $E$}\index{restriction of measure}.
\end{definition}

Given a measure $\mu'$ on $S\cap E$ we can enlarge its domain to obtain a measure $\mu$ on $S$ by letting $\mu(F)=\mu'(F\cap E)$ for all $F\in S$. Note that if we start with $\mu$ on $S$, restrict $\mu$ to $E$, and then enlarge back to a measure on $S$, we do not necessarily obtain $\mu$ back again. In particular, there will in general be many other ways of enlarging the domain of $\mu'$ to obtain a measure on $S$

\begin{proposition}\label{prop:sring generated by intersection}
If $X$ is a set, $P$ is a family of subsets of $X$ and $E\subseteq X$, then $\sring{P\cap E}=\sring{P}\cap E$
\end{proposition}


\begin{proof}
Since $\sring{P}\cap E$ is a $\sigma$-ring which contains $P\cap E$, it follows that $\sring{P\cap E}\subseteq\sring{P}\cap E$. We must show the reverse inclusion. Let $T$ be the class of all sets of the form $A\du(B\sd E)$ where $A\in\sring{P\cap E}$ and $B\in\sring{P}$. Symbolically we may write $T=\sring{P\cap E}\du(\sring{P}\cap\comp{E})$. It is easy to verify that $T$ is a $\sigma$-ring. If $F\in P$, then the relation $F=(F\cap E)\du(F\sd E)$ and the fact that $F\cap E\in P\cap E\subseteq\sring{P\cap E}$ show that $F\in T$, and therefore that $P\subseteq T$. It follows that $\sring{P}\subseteq T$, and therefore that $\sring{P}\cap E\subseteq T\cap E$. Since, however, it is clear that $T\cap E=\sring{P\cap E}$, it follows that $\sring{P}\cap E\subseteq\sring{P\cap E}$.
\end{proof}

For example, this proposition shows that the two natural ways of defining the Borel sets in the interval $[0,1]$ coincide; we can either take the intersections with $[0,1]$ of the Borel sets in $\bR$, or we can apply directly to $[0,1]$ the definition of the Borel sets of a topological space.

\begin{corollary}
If $\mu$ is a $\sigma$-finite premeasure on a semiring $P$, if $E\in P$ and if $\overline{\mu}$ is the extension of $\mu$ to $\sring{P}$ while $\widehat{\mu}$ is the extension to $\sring{P\cap E}$ of $\mu$ restricted to $P\cap E$, then $\widehat{\mu}$ is just the restriction of $\overline{\mu}$ to $\sring{P}\cap E$.
\end{corollary}

For example, this corollary shows that the two ways in which one might define Lebesgue measure on the interval $[0, 1)$ coincide.

\section{The Total Variation of a Measure}

In Chapter \ref{ch:measures}, most of the results which we proved involved non-negative measures. The purpose of this section is to define the total variation of an arbitrary measure. This gives us a way to obtain a non-negative measure which is closely related to a given arbitrary measure, and this will enable us to extend some of our earlier definitions and results about non-negative measures to arbitrary measures.

\begin{definition}
If $\mu$ is an arbitrary measure on a $\sigma$-ring $S$, then the \defline{total variation}\index{total variation}, $|\mu|$, of $\mu$ is defined by \[|\mu|(E)=\sup\brc{\sum_{i=1}^n\norm{\mu(E_i)}: E=\bigdu_{i=1}^nE_i, E_i\in S}\] for each $E\in S$. (Of course, if $\mu$ is extended real-valued, we let $\norm{\infty}=\infty$.)
\end{definition}


\begin{theorem}\label{thm:total var is measure}
The total variation, $|\mu|$, of a measure $\mu$ is a non-negative measure.
\end{theorem}

\begin{proof}
It is clear from its definition that $|\mu|$ is a non-negative extended real valued function, on $S$. We remark that it is also clear that $|\mu|$ is monotone, that is, that if $E\subseteq F$, where $E, F\in S$, then $|\mu|(E)\leq|\mu|(F)$. To prove the theorem we need to show that $|\mu|$ is countably additive.

Suppose first that $E=F\du G$ with $F,G\in S$. If $F=\bigdu_{i=1}^mF_i$ and $G=\bigdu_{j=1}^nG_j$, then $E=\bigdu_{i=1}^mF_i\du\bigdu_{j=1}^nG_j$, and so \[|\mu|(E)\geq\sum_{i=1}^m\norm{\mu(F_i)}+\sum_{j=1}^n\norm{\mu(G_j)}.\] It follows that $|\mu|(E)\geq|\mu|(F)+|\mu|(G)$. By induction it follows that if $E=\bigdu_{i=1}^nE_i$, then $|\mu|(E)\geq\sum_{i=1}^n|\mu|(E_i)$.

Suppose now that $E=\bigdu_{i=1}^\infty E_i$. Then $E\supseteq\bigdu_{i=1}^nE_i$ for all $n$, and so, since $|\mu|$ is monotone, $|\mu|(E)\geq|\mu|\br{\bigdu_{i=1}^nE_i}=\sum_{i=1}^n|\mu|(E_i)$ for all $n$. Thus $|\mu|(E)\geq\sum_{i=1}^\infty|\mu|(E_i)$.

To prove the opposite inequality, suppose that $E=\bigdu_{j=1}^\infty E_j$ and also that $E=\bigdu_{i=1}^nF_i$. Then
\begin{align*}
    \sum_{i=1}^n\norm{\mu(F_i)}&=\sum_{i=1}^n\norm{\mu\br{F_i\cap\bigdu_{j=1}^\infty E_j}}=\sum_{i=1}^n\norm{\mu\br{\bigdu_{j=1}^\infty F_i\cap E_j}}\\
    &=\sum_{i=1}^n\norm{\sum_{j=1}^\infty\mu(F_i\cap E_j)}\leq\sum_{i=1}^n\sum_{j=1}^\infty\norm{\mu(F_i\cap E_j)}\\
    &=\sum_{j=1}^\infty\sum_{i=1}^n\norm{\mu(F_i\cap E_j)}\leq\sum_{j=1}^\infty|\mu|(E_j),
\end{align*}
since $E_j=\bigdu_{i=1}^n(F_i\cap E_j)$ for each $j$. Thus $|\mu|(E)\leq\sum_{j=1}^\infty|\mu|(E_j)$, and so $|\mu|$ is countably additive.
\end{proof}

Note that $\norm{\mu(E)}\leq|\mu|(E)$ for all $E\in S$, so that $|\mu|$ may be thought of as a non-negative measure which in a sense dominates $\mu$. In Exercise \ref{exer:total var smallest} at the end of this chapter you will be asked to show that $|\mu|$ is the smallest non-negative measure having this property.

As a result of Theorem \ref{thm:total var is measure} we can extend some of our earlier definitions which were originally made only for non-negative measures.

\begin{definition}
An arbitrary measure $\mu$ is said to be \defline{$\sigma$-finite}\index{sigma-finite@$\sigma$-finite} (\defline{totally $\sigma$-finite}\index{totally sigma-finite@totally $\sigma$-finite}) if $|\mu|$ is $\sigma$-finite (totally $\sigma$-finite).
\end{definition}

\begin{definition}\label{def:general sigma finite null set}
If $\mu$ is an arbitrary measure on a $\sigma$-ring $S$, then a set $E\subseteq X$ is said to be (\defline{locally}\index{locally mu-measurable@locally $\mu$-measurable}) \defline{$\mu$-measurable}\index{mu-measurable@$\mu$-measurable} if it is (locally) measurable (see Definition \ref{def:measure space set locally}). A set $F\in\hring{S}$ is called a \defline{$\mu$-null set}\index{null set} (or just a \defline{null set}) if $|\mu|^*(F)=0$. We denote the family of $\mu$-null sets by $\nring{\mu}$.
\end{definition}

We remark again that Theorem \ref{thm:null set completion sigma finite} shows that if $\mu$ is $\sigma$-finite, then the $\mu$-measurable sets are exactly the sets which are measurable with respect to the outer measure determined by $|\mu|$.

\begin{definition}
An arbitrary measure $\mu$ is said to be \defline{complete}\index{complete measure} if every $\mu$-null set is in the domain of $\mu$.
\end{definition}

The fact that we have used the term $\mu$-measurable in Definition \ref{def:general sigma finite null set} suggests that we should be able to extend $\mu$ to a measure on the $\sigma$-ring of $\mu$-measurable sets. The following theorem, which generalizes Proposition \ref{prop:null set completion}, shows that this is in fact the case.

\begin{theorem}\label{thm:completion of general measure}
Let $(X,S,\mu)$ be an arbitrary measure space. Define $\widehat{\mu}$ on the $\sigma$-ring of $\mu$-measurable sets, $S\du\nring{\mu}$, by $\widehat{\mu}(E\du F)=\mu(E)$ where $E\in S$ and $F\in\nring{\mu}$. Then $\widehat{\mu}$ is a well defined complete measure which extends $\mu$.
\end{theorem}

\begin{proof}
If $E\in S$ and $|\mu|(E)=0$, then $\mu(E)=0$. As a consequence, it is easily seen that the proof of Proposition \ref{prop:null set completion} applies to this case also.
\end{proof}

\section{Bounded Measures}
\begin{definition}
A measure $\mu$ is said to be \defline{bounded}\index{bounded measure} if it does not take the value $+\infty$, that is, if all its values are in a Banach space. If $\mu$ is bounded and if in addition $X\in S$ (so that $\norm{\mu(X)}<\infty$), then $\mu$ is said to be \defline{totally bounded}\index{totally bounded measure}.
\end{definition}

We remark that a bounded measure need not be $\sigma$-finite. An example of such a measure will be given in Exercise of Chapter \ref{ch:Lp spaces}.

The following proposition justifies the name ``bounded''.

\begin{proposition}\label{prop:bounded measure are bounded}
If $\mu$ is a bounded measure, then there exists a constant $K<\infty$ such that $\norm{\mu(E)}<K$ for all $E\in$ domain $\mu$.
\end{proposition}

\begin{proof}
Suppose not. Then for each $n$ we can find a measurable set $E_n$ such that $\norm{\mu(E_n)}\geq n$. Let $E=\bigcup_{n=1}^\infty E_n$, so that $E\in S$. By construction, $\mu$ is unbounded on $E$, that is, for each $a\in\bR$ there exists a measurable set $F\subseteq E$ such that $\norm{\mu(F)}\geq a$. If the $E_n$ were disjoint we would easily get a contradiction. We now construct three sequences of measurable sets, $F_n, G_n$ and $H_n$, by induction, so that the $G_n$ acts like the $E_n$ but in addition are disjoint, and so that $\mu$ is unbounded on each $H_n$. Choose $F_1\subseteq E$ such that $\norm{\mu(F_1)}\geq\norm{\mu(E)}+1$. Then $\norm{\mu(F_1)}\geq1$ and $\norm{\mu(E\sd F_1)}\geq1$ (since $\norm{\mu(E\sd F_1)}=\norm{\mu(E)-\mu(F_1)}\geq|\norm{\mu(E)}-\norm{\mu(F_1)}|\geq1$.) It is easily seen that $\mu$ must be unbounded on either $F_1$ or $E\sd F_1$ since otherwise it would not be unbounded on $E$. Let $H_1$ be one of $F_1$ or $E\sd F_1$ in such a way that $\mu$ is unbounded on $H_1$, and let $G_1$ be the other of the two sets. We have thus defined $F_1$, $G_1$, and $H_1$. If $F_{n-1}$, $G_{n-1}$, and $H_{n-1}$ have been chosen, choose $F_n\subseteq H_{n-1}$ such that $\norm{\mu(F_n)}\geq\norm{\mu(H_{n-1})}+1$. Then, as before, $\norm{\mu(F_n)}\geq1$, $\norm{\mu(H_{n-1}\sd F_n)}\geq1$, and $\mu$ is unbounded on either $F_n$ or $H_{n-1}\sd F_n$. Let $H_n$ be one of $F_n$ or $H_{n-1}\sd F_n$ in such a way that $\mu$ is unbounded on $H_n$, and let $G_n$ be the other of the two sets. The important thing to notice is that not only is $\norm{\mu(G_n)}\geq1$ for each $n$, but also the $G_n$ are all disjoint. Let $G=\bigdu_{n=1}^\infty G_n$. Then $\mu(G)=\sum_{n=1}^\infty\mu(G_n)$, and so $\sum_{n=1}^\infty\mu(G_n)$ is a series which converges to a finite (since $\mu$ is assumed bounded) value, but this is impossible since $\norm{\mu(G_n)}\geq1$ for each $n$.
\end{proof}

\begin{definition}
Let $\mu$ be a measure on a $\sigma$-ring $S$. A locally $S$-measurable set $E$ is said to \defline{carry}\index{carry, lives} $\mu$ if $\mu(E)=\mu(F\cap E)$ for all $F\in S$, that is, if $F\cap E=\varnothing$ implies that $\mu(F)=0$. We also sometimes say that $\mu$ \defline{lives} on $E$.
\end{definition}

\begin{proposition}\label{prop:meas set it lives on}
Let $\mu$ be a measure on a $\sigma$-ring $S$. If $\mu$ is a bounded measure, then there exists $E\in S$ on which $\mu$ lives.
\end{proposition}

\begin{proof}
By Proposition \ref{prop:bounded measure are bounded} we know that $\sup\brc{\norm{\mu(F)}: F\in S}<\infty$. We define a sequence, $E_n$, of elements of $S$ by induction. Choose $E_1\in S$ so that $\norm{\mu(E_1)}\geq\frac12\sup\brc{\norm{\mu(F)}: F\in S}$. If $E_1,\dots,E_{n-1}$ have been chosen, choose $E_n$ so that $E_n\cap\bigdu_{i=1}^{n-1}E_i=\varnothing$ and $\norm{\mu(E_n)}\geq\frac12\sup\brc{\norm{\mu(F)}: F\cap\bigdu_{i=1}^{n-1}E_i=\varnothing,F\in S}$. Let $E=\bigdu_{n=1}^{\infty} E_n\in S$. Then $\mu(E)=\sum_{n=1}^\infty\mu(E_n)$, so the series converge to a finite value, and so $\norm{\mu(E_n)}$ converges to $0$ as $n$ goes to $\infty$. We show that $E$ carries $\mu$. Suppose that $F\in S$ and $F\cap E=\varnothing$. Then $F\cap E_n=\varnothing$ for each $n$, and so, by the definition of the $E_n$, we have $\norm{\mu(E_n)}\geq\frac12\norm{\mu(F)}$ for every $n$. Since the $\norm{\mu(E_n)}$ converges to $0$, it follows that $\mu(F)=0$.
\end{proof}

In view of Proposition \ref{prop:meas set it lives on}, it is natural to extend a bounded measure $\mu$ to a measure $\mu'$ on the $\sigma$-field of all locally measurable sets $F$ by setting $\mu'(F)=\mu(F\cap E)$, where $E$ carries $\mu$. It is then easily seen that if $\mu$ is $\sigma$-finite so is $\mu'$. Thus, in a sense, the only time we must work with $\sigma$-rings instead of $\sigma$-fields in order to preserve $\sigma$-finiteness is when we have an extended real valued measure.

If $\mu$ is a non-negative measure then it too can be extended to the $\sigma$-field of locally measurable sets $F$ by letting \[\mu'(F)=\sup\brc{\mu(E):E\subseteq F,E\in\text{domain }\mu}.\] However, in general this $\mu'$ will not be $\sigma$-finite even if $\mu$ is $\sigma$-finite, and so it is usually not useful to make this extension. 

\section{Convergence Properties of Measures}

The following two propositions will be very useful in later chapters.

\begin{proposition}\label{prop:increase limit of measures}
Let $\mu$ be a measure on a $\sigma$-ring $S$. If $\brc{E_n}_{n=1}^\infty$ is a sequence of elements of $S$, and if $E_n\uparrow E$, that is, if $E_n\subseteq E_{n+1}$ for each $n$ and $\bigcup_{n=1}^\infty E_n=E$, then $\mu(E_n)$ converges to $\mu(E)$ as $n$ goes to $\infty$. 
\end{proposition}

\begin{proof}
Clearly $E=E_1\du\bigdu_{i=1}^\infty(E_i\sd E_{i-1})$ and $E_n=E_1\du\bigdu_{i=1}^n(E_i\sd E_{i-1})$, and so
\begin{align*}
\mu(E)&=\mu(E_1)+\sum_{i=1}^\infty\mu(E_i\sd E_{i-1})\\
&=\lim_n\br{\mu(E_1)+\sum_{i=1}^n\mu(E_i\sd E_{i-1})}=\lim_n\mu(E_n)
\end{align*}
as desired.
\end{proof}

\begin{proposition}\label{prop:decrease limit of measures}
Let $\mu$ be a measure on a $\sigma$-ring $S$. If $\brc{E_n}_{n=1}^\infty$ is a sequence of elements of $S$ such that $E_n\downarrow E$, that is, $E_n\supseteq E_{n+1}$ for each $n$ and $\bigcap_{n=1}^\infty E_n=E$, and if $\norm{\mu(E_k)}<\infty$ for some $k$, in case $\mu$ is an extended real-valued measure, then $\mu(E_n)$ converges to $\mu(E)$ as $n$ goes to $\infty$.
\end{proposition}

\begin{proof}
We can assume that $\norm{\mu(E_1)}<\infty$ since we can ignore a finite number of terms if we wish. But $(E_1\sd E_n)\uparrow(E_1\sd E)$, so by Proposition \ref{prop:increase limit of measures}, $\mu(E_1\sd E_n)$ converges to $\mu(E_1\sd E)$ as $n$ goes to $\infty$. Thus, since $\mu(E_1)-\mu(E_n)=\mu(E_1\sd E_n)$ and $\mu(E_1)-\mu(E)=\mu(E_1\sd E)$, we find that $\mu(E_1)-\mu(E_n)$ converges to $\mu(E_1)-\mu(E)$ as $n$ goes to $\infty$. Since $\mu(E_1)$ is assumed finite, it follows that $\mu(E_n)$ converges to $\mu(E)$ as $n$ goes to $\infty$.
\end{proof}

Note that Proposition \ref{prop:decrease limit of measures} is not true without the hypothesis that $\norm{\mu(E_n)}<\infty$ for some $n$. As an example, let $\mu$ be Lebesgue measure on $\bR$ and let $E_n=[n,\infty)$. 

\section{Exercises}
\begin{enumerate}[label=\arabic*),ref=\arabic*]
\item\label{exer:total var smallest}
If $\mu$ is a vector valued measure, show that $|\mu|$ is the smallest extended real valued measure such that $\norm{\mu(E)}\leq|\mu|(E)$ for all $E\in S$, that is, if $\nu$ is a non-negative measure on the domain of $\mu$ such that $\norm{\mu(E)}\leq\nu(E)$ for all $E$ in the domain $\mu$, then $|\mu|(E)\leq\nu(E)$ for all $E$ in the domain $\mu$

\item\label{exer:total var of measure in banach}
A measure $\mu$ is said to be of \defline{bounded variation}\index{bounded variation} if $|\mu|$ is a bounded measure. Show that any measure with values in a finite dimensional Banach space is of bounded variation. (You may assume that $B=\bR^n$ with the usual Euclidean norm, since it can be shown that all norms on a finite dimensional vector space are equivalent, that is, any two norms, $\norm{\imarg}$ and $\norm{\imarg}_0$, satisfy $k\norm{\imarg}\leq\norm{\imarg}_0\leq K\norm{\imarg}$ for suitable constants $k>0$ and $K$.)

\item Compute the total variations of those set functions in Exercise \ref{exer:ex of measures} of Chapter \ref{ch:measures} which are measures. How does you result compare with Exercise \ref{exer:total var of measure in banach} above?

\item Let $B$ be a Banach space and let $(X, S)$ be a measurable space. Then the collection of $B$-valued measures on $S$ forms a vector space when $\mu+\nu$ is defined by $(\mu+\nu)(E)=\mu(E)+\nu(E)$ for all $E\in S$, and $\alpha\mu$ is defined by $(\alpha\mu)(E)=\alpha(\mu(E))$ for $E\in S$. (The field of scalars for the vector space of measures is taken to be the same as the field scalars for $B$. Note that we cannot form a vector space in this way if we admit measures taking the value $+\infty$.) Let $M$ be the collection of all $B$-valued measures on $S$ which are of bounded variation. It is easy to see that $M$ is a subspace of the vector space of all $B$-valued measures on $S$. Furthermore, we can define a function $\norm{\imarg}$ from $M$ to $\bR$ by \[\norm{\mu}=\sup\brc{|\mu|(E):E\in S}.\] Show that $\norm{\imarg}$ is a norm on $M$. (This is called the total variation norm.), and show that $(M,\norm{\imarg})$ is a Banach space. (Be sure to show that the set functions which you claim are measures really are.)

\item Two measures $\mu$ and $\nu$ on $(X, S)$ are said to be \defline{mutually singular}\index{mutally singular} if there are disjoint locally measurable sets $E$ and $F$ such that $E$ carries $\mu$ and $F$ carries $\nu$.
\begin{enumerate}
\item Show that if $\mu$ and $\nu$ are mutually singular then so are $|\mu|$ and $|\nu|$, and that if $\mu$ and $\nu$ have bounded variation, then $\norm{\mu+\nu}=\norm{\mu}+\norm{\nu}=\norm{\mu-\nu}$.
\item Find Borel measures (i.e. measures on Borel sets) $\mu$ and $\nu$ on $[0,1]$ which are mutually singular but are not carried on disjoint closed sets.
\end{enumerate}
\end{enumerate}

\chapter{Measurable Functions}

\section{The Definitions of Measurable Functions}

Let $(X,S,\mu)$ be a measure space, and let $B$ be a Banach space (we are not assuming that the range of $\mu$ is in $B$). In this section we will define certain classes of functions from $X$ to $B$ which are nicely related to $S$ and $\mu$. Eventually these classes will provide us with functions which we can try to integrate with respect to $\mu$.

The simplest functions from $X$ to $B$ which are nicely related to $S$ are the functions which are constant on a measurable set $E$ and have value 0 elsewhere. These functions can be written as $b\idf{E}$ where $b$ is the value of the function on $E$ and $\idf{E}$ is the characteristic function of $E$, that is, the real valued function which has value $1$ on $E$ and 0 elsewhere. Since the functions of the form $b\idf{E}$ take values in a Banach space, we can add these functions (pointwise). It is not difficult to convince oneself that the functions obtained by taking finite sums of functions of the type described above are exactly those described in the following definition.

\begin{definition}
A function, $f$, from $X$ to $B$ is called a \defline{simple $S$-measurable function} (or, for brevity, an \defline{$S$-simple function}) if the range of $f$ is a finite set, $\brc{b_1,\dots,b_n}$, in $B$, and if $f^{-1}(b_i)\in S$ for each $b_i\neq0$.
\end{definition}

It is clear from the definition that the family of $B$-valued simple $S$-measurable functions is also precisely the collection of all functions $f$ from $X$ to $B$ of the form $f=\sum_{i=1}^nb_i\idf{E_i}$ where $b_i$'s are distinct non-zero elements of $B$ and the $E_i$'s are disjoint elements of $S$.

We will let the reader convince themselves that the $B$-valued simple $S$-measurable functions form a vector space under pointwise addition and scalar multiplication by the scalars which act on $B$.

In a Banach space, we have, in addition to vector space structure, a metric space structure, and so we can take pointwise limits of functions which have values in a Banach space.

\begin{definition}
A function, $f$, from $X$ to $B$ is said to be \defline{$S$-measurable} if there is a sequence, $f_n$, of simple $S$-measurable functions which converges to $f$ pointwise, that is, such that for every $x\in X$ the sequence $f_n(x)$ converges to $f(x)$.
\end{definition}

The definition of $S$-measurable functions does not depend on $\mu$. By using $\mu$ we can define a larger class of measurable functions. But we first state a general definition whose use pervades measure theory.

\begin{definition}\label{def:a.e.}
Let $(X,S,\mu)$ be a measure space, and let $P$ be a property which may or may not hold for various points of $X$. Then $P$ is said to hold, or be true, \defline{almost everywhere} on $X$ (abbreviated a.e.) if there is a null set $N$ for $\mu$ such that the property $P$ is true for all points in $X$ except possibly those in $N$.
\end{definition}

This definition will be illustrated in Definition \ref{def:mu measurable function}.

\begin{definition}
A function, $f$, from $X$ to $B$ is called a \defline{simple $\mu$-measurable function} if $f$ is a simple $(S\du\nring\mu)$-measurable function. (The definition of $S\du\nring\mu$ was given in Theorem \ref{thm:completion of general measure}.)
\end{definition}

\begin{definition}\label{def:mu measurable function}
A function, $f$, with values in $B$, which is defined almost everywhere on $X$, is said to be \defline{$\mu$-measurable} if there is a sequence, $f_n$, of simple $\mu$-measurable functions which converges pointwise to $f$ almost everywhere.
\end{definition}

Thus in Definition \ref{def:a.e.}, we allow $f$ to be undefined on a null set of $\mu$, and we allow there to be a (possibly larger) null set on which the sequence $f_n$ does not necessarily converge to $f$ pointwise.

We note that since any simple $S$-measurable function is clearly a simple $\mu$-measurable function, it follows that any $S$-measurable function is $\mu$-measurable.

The following proposition serves to clarify further the relation between $S$-measurable and $\mu$-measurable functions.

\begin{proposition}
A function, $f$, defined almost everywhere on $X$, is $\mu$-measurable if and only if there exists a sequence of simple $S$-measurable functions which converges pointwise to $f$ almost everywhere.
\end{proposition}

\begin{proof}
If $f$ is $\mu$-measurable, then there exist a sequence, $f_n$, of simple $\mu$-measurable functions and a null set $N_0$, such that $f_n$ converges to $f$ off $N_0$. Since the functions $f_n$ are simple $\mu$-measurable functions it follows that for each $n$ we have $f_n=\sum_{i=1}^{k_n}b_i^n\idf{F_i^n}$ where each $b_i^n\in B$ and each $F_i^n\in S\du\nring{\mu}$. Now $F_i^n=E_i^n\du N_i^n$ where $E_i^n\in S$ and $N_i^n\in\nring{\mu}$. If we let $N=N_0\cup\bigcup_{i, n} N_i^n$, then $N$ is a null set. Furthermore, if we let $g_n=\sum_{i=1}^{k_n}b_i^n\idf{E_i^n}$, then each $g_n$ is a simple $S$-measurable function. Finally, since $g_n=f_n$ off $N$, it follows that the $g_n$ converge to $f$ a.e. The converse is clear.
\end{proof}

We noted above that any $S$-measurable function is $\mu$-measurable. It follows that $S$-measurable functions are simultaneously $\mu$-measurable for all measures $\mu$ on $S$, and for this reason it is generally in contexts in which one is working with many measures on $S$ that the class of $S$-measurable functions is useful. On the other hand, it is usually in contexts in which one is working with only one measure on $S$ that it is convenient to use the larger class of all $\mu$-measurable functions.

We remark that when one is working with real valued $\mu$-measurable functions, it is occasionally convenient to permit these functions to take the value $+\infty$ (for example, when using Fubini's theorem). Usually this value will be taken only at the points of some null set, and so we can treat such a function by considering it to be undefined at such points. However, if one prefers, much of measure theory can be developed so as to include functions which take value $+\infty$. Almost all of the results of the rest of this chapter hold for such functions (some statements may have to be modified slightly), but most proofs must include special arguments for the set on which the value $+\infty$ is taken. We will not allow such functions in our development of the theory, but in several places we will indicate where it might be convenient to use such functions.

The next proposition collects together some elementary facts about measurable functions

\begin{proposition}\label{prop:sum product of meas function}
Let $f$ and $g$ be $S$-measurable ($\mu$-measurable) functions from $X$ to $B$, and let $r$ be a scalar. Then $f+g, rf$, and $\norm{f(\imarg)}$ are $S$-measurable ($\mu$-measurable) functions. If, instead, $f$ is scalar-valued, then $fg$ is $S$-measurable ($\mu$-measurable). If $f$ and $g$ are real-valued, then $\max(f, g)$ and $\min(f, g)$ are $S$-measurable ($\mu$-measurable).
\end{proposition}

\begin{proof}
The first four facts follow almost immediately from the definitions of measurable functions. For example, if $f_n$ and $g_n$ are sequences of simple measurable functions which converge to $f$ and $g$ respectively, then it is clear that $f_n+g_n$ is a sequence of simple measurable functions which converges to $f+g$. The final two facts follow from the previous ones and the facts that $\max(f, g)=(f+g+|f-g|)/2$ and $\min(f, g)=(f+g-|f-g|)/2$.
\end{proof}

\section{Characterizations of Measurable Functions}

Unlike the facts stated in the last proposition, it is not at all clear that the pointwise limit of measurable functions is again measurable. In this section we will obtain a quite different characterization of measurable functions which, in addition to making this fact clear, is a very useful way of verifying that functions are measurable in many situations.

\begin{lemma}\label{lem:preimage of open set as sigma union intersection}
If $f_n$ is a sequence of functions from $X$ to $B$ which converges pointwise to a function $f$, and if for any open subset, $U$, of $B$ we define $U_n$ to be $\brc{y\in U:d(y,\comp{U})>\frac1n}$ (where $\comp{U}$ is the complement of $U$ in $B$, and $d(y,\comp{U})=\inf\brc{\norm{y-z}:z\in\comp{U}}$), then $f^{-1}(U)=\bigcup_{n=1}^\infty\bigcup_{K=1}^\infty\bigcap_{k=K}^\infty f_k^{-1}(U_n)$ for every open subset, $U$, of $B$.
\end{lemma}

\begin{proof}
The proof is given by the following chain of equivalent statements (where ``iff'' stands for ``if and only if''): $x\in f^{-1}(U)$ iff $f(x)\in U$, iff there exist integers $n$ and $K$ such that $f_k(x)\in U_n$ for all $k\geq K$, iff there exist $n$ and $K$ such that $x\in f_k^{-1}(U_n)$ for all $k\geq K$, iff there exist $n$ and $K$ such that $x\in\bigcap_{k=K}^\infty f_k^{-1}(U_n)$, iff $x\in\bigcup_{n=1}^\infty\bigcup_{K=1}^\infty\bigcap_{k=K}^\infty f_k^{-1}(U_n)$.
\end{proof}

\begin{definition}
If $f$ is a function from $X$ to $B$, then we define the \defline{carrier} of $f$ to be $\brc{x\in X: f(x)\neq 0}$. We will denote the carrier of $f$ by $\car{f}$. 
\end{definition}

Recall that a subset of a metric space is said to be separable if it contains a countable dense subset. The following theorem was shown to us by S. Newberger.

\begin{theorem}\label{thm:characterization of meas func}
Let $f$ be a function from $X$ to $B$. Then $f$ is $S$-measurable if and only if
\begin{enumerate}[label=\arabic*),ref=\arabic*)]
    \item\label{thm:item:sep range} $f(X)$ (the range of $f$) is a separable subset of $B$, and
    \item\label{thm:item:car} $f^{-1}(U)\cap\car{f}\in S$ for every open ball, $U$, in $B$.
\end{enumerate}
\end{theorem}

\begin{proof}
Suppose that $f$ is an $S$-measurable function. Then there exists a sequence, $f_n$, of simple $S$-measurable functions which converges to $f$ pointwise. For each $n$ let $\brc{b_1^n,\dots,b_{k_n}^n}$ be the range of $f_n$, and let $K$ be the closure of the set of all the $b_i^n$ for all $n$ and all $i$. Clearly $K$ is separable and $f(X)\subseteq K$, and so $f(X)$ is separable. Now let $U$ be an open ball, or, in fact, any open subset of $B$. We need to show that $f^{-1}(U)\cap\car{f}\in S$. But $f^{-1}(U)\cap\car{f}=f^{-1}(U\sd\brc{0})$. Thus we need merely to show that if $U$ is any open set not containing 0, then $f^{-1}(U)\in S$. Now this is obviously true for simple $S$-measurable functions. It follows then from Lemma \ref{lem:preimage of open set as sigma union intersection} that it is true for $f$, since each $U_n$ of that lemma will also be an open set not containing 0.

Conversely, suppose that $f$ satisfies properties \ref{thm:item:sep range} and \ref{thm:item:car}. Then we can choose a sequence, $b_i$, of elements of $B$ which is dense in $f(X)$. We wish to define a sequence, $f_n$, of simple $S$-measurable functions which converges pointwise to $f$. We will define the $f_n$ so that in fact $f_n$ will have its values in $\brc{0,b_1,\dots,b_n}$ for each $n$. In order to do this, let $$C_{ij}=\brc{x\in X:x\in\car{f}\text{ and }\norm{f(x)-b_i}<1/j}$$ for all positive integers $i$ and $j$. Note that by property \ref{thm:item:car} each $C_{ij}$ is in $S$. We might at first glance want to define $f_n$ to have value $b_i$ on $C_{ij}$, but for this to make sense we must first disjointize the $C_{ij}$. However, in doing this we must be careful to keep as much of the $C_{ij}$ for which $j$ is large, since these are the sets on which $f$ is most closely approximated. To accomplish this we let $E_{i j}^n=C_{ij}\sd\bigcup_{(i,j)<(k,l)\leq(n,n)}C_{kl}$ for $1\leq i,j\leq n$, where the pairs $(i,j)$ are totally ordered ``antilexicographically'', that is, we say that $(i,j)\leq(k,l)$ if $j<l$, or if $j=l$ and $i\leq k$. Then for each fixed $n$ the sets $E_{i j}^n$ are disjoint and $E_{ij}^n\subseteq C_{ij}$ for all $i$ and $j$. Let $f_n=\sum_{i,j=1}^nb_i\idf{E_{ij}^n}$. We show that the sequence $f_n$ converges to $f$ pointwise. Thus suppose we are given $x\in X$ and $\ep>0$. We will assume that $x\in\car{f}$, since if $f(x)=0$ then $f_n(x)=0$ for all $n$, and so we are done. Choose $m_0$ such that $1/m_0<\ep$, and choose $i_0$ so that $\norm{f(x)-b_{i_0}}<1/m_0$. If we let $N=\max\brc{i_0, m_0}$, then we claim that $\norm{f(x)-f_n(x)}<\ep$ whenever $n>N$. To see this we note that $x\in C_{i_0m_0}$ by the definition of $i_0$ and $m_0$, and so if $n>N$, then $x\in E^n_{kl}$, where $(k,l)=\max\brc{(i,j):x\in C_{ij},(i_0,m_0)\leq(i,j)\leq(n, n)}$. It follows that $f_n(x)=b_k$ and that $\norm{f(x)-b_k}<1/l\leq 1/m_0<\ep$. Thus $\norm{f(x)-f_n(x)}<\ep$.
\end{proof}

\begin{corollary}\label{cor:characterization of meas func}
Let $f$ be a $B$-valued function defined almost everywhere on $X$. Then $f$ is $\mu$-measurable if and only if
\begin{enumerate}[label=\arabic*),ref=\arabic*)]
    \item\label{cor:item:sep range}
    there exists a null set $N$ such that $f(X\sd N)$ is separable, and
    \item\label{cor:item:car}
    $f^{-1}(U)\cap\car{f}\in S\du\nring{\mu}$ for all open balls, $U$, in $B$.
\end{enumerate}
\end{corollary}
\begin{proof}
Suppose that $f$ is $\mu$-measurable. Then by definition there exist a sequence, $f_n$, of simple $S\du\nring{\mu}$-measurable functions and a null set $N$ such that $f_n$ converges to $f$ pointwise off $N$. Thus $f_n\idf{X\sd N}$ is a sequence of simple $(S\du\nring{\mu})$-measurable functions which converges pointwise to $f\idf{X\sd N}$ (which we take to have value 0 where $f$ is undefined), and so $f\idf{X\sd N}$ is an $(S\du\nring{\mu})$-measurable function. The fact that $f$ satisfies properties \ref{cor:item:sep range} and \ref{cor:item:car} now follow easily from Theorem \ref{thm:characterization of meas func}.

Conversely, suppose that $f$ satisfies properties \ref{cor:item:sep range} and \ref{cor:item:car}, and let $N$ be a null set such that $f(X\sd N)$ is separable. Then clearly, $f\idf{X\sd N}$ satisfies \ref{thm:item:car} and has separable range, so that $f\idf{X\sd N}$ is $(S\du\nring{\mu})$-measurable by Theorem \ref{thm:characterization of meas func}. It follows easily that $f$ is $\mu$-measurable.
\end{proof}

\begin{lemma}
Let $f$ be a function from $X$ to $B$. Then the following are equivalent:
\begin{enumerate}[label=\arabic*),ref=\arabic*)]
    \item\label{lem:item:car cond1}
    $f^{-1}(C)\cap\car{f}$ is in $S$ for every closed set, $C$, in $B$
    \item\label{lem:item:car cond2}
    $f^{-1}(U)\cap\car{f}$ is in $S$ for every open set, $U$, in $B$
    \item\label{lem:item:car cond3}
    $f^{-1}(W)\cap\car{f}$ is in $S$ for every Borel set, $W$, in $B$
\end{enumerate}
\end{lemma}
\begin{proof}
\ref{lem:item:car cond1} implies \ref{lem:item:car cond2}: Let $U$ be an open subset of $B$ and let $C_n=\brc{y\in B:d(x,\comp{U})\geq\frac1n}$, where $d(y,\comp{U})$ is defined as in Lemma \ref{lem:preimage of open set as sigma union intersection}. Then $C_n$ is closed for each $n$, $U=\bigcup_{n=1}^\infty C_n$, ad $f^{-1}(U)=\bigcup_{n=1}^\infty f^{-1}(C_n)$, from which the implication follows immediately.

\ref{lem:item:car cond2} implies \ref{lem:item:car cond3}: $f^{-1}$ is a Boolean algebra homomorphism, that is, $f^{-1}\br{\bigcup_{\alpha\in A}Y_\alpha}=\bigcup_{\alpha\in A}f^{-1}(Y_\alpha)$, $f^{-1}\br{\bigcap_{\alpha\in A}Y_\alpha}=\bigcap_{\alpha\in A}f^{-1}(Y_\alpha)$ and $f^{-1}(B\sd Y)=X\sd f^{-1}(Y)$, where $Y,Y_\alpha\subseteq B$ for all $\alpha\in A$, where $A$ is an arbitrary indexing set. Thus the family of sets $Y$ having the property that $f^{-1}(Y)\cap\car{f}\in S$, is a $\sigma$-field which by assumption contains the open sets of $B$, and so contains the $\sigma$-field of Borel sets of $B$.

\ref{lem:item:car cond3} implies \ref{lem:item:car cond1}: Closed sets are Borel sets.
\end{proof}

\begin{corollary}
In the statements of Theorem \ref{thm:characterization of meas func} and Corollary \ref{cor:characterization of meas func} we can substitute the words ``open subset'', ``closed subset'' or ``Borel subset'' for the words ``open ball''.
\end{corollary}

\begin{proof}
In the first part of the proof of Theorem \ref{thm:characterization of meas func} it was seen that if $f$ is $S$-measurable, then condition \ref{thm:item:car} holds for any open set $U \subset B$. The rest is easily verified.
\end{proof}

We are now in a position to prove the result about the limit of a sequence of measurable functions which was mentioned at the beginning of this section.

\begin{proposition}
If a sequence, $f_n$, of $S$-measurable ($\mu$-measurable) functions converges pointwise (a.e.) to a function $f$, then $f$ is $S$-measurable ($\mu$-measurable).
\end{proposition}

\begin{proof}
We give the proof for the case of $S$-measurable functions only, since the proof for $\mu$-measurable functions is very similar.

Since $f_n$ is $S$-measurable, $f_n(X)$ is separable. Thus the closure of $\bigcup_{n=1}^\infty f_n(x)$ is separable. But it is clear that $f(X)$ is contained in this closure, and so $f(X)$ is separable. Thus $f$ satisfies hypothesis \ref{thm:item:sep range} of Theorem \ref{thm:characterization of meas func}; That $f$ satisfies hypothesis \ref{thm:item:car} follows immediately from Lemma \ref{lem:preimage of open set as sigma union intersection}.
\end{proof}

\section{Almost Uniform Convergence}

For many purposes pointwise convergence is an insufficiently strong type of convergence. The next theorem shows that if the functions considered are measurable, then pointwise convergence actually implies a stronger kind of convergence.

\begin{theorem}[Egoroff (1911)]\label{thm:egoroff}
Let $\mu$ be a non-negative measure on a $\sigma$-ring $S$. If $E\in S$ and $\mu(E)<\infty$, and if $f_n$ is a sequence of $\mu$-measurable functions which converges to a function $f$ a.e. on $E$, then for every $\ep>0$ there exists a measurable set $F\subseteq E$ such that $\mu(E\sd F)<\ep$ and the sequence $f_n$ converges to $f$ uniformly on $F$.
\end{theorem}

\begin{proof}
By removing a null set, we can assume that $f_n$ converges to $f$ everywhere on $E$. For any positive integers $m$ and $n$ let $$E_n^m=\brc{x\in E:\norm{f(x)-f_k(x)}\geq 1/m\text{ for some }k\geq n}.$$ Since each $f_n$, and so $f$, is $\mu$-measurable, it is easily seen by using Proposition \ref{prop:sum product of meas function} and the easy half of Corollary \ref{cor:characterization of meas func} that each $E_n^m$ is in $S \du\nring{\mu}$. Now for fixed $m$ we have $E_n^m\downarrow_n\varnothing$ because $f_n$ converges to $f$. Since $E$, and thus the $E_n^m$, have finite measure, it follows from Proposition \ref{prop:increase limit of measures} that for fixed $m$ the sequence $\mu(E_n^m)$ converges to $0$ as $n$ goes to $\infty$.

Let $\ep>0$ be given. For each $m$ choose $n(m)$ such that $\mu(E_{n(m)}^m)<\ep/2^m$, and let $F=E\sd\bigcup_{m=1}^\infty E_{n(m)}^m$. Then $\mu(E\sd F)\leq\sum_{m=1}^\infty\mu(E_{n(m)}^m)<\ep$. We show that $f_n$ converges to $f$ uniformly on $F$. Given $\delta>0$, choose $m$ so that $1/m<\delta$, and let $N=n(m)$. Now if $x\in F$, then $x\notin E_N^m$, and so for $k\geq N$ we have $\norm{f(x)-f_k(x)}<1/m<\delta$.

Finally, the $F$ just obtained is in $S\du\nring{\mu}$, but if it is desired we can remove a null set so that $F\in S$.
\end{proof}

As is the case with many important theorems, Egoroff's theorem leads naturally to a definition.

\begin{definition}
Let $\mu$ be a non-negative measure on a measure space $(X,S)$, and let the corresponding outer measure be $\mu^*$. A sequence $f_n$ of arbitrary functions on $X$ is said to converge \defline{almost uniformly} on $E\subseteq X$ to a function $f$ if for every $\ep>0$ there exists $F\subseteq E$ such that $E\sd F$ is in the domain of $\mu^*$, $\mu^*(E\sd F)<\ep$, and $f_n$ converges to $f$ uniformly on $F$. (We will abbreviate ``almost uniformly'' by ``a.u.''.)
\end{definition}

Using this definition, Egoroff's theorem states that if $f_n$ is any sequence of $\mu$-measurable functions which converges a.e. to a function, $f$, on a set $E$ of finite measure, then $f_n$ converges to $f$ a.u. on $E$. It is not difficult to find example which show that the condition that $E$ of finite measure is necessary.

The next result shows that even for arbitrary functions on arbitrary sets, convergence a.u. is stronger than convergence a.e.

\begin{proposition}\label{prop:au implies ae}
If $f_n$ is a sequence of arbitrary functions which converges to $f$ a.u. on arbitrary $E\subseteq X$, then $f_n$ converges to $f$ a.e.
\end{proposition}

\begin{proof}
For each $m$ choose $F_m\subseteq E$ such that $\mu^*(E\sd F_m)<1/m$ and $f_n$ converges to $f$ uniformly on $F_m$. Then $f_n$ converges pointwise to $f$ on $\bigcup_mF_m$, and $\mu^*\br{E\sd\bigcup_m F_m}=0$.
\end{proof}

\begin{definition}
A sequence, $f_n$, of arbitrary functions on $X$ is said to be \defline{almost uniformly Cauchy} on $E\subseteq X$ if for each $\ep>0$ there exist $F\subseteq E$ such that $E\sd F$ is in the domain of $\mu^*$, $\mu^*(E\sd F)<\ep$, and the $f_n$ are uniformly Cauchy on $F$ (that is, given $\delta>0$ there exists $N$ such that if $m,n\geq N$, then $\norm{f_m(x)-f_n(x)}<\delta$ for all $x\in F$).
\end{definition}

The next proposition, which is a preliminary completeness result, is probably the first place at which it is important that the functions which we are considering take values in a Banach (hence complete) space.

\begin{proposition}\label{prop:au cauchy implies limit}
If $f_n$ is an almost uniformly Cauchy sequence of functions on $E\subseteq X$, then there exists a function $f$ such that $f_n$ converges to $f$ a.u. on $E$. Furthermore, $f$ is unique a.e. on $E$ (that is, if $f$ and $f'$ are two such limit functions, then $f$ and $f'$ coincide except on a null set in $E$).
\end{proposition}

\begin{proof}
For each integer $m$ choose $F_m\subset E$ such that $\mu^*(E\sd F_m)<1/m$, and $f_n$ is uniformly Cauchy on $F_m$. Let $F=\bigcup_mF_m$, so that $\mu^*(E\sd F)=0$. If $x\in F$, then $x\in F_m$ for some $m$, and so $f_n(x)$ is a Cauchy sequence. Define $f$ by $f(x)=\lim f_n(x)$ for $x\in F$, and let $f$ be anything (e.g. 0) on $E\sd F$. We show that $f_n$ converges to $f$ a.u. on $E$. Given $\ep>0$, choose $m$ so $1/m<\ep$. Then $\mu^*(E\sd F_m)<1/m<\ep$ and $f_n$ is uniformly Cauchy on $F_m$ and converges pointwise, and so uniformly, to $f$ on $F_m$. The uniqueness a.e. of $f$ follows from Proposition \ref{prop:au implies ae}.
\end{proof}

\section{Convergence in Measure}

In this section we study another type of convergence for functions, which will be of considerable use later.
\begin{definition}
Let $\mu$ be a non-negative measure on $(X,S)$ with corresponding outer measure $\mu^*$. A sequence, $f_n$, of arbitrary functions on $X$ is said to converges to a function $f$ \defline{in measure} on $E\subseteq X$ if the intersections of $E$ with $\car{f}$ and the $\car{f_n}$ are in the domain of $\mu^*$, and if for every $\ep>0$, $$\lim_{n\to+\infty}\mu^*(\brc{x\in E:\norm{f(x)-f_n(x)}\geq\ep})=0.$$
\end{definition}

Our first result shows that limits in measure are unique a.e.

\begin{proposition}\label{prop:uniqueness of in measure}
If $f_n$ converges in measure on $E$ to both $f$ and $g$, then $f(x)=g(x)$ a.e. on $E$.
\end{proposition}

\begin{proof}
Using the triangle inequality, we see that for any $\ep>0$ and $n$ we have
\begin{align*}
&\brc{x\in E:\norm{f(x)-g(x)}\geq\ep}\\
&\quad\subseteq\brc{x\in E:\norm{f(x)-f_n(x)}\geq\ep/2}\cup\brc{x\in E:\norm{g(x)-f_n(x)}\geq\ep/2}.
\end{align*}
Applying $\mu^*$ to both sides and letting $n$ go to $\infty$, we find that for each $\ep>0$ $$\mu^*(\brc{x\in E:\norm{f(x)-g(x)}\geq\ep})=0.$$ Since $\brc{x\in E:\norm{f(x)-g(x)}\neq0}=\bigcup_{m=1}^\infty\brc{x\in E:\norm{f(x)-g(x)}\geq1/m}$, we are done.
\end{proof}

We leave the proof of the next proposition to the reader.

\begin{proposition}
If $f_n$ converges to $f$ in measure and $g_n$ converges to $g$ in measure, then
\begin{enumerate}[label=\arabic*)]
    \item $f_n+g_n$ converges to $f+g$ in measure,
    \item for any scalar $c$, $cf_n$ converges to $cf$ in measure,
    \item $\norm{f_n(\imarg)}$ converges to $\norm{f(\imarg)}$ in measure,
    \item for any $F\subseteq X$, $\idf{F}f_n$ converges to $\idf{X}f$ in measure.
\end{enumerate}
\end{proposition}

It is not difficult to find examples which show that convergence in measure does not imply converges a.e., much less converges a.u. (although Corollary \ref{cor:in measure implies subsequence ae} will show that there is a partial implication in this direction). Examples can also be found to show that convergence a.e. does not imply convergence in measure, at least on infinite sets. However we do have:

\begin{proposition}\label{prop:au implies in measure}
If $f_n$ converges a.u. to $f$ on $E$, and if the intersections of $E$ with $\car{f}$ and the $\car{f_n}$ are in the domain of $\mu^*$, then $f_n$ converges to $f$ in measure. 
\end{proposition}

\begin{proof}
Let $\delta>0$ be given, and choose $F\subseteq E$ so that $E\sd F$ is in the domain of $\mu^*$, $\mu^*(E\sd F)<\delta$, and $f_n$ converges to $f$ uniformly on $F$. Then for any $\ep>0$ choose $N$ so that $\norm{f(x)-f_n(x)}<\ep$ for all $n>N$ and for all $x\in F$. It follows that for any $n>N$ we have $$\mu^*(\brc{x\in E:\norm{f(x)-f_n(x)}\geq\ep})\leq\mu^*(E\sd F)<\delta.$$
\end{proof}

\begin{corollary}
If $f_n$ is a sequence of $\mu$-measurable functions which converges a.e. to $f$ on a set $E$ of finite measure, then $f_n$ converges to $f$ in measure on $E$.
\end{corollary}

\begin{proof}
This follows from Egoroff's theorem and Proposition \ref{prop:au implies in measure}.
\end{proof}

\begin{definition}
A sequence $f_n$ of functions is said to be \defline{Cauchy in measure} on $E\subseteq X$ if the intersection of $E$ with each $\car{f_n}$ is in the domain of $\mu^*$, and if for every $\ep>0$ $$\lim_{m,n\to+\infty}\mu^*(\brc{x\in X:\norm{f_m(x)-f_n(x)}\geq\ep})=0.$$
\end{definition}

We now come to the most important completeness result of this chapter. It is the crux cf the proof that certain function spaces (the $L^p$ spaces) defined in Chapter \ref{ch:Lp spaces} are complete, and so are Banach spaces.

\begin{theorem}[Riesz-Weyl]\label{thm:riesz weyl}
If a sequence, $f_n$, of functions is Cauchy in measure on $E\subseteq X$, then there exists a subsequence, $f_{n_k}$, which is almost uniformly Cauchy on $E$. Thus there exists a function $f$ such that $f_{n_k}$ converges to $f$ a.u. on $E$, and hence so that $f_n$ converges to $f$ in measure on $E$. Furthermore, $f$ is unique a.e. on $E$.
\end{theorem}

\begin{proof}
We define the sequence of integers, $n_k$, by induction. Let $n_1=1$. For $k>1$ choose $n_k$ so that $n_k>n_{k-1}$ and so that if $m,n\geq n_k$, then $$\mu^*(\brc{x\in E:\norm{f_m(x)-f_n(x)}\geq2^{-k}})\leq 2^{-k}.$$ For each $k$ we let $g_k=f_{n_k}$, and we show that this subsequence is almost uniformly Cauchy on $E$. Given $\ep>0$, choose $K$ so that $\sum_{k=K}^\infty 2^{-k}=2^{-(k-1)}<\ep$. Let $F=E\sd\bigcup_{k=K}^\infty\brc{x\in E:\norm{g_k(x)-g_{k+1}(x)}\geq2^{-k}}$. Then by the choice of the $n_k$ we have $\mu^*(E\sd F)<\ep$. We must show that $g_k$ is uniformly Cauchy on $F$. Given $\delta>0$ choose $N$ so that $N\geq K$ and $2^{-(N-1)}<\delta$. Then for any $j>l>N$ and $x\in F$ we have
\begin{align*}
&\norm{g_j(x)-g_l(x)}\\
&\quad=\norm{g_j(x)-g_{j-1}(x)+g_{j-1}(x)-g_{j-2}(x)+\dots+g_{l+1}(x)-g_l(x)}\\
&\quad\leq\norm{g_j(x)-g_{j-1}(x)}+\norm{g_{j-1}(x)-g_{j-2}(x)}+\dots+\norm{g_{l+1}(x)-g_l(x)}.
\end{align*}
From the definition of the $n_k$ and the fact that $j,l>k$ and that $x\in F$, it follows that $\norm{g_j(x)-g_l(x)}\leq\sum_{k=l}^{j-1}2^{-k}<2^{-(N-1)}<\delta$, so that the sequence $g_k$ is uniformly Cauchy on $F$.

From Proposition \ref{prop:au cauchy implies limit} it follows that $g_k$ converges a.u. to some function $f$. From Proposition \ref{prop:au implies in measure} it follows that $g_k$ converges to $f$ in measure on $E$. From the fact that $f_n$ is Cauchy in measure, and from the inclusion
\begin{align*}
&\brc{x\in E:\norm{f(x)-f_n(x)}\geq\ep}\\
&\quad\subseteq\brc{x\in E:\norm{f(x)-g_k(x)}\geq\ep/2}\cup\brc{x\in E:\norm{f_n(x)-g_k(x)}\geq\ep/2}
\end{align*}
it follows that, in fact, $f_n$ converges to $f$ in measure on $E$. The uniqueness of $f$ a.e. follows from Proposition \ref{prop:uniqueness of in measure}.
\end{proof}

We remark that, of course, if each of the $f_n$ is $\mu$-measurable, then $f$ will be $\mu$-measurable.

\begin{corollary}\label{cor:in measure implies subsequence ae}
If a sequence, $f_n$, of functions converges to a function $f$ in measure on $E\subseteq X$, then there exists a subsequence, $f_{n_k}$, which converges to $f$ almost uniformly on $E$.
\end{corollary}

\begin{proof}
From the fact that the sequence $f_n$ converges in measure to $f$ on $E$, it is easily seen that $f_n$ is Cauchy in measure on $E$. From Theorem \ref{thm:riesz weyl} it follows that there is a subsequence, $f_{n_k}$, and a function $g$ such that $f_{n_k}$ converges to $g$ a.u. Then $f_{n_k}$ converges in measure to both $f$ and $g$ by Proposition \ref{prop:au implies in measure}, and so $f=g$ a.e. by Proposition \ref{prop:uniqueness of in measure}.
\end{proof}

The definitions of almost uniform convergence and of convergence in measure were given for non-negative measures. We extend these definitions to an arbitrary measure $\mu$ simply by applying them to $|\mu|$.

\begin{definition}
Let $\mu$ be an arbitrary measure on a measurable space $(X, S)$, and let $E\subseteq X$. Then a sequence, $f_n$, of functions will be said to converge on $E$ to a function $f$ \defline{almost uniformly} (resp. \defline{in measure}) with respect to $\mu$ if $f_n$ converges to $f$ on $E$ almost uniformly (resp. in measure) with respect to $|\mu|$. We define what is meant by a sequence of functions which is \defline{almost uniformly Cauchy} (resp. \defline{Cauchy in measure}) with respect to $\mu$, in analogous fashion. 
\end{definition}

\section{Exercises}

\begin{enumerate}[label=\arabic*),ref=\arabic*]
\item\label{exer:non sep range}
Let $X$ be an uncountable set, and let $S$ be the $\sigma$-field of all subsets of $X$. Let $\ell^\infty(X)$ be the Banach space of all bounded real-valued functions on $X$ (with the supremum norm). Find a function from $X$ to $\ell^\infty(X)$ which has the property that $f^{-1}(U)\in S$ for every open subset $U$ of $\ell^\infty(X)$, but such that $f$ is not $S$-measurable (Thus the condition that the range be separable cannot be omitted from the characterization of $S$-measurable functions given in Theorem \ref{thm:characterization of meas func}.)

\item\label{exer:cts on R meas}
Prove that if $f$ is a continuous function on the real line into a Banach space $B$, then $f$ is Borel measurable (that is, measurable with respect to the $\sigma$-field of Borel subsets of the real line). To what more general class of topological spaces than the real line can you extend this result?

\item Show that if $(X,S)$ is a measurable space, and if $S$ has cardinality $2^\alpha$ where $\alpha$ is the countable cardinal (for example, if $S$ is the $\sigma$-field of Borel subsets of the line, see Exercise \ref{exer:construct borel for reals} of Chapter \ref{ch:measures}), and if $f$ is a function on $X$ with values in a Banach space $B$ such that $f^{-1}(U)\in S$ for every open set $U$ in $B$, then $f$ in $S$-measurable. Hint: Show that the range of $f$ must be separable. To do this, show that in any non-separable metric space there is an uncountable collection of disjoint open balls. You may then invoke the continuum hypothesis. (However, those who are quite familiar with the theory of analytic sets will be able to avoid the use of the continuum hypothesis, as was pointed out to us by R. Solovay.) Compare this result with Exercise \ref{exer:non sep range}.

\item Let $\mu$ be Lebesgue measure on the real line. Find sequences of real-valued measurable functions on the real line which:
\begin{enumerate}[label=\alph*),ref=\theenumi\alph*)]
    \item Converge a.e. but not a.u. Thus Egoroff's theorem does not hold if the set under consideration has infinite measure.
    \item Converges a.e. but not in measure.
    \item\label{exer:item:in measure not ae}
    Converges in measure but not a.e.
\end{enumerate}

\item Let $(X,S,\mu)$ be a measure space. Prove that any $S$-measurable function from $X$ to a Banach space $B$ locally almost has compact range, that is, given any $E\in S$ with $|\mu|(E)<\infty$ (``locally'') and given $\ep>0$, there is a measurable $F\subseteq E$ with $|\mu|(E\sd F)<\ep$ (``almost''), such that the range of $f$ restricted to $F$ is a precompact subset of $B$. Hint: Use the definition of compactness in terms of total boundedness. (This result is related to the Radon-Nikodym theorem which we will prove in Chapter 7.) %FIXME linking can't be resolved now

\item\label{exer:essential range}
When working with $\mu$-measurable functions and when identifying functions which agree a.e., one would like to be able to talk about the ranges of such functions, but in such a way that the range of two functions which agree a.e. are the same. The appropriate definition is that of the \defline{essential range} of a $\mu$-measurable function $f$. If $E$ is locally measurable, then the essential range of $f$ on $E$, $\er{f}{E}$, is defined to be the set of $b \in B$ such that for all $\ep>0$, the set $\brc{x\in E:\norm{f(x)-b}<\ep}$ contains a measurable set of strictly positive $|\mu|$-measure. %I just changed it lol
\begin{enumerate}
    \item Verify the following simply properties of $\er{f}{E}$
    \begin{enumerate}[label=\arabic*)]
        \item $\er{f}{E}$ is closed,
        \item $\er{f}{E}\subseteq\text{closure of }f(E)$,
        \item If $f=g$ a.e. on $E$, then $\er{f}{E}=\er{g}{E}$,
        \item If $F\subseteq E$, then $\er{f}{F}\subseteq\er{f}{E}$,
        \item If $|\mu|(E)=0$, then $\er{f}{E}=\varnothing$.
    \end{enumerate}
    
    \item Show that if $E$ is measurable and $|\mu|(E)\neq 0$, then $\er{f}{E}\cap f(E)\neq\varnothing$ and, in fact, that $\brc{x \in E: f(x)\notin\er{f}{E}}$ is a null set in $E$. Thus $f$ can be changed on a null set so that $f(E) \subseteq\er{f}{E}$ (unless $|\mu|(E)=0)$. But show by example that if two sets, $E$ and $F$, are given, then it may not be possible to change $f$ so that $f(E)\subseteq\er{f}{E}$ and $f(F)\subseteq\er{f}{F}$ simultaneously.
    
    \item Show that if $E=\bigcup_{i=1}^\infty E_i$, then $\er{f}{E}=\text{closure}\br{\bigcup_{i=1}^\infty\er{f}{E_i}}$.
\end{enumerate}

\item\label{exer:essentially bounded}
A measurable function on $X$ is said to be essentially bounded if $\er{f}{X}$ is a bounded set. Traditionally the vector space (why?) of essentially bounded functions with values in the Banach space $B$ is denoted by $\cL^\infty(X, B)$. A seminorm, $\norm{\imarg}_\infty$, is defined on $\cL^\infty(X, B)$ by $$\norm{f}_\infty=\sup\brc{\norm{b}: b\in\er{f}{X}}.$$
\begin{enumerate}
    \item Show that the set of functions, $f$, such that $\norm{f}_\infty=0$ is a subspace, $N$, of $\cL^\infty$. Show that if $\norm{f}_\infty=0$ then $f=0$. a.e.
    
    \item The quotient space $\cL^\infty/N$ is traditionally denoted by $L^\infty$. Show that the elements of $L^\infty$ are the equivalence classes of functions in $\cL^\infty$ which agree a.e. Show that $\norm{\imarg}_\infty$ defines a norm on $L^\infty$ for which $L^\infty$ is a Banach space.

    \item Show that the simple measurable functions (more precisely, their equivalence classes) are dense in $L^\infty(X,\bR)$, but are not dense in $L^\infty(X, B)$ where $B=\ell^1$, for example.

    \item Show that if either $B=\bR$ or $B=\bC$ and if multiplication is defined in $L^\infty(X, B)$ to be pointwise multiplication, then $L^\infty(X, B)$ is a Banach algebra, that is, an algebra for which $\norm{fg}_\infty\leq\norm{f}_\infty\norm{g}_\infty$.
\end{enumerate}

\item Let $(X, S, \mu)$ be a measure space with $\mu$ finite and positive, and let $B$ be a fixed Banach space. Let $F$ be the vector space of all functions from $X$ to $B$, and define the distance between any two elements, $f$ and $g$, of $F$ to be $$d(f,g)=\inf_{\ep>0}\frac{\ep+\mu^*(\brc{x:\norm{g(x)-f(x)}\geq\ep})}{1+\ep+\mu^*(\brc{x:\norm{g(x)-f(x)}\geq\ep})}.$$
Show that $d$ is a pseudometric on $F$ which defines convergence in measure, that is, a sequence, $f_n$, of elements of $F$ converges in measure to $f\in F$ iff $d(f_n, f)$ converges to 0. (It is a curious fact that the smallest convex set in $F$ containing any $d$-neighborhood of 0 is all of $F$ itself if, for example, $\mu$ is Lebesgue measure.)

\item\label{exer:precpct range seq of simple func}
Show that if $f$ is a $B$-valued measurable function and if $E$ is a locally measurable set such that $f(E)$ is a precompact subset of $B$, then there exists a sequence of simple measurable functions which converges \defline{uniformly} to $f$ on $E$.
\end{enumerate}

\chapter{Integration}


Throughout the first four sections of this chapter $\mu$ will always denote a real or complex valued measure on a measure space $(X,S)$, and $B$ will denote a Banach space. If $\mu$ takes complex values, then $B$ will be assumed to be over the field of complex numbers. We will write scalars on the right as well as on the left of elements of $B$. All measurable functions will be assumed to have values in $B$ unless the contrary is explicitly stated. By measurable sets we will always mean $\mu$-measurable, and similarly for locally measurable sets. We will not distinguish between $\mu$ and its extension to a complete measure on $\mu$-measurable sets.

\section{Integrable Simple Functions}

\begin{definition} 
An \defline{integrable simple function} (ISF) with respect to $\mu$ is a simple $\mu$-measurable function $f$ whose carrier has finite $|\mu|$-measure. Thus an ISF can be represented as $\sum_{i=1}^nb_i\idf{E_i}$ where $E_i$ are disjoint $\mu$-measurable sets of finite $|\mu|$-measure, and the $b_i$ are in $B$.
\end{definition}

We would like to define the integral of $f$ with respect to $\mu$ to be $\sum_ib_i\mu(E_i)$, but we must first show that this quantity is independent of the representation of $f$

\begin{lemma}
If $\sum_i^mb_i\idf{E_i}=\sum_j^nc_j\idf{F_j}$ as functions, where the $E_i$ are disjoint and the $F_j$ are disjoint, then $\sum_i^mb_i\mu(E_i)=\sum_j^nc_j\mu(F_j)$.
\end{lemma}

\begin{proof}
We may assume that the $b_i$ and $c_j$ are non-zero. Since the two functions are equal, their carriers are equal, and so $\bigdu_i E_i=\bigdu_j F_j$. It follows that $E_i=\bigdu_j E_i \cap F_j$ for each $i$, and that $F_j=\bigdu_iE_i \cap F_j$ for each $j$. Thus we have \[\sum_{i, j}b_i\idf{E_i \cap F_j}=\sum_i b_i\idf{E_i}=\sum_jc_j\idf{F_j}=\sum_{i, j} c_j\idf{E_j\cap F_j}.\] Since the $E_i \cap F_j$ are disjoint, we must have $b_i=c_j$ if $E_i \cap F_j \neq \varnothing$. It follows that \[\sum_ib_i\mu(E_i)=\sum_{i,j}b_i\mu(E_i\cap F_j)=\sum_{i,j}c_j\mu(E_i\cap F_j)=\sum_jc_j\mu(F_j).\]
\end{proof}


If $f$ is an ISF represented by $\sum_ib_i\idf{E_i}$, and if $E$ is a locally $\mu$-measurable set, then $\idf{E}f$ is clearly an ISF represented by $\sum_ib_i\idf{E\cap E_i}$.

\begin{definition}
If $f$ is an ISF which is represented as $\sum_i b_i\idf{E_i}$ with the $E_i$ disjoint, and if $E$ is a locally measurable set, then the \defline{integral} of $f$ over $E$ with respect to $\mu$ is defined to be $\sum_ib_i \mu(E \cap E_i)$. It will be denoted by $\int_Ef(x)\dd\mu(x)$, or $\int_Ef\dd\mu$. If $E=X$ we may write $\int f\dd\mu$ instead of $\int_Xf\dd\mu$.
\end{definition}

\begin{lemma}\label{lem:linearity of ISF integral}
If $f$ and $g$ are ISF, then so is $f+g$, and for any locally measurable set $E$ we have $\int_E(f+g)\dd\mu=\int_Ef\dd\mu+\int_Eg\dd\mu$.
\end{lemma}

\begin{proof}
It is clear that $f+g$ is an ISF. If $f$ and $g$ are represented by $\sum_ib_i\idf{E_i}$ and $\sum_jc_j\idf{F_j}$ respectively, where the $E_i$ are disjoint and so are the $F_j$, and where $\bigdu_iE_i=\bigdu_jF_j$ but we allow the $b_i$ or the $c_j$ to have value 0, then $f=\sum_{i, j}b_i\idf{E_i\cap F_j}$ and $g=\sum_{i, j}c_j \idf{E_i\cap F_j}$, so that $f+g=\sum_{i, j}(b_i+c_j)\idf{E_i\cap F_j}$. Since the $E_i \cap F_j$ are disjoint, we have
\begin{align*}
    \int_E(f+g)\dd\mu&=\sum_{i,j}(b_i+c_j)\mu(E\cap E_i\cap F_j)\\
    &=\sum_{i,j}b_i\mu(E \cap E_i \cap F_j)+\sum_{i,j}c_j\mu(E\cap E_i\cap F_j)\\
    &=\int_Ef\dd\mu+\int_Eg\dd\mu.
\end{align*}
\end{proof}

\begin{corollary}
If $f$ is an ISF, and if $f$ is represented by $\sum_ib_i\idf{E_i}$ where the $E_i$ are no longer required to be disjoint, then for any locally measurable set $E$ it is still true that $\int_Ef\dd\mu=\sum_ib_i\mu(E\cap E_i)$.
\end{corollary}

We will let the reader supply the proofs of the following simple properties of the integral of ISF:

\begin{lemma}
Let $f$ be an ISF and let $E$ be a locally measurable set.
\begin{enumerate}
    \item\label{lem:item:ISF integral homogeneity}
    If $r$ is a scalar, then $rf$ is an ISF, and $\int_E(rf)\dd\mu=r\int_Ef\dd\mu$,
    
    \item $\norm{f(\imarg)}$ is an ISF,
    
    \item If $f$ is non-negative real-valued, then $\int_Ef\dd|\mu|\geq0$,
    
    \item\label{lem:item:ISF integral order}
    If $f$ and $g$ are real-valued ISF, and if $f \geq g$, then $\int_Ef\dd|\mu|\geq\int_Eg\dd|\mu|$,
    
    \item If $f$ is non-negative and $F \subseteq E$ is locally measurable, then $\int_Ff\dd|\mu|\leq\int_Ef\dd|\mu|$,

    \item If $E=F\du G$ with $F$ and $G$ locally measurable, then $\int_Ef\dd\mu=\int_Ff\dd\mu+\int_Gf\dd\mu$.
\end{enumerate}
\end{lemma} 

We remark that Lemma \ref{lem:item:ISF integral homogeneity} together with Lemma \ref{lem:linearity of ISF integral} shows that the ISF form a vector space, on which the integral is a linear functional.

The next result will be of crucial importance in extending the domain of the integral of a wider class of functions.

\begin{lemma}\label{lem:triangle inequality ISF integral}
If $f$ is an ISF and if $E$ is a locally measurable set, then \[\norm{\int_Ef\dd\mu}\leq\int_E\norm{f(x)}\dd|\mu|(x).\]
\end{lemma}

\begin{proof}
If $f=\sum_ib_i\idf{E_i}$ with the $E_i$ disjoint, then $\norm{f(\imarg)}=\sum_i\norm{b_i}\idf{E_i}(\imarg)$, and so
\begin{align*}
    \norm{\int_Ef\dd\mu}&=\norm{\sum_ib_i\mu(E\cap E_i)}\leq\sum_i\norm{b_i}|\mu(E\cap E_i)|\\
    &\leq\sum_i\norm{b_i}|\mu|(E\cap E_i)=\int_E\norm{f(x)}\dd|\mu|(x).
\end{align*}
\end{proof}

\begin{definition}
On the vector space of ISF we define the function $\norm{\imarg}_1$ by $\norm{f}_1=\int\norm{f(x)}\dd|\mu|(x)$. We will call $\norm{f}_1$ the \defline{$L^1$-norm} of $f$.
\end{definition}

\begin{lemma}\label{lem:L1 seminorm on ISF}
The function $\norm{\imarg}_1$ is a seminorm on the vector space of ISF, that is, it satisfies all of the properties of a norm except that if $\norm{f}_1=0$ it does not follow that $f=0$. (we can only conclude that $f(x)=0$ a.e.)
\end{lemma}
\begin{proof}
If $f$ and $g$ are ISF, then
\begin{align*}
    \norm{f+g}_1&=\int\norm{f(x)+g(x)}\dd|\mu|(x)\leq\int(\norm{f(x)}+\norm{g(x)})\dd|\mu|(x)\\
    &=\int\norm{f(x)}\dd|\mu|(x)+\int\norm{g(x)}\dd|\mu|(x)=\norm{f}_1+\norm{g}_1
\end{align*}
We will let the reader verify that $\norm{\imarg}_1$ also satisfies the other properties of a seminorm.
\end{proof}

By means of the seminorm $\norm{\imarg}_1$, we define a semimetric, $d$, on the vector space of ISF by $d(f,g)=\norm{f-g}_1$. Thus, as mentioned by Chapter \ref{ch:preliminary}, $d$ is a function which satisfies all of the properties of a metric except that if $d(f,g)=0$ if does not necessarily follow that $f=g$. This semimetric defines a topology, that is, a collection of open sets, on the vector space of ISF, but this topology need not to be Hausdorff, that is, limits of sequences need not to be unique (but we will find that they are unique a.e.). Furthermore, the usual definition of uniform continuity with respect to a metric is equally applicable to semimetric, and in the present case we have:

\begin{lemma}\label{lem:integral on E uniform cts on ISF}
For any locally measurable set $E$, the function $f\mapsto\int_Ef\dd\mu$ is a uniformly continuous function on the vector space of ISF.
\end{lemma}

\begin{proof}
This is an immediate application of Lemma \ref{lem:triangle inequality ISF integral}, for if $f$ and $g$ are ISF, then we have
\begin{align*}
    \norm{\int_Ef\dd\mu-\int_Eg\dd\mu}&=\norm{\int_E(f-g)\dd\mu}\\
    &\leq\int_E\norm{f(x)-g(x)}\dd|\mu|(x)\\
    &\leq\int_X\norm{f(x)-g(x)}\dd|\mu|(x)\\
    &=\norm{f-g}_1
\end{align*}
\end{proof}

The usual definition of a Cauchy sequence in a metric space is equally applicable to a semimetric space, and, of course, a semimetric space is said to be complete if every Cauchy sequence has a limit (which need not be unique). One can form the completion of a semimetric space in the usual way by taking equivalence classes of Cauchy sequences, and it will follow that uniformly continuous functions from a semimetric space into a complete space will extend to the completion. The objective of the next section is to apply these ideas to the vector space of ISF with the $L^1$-norm, and to the uniformly continuous functions consisting of taking the integrals over locally measurable sets.

\section{Integrable Functions and Convergence in Mean}

\begin{definition}
A sequence, $f_n$, of ISF will be said to be a \defline{mean Cauchy sequence} if it is a Cauchy sequence with respect to the $L^1$-norm, that is, if $\lim_{m, n}\norm{f_m-f_n}_1=0$.
\end{definition}

The completion of the vector space of ISF with respect to the $L^1$-norm is by definition the collection of equivalence classes of mean Cauchy sequences of ISF, where two mean Cauchy sequences, $f_n$ and $g_n$, are said to be equivalent if $\lim_n\norm{f_n-g_n}_1=0$. An ISF $f$ will be identified with the element of the completion which is represented by the constant sequence each of whose terms is $f$. The main objective of this section is to show that each element of the completion can be identified with a measurable function, which is unique a.e. Thus the completion of the vector space of ISF can be identified with a certain space of measurable functions, whose elements will be called integrable functions. The principle tool for making this identification is the Riesz-Weyl Theorem (Theorem \ref{thm:riesz weyl}). To place us in a position to apply this theorem, we need:

\begin{lemma}\label{lem:ISF mean cauchy implies cauchy in measure}
If a sequence of ISF is a mean Cauchy sequence, then it is Cauchy in measure.
\end{lemma}

\begin{proof}
Let $f_n$ be a mean Cauchy sequence of ISF, and let $\ep>0$ be given. Let \[E_{mn}=\brc{x\in X:\norm{f_n(x)-f_m(x)}\geq\ep}.\] Because the $f_n$ are ISF, it is clear that each $E_{mn}$ is a measurable set of finite measure, so that each $\idf{E_{mn}}$ is an ISF. We must show that $\lim_{m,n}|\mu|(E_{mn})=0$. But it is clear that $\idf{E_{mn}}\leq\norm{f_n(\imarg)-f_m(\imarg)}/\ep$, and so by Lemma \ref{lem:item:ISF integral order} we have \[|\mu|(E_{mn})=\int\idf{E_{mn}}\dd|\mu|\leq\int\frac{\norm{f_n(x)-f_m(x)}}{\ep}\dd|\mu|(x)=\norm{f_n-f_m}_1/\ep.\] The result now follows form the fact that the $f_n$ form a mean Cauchy sequence.
\end{proof}

It follows from this proposition and the Riesz-Weyl theorem that if $f_n$ is a mean Cauchy sequence of ISF, then there is a measurable function, $f$, to which $f_n$ converges in measure, and $f$ is unique a.e. We would like to identify $f$ with the elements of the completion of the vector space of ISF corresponding to the sequence $f_n$. But before we can do this, there are two things that we must check. To begin with, we must be sure that two equivalent Cauchy sequence converge in measure to the same function.

\begin{lemma}
If $f_n$ and $g_n$ are equivalent mean Cauchy sequences, and if $f_n$ converges in measure to $f$, then so does $g_n$.
\end{lemma}

\begin{proof}
The sequence $f_1,g_1,f_2,g_2,\dots$ of ISF is easily seen to be a mean Cauchy sequence, and so is Cauchy in measure by Lemme \ref{lem:ISF mean cauchy implies cauchy in measure}. But it has a subsequence which converges in measure to $f$. By the same argument as that used at the end of the proof of Theorem \ref{thm:riesz weyl} it follows that the whole sequence, and so the sequence $g_n$, converges in measure to f.
\end{proof}

We must also show that a given measurable function cannot represent more than one element of the completion of the vector space of ISF. That is, we must show that:

\begin{keylemma}\label{lem:ISF key lem}
If $f_n$ and $g_n$ are two mean Cauchy sequences of ISF which converge in measure to a given measurable function $f$, then they are equivalent sequences, that is, $\lim_n\norm{f_n-g_n}_1=0$.
\end{keylemma}

\begin{proof}
It is easily seen that it suffices to show that $f_n$ and $g_n$ have subsequences which are equivalent. Now by Corollary \ref{cor:in measure implies subsequence ae} there are subsequences, $f_{n_k}$ and $g_{n_k}$ respectively, which converge to $f$ a.u. We propose to show that these subsequences are equivalent. Let $h_k=f_{n_k}-g_{n_k}$. It is easily seen that $h_n$ is a mean Cauchy sequence of ISF which converges a.u. to the function 0. The proof of this lemma reduces to showing that $\norm{h_n}_1$ converges to $0$.

Let $\ep>0$ be given, and choose $N$ so that if $m,n\geq N$ then $\norm{h_n-h_m}_1<\ep/8$. We show that if $n\geq N$ then $\norm{h_n}_1<\ep$. To do this, it suffices to show that $\norm{h_N}_1<\ep/2$, for then $\norm{h_n}_1\leq\norm{h_n-h_N}_1+\norm{h_N}_1<\ep$.

Let $E$ be the carrier of $h_N$. Since $h_N$ is an ISF, we have $|\mu|(E)<\infty$. Furthermore, $h_N$ is bounded, that is, there is a constant, $c$, such that $\norm{h_N(x)}\leq c$ for all $x$. Since $h_n$ converges $0$ a.u., we can find a measurable set $F\subseteq E$ such that $|\mu|(E\sd F)<\ep/4c$ and $h_n$ converges uniformly to $0$ on $F$. Then \[\int_{E\sd F}\norm{h_N(x)}\dd|\mu|(x)\leq\int c\idf{E\sd F}\dd|\mu|=c|\mu|(E\sd F)<\ep/4.\]

Now, since $h_n$ converges to $0$ uniformly on $F$ and since $|\mu|(F)<\infty$, we can find an integer $m\geq N$ such that $\int_F\norm{h_m(x)}\dd|\mu|(x)<\ep/8$. But
\begin{align*}
    \abs{\int_F\norm{h_N(x)}\dd|\mu|(x)-\int_F\norm{h_m(x)}\dd|\mu|(x)}&\leq\int_F|\norm{h_N(x)}-\norm{h_m(x)}|\dd|\mu|(x)\\
    &\leq\int_F\norm{h_N(x)-h_m(x)}\dd|\mu|(x)\\
    &<\norm{h_N-h_m}_1<\ep/8.
\end{align*}
It follows that $\int_F\norm{h_N(x)}\dd|\mu|(x)<\ep/4$. Thus \[\norm{h_N}_1=\int_{E\sd F}\norm{h_N(x)}\dd|\mu|(x)+\int_F\norm{h_N(x)}\dd|\mu|(x)\leq\ep/4+\ep/4=\ep/2,\] as desired.
\end{proof}

\begin{definition}\label{def:bochner integrable}
A measurable function $f$ will be said to be \defline{Bochner-Lebesgue integrable} (or \defline{Bochner integrable}, or just \defline{integrable}) if there is a mean Cauchy sequence of ISF which converges to $f$ in measure. If $f$ is real-valued, we will say simply that $f$ is \defline{Lebesgue integrable}.
\end{definition}

Thus the integrable functions are those which correspond to the points of the completion of the vector space of ISF. Of course, the integral of the ISF over any fixed locally measurable set, being a uniformly continuous function, extends to a function on the collection of integrable functions (because we are assuming that the integral takes values in a Banach, that is complete, space). Specifically if $f_n$ is a mean Cauchy sequence of ISF which converges to $f$ in measure, then $\int_Ef_n\dd\mu$ is a Cauchy sequence (see Lemma \ref{lem:integral on E uniform cts on ISF}). Furthermore its limit depends only on $f$, for, if $g_n$ is another mean Cauchy sequence which converges to $f$ in measure, then $f_n$ and $g_n$ are equivalent Cauchy sequences by Lemma \ref{lem:ISF key lem}, and so $\int_Ef_n\dd\mu$ and $\int_Eg\dd\mu$ will also be equivalent Cauchy sequences (again see Lemma \ref{lem:integral on E uniform cts on ISF}), and so will have the same limit. Thus we are free to make:

\begin{definition}
If $f$ is an integrable function, then its \defline{Bochner Lebesgue integral} (or \defline{Bochner integral}, or \defline{integral}) over a locally measurable set $E$ is defined to be the limit of $\int_Ef_n\dd\mu$ where $f_n$ is any mean Cauchy sequence of ISF which converges to $f$ in measure. This limit will be denoted by $\int_Ef\dd\mu$ or $\int_Ef(x)\dd\mu(x)$. If $E=X$, we will denote the integral simply by $\int f\dd\mu$, or $\int f(x)\dd\mu(x)$. If $f$ is real-valued, then $\int_Ef\dd\mu$ is called simply the \defline{Lebesgue integral} of $f$ over $E$.
\end{definition}

The Lebesgue integral was defined by Lebesgue in his thesis in 1902 for real-valued functions on the real line. The generalization of the Lebesgue integral to real-valued functions on an arbitrary measure space was the work of a number of mathematicians. The extension of the Lebesgue integral to functions with values in a Banach space was developed by Bochner in 1933.

For the purpose of identifying integral functions with the completion of the vector space of ISF it was important to use convergence in measure, since a mean Cauchy sequence of ISF always converges in measure as we have already seen, but need not converge a.e., much less a.u. However, for the purposes of defining integrable functions and their integrals, these other types of converges are equally satisfactory, as the following proposition shows.

\begin{proposition}\label{prop:3 equiv ISF seq}
Let $f$ be a measurable function. Then the following three conditions are equivalent: There exists a mean Cauchy sequence $f_n$ of ISF which converges to $f$
\begin{enumerate}[label=\arabic*),ref=\arabic*)]
    \item\label{prop:item:ISF seq in measure}
    in measure
    \item\label{prop:item:ISF seq au}
    a.u.
    \item\label{prop:item:ISF seq ae}
    a.e.
\end{enumerate}
In all three cases $f$ is integrable, and $\int_Ef_n\dd\mu$ converges to $\int_Ef\dd\mu$.
\end{proposition}

\begin{proof}
The fact that \ref{prop:item:ISF seq in measure} implies \ref{prop:item:ISF seq au} follows from Corollary \ref{cor:in measure implies subsequence ae}. The fact that \ref{prop:item:ISF seq au} implies \ref{prop:item:ISF seq ae} follows from Proposition \ref{prop:au implies ae}. Finally, suppose that $f_n$ converges, to $f$ a.e. Since $f_n$ is a mean Cauchy sequence, and so Cauchy in measure by Lemma \ref{lem:ISF mean cauchy implies cauchy in measure}, it follows from the Riesz-Weyl theorem (Theorem \ref{thm:riesz weyl}) that there is a subsequence, $f_{n_k}$ which converges a.u., and so both a.e. and in measure, to a function $g$. But $f_{n_k}$ converges to $f$ a.e., and so $f=g$ a.e., so that $f_{n_k}$ converges to $f$ in measure. Thus $f_{n_k}$ is a sequence satisfying \ref{prop:item:ISF seq in measure}. The proof of the rest of Proposition \ref{prop:3 equiv ISF seq} should be clear from the above considerations.
\end{proof}

\section{Properties of Bochner-Lebesgue Integrals}

In this section we show that most of the properties which we proved at the beginning of this chapter for ISF actually hold for integrable functions. In fact, most of the proofs consist simply of combining the properties for ISF with the definition of integrable functions in terms of mean Cauchy sequences of ISF.

\begin{proposition}
Let $f$ and $g$ be integrable functions, and let $E$ be a locally measurable set. Then
\begin{enumerate}
    \item\label{prop:item:integral linear}
    $f+g$ is integrable, and $\int_E(f+g)\dd\mu=\int_Ef\dd\mu+\int_Eg\dd\mu$.

    \item\label{prop:item:integral homogeneity}
    If $r$ is a scalar, then $rf$ is integrable and $\int_E(rf)\dd\mu=r\int_Ef\dd\mu$.

    \item $\norm{f(\imarg)}$ is integrable, and $\norm{\int_Ef\dd\mu}\leq\int_E\norm{f(x)}\dd|\mu|(x)$. (Note that a function is $\mu$-integrable iff it is $|\mu|$-integrable).

    \item\label{prop:item:pos func pos integral}
    If $f$ is real-valued and if $f\geq0$ a.e., then $\int_Ef\dd|\mu|\geq0$.

    \item\label{prop:item:integral preserve order}
    If $f$ and $g$ are real-valued and if $f\geq g$ a.e., then $\int_Ef\dd|\mu|\geq\int_Eg\dd|\mu|$.

    \item If $f$ is real-valued and $f\geq0$ a.e., and if $F\subseteq E$ is locally measurable, then $\int_Ff\dd|\mu|\leq\int_Ef\dd|\mu|$

    \item\label{prop:item:integral on disjoint sets}
    If $E=F\du G$ with $F$ and $G$ locally measurable, then $\int_Ef\dd\mu=\int_Ff\dd\mu+\int_Gf\dd\mu$
\end{enumerate}
\end{proposition}

\begin{proof}
We have seen that all of the above properties hold for ISF. To prove \ref{prop:item:integral linear} let $f_n$ and $g_n$ be mean Cauchy sequences of ISF which converge a.e. to $f$ and $g$ respectively. (We use condition \ref{prop:item:ISF seq ae} of Proposition \ref{prop:3 equiv ISF seq}.) Then $f_n+g_n$ is easily seen to be a mean Cauchy sequence of ISF which converges to $f+g$ a.e., and so
\begin{align*}
    \int_E(f+g)\dd\mu&=\lim_n\int_E(f_n+g_n)\dd\mu\\
    &=\lim_n\int_Ef_n\dd\mu+\lim_n\int_Eg_n\dd\mu\\
    &=\int_Ef\dd\mu+\int_E\dd\mu.
\end{align*}
The proofs of the other facts are quite similar, and we leave them to the reader. We remark that to prove \ref{prop:item:pos func pos integral} one must show that if $f_n$ is a mean Cauchy sequence of real-valued ISF which converges to the non-negative function $f$ a.e., then so is $|f_n(\imarg)|$, so that $f$ can be approximated by a sequence of non-negative ISF.
\end{proof}

Properties \ref{prop:item:integral linear} and \ref{prop:item:integral homogeneity} above show that the integrable functions form a vector space, and that the integral over any locally measurable set is a linear function on this vector space.

\begin{definition}
We will denote the vector space of $\mu$-integrable $B$-valued functions by $\cL^1(X,S,\mu,B)$, (or appropriate abbreviations of this, such as $\cL^1$, when this involves no ambiguity)
\end{definition}

\begin{definition}
On $\cL^1$ we define function, $\norm{\imarg}_1$, by \[\norm{f}_1=\int\norm{f(x)}\dd|\mu|(x)\] for all $f\in\cL^1$. We will call $\norm{f}_1$ the \defline{$\cL^1$-norm} of $f$.
\end{definition}

\begin{proposition}
The function $\norm{\imarg}_1$ is a seminorm on $\cL^1$.
\end{proposition}

\begin{proof}
The proof that $\norm{\imarg}_1$ is a seminorm is very straightforward, and similar to the proof of Lemma \ref{lem:L1 seminorm on ISF}, and so we leave it to the reader.
\end{proof}

We would like to see under what conditions an integrable function $f$ has the property that $\norm{f}_1=0$, but we first need some preliminary results which are of some independent interest.

\begin{proposition}
If $f$ is an integrable function, then the carrier of $f$ is $\sigma$-finite.
\end{proposition}

\begin{proof}
Let $f_n$ be a mean Cauchy sequence of ISF which converges to $f$ a.e., and let $E_n$ be the carrier of $f_n$. Since $f_n$ is an ISF, $E_n$ has finite measure. But the carrier of $f$ is contained in the union of all the $E_n$ together possibly with a null set.
\end{proof}

\begin{proposition}\label{prop:integrable idf}
Let $f$ be a non-negative integrable function, and let $E$ be a measurable set. If $f\geq\idf{E}$ a.e., then $|\mu|(E)<\infty$. In particular, $\idf{E}$ is integrable.
\end{proposition}

\begin{proof}
It is clear that, except for a null set, $E$ is contained in the carrier of $f$, and so is $\sigma$-finite. Thus there is a sequence, $E_n$ of subsets of $E$ of finite measure which increases up to $E$, so that $\mu(E_n)$ increases to $\mu(E)$ by Proposition \ref{prop:increase limit of measures}. But $\idf{E_n}$ is integrable for each $n$ and $f\geq\idf{E_n}$ a.e., so that according to Proposition \ref{prop:item:integral preserve order} we have \[|\mu|(E_n)=\int\idf{E_n}\dd|\mu|\leq\int f\dd|\mu|.\] It follows that \[|\mu|(E)\leq\int f\dd|\mu|.\]
\end{proof}

\begin{definition}
A sequence, $f_n$, of integrable functions is said to be \defline{Cauchy in mean} (or a \defline{mean Cauchy sequence}) if it is a Cauchy sequence with respect to the $L^1$-norm, that is, if $\lim_{m, n}\norm{f_n-f_m}_1=0$. A sequence, $f_n$, of integrable functions is said to \defline{converge in mean} to an integrable function $f$ if $\lim_n\norm{f-f_n}_1=0$.
\end{definition}

\begin{proposition}\label{prop:integrable in mean implies measure}
If a sequence of integrable functions is Cauchy in mean, then it is Cauchy in measure. If a sequence of integrable functions converges in mean to an integrable function $f$, then it converges in measure to $f$.
\end{proposition}

\begin{proof}
For mean Cauchy sequences the proof is the same as the proof of Lemma \ref{lem:ISF mean cauchy implies cauchy in measure} except that we must use Proposition \ref{prop:integrable idf} to be sure that $\idf{E_{mn}}$ is integrable, where $E_{mn}$ is defined as in the proof of Lemma \ref{lem:ISF mean cauchy implies cauchy in measure}. The proof for the case of a mean convergent sequence is just a slight variation of the proof for the case of a mean Cauchy sequence.
\end{proof}

\begin{proposition}\label{prop:zero L1 norm}
Let $f$ be an integrable function. Then $\norm{f}_1=0$ if and only if $f=0$ a.e.
\end{proposition}

\begin{proof}
It is clear that if $f=0$ a.e. then $\norm{f}_1=0$, for the mean Cauchy sequence of ISF each of whose terms is the function $0$ will converge to $f$ a.e. Conversely, if $\norm{f}_1=0$, then the sequence each of whose terms is the function 0 converges to $f$ in mean, and so in measure by Proposition \ref{prop:integrable in mean implies measure}. But this sequence also converges to $0$ in measure, and so $f=0$ a.e. by Proposition \ref{prop:uniqueness of in measure}.
\end{proof}

It is easily seen that the collection of functions whose $L^1$-norm is $0$ forms a subspace of $\cL^1$, and that $\norm{\imarg}_1$ defines an actual norm on the factor space obtained by factoring $\cL^1$ by this space of functions whose $L^1$-norm is zero (that is, on the space obtained by identifying any two functions the $L^1$-norm of whose difference is zero). Proposition \ref{prop:zero L1 norm} shows that this factor space consists simply of the equivalence classes of integrable functions which agree a.e.

\begin{definition}
The normed space consisting of the equivalence classes of integrable functions which agree a.e. will be denoted by $L^1(X,S,\mu,B)$, (or appropriate abbreviations of this, such as $L^1$). We will denote the norm on $L^1$ again by $\norm{\imarg}_1$, and refer to it as the $L^1$-norm.
\end{definition}

Our motivation for defining $\cL^1$ was to obtain a completion of the vector space of ISF. Thus we would expect $\cL^1$, and so $L^1$, to be complete, so that $L^1$ is a Banach space. We will now begin to verify this fact. The next few results essentially parallel the usual proof of the fact that the completion of a metric space is in fact complete, but they also yield some other useful pieces of information. The first of these results essentially shows that the $L^1$-norm on $\cL^1$ gives the same metric as that which we would have obtained by the usual process of extending the metric on a metric space to a metric on its completion. Note that we did not quite define the $L^1$-norm by this process.

\begin{lemma}\label{lem:ISF converge in measure implies mean}
If $f_n$ is a mean Cauchy sequence of ISF which converges to $f$ in measure (or a.u., or a.e.), then $f_n$ converges to $f$ in mean.
\end{lemma}

\begin{proof}
It is easily seen that for each fixed $n$ the sequence $\norm{f_m(\imarg)-f_n(\imarg)}$ is a mean Cauchy sequence of ISF which converges to $\norm{f(\imarg)-f_n(\imarg)}$ in measure (or a.u., or a.e.), so that
\begin{align*}
    \norm{f-f_n}_1&=\int\norm{f(x)-f_n(x)}\dd|\mu|(x)\\
    &=\lim_m\int\norm{f_m(x)-f_n(x)}\dd|\mu|(x)=\lim_m\norm{f_m-f_n}_1.
\end{align*}
If for a given $\ep>0$ we choose $N$ large enough so that $\norm{f_m-f_n}_1<\ep$ whenever $m,n>N$, then for $n>N$ it follows that $\norm{f-f_n}_1<\ep$.
\end{proof}

\begin{corollary}\label{cor:ISF dense in L1}
The vector space of ISF is dense in $\cL^1$ with respect to the $L^1$-norm.
\end{corollary}

This corollary is, of course, just a special case of the fact that a metric space is dense in its completion.

\begin{theorem}
$\cL^1$, and so $L^1$, is complete.
\end{theorem}

\begin{proof}
Let $f_n$ be a mean Cauchy sequence of elements of $\cL^1$. Using Corollary \ref{cor:ISF dense in L1} choose for each $n$ an ISF $g_n$ such that $\norm{f_n-g_n}_1<1/n$. It is easily seen that $g_n$ is a mean Cauchy sequence, and so by the Riesz-Weyl theorem (Theorem \ref{thm:riesz weyl}) $g_n$ converges in measure to a measurable function, $f$, which must thus be integrable. By Lemma \ref{lem:ISF converge in measure implies mean}, $g_n$ converges to $f$ in mean. It is easily seen that this implies that $f_n$ converges to $f$ in mean (basically because $f_n$ and $g_n$ are equivalent Cauchy sequences).
\end{proof}

We conclude this section with some consequences of Corollary \ref{cor:ISF dense in L1}.

\begin{proposition}\label{prop:integral small outside finite set}
Let $f$ be an integrable function. Then for every $\ep>0$ there is a measurable set $E$ of finite measure such that \[\int_{X\sd E}\norm{f(x)}\dd|\mu|(x)<\ep.\]
\end{proposition}

\begin{proof}
Given $\ep>0$, we can, by Corollary \ref{cor:ISF dense in L1}, find an ISF, g, such that $\norm{f-g}_1<\ep$. Let $E$ be the carrier of $g$, so that $E$ is a measurable set of finite measure. Then \[\int_{X\sd E}\norm{f(x)}\dd|\mu|(x)=\int_{X\sd E}\norm{f(x)-g(x)}\dd|\mu|(x)\leq\norm{f-g}_1<\ep.\]
\end{proof}

\begin{proposition}
Let $f$ be an integrable function, and let $E$ be a locally measurable set. Then \[\int_Ef\dd\mu=\int\idf{E}f\dd\mu.\]
\end{proposition}

\begin{proof}
It is easily verified that this equality is true whenever $f$ is an ISF. It is also easily seen that both sides of the equality are continuous functions of $f$ with respect to the $L^1$-norm. From the fact that the ISF are dense it follows that the equality holds for all $f$.
\end{proof}

\section{The Indefinite Integral of an Integrable Function}

Let $f$ be an integrable function. In this section we will study the properties of $\int_Ef\dd\mu$ as a function of $E$.

\begin{definition}
Let $f\in\cL^1(X,S,\mu,B)$. Then the set function $\mu_f$, defined on the $\sigma$-field of locally measurable sets by \[\mu_f(E)=\int_Ef\dd\mu\] (and so having values in B) will be called the \defline{indefinite integral} of $f$.
\end{definition}

\begin{theorem}\label{thm:indef integral is measure}
For every integrable function $f$ the set function $\mu_f$ is a measure.
\end{theorem}

\begin{proof}

We must show that $\mu_f$ is countably additive. We will show this by first verifying it when $f$ is an ISF, and then, in the general case, by approximating $f$ by ISF.

Assume now that $f$ is an ISF. If $f=b\idf{F}$ for some $b\in B$ and some measurable set $F$ of finite measure, then $\mu_f(E)=b\mu(E\cap F)$ for every locally measurable set $E$. Thus the countable additivity of $\mu_f$ follows immediately from the countably additivity of $\mu$. But every ISF is just a finite sum of terms of the form $b\idf{F}$, and so the theorem is true for any ISF.

The proof in the general case depends on the following inequality:

\begin{lemma}
If $f$ and $g$ are integrable functions, then $\norm{\mu_f(E)-\mu_g(E)}\leq\norm{f-g}_1$ for every locally measurable set $E$.
\end{lemma}

\begin{proof}
$\norm{\mu_f(E)-\mu_g(E)}=\norm{\int_Ef\dd\mu+\int_Eg\dd\mu}\leq\int_E\norm{f(x)-g(x)}\dd|\mu|(x)\allowbreak\leq\norm{f-g}_1$.
\end{proof}

We return to the proof of Theorem \ref{thm:indef integral is measure}. Let a locally measurable set $E$ be given, and let $E=\bigdu_{i=1}^\infty E_i$, with the $E_i$ locally measurable. We have already seen that $\mu_f$ is finitely additive (Proposition \ref{prop:item:integral on disjoint sets}), and so to prove countable additivity it suffices to show that $\mu_f\br{\bigdu_{i=1}^n E_i}$ converges to $\mu_f(E)$ as $n$ goes to $\infty$. Let $\ep>0$ be given. Choose an ISF $g$ such that $\norm{f-g}_1<\ep/3$. We have just seen that $\mu_g$ is countably additive, and so we can choose $N$ so that if $n>N$ then $\norm{\mu_g(E)-\mu_g\br{\bigdu_{i=1}^nE_i}}<\ep/3$. Then for $n>N$ we have
\begin{align*} %FIXME
    \norm{\mu_g(E)-\mu_g\br{\bigdu_{i=1}^nE_i}}&\leq\norm{\mu_f(E)-\mu_g(E)}+\norm{\mu_g(E)-\mu_g\br{\bigdu_{i=1}^nE_i}}\\
    &\quad+\norm{\mu_g\br{\bigdu_{i=1}^nE_i}-\mu_f\br{\bigdu_{i=1}^nE_i}}\\
    &\leq\norm{f-g}_1+\ep/3+\norm{f-g}_1\leq\ep.
\end{align*}
\end{proof}

The next proposition shows how to compute the total variation of $\mu_f$.

\begin{proposition}\label{prop:total var of indef integral}
Let $f$ be an integrable function. Then for each locally measurable set $E$ we have \[|\mu_f|(E)=\int_E\norm{f(x)}\dd|\mu|(x).\]
\end{proposition}

\begin{proof}
Let $E$ be given, and suppose that $E=\bigdu_{i=1}^nE_i$. Then \begin{align*}
    \sum_{i=1}^n\norm{\mu_f(E_i)}&=\sum_{i=1}^n\norm{\int_{E_i}f\dd\mu} \leq \sum_{i=1}^n\int_{E_i}\norm{f(x)}\dd|\mu|(x)\\
    &=\int_E\norm{f(x)}\dd|\mu|(x).
\end{align*} It follows that $|\mu_f|(E)\leq\int_E\norm{f(x)}\dd|\mu|(x)$.

To prove the reverse inequality we first show that it is true when $f$ is an ISF, and then obtain the general case from the fact that the ISF are dense in $\cL^1$.

Suppose now that $f$ is an ISF, and let $f=\sum_i^kb_i\idf{F_i}$ where the $F_i$ are disjoint. Let $F=\bigdu_i^kF_i$. In the proof we must take into account the definition of $|\mu_f|$, and so for each $i$ suppose that measurable sets $G_{ij}$ are given such that $E\cap F_i=\bigdu_j^{n_i}G_{ij}$. Then
\begin{align*}
    |\mu_f|(E)&=|\mu_f|(E\cap F)\geq\sum_{i,j}\norm{\mu_f(G_{ij})}\\
    &=\sum_{i,j}\norm{\int_{G_{ij}}f\dd\mu}=\sum_{i,j}\norm{b_i}|\mu(G_{ij})|\\
    &=\sum_i\norm{b_i}\br{\sum_j|\mu(G_{i j})|}.
\end{align*}
It follows that \[|\mu_f|(E)\geq\sum_i\norm{b_i}|\mu|(F_i\cap E)=\int_E\norm{f(x)}\dd|\mu|(x),\] so that the theorem is true whenever $f$ is an ISF.

To complete the proof we need the following Lemma:

\begin{lemma}
Let $\mu$ and $\nu$ be measures on a measurable space $(X,S)$ with values in $B$ (so that $\mu+\nu$ is defined by $(\mu+\nu)(E)=\mu(E)+\nu(E)$). Then $|\mu+\nu|\leq|\mu|+|\nu|$, in the sense that for each $E\in S$ we have $|\mu+\nu|(E)\leq|\mu|(E)+|\nu|(E)$. Consequently, $||\mu|(E)-|\nu|(E)|\leq|\mu-\nu|(E)$
\end{lemma}

\begin{proof}
If $E=\bigdu_i^nE_i$, the \[\sum_i^n\norm{(\mu+\nu)(E_i)}\leq\sum_i^n\norm{\mu(E_i)}+\sum_i^n\norm{\nu(E_i)}\leq|\mu|(E)+|\nu|(E).\] The desired inequalities follow immediately.
\end{proof}

We now return to the proof of Proposition \ref{prop:total var of indef integral}. Suppose now that $f$ and $g$ are integrable functions and that $E$ is a locally measurable set. Then
\begin{align*}
    ||\mu_f|(E)-|\mu_g|(E)|&\leq|\mu_f-\mu_g|(E)=|\mu_{f-g}|(E)\\
    &\leq\int_E\norm{(f-g)(x)}\dd|\mu|(x)\leq\int\norm{(f-g)(x)}\dd|\mu|(x)\\
    &=\norm{f-g}_1.
\end{align*}
It follows that $|\mu_f|(E)$ is a continuous function of $f$ with respect to the $L^1$-norm for each fixed $E$, as is $\int_E\norm{f(x)}\dd|\mu|(x)$. Since we have seen that the equality which we wish to prove is true for the ISF, and since these are dense in $\cL^1$, it follows that the equality is true for all integrable functions.

\end{proof}

The indefinite integral of a $\mu$-integrable function is closely related to the measure $\mu$. The following definition and proposition provide one useful aspect of this relation.

\begin{definition}
Let $m$ and $\mu$ be arbitrary measures on a measurable space $(X,S)$. We say that $m$ is \defline{strongly absolutely $\mu$-continuous} if for each $\ep>0$ there exists a $\delta>0$ such that $|m|(E)<\ep$ for all $E\in S$ such that $|\mu|(E)<\delta$.
\end{definition}

We will examine some related types of $\mu$-continuity in chapter 7. %FIX unresolved linking

\begin{proposition}\label{prop:indef int mu strong abs cts}
Let $\mu$ be a scalar-valued measure and let $f$ be a $\mu$-integrable function. Then the indefinite integral, $\mu_f$, of $f$ (viewed as defined only as measurable sets) is strongly absolutely $\mu$-continuous.
\end{proposition}

\begin{proof}
Again the proof involves approximating $f$ by ISF. Let $\ep>0$ be given. Choose an ISF $g$ such that $\norm{f-g}_1<\ep/2$. Since $g$ is an ISF it is bounded, so there is a constant, $c$, such that $\norm{g(x)}\leq c$ for all $x$. Let $\delta=\ep/(2c)$. Suppose now that $E$ is a measurable set such that $|\mu|(E)<\delta$. Then
\begin{align*}
    |\mu_f|(E)&=\int_E\norm{f(x)}\dd|\mu|(x)\\
    &\leq\int_E\norm{f(x)-g(x)}\dd|\mu|(x)+\int_E\norm{g(x)}\dd|\mu|(x)\\
    &\leq\norm{f-g}_1+c|\mu|(E)\leq\ep.
\end{align*}
\end{proof}

A typical application of this proposition can be found in the proof of Theorem \ref{thm:dct}.

\section{Some Convergence Theorems}

In this section we derive some theorems which are very useful in determining whether a sequence of integrable functions converges in mean to a given integrable function. As a corollary of the first of these theorems we will obtain a very useful characterization of integrable functions.

\begin{theorem}[Lebesgue Dominated Convergence Theorem]\label{thm:dct}
Let $f_n$ be a sequence of integrable function which converges a.e. to a (necessarily measurable) function $f$. If there exists a real-valued integrable function $g$ such that $\norm{f_n(x)}\leq g(x)$ a.e. for each $n$, then the $f_n$ form a mean Cauchy sequence, $f$ is integrable, and $f_n$ converges to $f$ in mean.
\end{theorem}

\begin{proof}
Let $\ep>0$ be given. By Proposition \ref{prop:integral small outside finite set} choose a measurable set $E$ of finite measure such that $\int_{X\sd E}g(x)\dd|\mu|(x)<\ep/6$. (Note that $g \geq 0$ a.e. so that we don't need to take absolute values). Then for all $m$ and $n$ we have
\begin{align*}
    \int_{X\sd E}\norm{f_n(x)-f_m(x)}\dd|\mu|(x)&\leq\int_{X\sd E}\norm{f_n(x)}\dd|\mu|(x)+\int_{X\sd E}\norm{f_m(x)}\dd|\mu|(x)\\
    &\leq2\int g(x)\dd|\mu|(x)<\ep/3.
\end{align*}

We saw in the last section that $\mu_g$ is strongly absolutely $\mu$-continuous, so we choose $\delta>0$ such that if $|\mu|(G)<\delta$ then $|\mu_g|(G)<\ep/6$. Since $E$ has finite measure, it follows from Egoroff's theorem that the sequence $f_n$ converges to $f$ a.u. on $E$. Thus we can choose a measurable set $F\subseteq E$ such that $|\mu|(E\sd F)<\delta$ and the sequence $f_n$ converges to $f$ uniformly on $F$. Because of the way in which $\delta$ was chosen, and because $|\mu|(E\sd F)<\ep$, we have for all $m$ and $n$ 
\begin{align*}
    \int_{E\sd F}\norm{f_n(x)-f_m(x)}\dd|\mu|(x)&\leq\int_{E\sd F}\norm{f_n(x)}\dd|\mu|(x)+\int_{E\sd F}\norm{f_m(x)}\dd|\mu|(x)\\
    &\leq2\int_{E\sd F}g(x)\dd|\mu|(x)=2|\mu_g|(E\sd F)\leq\ep/3.
\end{align*}

Finally, since the sequence $f_n$ converges to $f$ uniformly on $F$, we can find $N$ such that if $m,n>N$, then $\norm{f_n(x)-f_m(x)}\leq\ep/(3|\mu|(F))$ for all $x\in F$. Then for all $m,n>N$ we have that \[\int_F\norm{f_n(x)-f_m(x)}\dd|\mu|(x)\leq\int_F\ep/(3|\mu|(F))\dd|\mu|(x)=\ep/3.\] It follows that for all $m,n>N$ we have \[\norm{f_n-f_m}_1=\br{\int_{X\sd E}+\int_{E\sd F}+\int_F}\norm{f_n(x)-f_m(x)}\dd|\mu|(x)\leq\ep,\] so that the $f_n$ form a mean Cauchy sequence as desired.

Since $\cL^1$ is complete, the sequence $f_n$ converges in mean to some integrable function. By Corollary \ref{cor:in measure implies subsequence ae} a subsequence of the $f_n$ converges a.e. to this same function, and so this function must equal $f$ a.e. Thus $f$ is integrable and $f_n$ converges to $f$ in mean.
\end{proof}

Bochner did not define integrable $B$-valued functions in the way that we did in Definition \ref{def:bochner integrable}. Rather, he assumed that the theory of the Lebesgue integral for real-valued functions was known, and he defined integrable $B$-valued functions to be measurable functions which are dominated in norm by an integrable real-valued function. The next theorem shows that his definition is equivalent to Definition \ref{def:bochner integrable}. But the main reason for our interest in this theorem is that it gives a very useful characterization of integrable functions (see for example the proof of Theorem \ref{thm:minkowski inequality}).

\begin{theorem}\label{thm:bochner characterization integrable}
Let $\mu$ be a scalar-valued measure on $(X,S)$, and let $f$ be a $\mu$-measurable $B$-valued function. If there is a real-valued $\mu$-integrable function $g$ such that $\norm{f(x)}\leq g(x)$ a.e., then $f$ is $\mu$-integrable.
\end{theorem}

\begin{proof}
We must produce a mean Cauchy sequence of ISF which converges to $f$ a.e. Since $f$ is measurable, we can find a sequence, $f_n$, of simple measurable functions which converges to $f$ a.e. For each $n$ let \[h_n(x)=\begin{cases}f_n(x)&\text{if }\norm{f_n(x)}\leq2g(x)\\0&\text{if }\norm{f_n(x)}>2g(x)\end{cases}.\]
Equivalently, if we let $E_n=\brc{x: 2g(x)-\norm{f_n(x)}\geq 0}$, then $h_n=f_n\idf{E_n}$. Since $E_n$ is clearly locally measurable, it follows that each $h_n$ is also a simple measurable function. It is easily seen that the sequence $h_n$ converges to $f$ a.e., and we have just arranged matters that for each $n$ we have $\norm{h_n(x)}\leq2g(x)$ for all $x$. But it is easily seen by using Proposition \ref{prop:integrable idf} that this inequality implies that each $h_n$ is integrable, which is something which did not need to be true for the $f_n$. It follows from the Lebesgue dominated convergence theorem (Theorem \ref{thm:dct}) that $h_n$ is a mean Cauchy sequence and that $f$ is integrable.
\end{proof}

The remaining theorems of this section involve the order properties of the real numbers, and so do not generalize to vector-valued functions. However these theorems are very useful for working with vector-valued functions, as we will see in the next chapter.

\begin{theorem}[The Monotone Convergence Theorem]\label{thm:mct}
Let $f_n$ be a sequence of real-valued integrable functions which is non-decreasing a.e. (that is, for each $n$ we have $f_n(x)\geq f_{n-1}(x)$ a.e.). If the sequence of the norms of the $f_n$ is bounded, that is, if there is a constant, $c$, such that $\norm{f_n}_1\leq c$ for all $n$, then $f_n$ is a mean Cauchy sequence, and there exists an integrable function $f$ such that $f_n$ converges to $f$ a.e. and in mean. In particular, $\int f_n\dd\mu$ converges to $\int f\dd\mu$. The same result holds if instead the $f_n$ are non-increasing a.e.
\end{theorem}

\begin{proof}
If the sequence $f_n$ is non-increasing, then the sequence $-f_n$ is non-decreasing, and so it suffices to consider only the case in which the sequence $f_n$ is non-decreasing. Then the numbers $\int f_n\dd|\mu|$ are non-decreasing, and are clearly bounded above by $c$ and so they form a Cauchy sequence. But $f_n-f_m$ is either positive a.e. or negative a.e., depending on whether or not $n$ is larger than $m$, and so we have \begin{align*}
    \norm{f_n-f_m}_1&=\int|f_n-f_m|\dd|\mu|=\abs{\int(f_n-f_m)\dd|\mu|}\\
    &=\abs{\int f_n\dd|\mu|-\int f_m\dd|\mu|}.
\end{align*} Thus $f_n$ is a mean Cauchy sequence also. Since $\cL^1$ is complete, there exists an $f\in\cL^1$ to which the sequence $f_n$ converges in mean, and hence in measure. But then by Corollary \ref{cor:in measure implies subsequence ae} there must be a subsequence of the $f_n$ which converges to $f$ a.e. Since the $f_n$ are non-decreasing, it follows that the sequence $f_n$ itself also converges to $f$ a.e.
\end{proof}

\begin{corollary}\label{cor:mct}
If $f_n$ is a sequence of real-valued integrable functions which is non-decreasing a.e. and which converges a.e. to a function $f$, and if the sequence of the norms of the $f_n$ is bounded, then $f$ is integrable and the sequence $f_n$ converges to $f$ in mean. In particular, $\int f_n\dd\mu$ converges to $\int f\dd\mu$. The same result holds if instead the sequence $f_n$ is non-increasing a.e.
\end{corollary} 

\begin{proof}
By Theorem \ref{thm:mct} the sequence $f_n$ will converge a.e. and in mean to an integrable function $h$, which must of course equal $f$ a.e.
\end{proof}

By making the appropriate definition of the integrals with respect to a non-negative measure of arbitrary non-negative extended real-valued functions which are measurable in a natural sense, we can conveniently state a corollary of the Monotone Convergence Theorem which is useful in certain situations. For a typical application of this corollary see the proof of Theorem \ref{thm:product premeasure is premeasure}.

\begin{definition}\label{def:extended real meas integrable}
Let $\mu$ be a non-negative measure, and let $f$ be a non-negative extended real-valued function on $X$. Let $E$ be the set where $f$ takes the value $\infty$. We say that $f$ is \defline{$\mu$-measurable} if $E$ is $\mu$-measurable and if $f$ is $\mu$-measurable in the usual sense on $X\sd E$. If $f$ is $\mu$-measurable, then we define the integral of $f$ with respect to $\mu$ as follows. If $\mu(E)>0$ then we set $\int f\dd\mu=\infty$. If $\mu(E)=0$, then we can view $f$ as being undefined on the null set $E$, and we then set $\int f\dd\mu=\infty$ if $f$ is not integrable in the usual sense, whereas if $f$ is integrable, then we let $\int f\dd\mu$ be the usual integral of $f$.
\end{definition}

\begin{corollary}\label{cor:mct extended real}
If $\mu$ is a non-negative measure and if $f_n$ is a non-decreasing sequence of $\mu$-measurable non-negative extended real-valued functions which converges a.e. to a non-negative extended real-valued function $f$, then $f$ is $\mu$-measurable and the sequence $\int f_n\dd\mu$ converges to $\int f\dd\mu$.
\end{corollary}

\begin{proof}
We will let the reader verify that $f$ must be $\mu$-measurable, since in the applications we make of this corollary in these notes we will always know in advance that $f$ is measurable. We remark next that if $g$ and $h$ are $\mu$-measurable non-negative extended real-valued functions such that $g\geq h$ a.e. then it follows from Theorem \ref{thm:bochner characterization integrable} and Proposition \ref{prop:item:integral preserve order} that $\int g\dd\mu\geq\int h\dd\mu$. Thus if $\int f_n\dd\mu=\infty$ for at least one $n$, then it is clear that the sequence $\int f_n\dd\mu$ converges to $\infty=\int f\dd\mu$. On the other hand, if all of the $f_n$ are integrable, then either the increasing sequence $\int f_n\dd\mu$ is unbounded, in which case it is clear again that it converges $\infty=\int f\dd\mu$, or else it is bounded, in which case we can apply Corollary \ref{cor:mct} to conclude that $f$ also is integrable and that $\int f_n\dd\mu$ again converges to $\int f\dd\mu$.
\end{proof}

We remark that the corresponding statement for the case in which the sequence $f_n$ is non-increasing is false in general unless at least one of the $f_n$ is integrable (in which case Corollary \ref{cor:mct} is applicable), for essentially the name reason as the fact that Proposition \ref{prop:decrease limit of measures} is false unless at least one of the $E_n$ has finite measure.

\begin{theorem}[Fatou's Lemma]\label{thm:fatou}
If $\mu$ is a non-negative measure, and if $f_n$ is a sequence of non-negative integrable functions, then \[\int_E(\liminf_nf_n)\dd\mu\leq\liminf_n\int f_n\dd\mu.\] In particular, if the right hand is finite, then $\liminf_nf_n$ is integrable.

\end{theorem}

\begin{proof}
Let $g_n=\inf\brc{f_i:n\leq i<\infty}$. Now $g_n$ is the limit as $m$ goes to $\infty$ of the decreasing sequence $\inf\brc{f_i:n\leq i\leq m}$. But from Proposition \ref{prop:sum product of meas function} it follows that the infimum of a finite number of measurable functions is measurable. Thus $g_n$ is measurable for each $n$. Since $\liminf_nf_n=\lim_ng_n$, we see that $\liminf_nf_n$ is also measurable. Now $g_n$ is a non-decreasing sequence, and so $\lim_n\int g_n\dd\mu=\int\liminf_nf_n\dd\mu$ by Corollary \ref{cor:mct extended real}. But $g_n\leq f_n$, and so $\int g_n\dd\mu\leq\int f_n\dd\mu$, for all $n$. Because the sequence $\int g_n\dd\mu$ is non-decreasing, we obtain the inequality \[\liminf_nf_n\dd\mu\geq\lim\int g_n\dd\mu=\int\liminf_nf_n\dd\mu\] as desired.
\end{proof}


\section{Exercises}
\begin{enumerate}[label=\arabic*),ref=\arabic*]
\item Show that if $f$ is a real-valued continuous function on $[a,b]$, then its Riemann integral on $[a,b]$ is equal to its Lebesgue integral on $[a,b]$.

\item\label{exer:integral of L1 func}
Let $\mu$ be Lebesgue measure or the whole real line, and let $B=L^1(\mu)$. Define a function $f$ from $[0,1]$ to $B$ by $f(t)=\idf{[t, t+1]}$ (or, more precisely, the equivalence class thereof). Show that $f$ is continuous, and so, by Exercise \ref{exer:cts on R meas} of Chapter \ref{ch:meas func}, is also measurable. Show that $f$ is $\mu$-integrable, and compute $\int_{[0,1]}f\dd\mu$ (that is, determine what function in $L^1(\mu)$ it is).

\item Show that $\lim _{a\to\infty}\int_0^a\frac{\sin x}{x}\dd x$ exists and is finite, but that $\frac{\sin x}{x}$ is not Lebesgue integrable on $[0,\infty)$.

\item\label{exer:integral compose linear functional}
Let $B$ and $B'$ be Banach spaces, and let $T$ be a bounded linear transformation from $B$ to $B'$, that is, a linear transformation such that there is a constant $K$ such that $\norm{T(b)}_{B'}\leq K\norm{b}_B$ for all $b\in B$. Let $\mu$ be a scalar measure on $(X,S)$. Show that if $f$ is a $B$-valued $\mu$-measurable function, then $T\circ f$ is a $B'$-valued $\mu$-measurable function, and that if $f$ is integrable then so is $T\circ f$ and $\int_E(T\circ f)\dd\mu=T\br{\int_Ef\dd\mu}$.

\item Show that if $f$ is a $\mu$-integrable scalar-valued function and $b\in B$, then the $B$-valued function $g(x)=f(x)b$ is $\mu$-integrable, and $\int g\dd\mu=\br{\int f\dd\mu}b$.

\item Let $f$ be a $B$-valued function which is integrable with respect to Lebesgue measure on $[0,1]$. If $\int_0^t f\dd\mu=0$ for all $t\in[0,1]$, what can you conclude about $f$?

\item Show that if $f$ is a $\mu$-integrable $B$-valued function and if \[\int f(x)g(x)\dd\mu(x)=0\] for all scalar-valued ISF $g$, then $f=0$ a.e.

\item A left-continuous non-decreasing real-valued function $\alpha$ defined on $\bR$ is said to be absolutely continuous if for every $\ep>0$ there is a $\delta>0$ such that $\sum_{i=1}^n|\alpha(b_i)-\alpha(a_i)|<\ep$ for every finite disjoint collection $\brc{[a_i, b_i): i=1,\dots, n}$ of intervals for which $\sum_{i=1}^n(b_i-a_i)<\delta$. If $\alpha$ is any left-continuous non-decreasing function, show that $\alpha$ is absolutely continuous if and only if the corresponding Stieltjes-Lebesgue measure is strongly absolutely continuous with respect to Lebesgue measure.

\item Find a sequence $f_n$ of integrable functions which converges a.e. to an integrable function $f$ but such that $\int f_n\dd\mu$ does not converge to $\int f\dd\mu$. This shows that some condition such as domination by an integrable function is necessary in the Lebesgue dominated convergence theorem.

\item Evaluate $\lim_{n\to\infty}\int_{-\infty}^\infty(1+x^2+n(x\sin(1/x))^2)^{-1}\dd x$. If you knew only the theory of the Riemann integral how would you go about doing this?

\item Show that $(x^3 \sin x)/(x^{10}+\log(2+|x|))^{1/2}$ is Lebesgue integrable on $(-\infty,\infty)$

\item\label{exer:mvt}
The Mean Value Theorem: This theorem is stated in terms of convex sets, and so we will need to discuss first the rudiments of the theory of convex sets.
\begin{enumerate}[label=\alph*)]
    \item A subset $A$ of a vector space is called convex if the line segment joining any two points of $A$ lies entirely in $A$, that is, whenever $a,b\in A$ then $ta+(1-t)b \in A$ for all $0\leq t\leq1$. Show that the intersection of any collection of convex sets in convex, so every set $A$ of a vector space is contained in a smallest convex set called the convex hull of $A$, which we will denote by $c(A)$. Show that \[c(A)=\brc{\sum_{i=1}^nt_ia_i:a_i \in A,t_i\geq0,\sum_{i=1}^nt_i=1}.\] If $A$ is contained in a Banach space and if $\overline{c}(A)$ denotes the closure on $c(A)$, show that $\overline{c}(A)$ is the smallest closed convex set containing $A$. It is called the closed convex hull of A.
    
    \item The Mean Value Theorem. Let $\mu$ be a non-negative measure, and let $f$ be a $B$-valued $\mu$-integrable function. Then for every measurable set $E$ of strictly positive finite measure we have \[\mu_f(E)/\mu(E)=(1/\mu(E))\int_Ef\dd\mu\in\overline{c}(\er{f}{E})\subseteq\overline{c}(\operatorname{range}f|_E).\] (where $\er{f}{E}$ is as defined in Exercise \ref{exer:essential range} of Chapter \ref{ch:meas func}). This is the analogue of the fact that if a real valued continuous function $f$ on $[a, b]$ has values in the interval $[m, M]$, then $\frac{1}{b-a}\int_a^b f(x)\dd x\in[m, M]$. Hint: As usual, prove this first for ISF and then approximate.
    
    \item Prove conversely that if $\mu$ is a non-negative measure, $f$ is a $B$-valued $\mu$-measurable function, $E$ is a locally measurable set and $K$ is a closed subset of $B$, then if $(\mu_f(F)/\mu(F))\in K$ for every measurable $F\subseteq E$ such that $0<\mu(F)<\infty$, it follows that $\er{f}{E}\subseteq K$. In particular, if $E$ is measurable then $f(x)\in K$ for almost all $x \in E$.
\end{enumerate}

\item
\begin{enumerate}[label=\alph*),ref=\theenumi\alph*)]
    \item Show that if $f$ is a $\mu$-integrable $B$-valued function then the range of $\mu_f$, that is $\brc{\mu_f(E): E\text{ is a locally measurable set}}$, is a relatively compact subset of $B$ (that is, has compact closure).
    
    \item\label{exer:item:non atomic convex closure range}
    Show that if $\mu$ is non-atomic then the closure of the range of $\mu_f$ is convex. Hint: Use Exercise \ref{exer:non atomic measure} of Chapter \ref{ch:measures}.

    \item\label{exer:item:countable range implies closed for indef int}
    Show that if the range of $f$ is countable, then the range of $\mu_f$ is closed (so that if $\mu$ is non-atomic the range of $\mu_f$ is convex). %Hint: Use exercise 15 of Chapter 1. %FIX there is no exercise 15
    (A classical theorem of Liapcunoff states that the range itself of a non-atomic measure with values in a finite dimensional vector space is convex, but no reasonably simple proof of this fact is known. The corresponding statement is false in the infinite dimensional case. In fact in in Exercise \ref{exer:item:non close range} we will give an example of a function, $f$, which is Bochner integrable with respect to Lebesgue measure, $\mu$, but such that the range of $\mu_f$ is not convex (and so also not closed)).
    
    \item Find an example of a finite real-valued Borel measure whose range is not convex.
\end{enumerate}

\item\label{exer:avg range}
If $\mu$ is a non-negative measure on $(X, S)$ and if $m$ is a $B$-valued measure on $(X,S)$ having the property that $m(E)=0$ whenever $\mu(E)=0$ (e.g. an indefinite integral), then for every locally measurable set $E$ define a subset $A_E$ (or more precisely, $A_E(m)$) of $B$ by $A_E=\brc{m(F)/\mu(F):F\subseteq E\text{ and }0<\mu(F)<\infty}$, called the \defline{average range} of $m$ on E. Show that if $f$ is a $\mu$-integrable $B$-valued function, then $\mu_f$ locally almost has compact average range, that is, for every measurable set $E$ with $\mu(E)<\infty$ and for every $\ep>0$ there exists $F\subseteq E$ with $\mu(E\sd F)<\ep$ such that $A_F$ is relatively compact. This and the next three exercises are closely related to the Radon-Nikodym theorem which we will consider in Chapter 7. %FIX lol no ch7
Hint: Use the results of Exercise \ref{exer:precpct range seq of simple func} of Chapter \ref{ch:meas func}, the mean value theorem (Exercise \ref{exer:mvt} above), and the fact that if $A$ is compact then so is $\overline{c}(A)$. To prove this last result, show that $\overline{c}(A)$ is totally bounded if $A$ is.

\item\label{exer:cone}
A subset $C$ of a vector space is called a cone (with vertex at 0) if $tc\in C$ for every $c\in C$ and every scalar $t\geq0$. If $A$ is a subset of a vector space, then the cone generated by $A$ is defined to be $\brc{ta:a\in A\text{ and }t\text{ is a non-negative scalar}}$. Show that if $\mu$ is a non-negative measure and if $f$ is a $\mu$-integrable $B$-valued function, then locally $\mu_f$ somewhere has compact direction, that is, for every measurable set $E$ with $0<\mu(E)<\infty$ there exists $F\subseteq E$ and a compact subset $K$ of $B$ not containing $0$ such that $\mu(F)>0$ and $\mu_f(G)$ is in the cone generated by $K$ for every $G\subseteq F$.

\item\label{exer:indef int locally avg range of small diam}
With $\mu$, $f$ and $A_E$ as in Exercise \ref{exer:avg range}, show that $\mu_f$ locally somewhere has average range of small diameter, that is, for every measurable set $E$ of finite measure and every $\ep>0$ there exists a measurable set $F\subseteq E$ such that $\mu(F)>0$ and the diameter of $A_E$ is less than $\ep$. Recall that the diameter of a set $K\subseteq B$ is $\sup\brc{\norm{b-b'}:b,b'\in K}$.

\item\label{exer:non indef integral}
Let $\mu$ be Lebesgue measure on $[0,1]$, and let $B=L^1([0,1],\mu)$. Define a function, $m$, on the $\sigma$-field of Lebesgue measurable subsets of [0, 1] with values in $B$, by $m(E)=\idf{E}$ (or, more precisely, the equivalence class thereof). Show that $m$ is a measure. Compute the total variation of $m$, Is $m$ strongly absolutely $\mu$-continuous? In spite of this, show that $m$ cannot be the indefinite integral of any $\mu$-integrable function, because it does not satisfy the properties described in Exercises \ref{exer:avg range}, \ref{exer:cone} and \ref{exer:indef int locally avg range of small diam} above. Also show that the closure of the range of $m$ is not convex. (Compare with Exercise \ref{exer:item:non atomic convex closure range} above.)

\item
\begin{enumerate}[label=\alph*),ref=\theenumi\alph*)]
    \item\label{exer:item:non atom avg range is convex}
    Let $\mu$ be a non-atomic non-negative measure, and let $f$ be a $\mu$-integrable $B$-valued function. Show that for any locally measurable set $E$ the closure of $A_E(\mu_f)$ is a convex set which, in fact, is equal to $\overline{c}(\er{f}{E})$. Hint: Use Exercise \ref{exer:non atomic measure} of Chanter \ref{ch:measures}.
    
    \item Show that if $m$ is the measure of Exercise \ref{exer:non indef integral}, then the closure of $A_E(m)$ is not convex.

    \item Find a real-valued function, $f$, integrable with respect to Lebesgue measure $\mu$ such that $A_R(\mu_f)$ is not closed.

    \item Find a measure $\mu$ and a real-valued $\mu$-integrable function for which the conclusion of part \ref{exer:item:non atom avg range is convex} fails.
\end{enumerate}

\item\label{exer:other def of bochner}
There are several ways of defining the integral of a function with values in a Banach space which are more general than the Bochner integral (in the sense that certain functions which are not measurable according to our definition can still be integrated), but which are less well understood then the Bochner integral. We will not try to give a precise description of these other definitions, but in this exercise and the next one we will give some suggestive examples of one of them. %(another can be found in exercise of Chapter 5). %FIX chapter 5 has no exercise lol

Let $\mu$ be Lebesgue measure on $[0,1]$, and let $B=L^\infty([0,1],\mu)$ (see Exercise \ref{exer:essentially bounded} of Chapter \ref{ch:meas func}). Define a function, $f$, from $[0,1]$ to $B$ by $f(t)=\idf{[0,t]}$ (or, more precisely, the equivalence class thereof).

\begin{enumerate}[label=\alph*),ref=\theenumi\alph*)]
    \item Show that $f$ is not measurable.

    \item We would nevertheless like to have a meaning for $\int_Ef\dd\mu$ for every $\mu$-measurable set $E$. To obtain a feeling for what the value of this integral should be, view $f$ as a function with values in $B'=L^1([0,1],\mu)$, and show that now it is measurable, and in fact integrable (see Exercise \ref{exer:integral of L1 func}). Thus $\mu_f$ is well defined as a measure with value in $B'$.

    \item Show that $\mu_f(E)$ is a continuous function (more precisely, the equivalence class of $\mu_f(E)$ contains a continuous function) for every measurable set $E$. Thus $\mu_f$ can be viewed as having values in $C([0,1])$ and so in $B$ (since $C([0,1])$ can be identified with a closed subspace of $L^\infty([0,1],\mu)$. Show that viewed in this way $\mu_f$ is still a measure. What is its total variation?

    \item\label{exer:item:integral w duality}
    Viewing $f$ as having values in $B$, one might then expect $\int_Ef\dd\mu$ to be $\mu_f(E)$ viewed as an element of $B$. The way to justify this is as follows: For $g \in B$ and $h\in B'$ note that $gh$ is integrable, and let $\brk{g,h}=\int gh\dd\mu$. Show that for any $h\in B'$ the function $t\mapsto\brk{f(t), h}$ is measurable (so we say that $f$ is weakly measurable for the duality $\brk{,}$). Then verify that \[\brk{\mu_f(E),h}=\int_E\brk{f(t),h}\dd\mu(t)\] for every measurable set $E$ and every $h \in B'$. Hint: Show it first when $h$ is an ISF. To do this note that $g\mapsto\brk{g,h}$ is a bounded linear functional and use Exercise \ref{exer:integral compose linear functional}. (We thus say that $f$ is weakly integrable for the duality, $\brk{,}$, between $B$ and $B'$, and that the weak integral of $f$ over any set $E$ is the vector $\mu_f(E)$. Thus $\mu_f$ viewed as having values in $B$ can be considered to be the indefinite integral of the weakly integrable function $f$.)
    
    \item\label{exer:item:duality non ind integral}
    Viewing $\mu_f$ as having values in $C([0,1])$, show that it is not the indefinite integral of any Bochner integrable function with values in $C([0,1])$, (basically because thus function would have to be $f$). This illustrates the fact that a measure with values in a certain space (e.g. $C([0,1])$) which is not an indefinite integral, may be an indefinite integral in at least a weak sense if the space in which it is viewed as taking its values is enlarged (e.g. to $B$). Note that the range of $\mu_f$ is relatively compact. (Why?)
    
    \item\label{exer:item:non close range}
    Show that the range of $\mu_f$, even when viewed as being in $B'$, is not closed. Hint: Show that the function $\alpha(t)=t/2$ is not in the range of $\mu_f$ but is in the convex hull of the range of $\mu_f$ (and so must be in the closure of the range of $\mu_f$). Thus Liapounoff's theorem (see comments in Exercise \ref{exer:item:countable range implies closed for indef int}) is not true for even the indefinite integrals of continuous Bochner integrable functions. (It would be interesting to have a characterization of those functions which are in the range of $\mu_f$.)
\end{enumerate}

\item Let $\mu$ be Lebesgue measure on $[0,1]$, and let $f_1,f_2,\dots$ be the characteristic functions of the intervals $[0,1/2],[1/2,1],[0,1/3],[1/3,2/3],\allowbreak[2/3,1],[0,1/4],\dots$ (you may have discovered this sequence of sets in answer to Exercise \ref{exer:item:in measure not ae} of Chapter \ref{ch:meas func}). Let $f$ be the function on $[0,1]$ with values in $\ell^\infty$ whose value at $t\in[0,1]$ is the sequence $(f_1(t),f_2(t),\dots)$

\begin{enumerate}[label=\alph*)]
    \item Show that $f$ is not measurable. Hint: Egoroff's theorem is helpful.
    \item Nevertheless we would like to integrate $f$. In fact we would expect that \[\int_E f\dd\mu=\br{\int_Ef_1\dd\mu,\int_Ef_2\dd\mu,\dots}\] for any measurable set $E$. If we define $\mu_f(E)=\int_Ef\dd\mu$ by the above formula, show that it is a measure, and has finite total variation.
    
    \item We justify the definition of the integral of $f$ suggested above in a manner entirely analogous to the justification provided in exercise \ref{exer:item:integral w duality}. Given $a\in\ell^\infty$ and $b\in\ell^1$ note that their pointwise product, $ab$, is in $\ell^1$, and let $\brk{a,b}=\sum_{i=1}^\infty a_ib_i$. Show that for every $b\in\ell^1$ the function $t\mapsto\brk{f(t),b}$ is measurable (so we say that $f$ is weakly measurable for the duality $\brk{,}$). Then verify that \[\brk{\mu_f(E), b}=\int_E\brk{f(t),b}\dd\mu(t)\] for every measurable set $E$ and every $b\in\ell^1$. (We thus say that $f$ is weakly integrable for the duality $\brk{,}$, and that the weak integral of $f$ over any set $E$ is the vector $\mu_f(E)$.)
    
    \item A Banach space of some interest is the subspace of $\ell^\infty$ consisting of all sequences which converge to 0. This space is traditionally denoted by $c_0$. Show that the range of $\mu_f$ is contained in $c_0$, and in fact is contained in a compact subset of $c_0$. But show that $\mu_f$ cannot be the indefinite integral of a function with values in $c_0$. This gives another example of the phenomenon described in exercise \ref{exer:item:duality non ind integral}.
\end{enumerate}

\item Let $\mu$ be Lebesgue measure on $\bR$. The Fourier transform of any function $f\in\cL^1(\mu,c)$ is defined to be the function $\cF(f)$ on $\bR$ whose value at $t\in\bR$ is \[\cF(f)(t)=\int f(x)\exp(ixt)\dd x.\]
\begin{enumerate}[label=\alph*]
    \item Prove that the Fourier transform of any function in $\cL^1$ is a bounded uniformly continuous function.

    \item Prove that if both $f$ and $x\mapsto xf(x)$ are in $\cL^1$, then $\cF(f)$ is differentiable and \[\dv{\cF(f)}{t}(s)=\cF(x\mapsto ixf(x))(s).\]
\end{enumerate}

\item Let $G$ be a topological croup, such as $\bR$, and let $B$ be a Banach space. Then a representation of $G$ on $B$ is a map, $R$, from $G$ to the set of bounded invertible linear operators on $B$ (the definition of a bounded operator was given in exercise \ref{exer:integral compose linear functional} above) such that $R(s+t)=R(s)R(t)$ for all $s,t\in G$ (composition of operators). The representation $R$ is called strongly continuous if $s\mapsto R(s)b$ is a (norm) continuous function on $G$ for each $b\in B$, and it is called uniformly bounded if there is a constant $c$ such that $\norm{R(s)b}\leq c\norm{b}$ for all $t\in G$ and $b\in B$.

\begin{enumerate}[label=\alph*),ref=\theenumi\alph*)]
    \item If $G=\bR$, if $\mu$ is Lebesgue measure on $\bR$ if $B=L^1(\mu,B')$ where $B'$ is any Banach space, and if $R$ is defined by $(R(s)f)(t)=f(t-s)$, show that $R$ is a uniformly bounded continuous representation of $R$. Hint: Use exercise \ref{exer:item:translation inv} of Chapter \ref{ch:measures}.
    
    \item\label{exer:item:representation induce bounded linear op}
    Let $R$ be a uniformly bounded strongly continuous representation of $\bR$ on a Banach space $B$, and let $\mu$ be Lebesgue measure on $\bR$. For each $f\in L^1(\mu,c)$ define a function, $R_f$, on $B$ by \[R_f(b)=\int f(s)(R(s)b)\dd\mu(s)\] for every $b \in B$. Show that $R_f$ is a bounded linear operator on $B$.

    \item With $R$ and $R_f$ as in \ref{exer:item:representation induce bounded linear op}, show that there is a sequence, $f_n$, of elements of $L^1(\mu)$ of norm one such that $\lim R_{f_n} b=b$ for all $b\in B$. Hint: Try an ``approximate $\delta$-function''.
\end{enumerate}
This exercise will be further developed in the exercises of Chapter \ref{ch:Lp spaces} and \ref{ch:product measure}.
\end{enumerate}

\chapter{The $L^p$ Spaces}

Throughout this chapter we will assume as we did in Chapter \ref{ch:integration} that $\mu$ is a scalar-valued measure on $(X,S)$ unless the contrary is explicitly stated.

\section{The \texorpdfstring{$L^p$}{Lp} Spaces}

In this section we define certain vector spaces of measurable functions called the $\cL^p$ spaces. For $1<p<\infty$ these spaces are equipped with a seminorm analogous to the $L^1$-norm on $\cL^1$. As was the case with $\cL^1$, we will see that if we identify functions of an $\cL^p$ space which agree a.e., then we obtain a normed space, which we will denote by $L^p$.

\begin{definition}
For $0<p<\infty$ we will denote the set of all $B$-valued $\mu$-measurable functions, $f$, such that $\norm{f(\imarg)}^p$ is integrable, by $\cL^p(X,S,\mu,B)$ (or suitable abbreviations thereof). We define a real-valued function, $\norm{\imarg}_p$, on $\cL^p$ by $$\norm{f}_p=\br{\int\norm{f(x)}^p\dd|\mu|(x)}^{1/p}.$$ For $1\leq p<\infty$ this function is called the \defline{$L^p$-norm}.
\end{definition}

As this definition suggests, we will show that $\cL^p$ is a vector space for $1 \leq p<\infty$ and that $\norm{\imarg}_p$ is a seminorm on this vector space. (For $0<p<1$, $\cL^p$ is still a vector space, but $\norm{\imarg}_p$ is no longer a norm, although $\norm{\imarg}_p^p$ can be used to define an interesting metric on $\cL^p$. We will not consider this case further).

\begin{proposition}
\label{prop:zero Lp norm}
If $f\in\cL^p$, then $\norm{f}_p=0$ iff $f=0$ a.e.
\end{proposition}

\begin{proof}
If $\norm{f}_p=0$ then $\int\norm{f(x)}^p\dd|\mu|(x)=0$ so that by Proposition \ref{prop:zero L1 norm} we see that $\norm{f(x)}^p=0$ a.e. It follows that $f=0$ a.e. The converse is clear.
\end{proof}

It is considerably more difficult to prove that $\cL^p$ is a vector space and that $\norm{\imarg}_p$ is a seminorm than was the case for $\norm{\imarg}_1$. The difficulty lies in proving the triangle inequality for $\norm{\imarg}_p$. We will need the next few results in order to prove it.

\begin{lemma}
\label{lem:pq inequality}
If $p>1,q>1$ and $(1/p)+(1/q)=1$, then for any positive real numbers $r$ and $s$ we have $$rs\leq(r^p/p)+(s^q/q).$$
\end{lemma}

\begin{proof}
Define a function $k$ by $k(t)=(t^p/p)+(t^{-q}/q)$ for all $t>0$. Then the derivative, $k'$, of $k$ is $k'(t)=t^{p-1}-t^{-q-1}$. Now $k'(1)=0$, so $k$ has a critical point at $t=1$. Furthermore, it is clear that if $t>1$ then $k'(t)>0$, whereas if $0<t<1$ then $k'(t)<0$. Thus $k$ has an absolute minimum at $t=1$. But $k(1)=1$, and so for every $t>0$ we have $1\leq(t^p/p)+(t^{-q}/q)$. Setting $t=r^{1/q}/s^{1/p}$ we obtain $1\leq r^{p/q}/ps+s^{q/p}/qr$, so that $rs\leq r^{(p/q)+1}+s^{(q/p)+1}=r^p/p+s^q/q$.
\end{proof}

\begin{theorem}[H\"older's inequality]
\label{thm:holders}
Suppose that $p>1, q>1$, and $1/p+1/q=1$. If $f\in\cL^p$ and $g\in\cL^q$, where $g$ is $B$-valued and $f$ is scalar valued, then $fg\in\cL^1$ and $\norm{fg}_1\leq\norm{f}_p\norm{g}_q$.
\end{theorem}

\begin{proof}
We have seen earlier that $fg$ is measurable. If either $f$ or $g=0$ a.e. then the result is obvious, so we assume that $\norm{f}_p>0$ and $\norm{g}_q>0$. Setting $r=|f(x)|/\norm{f}_p$ and $s=\norm{g(x)}/\norm{g}_q$ in Lemma \ref{lem:pq inequality}, we obtain $|f(x)|\norm{g(x)}/\norm{f}_p\norm{g}_q\leq|f(x)|^p/p\norm{f}_p^p+\norm{g(x)}^q/q\norm{g}_q^q$. By hypothesis, the right side is integrable, and so $f g\in \cL^1$ by Theorem \ref{thm:bochner characterization integrable}. Integrating both sides, we obtain $$\norm{fg}_1/\norm{f}_p\norm{g}_q\leq\norm{f}_p^p/p\norm{f}_p^p+\norm{g}_q^q/q\norm{g}_q^q=1$$
\end{proof}

By using H\"older's inequality, we can prove the triangle inequality for the $L^p$-norm.

\begin{theorem}[Minkowski's inequality]
\label{thm:minkowski inequality}
If $1<p<\infty$ and if $f$ and $g$ are in $\cL^p$, then $f+g\in\cL^p$ and $\norm{f+g}_p\leq\norm{f}_p+\norm{g}_p$.
\end{theorem}

\begin{proof}
To see that $f+g\in\cL^p$, we note that for each $x$
\begin{align*}
\norm{f(x)+g(x)}^p&\leq(\norm{f(x)}+\norm{g(x)}^p\leq(2\max(\norm{f(x)},\norm{g(x)}))^p. \\
&=2^p \max(\norm{f(x)}^p,\norm{g(x)}^p)\leq2^p(\norm{f(x)}^p+\norm{g(x)}^p).
\end{align*}
Since the right-hand end is assumed to be integrable, it follows from Theorem \ref{thm:bochner characterization integrable} that the left-hand end is also, so that $f+g\in\cL^p$.

Now let $q=p/(p-1)$, so that $(1/p)+(1/q)=1$ and $p-1=p/q$. Then, using the fact that $\norm{f(\imarg)+g(\imarg)}^{p/q}\in\cL^q$ since $\norm{f(\imarg)+g(\imarg)}\in\cL^p$, and applying H\"older's inequality, we obtain
\begin{align*}
    \norm{f+g}_p^p&=\int\norm{f(x)+g(x)}^p\dd|\mu|(x)\\
    &=\int\norm{f(x)+g(x)}\norm{f(x)+g(x)}^{p-1}\dd|\mu|(x)\\
    &\leq\int\norm{f(x)}\norm{f(x)+g(x)}^{p/q}\dd|\mu|(x)\\
    &\quad\quad\quad\quad\quad+\int\norm{g(x)}\norm{f(x)+g(x)}^{p/q}\dd|\mu|(x)\\
    &\leq\norm{f}_p\br{\int\norm{f(x)+g(x)}^p\dd|\mu|(x)}^{1/q}\\
    &\quad\quad\quad\quad\quad+\norm{g}_p\br{\int\norm{f(x)+g(x)}^p\dd|\mu|(x)}^{1/q}\\
    &=(\norm{f}_p+\norm{g}_p)\norm{f+g}_p^{p/q}=\br{\norm{f}_p+\norm{g}_p}\norm{f+g}_p^{p-1}.
\end{align*} From this Minkowski's inequality follows immediately.
\end{proof}

It is a trivial matter to show that $\cL^p$ is closed under scalar multiplication and that $\norm{\imarg}_p$ behaves as a seminorm should with respect to scalar multiplication. Thus

\begin{corollary}
$\cL^p$ is a vector space and $\norm{\imarg}_p$ is a seminorm on $\cL^p$, for $1\leq p<\infty$.
\end{corollary}

From Proposition \ref{prop:zero Lp norm} we see that the subspace of functions in $\cL^p$ whose $L^p$-norm is 0 is exactly the subspace of functions equal to 0 a.e. Thus, exactly as in the case of $\cL^1$, the $L^p$-norm drops to an actual norm on the factor space obtained by factoring $\cL^p$ by the subspace of functions which are equal to 0 a.e., and this factor space consists exactly of the equivalence classes of functions in $\cL^p$ which agree a.e.

\begin{definition}
For $1<p<\infty$ the normed space consisting of the equivalence classes of functions in $\cL^p$ which agree a.e. will be denoted by $L^p(X,S,\mu,B)$ (or appropriate abbreviations thereof). We will denote the norm on $L^p$ again by $\norm{\imarg}_p$ and we will call it the \defline{$L^p$-norm}.
\end{definition}

When working with $L^p$ it is convenient and traditional to speak as though its elements are functions rather than equivalence classes of functions. This seldom leads to confusion, and we will follow this practice in these notes.

\section{The Completeness of the \texorpdfstring{$L^p$}{Lp} Spaces}

In this section we will prove that the $\cL^p$ spaces are complete, and obtain some additional information about them. The proof of completeness is not quite as simple as that for $\cL^1$ and requires some of the convergence theorems proved at the end of the last chapter.

\begin{definition}
Convergence with respect to the $L^p$-norm will be called \defline{convergence in $p$-mean}. A sequence of functions in $\cL^p$ which is a Cauchy sequence with respect to the $L^p$-norm will be called a \defline{$p$-mean Cauchy sequence}.
\end{definition}

In analogy with the situation for $\cL^1$ (see Proposition \ref{prop:integrable in mean implies measure}) we have:

\begin{proposition}
\label{prop:p mean implies in measure}
A $p$-mean Cauchy sequence of functions in $\cL^p$ is Cauchy in measure. A sequence of functions in $\cL^p$ which converges to a function $f\in\cL^p$ in $p$-mean also converges to $f$ in measure.
\end{proposition}

\begin{proof}
The proof is only a slight variation of the proof of Proposition \ref{prop:integrable in mean implies measure} and Lemma \ref{lem:ISF mean cauchy implies cauchy in measure}. Let $f$ be a function in $\cL^p$, and let $\ep>0$ be given. Let $E=\brc{x\in X:\norm{f(x)}\geq\ep}$. It is easily seen that $\idf{E}\leq\norm{f(\imarg)}^p/\ep^p$, and so $|\mu|(E)\leq\br{\int\norm{f(x)}^p\dd|\mu|(x)}/\ep^p$. In other words $$|\mu|(\brc{x\in X:\norm{f(x)}\geq\ep})\leq\br{\norm{f}_p/\ep}^p.$$
This inequality is known as the Markov inequality, and for the case $p=2$ it is also called the Tchebychev inequality.

Suppose now that $f_n$ is a $p$-mean Cauchy sequence of functions in $\cL^p$. Applying the Markov inequality we obtain the fact that $$|\mu|(\brc{x\in X:\norm{f_m(x)-f_n(x)}\geq\ep})\leq(\norm{f_m-f_n}_p/\ep)^p.$$
It follows immediately that the sequence $f_n$ is Cauchy in measure.

The proof for the case in which the sequence $f_n$ converges in $p$-mean is only a slight variant of the above proof.
\end{proof}

\begin{theorem}
$\cL^p(X,S,\mu,B)$, and so $L^p(X,S,\mu,B)$, is complete.
\end{theorem}

\begin{proof}
Let $f_n$ be a $p$-mean Cauchy sequence of functions in $\cL^p$. Then the sequence $f_n$ is Cauchy in measure by Proposition \ref{prop:p mean implies in measure}, and so, by the Riesz-Weyl Theorem (Theorem \ref{thm:riesz weyl}), there is a measurable function, $f$, such that the sequence $f_n$ converges to $f$ in measure and a subsequence of the $f_n$ converges to $f$ a.e. We wish to show that $f\in\cL^p$ and that the sequence $f_n$ converges to $f$ in $p$-mean. Since for the latter it suffices to show that a subsequence of the $f_n$ converges to $f$ in $p$-mean, we will assume for the rest of the proof that the sequence $f_n$ converges to $f$ a.e.

Now $\norm{f(x)-f_n(x)}^p=\lim_m\norm{f_m(x)-f_n(x)}^p$ a.e., and so we can apply Fatou's lemma (Theorem \ref{thm:fatou}) to conclude that
\begin{align*}
    \int\norm{f(x)-f_n(x)}^p\dd|\mu|(x)&\leq\lim _m\int\norm{f_m(x)-f_n(x)}^p\dd|\mu|(x)\\
    &=\lim_m\norm{f_m-f_n}_p^p.
\end{align*} Because the sequence $f_n$ is Cauchy in $p$-mean, the limit on the right is finite, and so $\norm{f(\imarg)-f_n(\imarg)}^p$ is in $L^p$. Hence $f\in\cL^p$. It is also clear from the above inequality that $f_n$ converges to $f$ in $p$-mean.
\end{proof}

\begin{corollary}
\label{cor:converge in p mean is inside Lp}
If $f_n$ is a $p$-mean Cauchy sequence of functions in $\cL^p$ which converges to a function $f$ a.e., then $f\in\cL^p$ and $f_n$ converges to $f$ in $p$-mean.
\end{corollary}

\begin{proof}
A subsequence of the $f_n$ converges in $p$-mean and a.e. to a function in $\cL^p$, which must be equal to $f$ a.e.
\end{proof}

\begin{theorem}
\label{thm:ISF dense in Lp}
The ISF are dense in $\cL^p$.
\end{theorem}

\begin{proof}
We first note that if $g$ is an ISF then so is $\norm{g(\imarg)}^p$, and so $g\in\cL^p$. Now let $f$ be any element of $\cL^p$. Since $f$ is measurable, there is a sequence, $f_n$ of simple measurable functions which converges to $f$ a.e. For each $n$ define a function $h_n$ by $$h_n(x)=\begin{cases}f_n(x)&\text{if }\norm{f_n(x)}\leq2\norm{f(x)}\\ 0&\text{if }\norm{f_n(x)}>2\norm{f(x)}\end{cases}.$$ As was the case in the proof of Theorem \ref{thm:bochner characterization integrable}, it is easily seen that $h_n$ is a sequence of ISF which converges to $f$ a.e. and which has the property that $\norm{h_n(x)}\leq2\norm{f(x)}$ for all $x$. It follows that $\norm{f(x)-h_n(x)}\leq3\norm{f(x)}$, and so $\norm{f(x)-h_n(x)}^p\leq3^p\norm{f(x)}^p$, for all $x$. Since $\norm{f(\imarg)}^p$ is integrable and since the sequence $\norm{f(\imarg)-h_n(\imarg)}^p$ converges pointwise to 0, we can apply the Lebesgue dominated convergence theorem (Theorem \ref{thm:dct}) to conclude that $\int\norm{f(x)-h_n(x)}^p\dd|\mu|(x)$ converges to zero, so that $h_n$ converges to $f$ in $p$-mean as desired.
\end{proof}

\section{Some Convergence Theorems}

In this section we show that some of the convergence theorems proved at the end of Chapter \ref{ch:integration} have easy generalizations to the $\cL^p$ spaces, and we also prove a convergence theorem which was not mentioned in Chapter \ref{ch:integration}.

\begin{theorem}[Lebesgue Dominated Convergence Theorem]
Let $f_n$ be a sequence of functions in $\cL^p(\mu,B)$ which converges a.e. to a (necessarily measurable) function $f$. If there exists a real-valued function, $g$, in $\cL^p$ such that $\norm{f_n(x)}\leq g(x)$ a.e. for each $n$, then $f\in\cL^p(B)$ and the sequence $f_n$ converges to $f$ in $p$-mean.
\end{theorem}

\begin{proof}
It is clear that $\norm{f(x)}\leq g(x)$ a.e. and so $\norm{f(x)}^p \leq g(x)^p$ a.e. Since $g(\imarg)^p$ is assumed integrable it follows from Theorem \ref{thm:bochner characterization integrable} that $\norm{f(x)}^p$ is integrable, so that $f\in\cL^p$. The rest of the argument is almost identical to that in the second half of the proof of Theorem \ref{thm:ISF dense in Lp}, and, in fact, if we had preferred we could have proved the present theorem before Theorem \ref{thm:ISF dense in Lp} and then appealed to the present theorem for the second half of the proof of Theorem \ref{thm:ISF dense in Lp}. To repeat the argument, we have $\norm{f(x)-f_n(x)}\leq2g(x)$ a.e., and so $\norm{f(x)-f_n(x)}^p\leq2^pg(x)^p$ a.e. Since $g(\imarg)^p$ is assumed integrable and since the sequence $\norm{f(\imarg)-f_n(\imarg)}^p$ converges to 0 a.e., we can apply our first version of the Lebesgue dominated convergence theorem (Theorem \ref{thm:dct}) to conclude that $\int\norm{f(x)-f_n(x)}^p\dd|\mu|(x)$ converges to zero, so that $f_n$ converges to $f$ in $p$-mean.
\end{proof}

We generalize next the Monotone Convergence Theorem.

\begin{theorem}[The Monotone Convergence Theorem]
\label{thm:mct for Lp}
Let $f_n$ be a sequence of real-valued functions in $\cL^p$ which is non-decreasing a.e. (that is, for each $n$ we have $f_n(x)\geq f_{n-1}(x)$ a.e.). If the sequence of the $L^p$-norms of the $f_n$ is bounded, that is, if there is a constant, $c$, such that $\norm{f_n}_p\leq c$ for all $n$, then $f_n$ is a $p$-mean Cauchy sequence, and there exists an $f\in\cL^p$ such that $f_n$ converges to $f$ a.e. and in $p$-mean.
\end{theorem}

\begin{proof}
For the same reason as in the proof of Theorem \ref{thm:mct} we may assume that the $f_n$ are non-negative. We will show that the present theorem is a corollary of Theorem \ref{thm:mct}. To show this we need the following lemma:

\begin{lemma}
If $r$ and $s$ are real numbers such that $r\geq s\geq 0$, then $(r-s)^p\leq r^p-s^p$.
\end{lemma}

\begin{proof}
Define a function $k$ by $k(t)=t^p-s^p-(t-s)^p$. Then the derivative of $k$ is $k'(t)=pt^{p-1}-p(t-s)^{p-1}$, which for $t\geq s$ is non-negative, so that $k$ is non-decreasing for $t\geq s$. Since $k(s)=0$, the inequality follows immediately.
\end{proof}

We now return to the proof of Theorem \ref{thm:mct for Lp}. Since the $f_n$ are non-negative and in $\cL^p$, the $f_n^p$ form a non-decreasing sequence of functions in $L^1$, and the $L^1$-norms of the $f_n^p$ are bounded above. Then by Theorem \ref{thm:mct} there is a non-negative integrable function $g$ such that $f_n^p$ converges to $g$ a.e. and in mean. Define a function $f$ by $f(x)=g(x)^{1/p}$. It is then clear that $f\in\cL^p$, that $f_n$ converges to $f$ a.e., and that $f_n^p$ converges to $f^p$ in mean. Now, since $f \geq f_n$ a.e., it follows from Lemma \ref{cor:converge in p mean is inside Lp} that $(f(x)-f_n(x))^p\leq f(x)^p-f_n(x)^p$ a.e., so that
\begin{align*}
    \norm{f-f_n}_p^p&=\int(f(x)-f_n(x))^p\dd|\mu|(x) \\
    &\leq\int[f(x)^p-f_n(x)^p]\dd|\mu|(x)=\norm{f^p-f_n^p}_1.
\end{align*}
It follows that $f_n$ converges to $f$ in $p$-mean as desired.
\end{proof}

We conclude this section with one more convergence theorem (of which the Lebesgue dominated convergence theorem is, in fact, an easy consequence. The part of this theorem involving condition \ref{thm:item:vitali thm} is called Vitali's theorem, while the part involving condition \ref{thm:item:Lp convergence thm} is frequently called ``$L^p$ convergence theorem''.

\begin{theorem}
Let $f\in\cL^p$ and let $f_n$ be a sequence of functions in $\cL^p$.
\begin{enumerate}[label=\Roman*),ref=\Roman*)]
    \item \label{thm:cond:converge p mean}
    If $f_n$ converges to $f$ in $p$-mean, then
    \begin{enumerate}[label=\arabic*),ref=\arabic*)]
        \item \label{thm:item:indef int of p power uniform abs cts}
        the indefinite integrals with respect to $|\mu|$ of the functions $\norm{f_n(\imarg)}^p$ are uniformly absolutely $\mu$-continuous, that is, for every $\ep>0$ there is a $\delta>0$ such that if $|\mu|(E)<\delta$, then $\int_E\norm{f_n(x)}^p\dd|\mu|(x)<\ep$ for all $n$, and
        
        \item \label{thm:item:p power small outside finite set}
        for any $\ep>0$ there is a fixed measurable set $E$ of finite $|\mu|$-measure such that $\int_{X\sd E}\norm{f(x)}^p\dd|\mu|(x)<\ep$ for all $n$.
    \end{enumerate}
    
    \item \label{thm:cond:converge p mean converse}
    Conversely, if \ref{thm:item:indef int of p power uniform abs cts} and \ref{thm:item:p power small outside finite set} hold and either
    \begin{enumerate}[label=3\alph*),ref=3\alph*)]
        \item \label{thm:item:vitali thm}
        $f_n$ converges to $f$ a.e., or
        \item \label{thm:item:Lp convergence thm}
        $f_n$ converges to $f$ in measure, 
    \end{enumerate}
    then $f_n$ converges to $f$ in $p$-mean.
\end{enumerate}
\end{theorem}

\begin{proof}
We prove \ref{thm:cond:converge p mean} first. To show \ref{thm:item:indef int of p power uniform abs cts}, let $\ep>0$ be given. Choose $N$ such that if $n\geq N$ then $\norm{f-f_n}_p<\ep^{1/p}/2$, and choose $\delta'>0$ such that if $|\mu|(E)<\delta'$ then $\int_E\norm{f(x)}^p\dd|\mu|(x)<\ep/2^p$, as can be done by Proposition \ref{prop:indef int mu strong abs cts}. Then if $n\geq N$ and $|\mu|(E)<\delta'$, we have 
\begin{align*}
\int_E\norm{f_n(x)}^p\dd|\mu|(x)&=\norm{f_n\idf{E}}_p^p\leq(\norm{f_n\idf{E}-f\idf{E}}_p+\norm{f\idf{E}}_p)^p\\
&\leq\br{\norm{f-f_n}_p+\br{\int_E\norm{f(x)}^p\dd|\mu|(x)}^{1/p}}^p\\
&\leq(\ep^{1/p}/2+\ep^{1/p}/2)^p=\ep.
\end{align*}
Now choose $\delta''>0$ such that if $|\mu|(E)<\delta''$, then $\int_E\norm{f_i(x)}^p\dd|\mu|(x)<\ep$ for $1\leq i<N$. If we let $\delta=\min(\delta',\delta'')$, then we see that $\delta$ has the desired property.

The proof of \ref{thm:item:p power small outside finite set} is quite similar. Again let $\ep>0$ be given, and choose $N$ such that if $n \geq N$ then $\norm{f-f_n}_p<\ep^{1/p}/2$. By Proposition \ref{prop:integral small outside finite set} we can find a measurable set $E_0$ of finite $|\mu|$-measure such that $\int_{X\sd E_0}\norm{f(x)}^p\dd|\mu|(x)<\ep/2^p$. Then, by substituting $X\sd E_0$ for $E$ in the string of inequalities used to prove \ref{thm:item:indef int of p power uniform abs cts} above, we see that $\int_{X\sd E_0}\norm{f_n(x)}^p\dd|\mu|(x)<\ep$ for all $n\geq N$. But for each $i$ for which $1\leq i<N$ we can also find a measurable set $E_i$ of finite measure such that $\int_{X\sd E_i}\norm{f_i(x)}^p\dd|\mu|(x)<\ep$. If we let $E=\bigcup_{i=0}^{N-1}E_i$, then we see that $E$ has the desired property.

We now turn to the proof of part \ref{thm:cond:converge p mean converse}. Suppose that \ref{thm:item:indef int of p power uniform abs cts}, \ref{thm:item:p power small outside finite set}, and either \ref{thm:item:vitali thm} or \ref{thm:item:Lp convergence thm} hold. We wish to show that $f_n$ converges to $f$ in $p$-mean. To do this it suffices to show that $f_n$ is Cauchy in $p$-mean, for then if condition \ref{thm:item:vitali thm} holds we can apply Corollary $\ref{cor:converge in p mean is inside Lp}$, whereas if condition \ref{thm:item:Lp convergence thm} holds, then by the Riesz-Weyl Theorem (Theorem \ref{thm:riesz weyl}) there is a subsequence for which \ref{thm:item:vitali thm} holds, and so again Corollary \ref{cor:converge in p mean is inside Lp} is applicable.

Now let $\ep>0$ be given. According to condition \ref{thm:item:p power small outside finite set} there exists a measurable set $E$ of finite measure such that $\int_{X\sd E}\norm{f_n(x)}^p\dd|\mu|(x)<(\ep/4)^p$ for all $n$. Then $\norm{f_n\idf{X\sd E}-f_m\idf{X\sd E}}_p\leq\norm{f_n\idf{X\sd E}}_p+\norm{f_m\idf{X\sd E}}_p\leq\ep/2$ for all $n$, and so it suffices to show that $\norm{f_n\idf{E}-f_m\idf{E}}_p<\varepsilon/2$ for sufficiently large $n$.

By condition \ref{thm:item:indef int of p power uniform abs cts} choose a $\delta>0$ such that if $|\mu|(F)<\delta$ then $\int_F\norm{f_n(x)}^p\dd|\mu|(x)<(\ep/8)^p$ for all $n$, so that
$\norm{f_n\idf{F}-f_m\idf{F}}_p\leq\norm{f_n\idf{F}}_p+\norm{f_m\idf{F}}_p<\ep/4$ for all $n$.

Suppose now that condition \ref{thm:item:vitali thm} holds. Since $E$ has finite measure, we can apply Egoroff's theorem (Theorem \ref{thm:egoroff}), and so there is a measurable set $F\subseteq E$ such that $|\mu|(F)<\delta$ and $f_n$ converges uniformly to $f$ on $E\sd F$. Thus we can choose an $N$ such that if $m,n\geq N$ then $\norm{f_n(x)-f_m(x)}<\ep/(4(|\mu|(E))^{1/p})$ for all $x\in E\sd F$. For each $m$ and $n$ we define a set, $F_{mn}$, by $F_{mn}=F$.

On the other hand, if condition \ref{thm:item:Lp convergence thm} holds, let $$F_{mn}=\brc{x\in E:\norm{f_n(x)-f_m(x)}\geq\ep/(4(|\mu|(E))^{1/p})}.$$ Then there exists an $N$ such that $|\mu|(F_{mn})<\delta$ whenever $m, n>N$.

Thus in either case there exists an $N$ and measurable sets $F_{mn}\subseteq E$ such that if $m,n>N$ then $|\mu|(F_{mn})<\delta$ and $\norm{f_n(x)-f_m(x)}<\ep/(4(|\mu|(E))^{1/p})$ for $x\in E\sd F_{mn}$. Then for $m, n>N$ we have
\begin{align*}
    \norm{f_n\idf{E}-f_m\idf{E}}&\leq\norm{f_n\idf{E\sd F_{mn}}-f_m\idf{E\sd F_{mn}}}_p+\norm{f_n\idf{F_{mn}}-f_m\idf{F_{mn}}}_p\\
    &\leq\br{\int_{E\sd F_{mn}}\norm{f_n(x)-f_m(x)}^p\dd|\mu|(x)}^{1/p}+\ep/4\\
    &\leq\ep/2
\end{align*}
as desired.
\end{proof}

The reader should notice some similarities between the above proof and the proof of the Lebesgue dominated convergence theorem (Theorem \ref{thm:dct}).

\section{Dense Subspaces}

In this section we prove some theorems which will enable us to conclude such results as the fact that if $\mu$ is Lebesgue measure, then the continuous $B$-valued functions of compact support are dense in $\cL^p(\mu,B)$ for $1\leq p<\infty$.

We begin by showing how to introduce a semimetric on the collection of measurable subsets of finite measure of any measure space. If $(X,S,\mu)$ is a measure space, let $S_f$ denote the ring of elements of $S$ of finite measure. If $E$ and $F$ are any two sets, then their symmetric difference, denoted by $E \symd F$, is defined to be $(E\sd F)\du(F\sd E)$. (This definition was considered in Exercise \ref{exer:null set as ideal} of Chapter \ref{ch:measures}.)

\begin{proposition}
Let $(X,S,\mu)$ be a measure space, and define a function, $d$, on pairs of elements of $S_f$ by $$d(E, F)=|\mu|(E\symd F).$$ Then $d$ is a semimetric on $S_f$.
\end{proposition}

\begin{proof}
Only the triangle inequality is not entirely obvious. But suppose elements $E,F$ and $G$ of $S$ are given. Then $(E\sd F)\subseteq(E\sd G)\cup(G\sd F)$ and $(F\sd E)\subseteq(F\sd G)\cup(G\sd E)$, so that $E\symd F\subseteq(E\symd G)\cup(G\symd F)$. Thus $d(E,F)\leq d(E,G)+d(G,F)$.
\end{proof}

\begin{proposition}
\label{prop:set diff union are cts}
The operations of forming the union of any two elements of $S_f$ and of forming the difference of elements of $S_f$ are continuous with respect to the semimetric $d$.
\end{proposition}

\begin{proof}
Let $E,E_1,F,F_1$ be elements of $S_f$. Then it is easily seen that $(E\cup E_1)\sd(F\cup F_1)\subseteq(E\symd F)\cup(E_1\symd F_1)$. A similar inclusion is obtained if the roles of the $E$'s and $F$'s are interchanged. Combining these inclusions, we obtain $$d(E\cup E_1,F\cup F_1)\leq d(E,F)+d(E_1, F_1),$$
which shows that forming unions is continuous.

Similarly it is easily seen that $$(E\sd E_1)\symd(F\sd E_1)\subseteq E\symd F\text{ while }(E\sd E_1)\symd(E\sd F_1)\subseteq E_1\symd F_1.$$ An application of the triangle inequality then shows that $$d(E\sd E_1,F\sd F_1)\leq d(E, F)+d(E_1,F_1).$$
\end{proof}

The various density theorems in this section are all based on the following theorem.

\begin{theorem}
\label{thm:closure of subring}
Let $(X,S,\mu)$ be a finite measure space, and let $R$ be a subring of $S$. Then the closure of $R$ in $S$ with respect to the semimetric $d$ defined above is a $\sigma$-ring. In particular, if $R$ generates $S$, then $R$ is dense in $S$.
\end{theorem}

\begin{proof}
Let $\overline{R}$ denote the closure of $R$ in $S$. It suffices to show that $\overline{R}$ is a $\sigma$-ring. Now it follows immediately from Lemma \ref{prop:set diff union are cts} that $\overline{R}$ is a ring. It remains to show that $\overline{R}$ is closed under taking countable unions. Let $E_n$ be a sequence of elements of $\overline{R}$, and let $E=\bigcup_nE_n$. For each integer $n$ let $F_n=\bigcup_{i=1}^nE_i$. Since, as we have just seen, $\overline{R}$ is a ring, it follows that each $F_n$ is in $\overline{R}$. Furthermore, the $F_n$ increase up to $E$, and so $d(E,F_n)=|\mu|(E\sd F_n)$ goes to zero by Proposition \ref{prop:increase limit of measures} and the fact that $\mu$ is finite. Thus $E\in\overline{R}$.
\end{proof}

The key to applying the above theorem to the $L^p$ spaces is to remark that if $E$ and $F$ are any two elements of $S_f$, then $|\idf{E}(\imarg)-\idf{F}(\imarg)|=\idf{E\symd F}$, so that $\norm{\idf{E}-\idf{F}}_p=\br{\int\idf{E\symd F}\dd|\mu|}^{1/p}=(d(E,F))^{1/p}$. Using this remark we will prove:

\begin{theorem}
\label{thm:R ISF dense in Lp}
Let $(X,S,\mu)$ be a measure space, and let $R$ be a subring of $S_f$ which generates $S$. Then for $1\leq p<\infty$ the subspace of $R$-simple $B$-valued functions is dense in $\cL^p(X,S,\mu,B)$.
\end{theorem}

\begin{proof}
Since the $S$-simple integrable functions are dense in $\cL^p$ by Theorem \ref{thm:ISF dense in Lp}, it suffices to show that every $S$-simple integrable function can be approximated by $R$-simple functions. To do this it is easily seen to be sufficient to show that the characteristic function of any element of $S_f$ can be approximated in the $L^p$-norm by the characteristic functions of elements of $R$. But we show now that this follows from Theorem \ref{thm:closure of subring}.

Suppose first that $E$ is an element of $S_f$ which is contained in an element, $F$, of $R$. Let us denote by $R_F$ the subring of $R$ consisting of all those elements of $R$ which are contained in $R$. %Hm?
Then $E$ is contained in the $\sigma$-ring generated by $R_F$ according to Proposition \ref{prop:sring generated by intersection}. Restricting $\mu$ to the $\sigma$-ring generated by $R_F$, we can apply Theorem \ref{thm:closure of subring} to conclude that $E$ can be approximated by elements of $R_F$ with respect to the semimetric $d$. It then follows from the remarks immediately preceding the statement of Theorem \ref{thm:R ISF dense in Lp} that $X_E$ can be approximated in the $L^p$-norm by characteristic functions of elements of $R$.

Suppose now that $E$ is an arbitrary element of $S_f$. Since $R$ generates $S$, there is a sequence, $F_n$, of elements of $R$ whose union contains $E$. Since $R$ is a ring we may assume that the $F_n$ are increasing. Then $F_n\cap E$ converges up to $E$, and so, since $E$ has finite measure, $F_n\cap E$ converges to $E$ with respect to the semimetric $d$ (by the same argument as was used at the end of the proof of Theorem \ref{thm:closure of subring}). Thus $\idf{F_n\cap E}$ converges to $\idf{E}$ in the $L^p$-norm. But according to the results of the previous paragraph each $\idf{F_n\cap E}$ can be approximated in the $L^p$-norm by characteristic functions of elements of $R$.
\end{proof}

\begin{corollary}
\label{cor:lclosed ropen ISF dense in Lp}
Let $\mu$ be any Borel measure on the real line, and let $R$ be the ring of finite disjoint unions of finite left-closed right-open intervals. Then for $1\leq p<\infty$ the $R$-simple $B$-valued functions are dense in $\cL^p(\mu, B)$
\end{corollary}

\begin{definition}
Let $f$ be a vector-valued function on a topological space $X$. By the \defline{support} of $f$ is meant the smallest closed set outside of which the function everywhere has value $0$.
\end{definition}

\begin{theorem}
\label{thm:cpct support dense in Lp}
Let $\mu$ be any Borel measure on the real line. Then for $1\leq p<\infty$ the continuous $B$-valued functions of compact support are dense in $\cL^p(\mu, B)$
\end{theorem}

\begin{proof}
From Corollary \ref{cor:lclosed ropen ISF dense in Lp} it follows that it is sufficient to show that if $a$ and $b$ are any real numbers with $a<b$ then $\idf{[a, b)}$ can be approximated in the $L^p$-norm by continuous functions of compact support. But for each $n$ let $f_n$ be the continuous function of compact support whose graph is:
\begin{center}
\begin{tikzpicture}[
  declare function={
    func(\x)= or(\x < 2, \x > 12)     * (0)        +
              and(\x > 4, \x < 10)    * (4)        +
              and(\x >= 2, \x <= 4)   * (2*\x-4)   +
              and(\x >= 10, \x <= 12) * (-2*\x+24)
   ;
  }
]

    \begin{axis}[
        axis x line=bottom, axis y line=left,
        xtick = {0,2,4,10,12,14},
        xticklabels = {$ $,$a-(1/n)$,$a$,$b$,$b+(1/n)$,$ $},
        ytick = {0,4},
        yticklabels = {$0$,$1$},
        x post scale=1.3,
        y post scale=0.6,
        ymin=0, ymax=4.3,
        xmin=0, xmax=14, samples=8,
        domain=0:14
    ]
    \addplot [blue,thick] {func(x)};
    \end{axis}
\end{tikzpicture}
\end{center}
It is easily seen that the sequence $f_n$ converges in $p$-mean to $\idf{[a, b)}$.
\end{proof}

In Chapter 8 we will be able to generalize Corollary \ref{cor:lclosed ropen ISF dense in Lp} and Theorem \ref{thm:cpct support dense in Lp} to measures on an arbitrary locally compact space.

\chapter{Product Measures and Fubini's Theorem}

Note: Following Lemma 1.17 of Chapter $1$ we should have had:


The numbers of subsequent items in Chapter I must thus all be increased by 1 . In Chapter 6 references to items in Chapter $I$ will be numbered accordingly.

Also, pages $4.27$ and $4.28$ of Chapter 4 should be replaced by the following pages: non-decreasing, and are clearly bounded above by $c$, and so they form a Cauchy sequence. But $f_{n}-f_{m}$ is either positive a.e. or negative a.e., depending on whether or not $n$ is larger than $m$, and so we have

$$
\left\|f_{n}-f_{m}\right\|_{1}=\int f_{n}-f_{m}|\alpha| \mu||=\left|\int\left(f_{n}-f_{m}\right) d\right| \mu||=\left|\int f_{n} d\right| \mu\left|-\int f_{m} d\right| \mu|| .
$$

Thus $f_{n}$ is a mean Cauchy sequence also. Since $\mathcal{L}^{L}$ is complete, there exists an $f \in \mathcal{L}^{I}$ to which the sequence $f_{n}$ converges in mean, and hence in measure. But then by Corollary $3.27$ there must be a subsequence of the $f_{n}$ which converges to $f$ a.e. Since the $f_{n}$ are nondecreasing, it follows that the sequence $f_{n}$ itself also converges to $f$ a.e.ll

$4.43$ Corollary. If $f_{n}$ is a sequence of real-valued integrable functions which is non-decreasing a.e. and which converges a.e. to a function $f$, and if the sequence of the norms of the $f_{n}$ is bounded, then $f$ is integrable and the sequence $f_{n}$ converges to $f$ in mean. In particular, $\int f_{n} d \mu$ converges to $\int f d \mu$. The same result holds if instead the sequence $f_{n}$ is non-increasing a.e.

Proof. By Theorem $4.42$ the sequence $f_{n}$ will converge a.e. and in mean to an integrable function $h$, which must of course equal $f$ a.e. $/$ 

![](https://cdn.mathpix.com/cropped/2022_07_04_30ee69d6663b0900c88eg-03.jpg?height=244&width=1016&top_left_y=181&top_left_x=158)

Proof. We will let the reader verify that $f$ must be $\mu-$ measurable, since in the applications we make of this corollary in these notes we will always know in advance that $f$ is measurable. We remark next that if $g$ and $\mathrm{h}$ are $\mu$-measurable non-negative extended real-valued functions such that $g \geq h$ a.e. then it follows from Theorem $4.41$ and Proposition $4 .$ I 8 e that $\int g d \mu \geq \int$ haj. Thus if $\int f_{n} d \mu=\infty$ for at least one $n$, then it is 

![](https://cdn.mathpix.com/cropped/2022_07_04_30ee69d6663b0900c88eg-04.jpg?height=302&width=986&top_left_y=164&top_left_x=165)

$$
\int_{E}\left(1 i m \inf f_{n}\right) d \mu \leq \lim \inf \int f_{n} d \mu .
$$

In particular, if the right hand is finite, then $\lim \inf f_{n}$ is integrable.

Proof. Let $\left.g_{n}=i n f f_{i}: n \leq i<\infty\right\} .$ Now $g_{n}$ is the limit as $m$ goes to $\infty$ of the decreasing sequence $\inf \left\{f_{i}: n \leq i \leq m\right\}$. But $f r o m$ Proposition $3.7$ it follows that the infinum of a finite number of measurable functions is measurable. Thus $g_{n}$ is measurable for each $n$. Since lim inf $f_{n}=$ lim $g_{n}$, we see that lim inf $f_{n}$ is also measurable. Now $g_{n}$ is a non-decreasing sequence, and so Iim $\int g_{n} d \mu=\int$ lim inf $f\left(\mu_{n} b y\right.$ Corollary 4.44. But $g_{n} \leq f_{n}$, and so $\int g_{n} d \mu \leq \int f_{n} d \mu$, for all $n$. Because the sequence $\int g_{n} d \mu$ is non-decreasing, we obtain the inequality Iim inf $\int f_{n} d \mu \geq$ Iim $\int g_{n} d \mu=\int$ Iim inf $f_{n} d \mu$ 

\section{Chapter 6 - Product Measures and Fubini's Theorem}

![](https://cdn.mathpix.com/cropped/2022_07_04_30ee69d6663b0900c88eg-05.jpg?height=248&width=980&top_left_y=264&top_left_x=166)

$$
(\mu \otimes \nu)(E \times F)=\mu(E) \nu(F)
$$

for all $E \in S, F \in T$. We then derive some relations between integration with respect to $\mu \otimes v$ and integration with respect to $\mu$ and $v$.

\section{A. Product Measures}

6.I Definition. We will let SxT denote the family of subsets of $X \times Y$ of the form $E \times F$ where $E \in S$ and $F \in T$. The elements of $S \times I$ will be called the measurable rectangles of $X \times Y$. We will denote the $\sigma$-ring generated by $\mathrm{S} \times \mathrm{T}$ by $\mathrm{S} \otimes \mathrm{T} \mathrm{~ . ~ T h e ~ p a i r ~ ( X}$ product measurable space of the measurable spaces $(X, S)$ and (Y, $Y)$.

If we let $\rho$ be the function on $S \times T$ defined by $\rho(E \times F)=\mu(E) \nu(F)$ (with the convention that $0 \cdot \infty=0$ ), the objective of this section is to show that we can extend $\rho$ to a measure on $S \otimes T$. We do this by showing that the results of Chapter 1, particularly Theorem $1.33$, are applicable to the set function $p$. In order to do this it will be useful to have the definition and some properties of "sections".

6.2 Definition. If $A \subseteq X \times Y$ and if $x \in X$, then the $x$-section of $A$, which is denoted by ${ }^{x} A$, is defined by 

$$
x_{A}=\{y \in Y:(x, y) \in A\} .
$$

Note that an $\mathrm{x}$-section is a subset of $\mathrm{Y}$. Similarly, if $\mathrm{y} \in \mathrm{Y}$, then the $\underline{y \text {-section of } A \text {, denoted by } A^{\mathrm{y}} \text {, is defined by }$

$$
A^{y}=\{x \in X:(x, y) \in A\} .
$$

It is, of course, a subset of $\mathrm{X}$.

$6.3$ Proposition. Let $A_{n}$ be a sequence of subsets of $X \times Y$. Then $\left.\left.x_{n=1}^{\infty} \bigcup_{n}\right)=\bigcup_{n=1}^{\infty} x_{A_{n}}, x_{n=1}^{\infty} A_{n}\right)=\bigcap_{n=1}^{\infty} x_{A_{n}}$, and ${ }^{x}\left(A_{1}-A_{2}\right)=x_{A_{1}}-x_{A_{2}}$. Similar results hold for y-sections.

$\underline{\text { Proof. }}$. Let $\mathrm{x}^{j}$ denote the function from $Y$ to $X \times Y$ defined by $x^{j}(y)=(x, y)$. For any $A \subseteq X \times Y$ it is clear that $x_{A}=\left(x^{j}\right)^{-1}(A)$. The proposition now follows immediately from the fact that the formation of preimages preserves unions, intersections and differences.//

6.4 Definition. If $h$ is a function on $X \times Y$ and if $x \in X$, then the $\underline{x \text {-section of } h}$, which is denoted by $x, h$, is the function on $Y$ defined by $(x h)(y)=h(x, y)$. The definition of $y$-sections of $h$, which we denote by $h^{y}$, is similar.

6.5 Proposition. a) If $g$ and $h$ are B-valued functions on $X \times Y$ and if $r$ and $s$ are scalars, then for all $x \in X$ we have

$$
{ }^{x}(r g+s h)=r\left({ }^{x} g\right)+s\left({ }^{x} h\right) .
$$

b) If $h_{n}$ is a sequence of functions on $X \times Y$ which converges to $h$ pointwise, then for every $x$ the sequence $x_{h}$ converges to $x_{h}$ pointwise. c) If $A \subseteq X Y$, then ${ }^{x}\left(X_{A}\right)=\chi_{\left({ }^{x} A\right)}$.

Similar results hold for y-sections.

Proof. The proposition follows immediately from the observation that if $\mathrm{h}$ is a function on $X \times Y$ then $x_{h}=h 0_{x}^{j}$, where $x^{j}$ is the function defined in the proof of Proposition 6.3.//

6.6 Proposition. a) If $A \in S \otimes T$, then ${ }^{x} A \in T$ for every $x \in x$, and $A^{\mathrm{y}} \in S$ for every $y \in Y$.

b) If $h$ is an $(S \otimes T)$-measurable function on $X \times Y$, then $x_{h}$ is T-measurable for every $\mathrm{x} \in \mathrm{X}$ and $\mathrm{h}^{\mathrm{Y}}$ is S-measurable for every $\mathrm{y} \in \mathrm{Y}$.

Proof. Let $R$ be the set of those $A \in S \otimes T$ such that ${ }^{x_{A}} \in T$ for every $x \in X$ and $A^{y} \in S$ for every $y \in Y$. Clearly $S \times T \subseteq R$, and $R$ is a o-ring by Proposition 6.3. Therefore $R=S \otimes T$. Turning to the second part of the proposition, it follows from parts a) and c) of Proposition $6.5$ that x-sections of simple $\mathrm{S} \otimes$ T-measurable functions are simple T-measurable functions, and similarly for y-sections. The second part of the proposition now follows from part $b$ ) of Proposition $6.5$ and the definition of measurable functions.//

$\underline{6.7 \text { Lemma. }} . \quad \mathrm{S} \times \mathrm{T}$ is a semiring.

Proof. This is a consequence of the following easily verified set theoretic equalities.

a) $\bigcap_{n=1}^{\infty}\left(E_{n} \times F_{n}\right)=\left(\bigcap_{n=1}^{\infty} E_{n}\right) \times\left(\bigcap_{n=1}^{\infty} F_{n}\right)$. b) $E_{1} \times F_{1}-E_{2} \times F_{2}=\left[\left(E_{1}-E_{2}\right) \times F_{1}\right] \oplus\left[\left(E_{1} \cap E_{2}\right) \times\left(F_{1}-F_{2}\right)\right]$

$$
=\left[E_{1} \times\left(F_{1}-F_{2}\right)\right] \oplus\left[\left(E_{1}-E_{2}\right) \times\left(F_{1} \cap F_{2}\right)\right] . \|
$$

$6.8$ Theorem. Let $\mu$ and $v$ be arbitrary non-negative measures. Then the set function $\rho$ defined on $S \times T$ by $\rho(E \times F)=\mu(E) \nu(F)$, (with the convention that $0 \cdot \infty=0$ ) is a premeasure, and so extends to a non-negative measure on $\mathrm{S} \otimes \mathrm{T}$.

$\underline{\text { Proof. We must show that } \rho \text { is countably additive. Suppose that }$ $E \times F=\underset{n=1}{\oplus} E_{n} \times F_{n}$, where $E \times F$ and the $E_{n} \times F_{n}$ are in $S \times T$. We must show that $\rho(E \times F)=\sum_{n=1}^{\infty} \rho\left(E_{n} \times F_{n}\right)$. For each m let $A_{m}=\prod_{n=1}^{m} E_{n} \times F_{n}$, and let $A=E \times F$. Then $A_{m}$ increases to $A$, and so ${ }^{x} X_{A_{m}}$ increases to ${ }^{x} x_{A}$ for each $x$. If we use Definition $4.44$ for the integral of non-negative functions which are not necessarily integrable, and apply Corol.lary $4.45$, we find that $\int^{x} x_{A} d v$ increases to $\int{ }^{x} x_{A} d v$ for each $x$. Evaluating these integrals, we find that $\sum_{n=1}^{m} \chi_{E}(x) v\left(F_{n}\right)$ increases to $x_{E}(x) v(F)$ for each $x$ (where we must again use the convention that $0 \cdot \infty=0$ ). Note that these functions may now take the value $\infty$, but are clearly measurable in the sense of Definition 4.44. Furthermore it is easily seen that if we apply the definition of the integral given in Definition $4.44$, then we find that $\int\left(\chi_{E} v(F)\right) d \mu=\rho(E \times F)$ and $\int\left(\sum_{n=1}^{m} X_{E} v_{n} v\left(F_{n}\right)\right) d \mu=\sum_{m=1}^{m} \rho\left(E_{n} \times F_{m}\right)$. Applying Corollary $4.45$ again, we thus conclude that $\sum_{n=1}^{m} \rho\left(E n_{n} \times F_{m}\right)$ converges to $\rho(E \times F)$ as $m$ goes to $\infty$, so that $\rho$ is a premeasure. Theorem $1.33$ is now applicable. We invite the reader to contemplate the possibility of a proof of. Theorem $6.8$ which does not use integration theory (in the form of the Monotone Convergence Theorem) but only results of Chapter $I$.

6.9 Definition. The measure obtained from $\rho$ by applying Theorem $1.33$ is called the product of the measures $\mu$ and $v$. We will denote it by $\mu \otimes v$. The measure space $(X \times Y, S \otimes T, \mu \otimes v)$ is called the product of the measure spaces $(X, S, T)$ and $(Y, T, V)$.

In order to know that the extension of $\rho$ to a measure on $S \otimes T$ is unique (by applying Theorem 1.39) we must know that $\rho$ is $\sigma$-finite.

6.10 Proposition. If $\mu$ and $v$ are both $\sigma$-finite, then so is $\rho$.

$\underline{\text { Proof. }}$. Let $E \times F \in S \times T$. Since $\mu$ and $v$ are assumed $\sigma$-finite, we have $E=\bigcup_{m=1}^{\infty} E_{m}$ and $F=\bigcup_{n=1}^{\infty} F_{n}$ where the $E_{m}$ and the $F_{n}$ have finite measure. Then $E \times F \subseteq \bigcup_{m, n} E_{m} \times F_{n}$ and $\rho\left(E_{m} \times F{ }_{n}\right)<\infty$ for all $m$ and $n . \|$ From Proposition 6.10 and Theorem 1.39 we immediately obtain:

$\underline{6.11}$ Proposition. If $\mu$ and $v$ are $\sigma$-finite then so is $\mu \otimes v$, and it is the only measure on $S \otimes T$ which has the property that $(\mu \otimes V)(E \times F)$ $=\mu(E) \nu(F)$ for all $E \in S$ and $F \in T$.

$\underline{6.12 \text { Corollary. If } \mu \text { and } v \text { are arbitrary non-negative measures, }$ and if $E$ and $F$ are o-finite elements of $S$ and $T$ respectively, then the product of the restrictions of $\mu$ and $\nu$ to $E$ and $F$ respectively coincides with the restriction of $\mu \otimes v$ to $E \times F$. 6.13 Proposition. If $\mu$ and $v$ are both finite (or totally finite, or totally $\sigma$-finite), then so is $\mu \otimes v$.

Proof. Note that every element of $\mathrm{S} \otimes T$ is contained in a measurable rectangle (since the collection of sets having this property is clearly a o-ring containing $S \times T$ ). So if $A \in S \otimes T$, let $A \subseteq E \times F$ for some $E \in S$ and $F \in T$. Then if $\mu$ and $v$ are finite $(\mu \otimes \nu)(A) \leq(\mu \otimes V)(E \times F)$ $=\mu(E) \nu(F)<\infty$. In particular if $\mu(X)<\infty$ and $v(Y)<\infty$ then $(\mu \otimes v)(X \times Y)<\infty$, and similarly in the $\sigma$-finite case. $\|$ 

\section{B. Fubini's Theorem}

In this section we examine the relation between integration with respect to $\mu \otimes \nu$ and integration with respect to $\mu$ and $\nu$. The main result is known as Fubini's theorem.

In order to prove the main theorems we will need to make some finiteness assumptions. For this reason it is sufficient in the preliminary lemmas to assume that the measures $\mu$ and $v$ are finite.

Let $R$ denote the collection of all finite disjoint unions of elements of SXI. Then from Corollary $1.18$ we see that $R$ is just the ring generated by SXI. The first step in the proof of Fubini's theorem is to prove a version of it just for this ring.

6. I4 Lemma. Let $\mu$ and $v$ be finite. If $A$ is any element of $R$ (the ring generated by $S \times T$ ) then

1) ${ }^{x} X_{A}$ and $X_{A}^{y}$ are measurable, in fact integrable, functions with respect to $v$ and $\mu$ respectively,

2) $\int{ }^{\mathrm{x}} X_{A} d v$ and $\int x_{A}{ }^{y} d \mu$ are measurable, in fact integrable, functions of $x$ and $y$ respectively, and

3) $\int\left(\int x_{A} d v\right) d \mu(x)=\int x_{A} d(\mu \geq v)=\int\left(\int y_{A} d \mu\right) d v(y)$.

Proof. Let $A=\prod_{i=1}^{n} E_{i} \times F_{i}$ where $E_{i} \otimes F_{i} \in S \times T$ for each i. It follows that $X_{A}(x, y)=\sum_{i=1}^{n} X_{E_{i}}(x) X_{F_{i}}(y)$ for all $x$ and $y$. From this it is clear that each section is an ISF, which proves I). It is also clear . that $\int^{x} X_{A} d v=\sum_{i=1}^{n} X_{E_{i}}(x) v\left(F_{i}\right)$, and similarly for the integral with respect to $\mu$, and so they too are both ISF, proving 2). Finally, it is now clear that both iterated integrals turn out to be $\sum_{i=1}^{n} \mu\left(E_{i}\right) \cup\left(F_{i}\right)$, that is, $\int x_{A} d(\mu \otimes v) . \|$

The main step in the proof of Fubini's theorem is to extend Lemma $6.14$ to the case in which A is an arbitrary element of $\mathrm{S} \otimes \mathrm{T}$. But it is sufficient to still work with finite measures.

$6.15 \mathrm{Key}$ Lemma. Let $\mu$ and $\nu$ be finite and let $A \in S \otimes T$. Then 1) ${ }^{x} \chi_{A}$ and $x_{A}^{y}$ are measurable, in fact integrable, functions with respect to $\mu$ and $v$ respectively.

2) $\int x_{A}^{x} d \nu$ and $\int X_{A}^{y} d \mu$ are measurable, in fact integrable, functions of $x$ and $y$ respectively.

3) $\int\left(\int^{x} x_{A} d v\right) d \mu(x)=\int x_{A} d(\mu \otimes v)=\int\left(\int x_{A}^{y} d \mu\right) d v(y)$.

Proof. By Proposition 6.5c) and 6.6a), the functions in 1) are characteristic functions of measurable sets, and so, since $\mu$ and $v$ are finite measures, they are integrable, This proves property l).

Note that because the integrands in property 2) are characteristic functions and the measures are finite, the two functions in property 2) are bounded. Thus if we can prove that these functions are measurable, it will follow that they are integrable.

Let $M$ be the collection of all sets $A \in S \otimes T$ for which properties 1), 2) and 3) are true. We wish to show that $M=S \otimes T$. Let $R$ be the set of finite disjoint unions of elements of sxT. (so that by Corollary 1.18 $\vec{R}$ is the ring generated by $S \times T$ ). Then Lemma 6.14 says exactly that $R \subseteq M$. Let us investigate further the properties of $M$.

Suppose that $A_{n}$ is a sequence of elements of $M$, and that $A_{n}$ decreases to a set A. Then $A \in S \otimes T$ and so property l) holds for A. Now ${ }^{\mathrm{x}} \chi_{A_{n}}$ and $x_{A_{n}}^{y}$ will decrease to ${ }^{x} X_{A}$ and $x_{A}^{y}$, and so we can apply Corollary $4.43$ of the Monotone Convergence Theorem to conclude that

$$
\begin{aligned}
&\int{ }^{x_{A}} d v \text { decreases to } \int x^{x} x_{A} d v \\
&\int x_{A_{n}}^{y} d \mu \text { decreases to } \int x_{A}^{y} d \mu .
\end{aligned}
$$

Since property 2) holds for the sets $A_{n}$, it follows that the functions $\int^{x} x_{A} d \nu$ and $\int \chi_{A}^{y} d \mu$ are measurable. Thus property 2) holds for A also. Using Corollary $4.43$ again, we see that

$$
\begin{aligned}
&\int\left(\int x_{A_{n}}^{x_{1}} d v\right) d \mu(x) \text { decreases to } \int\left(\int^{x} x_{A} d v\right) d \mu(x) \text { and } \\
&\int\left(\int x_{A_{n}}^{y} d \mu\right) d \nu(y) \text { decreases to } \int\left(\int x_{A}^{y} d \mu\right) d v(y) .
\end{aligned}
$$

Since property 3) holds for the sets $A_{n}$, the left hand sides are just $\int x_{A_{n}} d(\mu \otimes v)$. But $\int x_{A_{n}} d(\mu \otimes v)$ decreases to $\int x_{A} a(\mu \otimes v)$ by Proposition 4.43, and so property 3) holds for A also. A similar proof shows that if instead the sequence $A_{n}$ increases to $A$, then again $A \in M$.

We are thus led to make the following definition:

6.16 Definition. A collection $M$ of sets is called a monotone class if it is closed under the formation of countable increasing unions and countable decreasing intersections. Thus what we have shown above is that the collection $M$ of the proof of Lemma $6.15$ is a monotone class which contains the ring generated by the semiring $S \times T$. It is then clear that the proof of Lemma $6.15$ will be completed once we have proven the following lemma:

6.17 Lemma (The Lerama on Monotone Classes). Let M be a monotone class and let $P$ be a semiring. If $M$ contains the ring generated by $P$, then $M$ contains the $\sigma$-ring generated by $P$.

Proof. Note first that the intersection of any collection of monotone classes is again a monotone class, so that any collection of sets is contained in a smallest monotone class, which it is said to generate. Thus we can (and willl) assume that $M$ is the monotone class generated by the ring, $R$, generated by $P$. Thus we wish to show that $M$ coincides with the $\sigma$-ring, $S(P)$, generated by $P$. For this it suffices to show that $M$ is a $\sigma$-ring, and for this it suffices to show that $M$ is closed under taking differences and finite unions (since $\bigcup_{n=1}^{\infty} F_{n}=\bigcup_{m=1}^{\infty}\left(\bigcup_{n=1}^{m} F_{n}\right.$ ) which is an increasing union).

Thus for each $E \in M$ define a subset, $K(E)$, of $M$ by $K(E)=\{F \in M: E-F, F-E$, and $E \cup F$ are in $M\}$.

What we need to show is that $K(E)=M$ for all $E \in M$. We divide the proof of this fact into 6 short steps.

1) If $F \in K(E)$ then $E \in K(F)$. This is clear from the definitions of $K(E)$ and $K(F)$. 2) If $E \in R$ then $R \subseteq K(E)$. This is clear from the definition of a ring.

3) $K(E)$ is a monotone class for every $E \in M$. To see this, suppose that $F_{n}$ is a sequence of elements of $K(E)$ which increases to a set $F$. Then $F_{n}-E$ increases to $F-E, E-F_{n}$ decreases to $E-F$, and $F_{n} \cup E$ increases to $F \cup E$, so that $F-E, E-F$ and $E \cup F$ are in $M$. Thus $F \in K(E)$. A similar argument works if instead the sequence $F_{n}$ decreases to $F$.

4) If $E \in R$ then $K(E)=M$. This follows from 2) and 3) and the fact that we have assumed that $M$ is the monotone class generated by $R$.

5) $R \subseteq K(E)$ for all $E \in M$. This follows from 4) and $I)$.

6) $K(E)=M$ for all $E \in M$. This follows from 5) and 3).//

The next step in the proof of Fubini's theorem is to extend Lemma $6.15$ to the case of non-negative measurable functions. We do not need to assume that $\mu$ and $\nu$ are finite any more, but we do need to assume that they are o-finite (see exercise ). Because we will want to use Corollary $4.43$ of the Monotone convergence theorem we need the following fact:

6.18 Proposition. If $f$ is a non-negative measurable function, then there is an increasing sequence of non-negative simple measurable functions which converges to $f$ pointwise.

$\underline{\text { Proof. }}$. Define $\mathrm{f}_{\mathrm{n}}$ by

$$
f_{n}(x)= \begin{cases}(k-1) / 2^{n} & \text { if }(k-1) / 2^{n} \leq f(x)<k / 2^{n} \quad k=1, \ldots, n 2^{n} \\ n & \text { if } n \leq f(x) .\end{cases}
$$

6.19 Theorem (Fubini, Tonelli). Let $f$ be a non-negative $S \otimes T-$ measurable function. Then the following conditions are equivalent:

1) $f$ is integrable,

2) $x_{f}$ is integrable for almost all $x \in X$, and $x \mapsto \int x_{f d \nu}$ is an (almost everywhere defined) $\mu$-integrable function on $\mathrm{X}$,

3) $f^{y}$ is integrable for almost all $y \in Y$, and $y \mapsto \int f^{y} d \mu$ is an (a)most everywhere defined) $v$-integrable function on $Y$.

If any of these three conditions holds then

$$
\int\left(\int x_{f d v}\right) d \mu=\int f d(\mu \otimes v)=\int\left(\int f^{y} d \mu\right) d v .
$$

Proof. Since $f$ is a measurable function, $C(f)$ is an $S \otimes T$-measurable set. Since we are assuming that the measures $\mu$ and $v$ are $\sigma$-finite, it is easy to see that there is a rectangle $E \times F \in S \times T$ and a sequence $E_{n} \times F$ of elements of $S \times T$ such that $C(f) \subseteq E \times F$, the sequence $E_{n} \times F_{n}$ increases to $E \times F$, and $\mu\left(E_{n}\right)<\infty, V\left(F_{n}\right)<\infty$ for all $n$. Using Proposition $6.18$ let $f_{n}$ be an increasing sequence of non-negative simple $s \otimes T$-measurable functions which converges to $f$ pointwise. If we let $g_{n}=\chi_{E_{n} \times F_{n}} f_{n}$ for each $n$, then the $g_{n}$ form a sequence of non-negative ISF increasing to $f$ pointwise.

For the moment fix $n$, and consider $\mu, v$ and $\mu \otimes \nu$ restricted to $E_{n}, F_{n}$ and $E_{n} \times F_{n}$ respectively. According to Proposition $6.13$ the product of the restrictions of $\mu$ and $\nu$ to $E_{n}$ and $F_{n}$ respectively coincides with $\mu \otimes \nu$ restricted to $E_{n} \times F_{n}$, and all these restrictions are finite. We can thus apply key Lemma $6.15$ to conclude that conditions 2) and 3) and the equality between integrals in Theorem $6.19$ are true for the characteristic function of any measurable subset of $E_{n} \times F_{n}$, in fact true even with the qualification "almost" omitted. Now since $C\left(g_{n}\right) \subseteq E_{n} \times F$ m and $g_{n}$ is just a finite sum of characteristic functions of measurable subsets of $E_{n} \times F_{n}$, it follows that conditions 2) and 3) and the equality between integrals in Theorem $6.19$ are true for. $g_{n}$ also, again even with the qualification "almost" omitted. In particular, $\int{ }_{g_{n}} d \nu$ is an integrable function of $x$ for each $n$, and the $\int{ }_{g_{n}} d \nu$ form an increasing sequence of non-negative integrable functions.

Now if condition 1 ) holds, then

$$
\int\left(\int \mathrm{g}_{\mathrm{n}} d v\right) d \mu=\int \mathrm{g}_{\mathrm{n}} d(\mu \otimes v) \leq \int f d(\mu \otimes v)
$$

for all n. On the other hand, if condition 2). holds, then $\int{ }^{x_{n}} d v \leq \int x_{f d v}$ a.e., and so

$$
\int\left(\int{ }_{g_{n}} d v\right) d \mu \leq \int\left(\int x_{f} d v\right) d \mu
$$

for all n. Thus in either case the sequence of the $L^{I}$-norms of the functions $\int x_{g_{n}} d v$ is bounded above, and so we can apply the Monotone Convergence Theorem (Theorem 4.42) to conclude that the sequence $\int{ }_{g_{n}} d v$ converges a.e. and in mean to an integrable function, $\mathrm{h}$. Let. $N$ be the null set off of which $\int x_{g_{n}} d v$ converges to $h(x)$ for all $x$. If $x \notin N$ then $\int{ }^{g_{n}} d \nu \leq h(x)$ for all $n$, and so the sequence of the norms of the ${ }^{x_{n}} g_{n}$ is bounded. But $\mathrm{x}_{\mathrm{g}_{\mathrm{n}}}$ increases to $\mathrm{x}_{f}$, and so $\mathrm{x}_{f}$ is integrable and $\int{ }^{x_{f d \nu}}=\lim \int{ }^{x_{n}} d \nu$ for all $x \notin N$ by Corollary 4.43. (Thus we see that $\mathbb{N}$ is exactly the set of those $x \in X$ for which ${ }_{x} f$ is not integrable.) 

![](https://cdn.mathpix.com/cropped/2022_07_04_30ee69d6663b0900c88eg-18.jpg?height=198&width=940&top_left_y=197&top_left_x=167)

longer natural to set $\int x_{\text {fav }}$ equal to $\infty$ for $x \in \mathbb{N}$. Thus even in the present setting we prefer simply to say that $\int{ }^{x}$ fd is undefined for $x \in \mathbb{N}$.

We also remark that the part of the above theorem which is usually called Tonelli's theorem is the fact that conditions 2) or 3) imply condition 1). For applications this is by far the most useful method of trying to show that a measurable function on a product space is integrable. Notice that Tonelli's theorem is immediately applicable to functions with values in a Banach space, since if $f$ is such a function and is measurable, then to show that $f$ is integrable it suffices by Theorem 4.4I to show that the nonnegative measurable function $\|f(\cdot)\|$ is integrable. A counter-example for Tonelli's theorem when one of the measures $\mu$ and $\nu$ is not $\sigma$-finite can be found in exercise .

Notation which is more commoniy used in stating Fubini's theorem than that which we have used above is as follows:

6.20 Definition. 'If $f$ is a measurable function on $X \times Y$, if $\int x_{f d v}$ is defined a.e. and measurable, and if $\int\left(\int x_{f d v) d \mu}\right.$ exists, then we will write $\int f(x, y) d \nu(y) d \mu(x)$ or $\int f d v d \mu$ instead of $\int\left(\int x_{f d v}\right) d \mu$. Under similar conditions we will write $\int f(x, y) d \mu(x) d \nu(y)$ or $\int f d \mu d \nu$ instead of $\int\left(\int f^{y} d \mu\right) d v$. These expressions are called iterated integrals to distinguish them from $\int f(\mu \otimes \nu)$, which is called the double integral of $f$.

From Fubini's theorem we obtain the following convenient characterization of null sets with respect to $\mu \otimes \nu$ : 6.21 Corollary. If. $C \in \mathbb{N}(\mu \otimes v)$ (the collection of null sets for $\mu \otimes v)$, then $\mathrm{x}_{\mathrm{C}} \in \mathbb{N}(v)$ for almost all $\mathrm{x} \in \mathrm{X}$ and $C^{\mathrm{y}} \in \mathbb{N}(\mu)$ for almost all $y \in Y$. Conversely if $A \in S \otimes T$ and ${ }^{x} A \in \mathbb{N}(\nu)$ a.e. (or $A^{y} \in \mathbb{N}(\mu)$ a.e.), then $A \in \mathbb{N}(\mu \otimes v)$.

Proof. Since $c \in \mathbb{N}(\mu \otimes v)$, there exists $A \in S \otimes T$. such that $c \subseteq A$ and $(\mu \otimes \nu)(A)=0$. By Theorem $6.19$ we have

$$
\int\left(\int x^{x} X_{A} d \nu\right) d \mu=\int x_{A} d(\mu \otimes \nu)=(\mu \otimes \nu)(A)=0 .
$$

Thus $v\left({ }^{\mathrm{x}} \mathrm{A}\right)=\int{ }^{x_{A}} X_{A} v=0$ for almost all $\mathrm{x} \in \mathrm{X}$. Since ${ }^{\mathrm{x}_{C}} \subseteq{ }^{\mathrm{x}_{A}}$ for all $\mathbf{x} \in \mathrm{X}$ we are done. The converse is clear.ll

We now come to our final version of Fubini's theorem, which involves functions with values in a Banach space.

6.22 Theorem (Fubini). If $\mu$ and $v$ are $\sigma$-finite non-negative measures and if $f$ is a B-valued $(\mu \otimes v)$-integrable function, then

1) $x_{f}$ and $f^{y}$ are integrable functions for almost all $x \in x$ and almost all y $\in Y$,

2) $\int x^{x} d v$ and $\int f^{y} d \mu$ are (almost everywhere defined) integrable functions of $\mathrm{x}$ and $\mathrm{y}$ respectively, and

3) $\iint f d v d \mu=\int f(\mu \otimes v)=\iint f d \mu d \nu$.

Proof. Since $f$ is $(\mu \otimes v)$-integrable, $\|f(\cdot)\|$ is a non-negative ( $\mu \otimes v$ )-integrable function. Choose (as in the proof of Theorem 4.4I) a sequence, $f_{n}$, of simple $S \otimes T$-measurable functions which converges to $f$ a.e. and such that $\left\|_{n}\right\| \leq 2\|f\|$ for all $n$. Then by the Lebesgue Dominated 

![](https://cdn.mathpix.com/cropped/2022_07_04_30ee69d6663b0900c88eg-21.jpg?height=196&width=912&top_left_y=156&top_left_x=166)

$$
\left\|\int_{f_{n}} d v\right\| \leq 2 \int^{x}(\|f\|) d v
$$

for all $\mathrm{x} \notin \mathrm{N}_{1}$ and $a l l \mathrm{n}$, and since by Theorem $6.19$ the right hand side is a $\mu$-integrable function of $\mathrm{x}$, we can again apply the Lebesgue Dominated Convergence Theorem to conclude that $\int \cdot{ }^{x} f d \nu$ is a $\mu$-integrable function on $X$ (proving part 2)), and that

$$
\iint f d v d \mu=\lim \iint f_{n} d v d \mu
$$

But as was noted before, Fubini's Theorem holds for the ISF, and so

$$
\int f d(\mu \otimes v)=\lim \int f_{n} d(\mu \otimes v)=\lim \iint f_{n} d v d \mu .
$$

Thus part 3) and the theorem are proved.II

![](https://cdn.mathpix.com/cropped/2022_07_04_30ee69d6663b0900c88eg-22.jpg?height=88&width=930&top_left_y=292&top_left_x=221)

$$
\mu^{+}=\frac{|\mu|+\mu}{2} \text { and } \mu^{-}=\frac{|\mu|-\mu}{2} \text {. }
$$

Then $\mu^{+}$and $\mu^{-}$are easily seen to be non-negative measures and $\mu=\mu^{+}-\mu^{-}$. Thus if both $\mu$ and $v$ are real measures it is natural to define $\mu \otimes v$ by

$$
\mu \otimes \nu=\mu^{+} \otimes \nu^{+}-\mu^{+} \otimes v^{-}-\mu^{-} \otimes v^{+}+\mu^{-} \otimes v^{-} .
$$

It is also easy to show that this is what the product measure should be, in the sense that it does the right thing on rectangles, and that by linearity Fubini's theorem is true with this definition of $\mu \otimes v$.

Similarly, if $\mu$ is a complex measure, define $\mu_{r}$ and $\mu_{i}$ by

$$
\mu_{r}=\frac{\mu+\bar{\mu}}{2} \text { and } \mu_{i}=\frac{\mu-\bar{\mu}}{2 i}
$$

(where the bar denotes complex conjugation and $\bar{\mu}$ is defined by $\left.\bar{\mu}(E)=(\mu(E))^{-}\right)$.

Then $\mu_{r}$ and $\mu_{i}$ are real measures and $\mu=\mu_{r}+i_{i_{i}}$. Thus if $\mu$ and $v$ are both complex measures, we define $\mu \otimes \nu$ by

$$
\mu \otimes v_{r} \mu_{r} \otimes v_{r}-\mu_{i} \otimes v_{i}+i\left(\mu_{r} \otimes v_{i}+\mu_{i} \otimes v_{r}\right) \text {. }
$$

Again it is easily seen that this is the appropriate product measure and that Fubini's theorem holds for it.

\chapter{The Radon-Nikodym Theorem}
\chapter{Integration on Locally Compact Spaces}

\end{document}